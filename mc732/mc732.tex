\documentclass{article}
\usepackage{tpack}


\title{MC732 - Projeto Sistemas Computacionais}
\author{Guilherme Nunes Trofino}
\authorRA{217276}
\project{Resumo Teórico}


\begin{document}
    \maketitle
\newpage

    \tableofcontents
\newpage
\section{}


\section{RISC-V Processors}
\subsection{Design Principles}
    \begin{enumerate}
        \item \textbf{Simplicity} favours regularity;
        \item \textbf{Smaller} is faster;
        \item Good design demands good \textbf{Compromises};
    \end{enumerate}

\subsection{Registers}
    \begin{table}[H]
        \centering\begin{tabular}{rl | l}\hline
            x0              & zero  & Constant Value 0\\
            x1              & ra    & Return Address\\
            x2              & sp    & Stack Pointer\\
            x3              & gp    & Global Pointer\\
            x4              & tp    & Thread Pointer\\
            x5-x7, x28-x31  & t0-t6 & Temporaries\\
            x8              & fp    & Frame Pointer\\
            x9,    x18-x27  & s0-s11& Saved Registers\\
            x10-x17         & a0-a7 & Function Arguments\\\hline
        \end{tabular}
        \caption{RISC-V Registers}
        \label{tab:riscvRegisers}
    \end{table}
\noindent Note que os temporaries registers podem ser manipulados livremente e não preocupa-se com "sujá-los".\\

\noindent Por outro lado, saved registers devem ser salvos e restaurados caso estes sejam utilizados pela função callee.

\subsection{Functions}
    \begin{enumerate}[rightmargin = \leftmargin]
        \item \textbf{Function Arguments:} Place parameters in registers x10 to x17;
        \item \textbf{Procedure Call:} Transfer control to procedure;
        \begin{scriptsize}
            \myStyleRISCV
            \begin{lstlisting}
    jal x1, ProcedureLabel  # jal ra, ProcedureLabel
            \end{lstlisting}
        \end{scriptsize}
        \item \textbf{Heap Store:} Acquire storage for procedure;
        \begin{scriptsize}
            \myStyleRISCV
            \begin{lstlisting}
    addi sp, sp, -16   # 16 = 4 * n, n: number of words
    sw   x20, 0(sp)
    sw   x19, 4(sp)
    sw   x18, 8(sp)
    sw   ra, 12(sp)
            \end{lstlisting}
        \end{scriptsize}
        \item \textbf{Function Execution:} Perform procedure's operations;
        \item \textbf{Heap Restore:} Place result in register for caller;
        \begin{scriptsize}
            \myStyleRISCV
            \begin{lstlisting}
    lw   ra, 12(sp)
    lw   x18, 8(sp)
    lw   x19, 4(sp)
    lw   x20, 0(sp)
    addi sp, sp, +16   # 16 = 4 * n, n: number of words
            \end{lstlisting}
        \end{scriptsize}
        \item \textbf{Procedure Return:} Return to place of call;
        \begin{scriptsize}
            \myStyleRISCV
            \begin{lstlisting}
    jalr x0, 0(x1)  # jalr zero, 0(ra)
            \end{lstlisting}
        \end{scriptsize}
    \end{enumerate}

\end{document}