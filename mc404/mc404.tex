\documentclass{article}

\usepackage[a4paper, hmargin={20mm, 20mm}, vmargin={25mm, 30mm}]{geometry}
\usepackage[utf8]{inputenc}
\usepackage[english, main=portuguese]{babel}

\usepackage[hidelinks]{hyperref}
\usepackage{bookmark}
\usepackage{cancel}
\usepackage{comment}

\usepackage{array}
\usepackage{indentfirst}
\usepackage{multicol}
\setlength{\multicolsep}{2pt}% 50% of original values
\usepackage{subfiles}

\usepackage{titlesec}

\usepackage{amsmath}
\usepackage{amssymb}
\usepackage{systeme}
\usepackage{float}
\usepackage{enumitem}
\usepackage[thinc]{esdiff} %parcial derivatives
\restylefloat{table}

\usepackage{graphicx}
\usepackage{subcaption}
\graphicspath{ {./images/} }

% Pacote para a definição de novas cores
\usepackage{xcolor}
% Definindo novas cores
\definecolor{darkgreen}{rgb}{0.0, 0.42, 0.24}
\definecolor{darkpurple}{rgb}{0.74, 0.2, 0.64}
\definecolor{darkblue}{rgb}{0.0, 0.28, 0.67}

%Configurando espaços entre paragrafos
%\setlength{\parskip}{0.5em}

\usepackage{chngcntr}
\counterwithin{figure}{section}
\counterwithin{table}{section}
\counterwithin{equation}{section}

%Configurando pacote de Gráficos plots
\usepackage{pgfplots}
\usepackage{tikz}

%Configurando pacote de circuitos
\usepackage{circuitikz}

\tikzset{
    mux2by1/.style={
        muxdemux, muxdemux def={
            Lh=4,
            Rh=2,
            w=2,
            NL=2,
            NB=1,
            NR=1,
            NT=0,
            inset Lh=0,
            inset Rh=0,
            inset w=0,
            square pins=0
        }
    }
}

\tikzset{
    demux2by1/.style={
        muxdemux, muxdemux def={
            Lh=2,
            Rh=4,
            w=2,
            NL=1,
            NB=1,
            NR=2,
            NT=0,
            inset Lh=0,
            inset Rh=0,
            inset w=0,
            square pins=0
        }
    }
}



%Configurando multiple files
\usepackage{filecontents}

%Configurando quotes
\usepackage{csquotes}


%Configurando layout para mostrar códigos
\usepackage{listings}
%Setting default style for code
\newcommand{\myStyleC}{
\lstset{
    language=C,                            % the language of the code
    basicstyle=\ttfamily\small,               % the size of the fonts that are used for the code
    keywordstyle=\color{darkpurple}\bfseries, %
    stringstyle=\color{darkblue},             %
    commentstyle=\color{darkgreen},           %
    morecomment=[s][\color{blue}]{/**}{*/},   %
    extendedchars=true,                       %
    showtabs=false,                           % show tabs within strings adding particular underscores
    showspaces=false,                         % show spaces adding particular underscores
    showstringspaces=false,                   % underline spaces within strings
    numbers=left,                             % where to put the line-numbers
    numberstyle=\tiny\color{gray},            % the style that is used for the line-numbers
    stepnumber=1,                             % the step between two line-numbers. If it's 1, each line will be numbered
    numbersep=5pt,                            % how far the line-numbers are from the code
    frame=single,                             % adds a frame around the code
    rulecolor=\color{black},                  % if not set, the frame-color may be changed on line-breaks within not-black text
    breaklines=true,                          % sets automatic line breaking
    backgroundcolor=\color{white},            % choose the background color
    breakatwhitespace=true,                   % sets if automatic breaks should only happen at whitespace
    breakautoindent=false,                    %
    captionpos=b,                             % sets the caption-position to bottom
    xleftmargin=0pt,                          %
    tabsize=2,                                % sets default tabsize to 2 spaces
}}

%setting bash setup
\newcommand{\myStyleBash}{
\lstset{
    language=Bash,                            % the language of the code
    basicstyle=\ttfamily\small,               % the size of the fonts that are used for the code
    keywordstyle=\color{darkpurple}\bfseries, %
    stringstyle=\color{darkblue},             %
    commentstyle=\color{darkgreen},           %
    morecomment=[s][\color{blue}]{/**}{*/},   %
    extendedchars=true,                       %
    showtabs=false,                           % show tabs within strings adding particular underscores
    showspaces=false,                         % show spaces adding particular underscores
    showstringspaces=false,                   % underline spaces within strings
    numbers=left,                             % where to put the line-numbers
    numberstyle=\tiny\color{gray},            % the style that is used for the line-numbers
    stepnumber=1,                             % the step between two line-numbers. If it's 1, each line will be numbered
    numbersep=5pt,                            % how far the line-numbers are from the code
    frame=single,                             % adds a frame around the code
    rulecolor=\color{black},                  % if not set, the frame-color may be changed on line-breaks within not-black text
    breaklines=true,                          % sets automatic line breaking
    backgroundcolor=\color{white},            % choose the background color
    breakatwhitespace=true,                   % sets if automatic breaks should only happen at whitespace
    breakautoindent=false,                    %
    captionpos=b,                             % sets the caption-position to bottom
    xleftmargin=0pt,                          %
    tabsize=2,                                % sets default tabsize to 2 spaces
}}
%setting VHDL setup
\newcommand{\myStyleVHDL}{
\lstset{
    language=VHDL,                            % the language of the code
    basicstyle=\ttfamily\small,               % the size of the fonts that are used for the code
    keywordstyle=\color{darkpurple}\bfseries, %
    stringstyle=\color{darkblue},             %
    commentstyle=\color{darkgreen},           %
    morecomment=[s][\color{blue}]{/**}{*/},   %
    extendedchars=true,                       %
    showtabs=false,                           % show tabs within strings adding particular underscores
    showspaces=false,                         % show spaces adding particular underscores
    showstringspaces=false,                   % underline spaces within strings
    numbers=left,                             % where to put the line-numbers
    numberstyle=\tiny\color{gray},            % the style that is used for the line-numbers
    stepnumber=1,                             % the step between two line-numbers. If it's 1, each line will be numbered
    numbersep=5pt,                            % how far the line-numbers are from the code
    frame=single,                             % adds a frame around the code
    rulecolor=\color{black},                  % if not set, the frame-color may be changed on line-breaks within not-black text
    breaklines=true,                          % sets automatic line breaking
    backgroundcolor=\color{white},            % choose the background color
    breakatwhitespace=true,                   % sets if automatic breaks should only happen at whitespace
    breakautoindent=false,                    %
    captionpos=b,                             % sets the caption-position to bottom
    xleftmargin=0pt,                          %
    tabsize=2,                                % sets default tabsize to 2 spaces
}}

\newcommand{\myStyleAssembly}{
\lstset{
    language=[x86masm]Assembler,                       % the language of the code
    basicstyle=\ttfamily\small,               % the size of the fonts that are used for the code
    keywordstyle=\color{darkpurple}\bfseries, %
    stringstyle=\color{darkblue},             %
    commentstyle=\color{darkgreen},           %
    morecomment=[s][\color{blue}]{/**}{*/},   %
    extendedchars=true,                       %
    showtabs=false,                           % show tabs within strings adding particular underscores
    showspaces=false,                         % show spaces adding particular underscores
    showstringspaces=false,                   % underline spaces within strings
    numbers=left,                             % where to put the line-numbers
    numberstyle=\tiny\color{gray},            % the style that is used for the line-numbers
    stepnumber=1,                             % the step between two line-numbers. If it's 1, each line will be numbered
    numbersep=5pt,                            % how far the line-numbers are from the code
    frame=single,                             % adds a frame around the code
    rulecolor=\color{black},                  % if not set, the frame-color may be changed on line-breaks within not-black text
    breaklines=true,                          % sets automatic line breaking
    backgroundcolor=\color{white},            % choose the background color
    breakatwhitespace=true,                   % sets if automatic breaks should only happen at whitespace
    breakautoindent=false,                    %
    captionpos=b,                             % sets the caption-position to bottom
    xleftmargin=0pt,                          %
    tabsize=2,                                % sets default tabsize to 2 spaces
}}

%\titleformat{<command>}[<shape>]{<format>}{<label>}{<sep>}{<before-code>}[<after-code>]
\titleformat
{\section} %comand
[block]  %shape
{\normalfont\LARGE} %format
{\thesection. } %label
{0mm} %sep
{} %before-code
[{\titlerule[0.1mm]}] %after-code

\titlespacing*{\section}{0mm}{0mm}{15mm}

\titleformat
{\subsection} %comand
[block]  %shape
{\normalfont\Large} %format
{\thesubsection. } %label
{0mm} %sep
{} %before-code
[] %after-code

\titlespacing*{\subsection}{0mm}{5mm}{2.5mm}


\begin{document}
    \begin{titlepage}
        \begin{center}
            \rule{450pt}{0.5pt}\\[4mm]
            {\Huge MC404 - Organição Básica de Computadores}\\
            \rule{450pt}{0.5pt}\\[2mm]
            {\Large Resumo Teórico}\\[200mm]
            \today\\
            \rule{250pt}{0.5pt}\\
            {\large Guilherme Nunes Trofino}\\
            {\large 217276}\\
        \end{center}
    \end{titlepage}
\newpage

    \tableofcontents
\newpage

    \paragraph{Instruções}Comandos necessários para que o simulador seja aberto a partir do terminal: 
        \begin{scriptsize}
            \myStyleBash
            \begin{lstlisting}
    cd classes/mc404/nand2tetris/tools/
    chmod 755 HardwareSimulator.sh
    ./HardwareSimulator.sh

    chmod 755 CPUEmulator.sh
    ./CPUEmulator.sh 
            \end{lstlisting}
        \end{scriptsize}
\newpage

    \section{Projeto 1}
        \subsection{Not}
            \paragraph{Definição}[Funcionamento]

            \paragraph{Tabela Verdade}Esta porta lógica possuirá a seguinte tabela verdade:
                \begin{table}[H]
                    \centering  
                    \begin{tabular}[]{c|c}\hline
                        in & out\\\hline
                        0  & 1\\
                        1  & 0\\\hline
                    \end{tabular}
                    \caption{Tabela Verdade Not}
                \end{table}

            \paragraph{Representação}Esta porta lógica pode ser expressa pelo seguinte circuito:
                \begin{figure}[H]
                    \centering
                    \begin{circuitikz}
                        \ctikzset{component text=left}
                        \draw
                        (0,0) node[not port] (myPort) {not}
                        (myPort.in 1) node [anchor = east] {in}
                        (myPort.out)  node [anchor = west] {out};
                    \end{circuitikz} 
                    \caption{Porta Lógica Not}
                \end{figure} \noindent

            \paragraph{Implementação}Esta porta lógica pode ser implementada de acordo com o seguinte código:
                \begin{scriptsize}
                    \myStyleVHDL
                    \lstinputlisting{./nand2tetris/projects/01/Not.hdl}
                \end{scriptsize}
\newpage

        \subsection{And}
            \paragraph{Definição}[Funcionamento]

            \paragraph{Tabela Verdade}Esta porta lógica possuirá a seguinte tabela verdade:
                \begin{table}[H]
                    \centering
                    \begin{tabular}[]{cc|c}\hline
                        a & b & out\\\hline
                        0 & 0 & 0\\
                        0 & 1 & 0\\
                        1 & 0 & 0\\
                        1 & 1 & 1\\\hline
                    \end{tabular}
                    \caption{Tabela Verdade And}
                \end{table}

            \paragraph{Representação}Esta porta lógica pode ser expressa pelo seguinte circuito:
                \begin{figure}[H]
                    \centering
                    \begin{circuitikz}
                        \ctikzset{component text=left}
                        \draw
                        (0,0) node[and port] (myPort) {and}
                        (myPort.in 1) node [anchor = east] {a}
                        (myPort.in 2) node [anchor = east] {b}
                        (myPort.out)  node [anchor = west] {out};
                    \end{circuitikz} 
                    \caption{Porta Lógica And}
                \end{figure} \noindent

            \paragraph{Implementação}Esta porta lógica pode ser implementada de acordo com o seguinte código:
                \begin{scriptsize}
                    \myStyleVHDL
                    \lstinputlisting{./nand2tetris/projects/01/And.hdl}
                \end{scriptsize}
\newpage

        \subsection{Or}
            \paragraph{Definição}[Funcionamento]

            \paragraph{Tabela Verdade}Esta porta lógica possuirá a seguinte tabela verdade:
                \begin{table}[H]
                    \centering
                    \begin{tabular}[]{cc|c}\hline
                        a & b & out\\\hline
                        0 & 0 & 0\\
                        0 & 1 & 1\\
                        1 & 0 & 1\\
                        1 & 1 & 1\\\hline
                    \end{tabular}
                    \caption{Tabela Verdade Or}
                \end{table}

            \paragraph{Representação}Esta porta lógica pode ser expressa pelo seguinte circuito:
                \begin{figure}[H]
                    \centering
                    \begin{circuitikz}
                        \ctikzset{component text=left}
                        \draw
                        (0,0) node[or port] (myPort) {or}
                        (myPort.in 1) node [anchor = east] {a}
                        (myPort.in 2) node [anchor = east] {b}
                        (myPort.out)  node [anchor = west] {out};
                    \end{circuitikz} 
                    \caption{Porta Lógica Or}
                \end{figure} \noindent

            \paragraph{Implementação}Esta porta lógica pode ser implementada de acordo com o seguinte código:
                \begin{scriptsize}
                    \myStyleVHDL
                    \lstinputlisting{./nand2tetris/projects/01/Or.hdl}
                \end{scriptsize}
\newpage

        \subsection{Xor}
            \paragraph{Definição}[Funcionamento]

            \paragraph{Tabela Verdade}Esta porta lógica possuirá a seguinte tabela verdade:
                \begin{table}[H]
                    \centering
                    \begin{tabular}[]{cc|c}\hline
                        a & b & out\\\hline
                        0 & 0 & 0\\
                        0 & 1 & 1\\
                        1 & 0 & 1\\
                        1 & 1 & 0\\\hline
                    \end{tabular}
                    \caption{Tabela Verdade Xor}
                \end{table}

            \paragraph{Representação}Esta porta lógica pode ser expressa pelo seguinte circuito:
                \begin{figure}[H]
                    \centering
                    \begin{circuitikz}
                        \ctikzset{component text=left}
                        \draw
                        (0,0) node[xor port] (myPort) {xor}
                        (myPort.in 1) node [anchor = east] {a}
                        (myPort.in 2) node [anchor = east] {b}
                        (myPort.out)  node [anchor = west] {out};
                    \end{circuitikz} 
                    \caption{Porta Lógica Xor}
                \end{figure} \noindent

            \paragraph{Implementação}Esta porta lógica pode ser implementada de acordo com o seguinte código:
                \begin{scriptsize}
                    \myStyleVHDL
                    \lstinputlisting{./nand2tetris/projects/01/Xor.hdl}
                \end{scriptsize}
\newpage

        \subsection{Mux}
            \paragraph{Definição}[Funcionamento]

            \paragraph{Tabela Verdade}Esta porta lógica possuirá a seguinte tabela verdade:
                \begin{table}[H]
                    \centering
                    \begin{tabular}[]{ccc|c}\hline
                        s & a & b & out\\\hline
                        0 & \textcolor{red}{0} & \textcolor{blue}{0} & \textcolor{red}{0}\\
                        0 & \textcolor{red}{0} & \textcolor{blue}{1} & \textcolor{red}{0}\\
                        0 & \textcolor{red}{1} & \textcolor{blue}{0} & \textcolor{red}{1}\\
                        0 & \textcolor{red}{1} & \textcolor{blue}{1} & \textcolor{red}{1}\\
                        1 & \textcolor{red}{0} & \textcolor{blue}{0} & \textcolor{blue}{0}\\
                        1 & \textcolor{red}{0} & \textcolor{blue}{1} & \textcolor{blue}{1}\\
                        1 & \textcolor{red}{1} & \textcolor{blue}{0} & \textcolor{blue}{0}\\
                        1 & \textcolor{red}{1} & \textcolor{blue}{1} & \textcolor{blue}{1}\\
                    \end{tabular}
                    \caption{Tabela Verdade Mux}
                \end{table}

            \paragraph{Representação}Esta porta lógica pode ser expressa pelo seguinte circuito:
                \begin{figure}[H]
                    \centering
                    \begin{circuitikz}
                        \ctikzset{component text=left}
                        \draw
                        (0,0) node[mux2by1] (myPort) {mux}
                        (myPort.lpin 1) node [anchor = east] {a}
                        (myPort.lpin 2) node [anchor = east] {b}
                        (myPort.bpin 1) node [anchor = north] {s}
                        (myPort.rpin 1)  node [anchor = west] {out};
                    \end{circuitikz} 
                    \caption{Porta Lógica Mux}
                \end{figure} \noindent

            \paragraph{Implementação}Esta porta lógica pode ser implementada de acordo com o seguinte código:
                \begin{scriptsize}
                    \myStyleVHDL
                    \lstinputlisting{./nand2tetris/projects/01/Mux.hdl}
                \end{scriptsize}
\newpage

        \subsection{DMux}
            \paragraph{Definição}[Funcionamento]

            \paragraph{Tabela Verdade}Esta porta lógica possuirá a seguinte tabela verdade:
                \begin{table}[H]
                    \centering
                    \begin{tabular}[]{cc|cc}\hline
                        s & in & a & b\\\hline
                        0 & \textcolor{blue}{0}  & \textcolor{blue}{0} & 0 \\
                        0 & \textcolor{blue}{1}  & \textcolor{blue}{1} & 0 \\
                        1 & \textcolor{blue}{0}  & 0                   & \textcolor{blue}{0}\\
                        1 & \textcolor{blue}{1}  & 0                   & \textcolor{blue}{1}\\\hline
                    \end{tabular}
                    \caption{Tabela Verdade DMux}
                \end{table}

            \paragraph{Representação}Esta porta lógica pode ser expressa pelo seguinte circuito:
                \begin{figure}[H]
                    \centering
                    \begin{circuitikz}
                        \ctikzset{component text=left}
                        \draw
                        (0,0) node[demux2by1] (myPort) {dmux}
                        (myPort.rpin 1) node [anchor = west] {a}
                        (myPort.rpin 2) node [anchor = west] {b}
                        (myPort.bpin 1) node [anchor = north] {s}
                        (myPort.lpin 1) node [anchor = east] {in};
                    \end{circuitikz} 
                    \caption{Porta Lógica DMux}
                \end{figure} \noindent

            \paragraph{Implementação}Esta porta lógica pode ser implementada de acordo com o seguinte código:
                \begin{scriptsize}
                    \myStyleVHDL
                    \lstinputlisting{./nand2tetris/projects/01/DMux.hdl}
                \end{scriptsize}
\newpage

        \subsection{Not16}
            \paragraph{Definição}[Funcionamento]

            \paragraph{Tabela Verdade}Esta porta lógica possuirá a seguinte tabela verdade:
                \begin{table}[H]
                    \centering
                    \begin{tabular}[]{c|c}\hline
                        in[i] & out[i]\\\hline
                        0     & 1\\
                        1     & 0\\\hline
                    \end{tabular}
                    \caption{Tabela Verdade Not16}
                \end{table}

            \paragraph{Representação}Esta porta lógica pode ser expressa pelo seguinte circuito:
                \begin{figure}[H]
                    \centering
                    \begin{circuitikz}
                        \ctikzset{component text=left}
                        \draw
                        (0,0) node[not port] (myPort) {\scriptsize not16}
                        (myPort.in 1) to[multiwire=16] (-1,0)
                        (-1,0) node [anchor = east] {in[16]}
                        (1,0) to[multiwire=16] (myPort.out)
                        (1,0) node [anchor = west] {out[16]};
                    \end{circuitikz} 
                    \caption{Porta Lógica Not16}
                \end{figure} \noindent

            \paragraph{Implementação}Esta porta lógica pode ser implementada de acordo com o seguinte código:
                \begin{scriptsize}
                    \myStyleVHDL
                    \lstinputlisting{./nand2tetris/projects/01/Not16.hdl}
                \end{scriptsize}
\newpage

    \subsection{And16}
        \paragraph{Definição}[Funcionamento]

        \paragraph{Tabela Verdade}Esta porta lógica possuirá a seguinte tabela verdade:
            \begin{table}[H]
                \centering
                \begin{tabular}[]{cc|c}\hline
                    a[i] & b[i] & out[i]\\\hline
                    0    & 0    & 0\\
                    0    & 1    & 0\\
                    1    & 0    & 0\\
                    1    & 1    & 1\\\hline
                \end{tabular}
                \caption{Tabela Verdade And16}
            \end{table}

        \paragraph{Representação}Esta porta lógica pode ser expressa pelo seguinte circuito:
            \begin{figure}[H]
                \centering
                \begin{circuitikz}
                    \ctikzset{component text=left}
                    \draw
                    (0,0) node[and port] (myPort) {\scriptsize and16}
                    (-2,0.275) node [anchor = east] {a}
                    (-2,0.275) to[multiwire=16] (myPort.in 1)

                    (-2,-0.275) node [anchor = east] {b}
                    (myPort.in 2) to[multiwire=16] (-2,-0.275)

                    (1,0) node [anchor = west] {out};
                \end{circuitikz}
                \caption{Porta Lógica And16}
            \end{figure} \noindent

        \paragraph{Implementação}Esta porta lógica pode ser implementada de acordo com o seguinte código:
            \begin{scriptsize}
                \myStyleVHDL
                \lstinputlisting{./nand2tetris/projects/01/And16.hdl}
            \end{scriptsize}
\newpage

        \subsection{OR16}
            \paragraph{Definição}[Funcionamento]

            \paragraph{Tabela Verdade}Esta porta lógica possuirá a seguinte tabela verdade:
                \begin{table}[H]
                    \centering
                    \begin{tabular}[]{cc|c}\hline
                        a & b & out\\\hline
                        0 & 1 & 1\\
                        0 & 1 & 1\\
                        0 & 1 & 1\\
                        0 & 1 & 1\\\hline
                    \end{tabular}
                    \caption{Tabela Verdade OR16}
                \end{table}

            \paragraph{Representação}Esta porta lógica pode ser expressa pelo seguinte circuito:
                \begin{figure}[H]
                    \centering
                    \begin{circuitikz}
                        \ctikzset{component text=left}
                        \draw
                        (0,0) node[or port] (myPort) {OR16}
                        (myPort.in 1) node [anchor = east] {a}
                        (myPort.in 2) node [anchor = east] {b}
                        (myPort.out)  node [anchor = west] {out};
                    \end{circuitikz} 
                    \caption{Porta Lógica OR16}
                \end{figure} \noindent

            \paragraph{Implementação}Esta porta lógica pode ser implementada de acordo com o seguinte código:
                \begin{scriptsize}
                    \myStyleVHDL
                    \lstinputlisting{./nand2tetris/projects/01/Or16.hdl}
                \end{scriptsize}
\newpage

    \subsection{MUX16}
        \paragraph{Definição}[Funcionamento]

        \paragraph{Tabela Verdade}Esta porta lógica possuirá a seguinte tabela verdade:
            \begin{table}[H]
                \centering
                \begin{tabular}[]{cc|c}\hline
                    a & b & out\\\hline
                    0 & 1 & 1\\
                    0 & 1 & 1\\
                    0 & 1 & 1\\
                    0 & 1 & 1\\\hline
                \end{tabular}
                \caption{Tabela Verdade MUX16}
            \end{table}

        \paragraph{Representação}Esta porta lógica pode ser expressa pelo seguinte circuito:
            \begin{figure}[H]
                \centering
                \begin{circuitikz}
                    \ctikzset{component text=left}
                    \draw
                    (0,0) node[muxdemux] (myPort) {MUX16};
                    % (myPort.in 1) node [anchor = east] {a}
                    % (myPort.in 2) node [anchor = east] {b}
                    % (myPort.out)  node [anchor = west] {out};
                \end{circuitikz} 
                \caption{Porta Lógica MUX16}
            \end{figure} \noindent

        \paragraph{Implementação}Esta porta lógica pode ser implementada de acordo com o seguinte código:
            \begin{scriptsize}
                \myStyleVHDL
                \lstinputlisting{./nand2tetris/projects/01/Mux16.hdl}
            \end{scriptsize}
\newpage

        \subsection{OR8WAY}
            \paragraph{Definição}[Funcionamento]

            \paragraph{Tabela Verdade}Esta porta lógica possuirá a seguinte tabela verdade:
                \begin{table}[H]
                    \centering
                    \begin{tabular}[]{cc|c}\hline
                        a & b & out\\\hline
                        0 & 1 & 1\\
                        0 & 1 & 1\\
                        0 & 1 & 1\\
                        0 & 1 & 1\\\hline
                    \end{tabular}
                    \caption{Tabela Verdade OR8WAY}
                \end{table}

            \paragraph{Representação}Esta porta lógica pode ser expressa pelo seguinte circuito:
                \begin{figure}[H]
                    \centering
                    \begin{circuitikz}
                        \ctikzset{component text=left}
                        \draw
                        (0,0) node[or port] (myPort) {OR8WAY}
                        (myPort.in 1) node [anchor = east] {a}
                        (myPort.in 2) node [anchor = east] {b}
                        (myPort.out)  node [anchor = west] {out};
                    \end{circuitikz} 
                    \caption{Porta Lógica OR8WAY}
                \end{figure} \noindent

            \paragraph{Implementação}Esta porta lógica pode ser implementada de acordo com o seguinte código:
                \begin{scriptsize}
                    \myStyleVHDL
                    \lstinputlisting{./nand2tetris/projects/01/Or8Way.hdl}
                \end{scriptsize}
\newpage

        \subsection{MUX4WAY16}
            \paragraph{Definição}[Funcionamento]

            \paragraph{Tabela Verdade}Esta porta lógica possuirá a seguinte tabela verdade:
                \begin{table}[H]
                    \centering
                    \begin{tabular}[]{cc|c}\hline
                        a & b & out\\\hline
                        0 & 1 & 1\\
                        0 & 1 & 1\\
                        0 & 1 & 1\\
                        0 & 1 & 1\\\hline
                    \end{tabular}
                    \caption{Tabela Verdade MUX4WAY16}
                \end{table}

            \paragraph{Representação}Esta porta lógica pode ser expressa pelo seguinte circuito:
                \begin{figure}[H]
                    \centering
                    \begin{circuitikz}
                        \ctikzset{component text=left}
                        \draw
                        (0,0) node[muxdemux] (myPort) {MUX4WAY16};
                        % (myPort.in 1) node [anchor = east] {a}
                        % (myPort.in 2) node [anchor = east] {b}
                        % (myPort.out)  node [anchor = west] {out};
                    \end{circuitikz} 
                    \caption{Porta Lógica MUX4WAY16}
                \end{figure} \noindent

            \paragraph{Implementação}Esta porta lógica pode ser implementada de acordo com o seguinte código:
                \begin{scriptsize}
                    \myStyleVHDL
                    \lstinputlisting{./nand2tetris/projects/01/Mux4Way16.hdl}
                \end{scriptsize}
\newpage

        \subsection{MUX8WAY16}
            \paragraph{Definição}[Funcionamento]

            \paragraph{Tabela Verdade}Esta porta lógica possuirá a seguinte tabela verdade:
                \begin{table}[H]
                    \centering
                    \begin{tabular}[]{cc|c}\hline
                        a & b & out\\\hline
                        0 & 1 & 1\\
                        0 & 1 & 1\\
                        0 & 1 & 1\\
                        0 & 1 & 1\\\hline
                    \end{tabular}
                    \caption{Tabela Verdade MUX8WAY16}
                \end{table}

            \paragraph{Representação}Esta porta lógica pode ser expressa pelo seguinte circuito:
                \begin{figure}[H]
                    \centering
                    \begin{circuitikz}
                        \ctikzset{component text=left}
                        \draw
                        (0,0) node[muxdemux] (myPort) {MUX8WAY16};
                        % (myPort.in 1) node [anchor = east] {a}
                        % (myPort.in 2) node [anchor = east] {b}
                        % (myPort.out)  node [anchor = west] {out};
                    \end{circuitikz} 
                    \caption{Porta Lógica MUX8WAY16}
                \end{figure} \noindent

            \paragraph{Implementação}Esta porta lógica pode ser implementada de acordo com o seguinte código:
                \begin{scriptsize}
                    \myStyleVHDL
                    \lstinputlisting{./nand2tetris/projects/01/Mux8Way16.hdl}
                \end{scriptsize}
\newpage

        \subsection{DMUX4WAY}
            \paragraph{Definição}[Funcionamento]

            \paragraph{Tabela Verdade}Esta porta lógica possuirá a seguinte tabela verdade:
                \begin{table}[H]
                    \centering
                    \begin{tabular}[]{cc|c}\hline
                        a & b & out\\\hline
                        0 & 1 & 1\\
                        0 & 1 & 1\\
                        0 & 1 & 1\\
                        0 & 1 & 1\\\hline
                    \end{tabular}
                    \caption{Tabela Verdade DMUX4WAY}
                \end{table}

            \paragraph{Representação}Esta porta lógica pode ser expressa pelo seguinte circuito:
                \begin{figure}[H]
                    \centering
                    \begin{circuitikz}
                        \ctikzset{component text=left}
                        \draw
                        (0,0) node[demux] (myPort) {DMUX4WAY};
                        % (myPort.in 1) node [anchor = east] {a}
                        % (myPort.in 2) node [anchor = east] {b}
                        % (myPort.out)  node [anchor = west] {out};
                    \end{circuitikz} 
                    \caption{Porta Lógica DMUX4WAY}
                \end{figure} \noindent

            \paragraph{Implementação}Esta porta lógica pode ser implementada de acordo com o seguinte código:
                \begin{scriptsize}
                    \myStyleVHDL
                    \lstinputlisting{./nand2tetris/projects/01/DMux4Way.hdl}
                \end{scriptsize}
\newpage

        \subsection{DMUX8WAY}
            \paragraph{Definição}[Funcionamento]

            \paragraph{Tabela Verdade}Esta porta lógica possuirá a seguinte tabela verdade:
                \begin{table}[H]
                    \centering
                    \begin{tabular}[]{cc|c}\hline
                        a & b & out\\\hline
                        0 & 1 & 1\\
                        0 & 1 & 1\\
                        0 & 1 & 1\\
                        0 & 1 & 1\\\hline
                    \end{tabular}
                    \caption{Tabela Verdade DMUX8WAY}
                \end{table}

            \paragraph{Representação}Esta porta lógica pode ser expressa pelo seguinte circuito:
                \begin{figure}[H]
                    \centering
                    \begin{circuitikz}
                        \ctikzset{component text=left}
                        \draw
                        (0,0) node[demux] (myPort) {DMUX8WAY};
                        % (myPort.in 1) node [anchor = east] {a}
                        % (myPort.in 2) node [anchor = east] {b}
                        % (myPort.out)  node [anchor = west] {out};
                    \end{circuitikz} 
                    \caption{Porta Lógica DMUX8WAY}
                \end{figure} \noindent

            \paragraph{Implementação}Esta porta lógica pode ser implementada de acordo com o seguinte código:
                \begin{scriptsize}
                    \myStyleVHDL
                    \lstinputlisting{./nand2tetris/projects/01/DMux8Way.hdl}
                \end{scriptsize}
\newpage

    \section{Projeto 2}
        \subsection{HalfAdder}
            \paragraph{Definição}[Funcionamento]

            \paragraph{Tabela Verdade}Esta porta lógica possuirá a seguinte tabela verdade:
                \begin{table}[H]
                    \centering
                    \begin{tabular}[]{cc|cc}\hline
                        a & b & sum & out\\\hline
                        0 & 0 & 0   & 0\\
                        0 & 1 & 1   & 0\\
                        1 & 0 & 1   & 0\\
                        1 & 1 & 0   & 1\\\hline
                    \end{tabular}
                    \caption{Tabela Verdade HalfAdder}
                \end{table}

            \paragraph{Representação}Esta porta lógica pode ser expressa pelo seguinte circuito:
                \begin{figure}[H]
                    \centering
                    \begin{circuitikz}
                        \ctikzset{component text=left}
                        \draw;
                        % (0,0) node[demux] (myPort) {HalfAdder};
                        % (myPort.in 1) node [anchor = east] {a}
                        % (myPort.in 2) node [anchor = east] {b}
                        % (myPort.out)  node [anchor = west] {out};
                    \end{circuitikz} 
                    \caption{Porta Lógica HalfAdder}
                \end{figure} \noindent

            \paragraph{Implementação}Esta porta lógica pode ser implementada de acordo com o seguinte código:
                \begin{scriptsize}
                    \myStyleVHDL
                    \lstinputlisting{./nand2tetris/projects/02/HalfAdder.hdl}
                \end{scriptsize}
\newpage
    
        \subsection{FullAdder}
            \paragraph{Definição}[Funcionamento]

            \paragraph{Tabela Verdade}Esta porta lógica possuirá a seguinte tabela verdade:
                \begin{table}[H]
                    \centering
                    \begin{tabular}[]{cc|c}\hline
                        a & b & out\\\hline
                        0 & 1 & 1\\
                        0 & 1 & 1\\
                        0 & 1 & 1\\
                        0 & 1 & 1\\\hline
                    \end{tabular}
                    \caption{Tabela Verdade FullAdder}
                \end{table}

            \paragraph{Representação}Esta porta lógica pode ser expressa pelo seguinte circuito:
                \begin{figure}[H]
                    \centering
                    \begin{circuitikz}
                        \ctikzset{component text=left}
                        \draw;
                        % (0,0) node[demux] (myPort) {HalfAdder};
                        % (myPort.in 1) node [anchor = east] {a}
                        % (myPort.in 2) node [anchor = east] {b}
                        % (myPort.out)  node [anchor = west] {out};
                    \end{circuitikz} 
                    \caption{Porta Lógica FullAdder}
                \end{figure} \noindent

            \paragraph{Implementação}Esta porta lógica pode ser implementada de acordo com o seguinte código:
                \begin{scriptsize}
                    \myStyleVHDL
                    \lstinputlisting{./nand2tetris/projects/02/FullAdder.hdl}
                \end{scriptsize}
\newpage

        \subsection{Add16}
            \paragraph{Definição}[Funcionamento]

            \paragraph{Tabela Verdade}Esta porta lógica possuirá a seguinte tabela verdade:
                \begin{table}[H]
                    \centering
                    \begin{tabular}[]{cc|c}\hline
                        a & b & out\\\hline
                        0 & 1 & 1\\
                        0 & 1 & 1\\
                        0 & 1 & 1\\
                        0 & 1 & 1\\\hline
                    \end{tabular}
                    \caption{Tabela Verdade Add16}
                \end{table}

            \paragraph{Representação}Esta porta lógica pode ser expressa pelo seguinte circuito:
                \begin{figure}[H]
                    \centering
                    \begin{circuitikz}
                        \ctikzset{component text=left}
                        \draw;
                        % (0,0) node[demux] (myPort) {HalfAdder};
                        % (myPort.in 1) node [anchor = east] {a}
                        % (myPort.in 2) node [anchor = east] {b}
                        % (myPort.out)  node [anchor = west] {out};
                    \end{circuitikz} 
                    \caption{Porta Lógica Add16}
                \end{figure} \noindent

            \paragraph{Implementação}Esta porta lógica pode ser implementada de acordo com o seguinte código:
                \begin{scriptsize}
                    \myStyleVHDL
                    \lstinputlisting{./nand2tetris/projects/02/Add16.hdl}
                \end{scriptsize}
\newpage

        \subsection{Inc16}
            \paragraph{Definição}[Funcionamento]

            \paragraph{Tabela Verdade}Esta porta lógica possuirá a seguinte tabela verdade:
                \begin{table}[H]
                    \centering
                    \begin{tabular}[]{cc|c}\hline
                        a & b & out\\\hline
                        0 & 1 & 1\\
                        0 & 1 & 1\\
                        0 & 1 & 1\\
                        0 & 1 & 1\\\hline
                    \end{tabular}
                    \caption{Tabela Verdade Inc16}
                \end{table}

            \paragraph{Representação}Esta porta lógica pode ser expressa pelo seguinte circuito:
                \begin{figure}[H]
                    \centering
                    \begin{circuitikz}
                        \ctikzset{component text=left}
                        \draw;
                        % (0,0) node[demux] (myPort) {HalfAdder};
                        % (myPort.in 1) node [anchor = east] {a}
                        % (myPort.in 2) node [anchor = east] {b}
                        % (myPort.out)  node [anchor = west] {out};
                    \end{circuitikz} 
                    \caption{Porta Lógica Inc16}
                \end{figure} \noindent

            \paragraph{Implementação}Esta porta lógica pode ser implementada de acordo com o seguinte código:
                \begin{scriptsize}
                    \myStyleVHDL
                    \lstinputlisting{./nand2tetris/projects/02/Inc16.hdl}
                \end{scriptsize}
\newpage

        \subsection{ALU}
            \paragraph{Definição}[Funcionamento]

            \paragraph{Tabela Verdade}Esta porta lógica possuirá a seguinte tabela verdade:
                \begin{table}[H]
                    \centering
                    \begin{tabular}[]{cc|c}\hline
                        a & b & out\\\hline
                        0 & 1 & 1\\
                        0 & 1 & 1\\
                        0 & 1 & 1\\
                        0 & 1 & 1\\\hline
                    \end{tabular}
                    \caption{Tabela Verdade ALU}
                \end{table}

            \paragraph{Representação}Esta porta lógica pode ser expressa pelo seguinte circuito:
                \begin{figure}[H]
                    \centering
                    \begin{circuitikz}
                        \ctikzset{component text=left}
                        \draw;
                        % (0,0) node[demux] (myPort) {HalfAdder};
                        % (myPort.in 1) node [anchor = east] {a}
                        % (myPort.in 2) node [anchor = east] {b}
                        % (myPort.out)  node [anchor = west] {out};
                    \end{circuitikz} 
                    \caption{Porta Lógica ALU}
                \end{figure} \noindent

            \paragraph{Implementação}Esta porta lógica pode ser implementada de acordo com o seguinte código:
                \begin{scriptsize}
                    \myStyleVHDL
                    \lstinputlisting{./nand2tetris/projects/02/ALU.hdl}
                \end{scriptsize}
\newpage

    \section{Projeto 3}
        \subsection{Bit}
            \paragraph{Definição}[Funcionamento]

            \paragraph{Tabela Verdade}Esta porta lógica possuirá a seguinte tabela verdade:
                \begin{table}[H]
                    \centering
                    \begin{tabular}[]{cc|c}\hline
                        a & b & out\\\hline
                        0 & 1 & 1\\
                        0 & 1 & 1\\
                        0 & 1 & 1\\
                        0 & 1 & 1\\\hline
                    \end{tabular}
                    \caption{Tabela Verdade Bit}
                \end{table}

            \paragraph{Representação}Esta porta lógica pode ser expressa pelo seguinte circuito:
                \begin{figure}[H]
                    \centering
                    \begin{circuitikz}
                        \ctikzset{component text=left}
                        \draw;
                        % (0,0) node[demux] (myPort) {HalfAdder};
                        % (myPort.in 1) node [anchor = east] {a}
                        % (myPort.in 2) node [anchor = east] {b}
                        % (myPort.out)  node [anchor = west] {out};
                    \end{circuitikz} 
                    \caption{Porta Lógica Bit}
                \end{figure} \noindent

            \paragraph{Implementação}Esta porta lógica pode ser implementada de acordo com o seguinte código:
                \begin{scriptsize}
                    \myStyleVHDL
                    \lstinputlisting{./nand2tetris/projects/03/a/Bit.hdl}
                \end{scriptsize}
\newpage

        \subsection{Register}
            \paragraph{Definição}[Funcionamento]

            \paragraph{Tabela Verdade}Esta porta lógica possuirá a seguinte tabela verdade:
                \begin{table}[H]
                    \centering
                    \begin{tabular}[]{cc|c}\hline
                        a & b & out\\\hline
                        0 & 1 & 1\\
                        0 & 1 & 1\\
                        0 & 1 & 1\\
                        0 & 1 & 1\\\hline
                    \end{tabular}
                    \caption{Tabela Verdade Register}
                \end{table}

            \paragraph{Representação}Esta porta lógica pode ser expressa pelo seguinte circuito:
                \begin{figure}[H]
                    \centering
                    \begin{circuitikz}
                        \ctikzset{component text=left}
                        \draw;
                        % (0,0) node[demux] (myPort) {HalfAdder};
                        % (myPort.in 1) node [anchor = east] {a}
                        % (myPort.in 2) node [anchor = east] {b}
                        % (myPort.out)  node [anchor = west] {out};
                    \end{circuitikz} 
                    \caption{Porta Lógica Register}
                \end{figure} \noindent

            \paragraph{Implementação}Esta porta lógica pode ser implementada de acordo com o seguinte código:
                \begin{scriptsize}
                    \myStyleVHDL
                    \lstinputlisting{./nand2tetris/projects/03/a/Register.hdl}
                \end{scriptsize}
\newpage

    \subsection{RAM8}
        \paragraph{Definição}[Funcionamento]

        \paragraph{Tabela Verdade}Esta porta lógica possuirá a seguinte tabela verdade:
            \begin{table}[H]
                \centering
                \begin{tabular}[]{cc|c}\hline
                    a & b & out\\\hline
                    0 & 1 & 1\\
                    0 & 1 & 1\\
                    0 & 1 & 1\\
                    0 & 1 & 1\\\hline
                \end{tabular}
                \caption{Tabela Verdade RAM8}
            \end{table}

        \paragraph{Representação}Esta porta lógica pode ser expressa pelo seguinte circuito:
            \begin{figure}[H]
                \centering
                \begin{circuitikz}
                    \ctikzset{component text=left}
                    \draw;
                    % (0,0) node[demux] (myPort) {HalfAdder};
                    % (myPort.in 1) node [anchor = east] {a}
                    % (myPort.in 2) node [anchor = east] {b}
                    % (myPort.out)  node [anchor = west] {out};
                \end{circuitikz} 
                \caption{Porta Lógica RAM8}
            \end{figure} \noindent

        \paragraph{Implementação}Esta porta lógica pode ser implementada de acordo com o seguinte código:
            \begin{scriptsize}
                \myStyleVHDL
                \lstinputlisting{./nand2tetris/projects/03/a/RAM8.hdl}
            \end{scriptsize}
\newpage

    \subsection{RAM64}
        \paragraph{Definição}[Funcionamento]

        \paragraph{Tabela Verdade}Esta porta lógica possuirá a seguinte tabela verdade:
            \begin{table}[H]
                \centering
                \begin{tabular}[]{cc|c}\hline
                    a & b & out\\\hline
                    0 & 1 & 1\\
                    0 & 1 & 1\\
                    0 & 1 & 1\\
                    0 & 1 & 1\\\hline
                \end{tabular}
                \caption{Tabela Verdade RAM64}
            \end{table}

        \paragraph{Representação}Esta porta lógica pode ser expressa pelo seguinte circuito:
            \begin{figure}[H]
                \centering
                \begin{circuitikz}
                    \ctikzset{component text=left}
                    \draw;
                    % (0,0) node[demux] (myPort) {HalfAdder};
                    % (myPort.in 1) node [anchor = east] {a}
                    % (myPort.in 2) node [anchor = east] {b}
                    % (myPort.out)  node [anchor = west] {out};
                \end{circuitikz} 
                \caption{Porta Lógica RAM64}
            \end{figure} \noindent

        \paragraph{Implementação}Esta porta lógica pode ser implementada de acordo com o seguinte código:
            \begin{scriptsize}
                \myStyleVHDL
                \lstinputlisting{./nand2tetris/projects/03/a/RAM64.hdl}
            \end{scriptsize}
\newpage

    \subsection{RAM512}
        \paragraph{Definição}[Funcionamento]

        \paragraph{Tabela Verdade}Esta porta lógica possuirá a seguinte tabela verdade:
            \begin{table}[H]
                \centering
                \begin{tabular}[]{cc|c}\hline
                    a & b & out\\\hline
                    0 & 1 & 1\\
                    0 & 1 & 1\\
                    0 & 1 & 1\\
                    0 & 1 & 1\\\hline
                \end{tabular}
                \caption{Tabela Verdade RAM512}
            \end{table}

        \paragraph{Representação}Esta porta lógica pode ser expressa pelo seguinte circuito:
            \begin{figure}[H]
                \centering
                \begin{circuitikz}
                    \ctikzset{component text=left}
                    \draw;
                    % (0,0) node[demux] (myPort) {HalfAdder};
                    % (myPort.in 1) node [anchor = east] {a}
                    % (myPort.in 2) node [anchor = east] {b}
                    % (myPort.out)  node [anchor = west] {out};
                \end{circuitikz} 
                \caption{Porta Lógica RAM512}
            \end{figure} \noindent

        \paragraph{Implementação}Esta porta lógica pode ser implementada de acordo com o seguinte código:
            \begin{scriptsize}
                \myStyleVHDL
                \lstinputlisting{./nand2tetris/projects/03/b/RAM512.hdl}
            \end{scriptsize}
\newpage

    \subsection{RAM4K}
        \paragraph{Definição}[Funcionamento]

        \paragraph{Tabela Verdade}Esta porta lógica possuirá a seguinte tabela verdade:
            \begin{table}[H]
                \centering
                \begin{tabular}[]{cc|c}\hline
                    a & b & out\\\hline
                    0 & 1 & 1\\
                    0 & 1 & 1\\
                    0 & 1 & 1\\
                    0 & 1 & 1\\\hline
                \end{tabular}
                \caption{Tabela Verdade RAM4K}
            \end{table}

        \paragraph{Representação}Esta porta lógica pode ser expressa pelo seguinte circuito:
            \begin{figure}[H]
                \centering
                \begin{circuitikz}
                    \ctikzset{component text=left}
                    \draw;
                    % (0,0) node[demux] (myPort) {HalfAdder};
                    % (myPort.in 1) node [anchor = east] {a}
                    % (myPort.in 2) node [anchor = east] {b}
                    % (myPort.out)  node [anchor = west] {out};
                \end{circuitikz} 
                \caption{Porta Lógica RAM4K}
            \end{figure} \noindent

        \paragraph{Implementação}Esta porta lógica pode ser implementada de acordo com o seguinte código:
            \begin{scriptsize}
                \myStyleVHDL
                \lstinputlisting{./nand2tetris/projects/03/b/RAM4K.hdl}
            \end{scriptsize}
\newpage

    \subsection{RAM16K}
        \paragraph{Definição}[Funcionamento]

        \paragraph{Tabela Verdade}Esta porta lógica possuirá a seguinte tabela verdade:
            \begin{table}[H]
                \centering
                \begin{tabular}[]{cc|c}\hline
                    a & b & out\\\hline
                    0 & 1 & 1\\
                    0 & 1 & 1\\
                    0 & 1 & 1\\
                    0 & 1 & 1\\\hline
                \end{tabular}
                \caption{Tabela Verdade RAM16K}
            \end{table}

        \paragraph{Representação}Esta porta lógica pode ser expressa pelo seguinte circuito:
            \begin{figure}[H]
                \centering
                \begin{circuitikz}
                    \ctikzset{component text=left}
                    \draw;
                    % (0,0) node[demux] (myPort) {HalfAdder};
                    % (myPort.in 1) node [anchor = east] {a}
                    % (myPort.in 2) node [anchor = east] {b}
                    % (myPort.out)  node [anchor = west] {out};
                \end{circuitikz} 
                \caption{Porta Lógica RAM16K}
            \end{figure} \noindent

        \paragraph{Implementação}Esta porta lógica pode ser implementada de acordo com o seguinte código:
            \begin{scriptsize}
                \myStyleVHDL
                \lstinputlisting{./nand2tetris/projects/03/b/RAM16K.hdl}
            \end{scriptsize}
\newpage

    \subsection{PC}
        \paragraph{Definição}[Funcionamento]

        \paragraph{Tabela Verdade}Esta porta lógica possuirá a seguinte tabela verdade:
            \begin{table}[H]
                \centering
                \begin{tabular}[]{cc|c}\hline
                    a & b & out\\\hline
                    0 & 1 & 1\\
                    0 & 1 & 1\\
                    0 & 1 & 1\\
                    0 & 1 & 1\\\hline
                \end{tabular}
                \caption{Tabela Verdade PC}
            \end{table}

        \paragraph{Representação}Esta porta lógica pode ser expressa pelo seguinte circuito:
            \begin{figure}[H]
                \centering
                \begin{circuitikz}
                    \ctikzset{component text=left}
                    \draw;
                    % (0,0) node[demux] (myPort) {HalfAdder};
                    % (myPort.in 1) node [anchor = east] {a}
                    % (myPort.in 2) node [anchor = east] {b}
                    % (myPort.out)  node [anchor = west] {out};
                \end{circuitikz} 
                \caption{Porta Lógica PC}
            \end{figure} \noindent

        \paragraph{Implementação}Esta porta lógica pode ser implementada de acordo com o seguinte código:
            \begin{scriptsize}
                \myStyleVHDL
                \lstinputlisting{./nand2tetris/projects/03/a/PC.hdl}
            \end{scriptsize}
\newpage

    \section{Projeto 4}
        \subsection{Linguagem de Máquina Hack}
            \paragraph{Definição}Linguagem básica utilizada para manipular as portas lógicas implementadas anteriormente, consistindo de comandos binários com palavras de 16-bit. Há duas possíveis instruções suportadas por esta linguagem, descritas a seguir:
                \begin{enumerate}[noitemsep]
                    \item \textbf{A Instructions:} Registrador A assume o valor de \texttt{value}, representado pelo seguinte comando:
                        \begin{scriptsize}
                            \myStyleAssembly
                            \begin{lstlisting}
    @value
                            \end{lstlisting}
                        \end{scriptsize}
                    Este comando também possuirá uma representação em binário de 16-bit, descrito pela seguinte notação:
                        \begin{equation}
                            \boxed{
                            \underbrace{0}_{
                                    \text{instrução \texttt{A}}
                                }
                                \hspace{5mm}
                                \underbrace{000000000000000}_{
                                    \text{\texttt{value} representado em 15-bit}
                                }
                            }
                        \end{equation}
                    \item \textbf{C Instructions:} Realiza a operação \texttt{comp}, armazena o resultado em \texttt{dest} e poderá realizar um deslocamento de acordo com a condição imposta por \texttt{jump};
                        \begin{scriptsize}
                            \myStyleAssembly
                            \begin{lstlisting}
    dest = comp ; jump
                            \end{lstlisting}
                        \end{scriptsize}
                    Este comando também possuirá uma representação em binário de 16-bit, descrito pela seguinte notação:
                        \begin{equation}
                            \boxed{
                            \underbrace{1}_{
                                    \text{instrução \texttt{C}}
                                }
                                \hspace{5mm}
                                \underbrace{11}_{
                                    \text{não utilizado}
                                }
                                \hspace{5mm}
                                \underbrace{\textcolor{darkblue}{\text{a c1 c2 c3 c4 c5 c6}}}_{
                                    \text{\texttt{comp} bits}
                                }
                                \hspace{5mm}
                                \underbrace{\textcolor{darkpurple}{\text{d1 d2 d3}}}_{
                                    \text{\texttt{dest} bits}
                                }
                                \hspace{5mm}
                                \underbrace{\textcolor{darkgreen}{\text{j1 j2 j3}}}_{
                                    \text{\texttt{jump} bits}
                                }
                            }
                        \end{equation}
                    Onde:
                        \begin{enumerate}
                            \item \texttt{comp}: Representa todas as possíveis operações que a \texttt{ALU} poderá realizar, expressas a seguir:
                                \begin{table}[H]
                                    \centering
                                    \begin{tabular}[]{l|cc|cccccc}\hline
                                        Operação     & Resultado  &            & c1& c2& c3& c4& c5&c6\\\hline
                                        Tornar Zero  & 0          &            & 1 & 0 & 1 & 0 & 1 & 0\\
                                        Tornar Um    & 1          &            & 1 & 1 & 1 & 1 & 1 & 1\\
                                        Tornar $-$Um & $-$1       &            & 1 & 1 & 1 & 0 & 1 & 0\\
                                        Manter       & D          &            & 0 & 0 & 1 & 1 & 0 & 0\\
                                                     & A          &  M         & 1 & 1 & 0 & 0 & 0 & 0\\
                                        Negar        & !D         &            & 0 & 0 & 1 & 1 & 0 & 1\\
                                                     & !A         & !M         & 1 & 1 & 0 & 0 & 0 & 1\\
                                        Oposto       & $-$D       &            & 0 & 0 & 1 & 1 & 1 & 1\\
                                                     & $-$A       & $-$M       & 1 & 1 & 0 & 0 & 1 & 1\\
                                        Incrementar  & D+1        &            & 0 & 1 & 1 & 1 & 1 & 1\\
                                                     & A+1        & M+1        & 1 & 1 & 0 & 1 & 1 & 1\\
                                        Decrementar  & D$-$1      &            & 0 & 0 & 1 & 1 & 1 & 0\\
                                                     & A$-$1      & M$-$1      & 1 & 1 & 0 & 0 & 1 & 0\\
                                        Somar        & D+A        & D+M        & 0 & 0 & 0 & 0 & 1 & 0\\
                                        Subtrair     & D$-$A      & D$-$M      & 0 & 1 & 0 & 0 & 1 & 1\\
                                                     & A$-$D      & M$-$D      & 0 & 0 & 0 & 1 & 1 & 1\\
                                        \texttt{AND} & D\&A       & D\&M       & 0 & 0 & 0 & 0 & 0 & 0\\
                                        \texttt{OR}  & D $\mid$ A & D $\mid$ M & 0 & 1 & 0 & 1 & 0 & 1\\\hline
                                                     & a==0       & a==1       &   &   &   &   &   &  \\\hline
                                    \end{tabular}
                                    \caption{Operações \texttt{ALU}}
                                \end{table}\noindent
                            Note que estas entradas corresponderam as entradas da \texttt{ALU}.
                            \item \texttt{dest}: Representa o destino do resultado da operação realizada, expressa a seguir:
                                \begin{table}[H]
                                    \centering
                                    \begin{tabular}[]{l|l|ccc}\hline
                                        Operação      & Armazena                                       & d1& d2& d3\\\hline
                                        \texttt{null} & Descarta Resultado                             & 0 & 0 & 0\\
                                        \texttt{M}    & \texttt{RAM[A]}                                & 0 & 0 & 1\\
                                        \texttt{D}    & Registrador D                                  & 0 & 1 & 0\\
                                        \texttt{MD}   & \texttt{RAM[A]} e Registrador D                & 0 & 1 & 1\\
                                        \texttt{A}    & Registrador A                                  & 1 & 0 & 0\\
                                        \texttt{AM}   & Registrador A e \texttt{RAM[A]}                & 1 & 0 & 1\\
                                        \texttt{AD}   & Registrador A e Registrador D                  & 1 & 1 & 0\\
                                        \texttt{ADM}  & Registrador A, \texttt{RAM[A]} e Registrador D & 1 & 1 & 1\\\hline
                                    \end{tabular}
                                    \caption{Destinos de \texttt{C}}
                                \end{table}\noindent
                            Note que estas entradas corresponderam respectivamente aos loads de cada armazenador; \texttt{A.load = d1}, \texttt{D.load = d2} e \texttt{M.load = d3}.
                            \item \texttt{jump}: Representa qual a condição para o fluxo do programa, expresso a seguir:
                                \begin{table}[H]
                                    \centering
                                    \begin{tabular}[]{l|l|ccc}\hline
                                        Operação      & Resultado          & j1& j2& j3\\\hline
                                        \texttt{null} & no jump            & 0 & 0 & 0\\
                                        \texttt{JGT}  & if out $>0$    jump & 0 & 0 & 1\\
                                        \texttt{JEQ}  & if out $=0$    jump & 0 & 1 & 0\\
                                        \texttt{JGE}  & if out $\ge 0$  jump & 0 & 1 & 1\\
                                        \texttt{JLT}  & if out $<0$    jump & 1 & 0 & 0\\
                                        \texttt{JNE}  & if out $\neq 0$ jump & 1 & 0 & 1\\
                                        \texttt{JLE}  & if out $\le 0$  jump & 1 & 1 & 0\\
                                        \texttt{JMP}  & unconditional jump & 1 & 1 & 1\\\hline
                                    \end{tabular}
                                    \caption{Condições de \texttt{jump}}
                                \end{table}\noindent
                            Note que estas entradas corresponderam respectivamente aos resultados da \texttt{ALU}; \texttt{not(ng) = j1}, \texttt{zr = j2} e \texttt{ng = j3}.
                        \end{enumerate}
                \end{enumerate}
\newpage

        \subsection{Mult}
            \paragraph{Definição}

            \paragraph{Implementação}Este código pode ser implementada de acordo com o seguinte código:
                \begin{scriptsize}
                    \myStyleAssembly
                    \lstinputlisting{./nand2tetris/projects/04/mult/Mult.asm}
                \end{scriptsize}
\newpage

        \subsection{Fill}
            \paragraph{Definição}

            \paragraph{Implementação}Este código pode ser implementada de acordo com o seguinte código:
                \begin{scriptsize}
                    \myStyleAssembly
                    \lstinputlisting{./nand2tetris/projects/04/fill/Fill.asm}
                \end{scriptsize}
\newpage

    \section{Projeto 5}
        \subsection{Arquitetura de Computadores}
            \paragraph{Definição}Estruturação e construção de microcontroladores, definida teoricamente por Alan Turing e estabelecida na prática por John Von Neumann, estabelecendo diferentes blocos básicos:
                \begin{figure}[H]
                    \centering
                    \includegraphics[height = 5cm]{ima0.png}
                    \caption{Hack Processor}\label{fig:hackProcessor}
                \end{figure} \noindent
            Onde:
                \begin{enumerate}[rightmargin = \leftmargin, noitemsep]
                    \item \textbf{Memória:} Responsável por armazenar variáveis do código, separadas em:
                        \begin{enumerate}[rightmargin = \leftmargin, noitemsep]
                            \item \texttt{Data};
                            \item \texttt{Programa};
                        \end{enumerate}
                    \item \textbf{CPU:} Controle das operações realizadas por meio dos seguintes componentes:
                        \begin{enumerate}[rightmargin = \leftmargin, noitemsep]
                            \item \texttt{Registradores};
                            \item \texttt{ALU};
                        \end{enumerate}
                \end{enumerate}
            Estes componentes se comunicam através de 3 vias principais, cada qual responsável por transportar uma parcela das informações do sistema:
                \begin{enumerate}[rightmargin = \leftmargin, noitemsep]
                    \item \textbf{Address Bus:} Transporta os enderenços envolvidos no comando;
                    \item \textbf{Control Bus:} Transporta as instruções a serem executadas;
                    \item \textbf{Data Bus:} Transporta as informações a serem utilizadas;
                \end{enumerate}

            \paragraph{Fetching}Armazenar a localização da próxima instrução na entrada do endereço de memória e obter a instrução através da leitura dessa memória alocada.
\newpage

        \subsection{Memory}
            \paragraph{Definição}[Funcionamento]

            \paragraph{Implementação}Este componente pode ser implementada de acordo com o seguinte código:
                \begin{scriptsize}
                    \myStyleVHDL
                    \lstinputlisting{./nand2tetris/projects/05/Memory.hdl}
                \end{scriptsize}
\newpage

        \subsection{CPU}
            \paragraph{Definição}[Funcionamento]

            \paragraph{Implementação}Este componente pode ser implementada de acordo com o seguinte código:
                \begin{scriptsize}
                    \myStyleVHDL
                    \lstinputlisting{./nand2tetris/projects/05/CPU.hdl}
                \end{scriptsize}
\newpage

        \subsection{Computer}
            \paragraph{Definição}[Funcionamento]

            \paragraph{Implementação}Este componente pode ser implementada de acordo com o seguinte código:
                \begin{scriptsize}
                    \myStyleVHDL
                    \lstinputlisting{./nand2tetris/projects/05/Computer.hdl}
                \end{scriptsize}
\newpage

    \section{Projeto 6}
        \paragraph{Definição}Conjunto gratuito e aberto RISC de Instruction Set Architecture, ou seja, conjunto de regras de desenvolvimento de software e hardware.

        \subsection{Registradores RISC-V}
            \paragraph{Definição}Processodores elaborados sobre esta Arquitetura possuirão 32 registradores responsáveis por desempenhar funções especificas, cuja descrição seguem abaixo:
                \begin{table}[H]
                    \centering
                    \begin{tabular}[]{l|l}\hline
                        Nome & Descrição\\\hline
                        zero   & Valor Fixo em 0\\
                        t0-t6  & Valores Temporários\\
                        s0-s11 & Valores Salvos\\
                        a0-a7  & Parâmetros e Valores de Retorno de Funções\\
                        ra     & Endereço de Retorno de Função\\
                        sp     & Apontador de Pilha\\\hline
                    \end{tabular}
                    \caption{Registradores RISC-V}
                \end{table}\noindent

        \subsection{Formato de Funções}
            \paragraph{Definição}Há diferentes estruturas de funções que podem ser empregadas, entre as principais estruturas de funções tem-se os seguintes formatos básicos onde \texttt{func} representa uma função genérica, como descrito a seguir:
                \begin{enumerate}[rightmargin = \leftmargin, noitemsep]
                    \item \textbf{3 Arguments Functions:} Registrador \texttt{s0} assume o valor de \texttt{s1 func s2}, representado pelo seguinte comando:
                        \begin{scriptsize}
                            \myStyleAssembly
                            \begin{lstlisting}
    func s0, s1, s2
                            \end{lstlisting}
                        \end{scriptsize}

                    \item \textbf{2 Arguments Functions:} Registrador \texttt{s0} assume o valor de \texttt{s1 func 1}, onde \texttt{1} será nomeado imediato, representado pelo seguinte comando:
                        \begin{scriptsize}
                            \myStyleAssembly
                            \begin{lstlisting}
    func s0, s1, 1
                            \end{lstlisting}
                        \end{scriptsize}

                    \item \textbf{1 Arguments Functions:} Registrador \texttt{s0} assume o valor de \texttt{1}, onde \texttt{1} será nomeado imediato, representado pelo seguinte comando:
                        \begin{scriptsize}
                            \myStyleAssembly
                            \begin{lstlisting}
    func s0, 1
                            \end{lstlisting}
                        \end{scriptsize}
                \end{enumerate}

        \subsection{Instruções Aritméticas}
            \paragraph{Definição}Há diferentes funções aritméticas que permitem realizar matemática aritmética simples na arquitetura \texttt{load/store}, entre as principais instruções tem-se como descrito a seguir:
                \begin{enumerate}[rightmargin = \leftmargin, noitemsep]
                    \item \textbf{ADD Instruction:} Armazenada a soma de \texttt{rs1 + rs2} no registrador \texttt{rd}, representado pelo seguinte comando:
                        \begin{scriptsize}
                            \myStyleAssembly
                            \begin{lstlisting}
    ADD rd, rs1, rs2
                            \end{lstlisting}
                        \end{scriptsize}

                    \item \textbf{ADDI Instruction:} Armazenada a soma de \texttt{rs1 + imm} no registrador \texttt{rd}, representado pelo seguinte comando:
                        \begin{scriptsize}
                            \myStyleAssembly
                            \begin{lstlisting}
    ADDI rd, rs1, imm
                            \end{lstlisting}
                        \end{scriptsize}

                    \item \textbf{SUB Instruction:} Armazenada a subtração de \texttt{rs1 - rs2} no registrador \texttt{rd}, representado pelo seguinte comando:
                        \begin{scriptsize}
                            \myStyleAssembly
                            \begin{lstlisting}
    SUB rd, rs1, rs2
                            \end{lstlisting}
                        \end{scriptsize}
                \end{enumerate}

        \subsection{Instruções Lógicas}
            \paragraph{Definição}Há diferentes funções lógicas que permitem realizar operações simples na arquitetura \texttt{load/store}, entre as principais instruções tem-se como descrito a seguir:
                \begin{enumerate}[rightmargin = \leftmargin, noitemsep]
                    \item \textbf{XOR Instruction:} Armazenada a lógica de \texttt{rs1 XOR rs2} no registrador \texttt{rd}, representado pelo seguinte comando:
                        \begin{scriptsize}
                            \myStyleAssembly
                            \begin{lstlisting}
    XOR rd, rs1, rs2
                            \end{lstlisting}
                        \end{scriptsize}

                    \item \textbf{OR Instruction:} Armazenada a lógica de \texttt{rs1 OR rs2} no registrador \texttt{rd}, representado pelo seguinte comando:
                        \begin{scriptsize}
                            \myStyleAssembly
                            \begin{lstlisting}
    OR rd, rs1, rs2
                            \end{lstlisting}
                        \end{scriptsize}

                    \item \textbf{OR Instruction:} Armazenada a lógica de \texttt{rs1 AND rs2} no registrador \texttt{rd}, representado pelo seguinte comando:
                        \begin{scriptsize}
                            \myStyleAssembly
                            \begin{lstlisting}
    AND rd, rs1, rs2
                            \end{lstlisting}
                        \end{scriptsize}
                \end{enumerate}
            Nota-se que todas as instruções acima descritas possuem variação imediata, ou seja, podem receber alternativamente um valor imediato para realizar a operação.


        \subsection{Instruções de Deslocamento}
            \paragraph{Definição}Há diferentes funções que permitem deslocar lateralmente bits na arquitetura \texttt{load/store}, entre as principais instruções tem-se como descrito a seguir:
                \begin{enumerate}[rightmargin = \leftmargin]
                    \item \textbf{SLL Instruction:} Armazena no registrador \texttt{rd} o deslocamento de \texttt{rs2} bits para esquerda do valor que se encontra em  \texttt{rs1}, representado pelo seguinte comando:
                        \begin{scriptsize}
                            \myStyleAssembly
                            \begin{lstlisting}
    SLL rd, rs1, rs2
                            \end{lstlisting}
                        \end{scriptsize}
                    Este comando multiplica o valor de \texttt{rs1} por $2^{x}$, onde $x$ representa o valor de \texttt{rs2}.

                    \item \textbf{SRL Instruction:} Armazena no registrador \texttt{rd} o deslocamento de \texttt{rs2} bits para direita do valor que se encontra em  \texttt{rs1}, representado pelo seguinte comando:
                        \begin{scriptsize}
                            \myStyleAssembly
                            \begin{lstlisting}
    SRL rd, rs1, rs2
                            \end{lstlisting}
                        \end{scriptsize}
                    Este comando divide o valor de \texttt{rs1} por $2^{x}$, onde $x$ representa o valor de \texttt{rs2}.

                    \item \textbf{SRA Instruction:} Armazena no registrador \texttt{rd} o deslocamento de \texttt{rs2} bits para direita do valor que se encontra em  \texttt{rs1}, representado pelo seguinte comando:
                        \begin{scriptsize}
                            \myStyleAssembly
                            \begin{lstlisting}
    SRA rd, rs1, rs2
                            \end{lstlisting}
                        \end{scriptsize}
                    Este comando, diferentemente do SRL, replica o valor mais significativo do valor, garantindo que o complemento de dois seja conservado.
                \end{enumerate}
            Nota-se que todas as instruções acima descritas possuem variação imediata, ou seja, podem receber alternativamente um valor imediato para realizar a operação, sendo o método normalmente mais utilizado.


        \subsection{Instruções de Memória}
            \paragraph{Definição}Há diferentes funções que permitem acessar e escrever na memória na arquitetura \texttt{load/store}, entre as principais instruções tem-se como descrito a seguir:
                \begin{enumerate}[rightmargin = \leftmargin]
                    \item \textbf{LW Instruction:} Armazena no registrador \texttt{rd} a leitura do endereço dentro de \texttt{rs1 + imm}, representado pelo seguinte comando:
                        \begin{scriptsize}
                            \myStyleAssembly
                            \begin{lstlisting}
    LW rd, rs1, imm
                            \end{lstlisting}
                        \end{scriptsize}

                    \item \textbf{SW Instruction:} Armazena no registrador \texttt{rd} a escrita do endereço de \texttt{rs1}, representado pelo seguinte comando:
                        \begin{scriptsize}
                            \myStyleAssembly
                            \begin{lstlisting}
    SW rd, rs1, imm
                            \end{lstlisting}
                        \end{scriptsize}
                \end{enumerate}
            Neste conjunto de instruções será necessário fornecer a localização do vetor utilizado com relação a seu início. Como \texttt{32 bits = 4 bytes} os imediatos fornecidos as funções serão múltiplos do tamanho de palavra que estiver sendo lida, como representado pela seguinte tabela:
                \begin{table}[H]
                    \centering
                    \begin{tabular}[]{c|cc}\hline
                        Linguagem C & Variáveis em RISC-V & Tamanho em Bytes\\\hline
                        bool        & byte             & 1\\
                        char        & byte             & 1\\
                        short       & halfword         & 2\\
                        int         & word             & 4\\
                        long        & word             & 4\\
                        void        & unsigned word    & 4\\\hline
                    \end{tabular}
                    \caption{Variavéis RISC-V}
                \end{table}\noindent

        \subsection{Instruções de Comparação}
            \paragraph{Definição}Há diferentes funções que permitem, limitadamente, comparar valores na arquitetura \texttt{load/store}, entre as principais instruções tem-se como descrito a seguir:
                \begin{enumerate}[rightmargin = \leftmargin]
                    \item \textbf{SLT Instruction:} Armazena no registrador \texttt{rd} a comparação se o valor em \texttt{rs1} é menor do que o valor em \texttt{rs2}, representado pelo seguinte comando:
                        \begin{scriptsize}
                            \myStyleAssembly
                            \begin{lstlisting}
    SLT rd, rs1, rs2
                            \end{lstlisting}
                        \end{scriptsize}
                    Este, comando apresentaram as variações imediatas, não sinalizadas e a combinação entre imediata e não sinalizada.
                \end{enumerate}


        \subsection{Instruções de Salto Condicional}
            \paragraph{Definição}Há diferentes funções que permitem, com base em uma comparação, ir para outra posição de memória na arquitetura \texttt{load/store}, entre as principais instruções tem-se como descrito a seguir:
                \begin{enumerate}[rightmargin = \leftmargin]
                    \item \textbf{BEQ Instruction:} Desloca-se para a posição de \texttt{imm} se o valor de \texttt{rs1 == rs2}, representado pelo seguinte comando:
                        \begin{scriptsize}
                            \myStyleAssembly
                            \begin{lstlisting}
    BEQ rs1, rs2, imm
                            \end{lstlisting}
                        \end{scriptsize}

                    \item \textbf{BNE Instruction:} Desloca-se para a posição de \texttt{imm} se o valor de \texttt{rs1 != rs2}, representado pelo seguinte comando:
                        \begin{scriptsize}
                            \myStyleAssembly
                            \begin{lstlisting}
    BNE rs1, rs2, imm
                            \end{lstlisting}
                        \end{scriptsize}

                    \item \textbf{BLT Instruction:} Desloca-se para a posição de \texttt{imm} se o valor de \texttt{rs1 < rs2}, representado pelo seguinte comando:
                        \begin{scriptsize}
                            \myStyleAssembly
                            \begin{lstlisting}
    BLT rs1, rs2, imm
                            \end{lstlisting}
                        \end{scriptsize}
                    Este, comando apresentará a variação não sinalizada.

                    \item \textbf{BGE Instruction:} Desloca-se para a posição de \texttt{imm} se o valor de \texttt{rs1 >=rs2}, representado pelo seguinte comando:
                        \begin{scriptsize}
                            \myStyleAssembly
                            \begin{lstlisting}
    BGE rs1, rs2, imm
                            \end{lstlisting}
                        \end{scriptsize}
                    Este, comando apresentará a variação não sinalizada.
                \end{enumerate}

        \subsection{Códigos Básicos}
            \paragraph{Definição}Há diferentes estruturas comumente empregadas em código, entre os principais métodos tem-se como descrito a seguir:
                \begin{enumerate}[rightmargin = \leftmargin]
                    \item \textbf{if Estrutura:} Realização de uma comparação e separação do código, representado pelos seguintes comandos:\\
                        \begin{minipage}[t]{0.45\linewidth}
                            \begin{scriptsize}
                                \myStyleC
                                \begin{lstlisting}
    #include<stdio.h>
    int main()
    {
        int t0 = 9;
        int t1 = 0;
        int t2 = 5;
    
        if (t0 == t2)
        {
            t1 += 7;
        } else {
            t1 += 15;
        }
    }

                                \end{lstlisting}
                            \end{scriptsize}
                        \end{minipage}
                            \hspace{12.5mm}
                        \begin{minipage}[t]{0.45\linewidth}
                            \begin{scriptsize}
                                \myStyleAssembly
                                \begin{lstlisting}
    main:
        addi t0, zero, 9
        addi t1, zero, 0
        addi t2, zero, 5

        bne t0, t2, else
        addi t1, t1, 7
        j fim

    else:
        addi t1, t1, 15

    fim:
        jr ra
                                \end{lstlisting}
                            \end{scriptsize}
                        \end{minipage}
                \item \textbf{while Estrutura:} Realização de \texttt{loop} do código, representado pelos seguintes comandos:\\
                    \begin{minipage}[t]{0.45\linewidth}
                        \begin{scriptsize}
                            \myStyleC
                            \begin{lstlisting}
    #include<stdio.h>
    int main()
    {
        int t0 = 20;
        int t1 = 10;
    
        while (t0 != t1)
        {
            t0 += 2;
            t1 += 3;
        }
    }
                            \end{lstlisting}
                        \end{scriptsize}
                    \end{minipage}
                        \hspace{12.5mm}
                    \begin{minipage}[t]{0.45\linewidth}
                        \begin{scriptsize}
                            \myStyleAssembly
                            \begin{lstlisting}
    main:
        addi t0, zero, 20
        addi t1, zero, 10

    while:
        beq t0, t1, fim
        addi t0, t0, 2
        addi t1, t1, 3
        j while

    fim:
        jr ra
                                \end{lstlisting}
                            \end{scriptsize}
                    \end{minipage}

                \item \textbf{for Estrutura:} Realização de \texttt{loop} do código, representado pelos seguintes comandos:\\
                    \begin{minipage}[t]{0.45\linewidth}
                        \begin{scriptsize}
                            \myStyleC
                            \begin{lstlisting}
    #include<stdio.h>
    int main()
    {
        int t0 = 0;
        int t1 = 100;
        int t2 = 0;
    
        for (t0 = 0; t0 < t1; t0++)
        {
            t2 += t0;
        }
    }
                            \end{lstlisting}
                        \end{scriptsize}
                    \end{minipage}
                        \hspace{12.5mm}
                    \begin{minipage}[t]{0.45\linewidth}
                        \begin{scriptsize}
                            \myStyleAssembly
                            \begin{lstlisting}
    main:
        addi t0, zero, 0
        addi t1, zero, 0
        addi t2, zero, 100

    for:
        bge t1, t2, fim
        addi t0, t0, t1
        addi t1, t1, 1
        j for

    fim:
        jr ra
                            \end{lstlisting}
                        \end{scriptsize}
                    \end{minipage}
            \end{enumerate}
\newpage

        \subsection{Programas}
            \begin{enumerate}[rightmargin = \leftmargin]
                \item \textbf{Triângulo}:
                    \begin{scriptsize}
                        \myStyleAssembly
                        \lstinputlisting{examples/lab06/mc404_tri.s}
                    \end{scriptsize}
\newpage
                \item \textbf{Multiplicação}:
                    \begin{scriptsize}
                        \myStyleAssembly
                        \lstinputlisting{examples/lab06/mc404_mul.s}
                    \end{scriptsize}
            \end{enumerate}
\newpage

    \section{Projeto 7}
        \paragraph{Definição}Assim como outras linguagens o RISC-V seguem uma sequência de convenções, regras estabelecida entre os usuários, para utilização e aplicação desta linguagem. Recomenda-se que sejam seguidas para garantir a compreensão de seu código.

        \subsection{Endereçamento da Memória}
            \paragraph{Definição}Sequência utilizada para armazenar uma palavra de \texttt{4 bytes}, ou seja \texttt{32 bits}, na memória, diferenciando a ordem de leitura e escrita através dos seguintes métodos:
                \begin{enumerate}[rightmargin = \leftmargin, noitemsep]
                    \item \textbf{Big Endian}: Aloca o \texttt{byte} mais significativo, \texttt{MSB}, primeiro;
                    \item \textbf{Little Endian}: Aloca o \texttt{byte} menos significativo, \texttt{LSB}, primeiro;
                \end{enumerate}
            Esta configuração demonstra apenas como o processor lerá cada palavra de sua memória, iniciando pelo $MSB$ ou pelo $LSB$. Desta maneira pode-se considerar o seguinte exemplo:
                \begin{equation}
                    \boxed{
                        \underbrace{\text{H}}_{
                            \text{0x48}
                        }
                        \hspace{5mm}
                        \underbrace{\text{E}}_{
                            \text{0x45}
                        }
                        \hspace{5mm}
                        \underbrace{\text{L}}_{
                            \text{0x4C}
                        }
                        \hspace{5mm}
                        \underbrace{\text{L}}_{
                            \text{0x4C}
                        }
                        \hspace{5mm}
                        \underbrace{\text{O}}_{
                            \text{0x4F}
                        }
                        \hspace{5mm}
                        \underbrace{\text{!}}_{
                            \text{0x21}
                        }
                        \hspace{5mm}
                        \underbrace{\text{}}_{
                            \text{0x00}
                        }
                    }
                \end{equation}\\
                \begin{center}
                    \begin{minipage}[t]{0.425\linewidth}
                        \textbf{Código Little Endian:}
                        \begin{scriptsize}
                            \myStyleAssembly
                            \begin{lstlisting}
    .word 0x4C4C4548
    .word 0x0000214F
                            \end{lstlisting}
                        \end{scriptsize}
                        \vspace{12.5mm}
                        \textbf{Código Big Endian:}
                        \begin{scriptsize}
                            \myStyleAssembly
                            \begin{lstlisting}
    .word 0x48454C4C
    .word 0x4F210000
                            \end{lstlisting}
                        \end{scriptsize}
                    \end{minipage}
                        \hspace{12.5mm}
                    \begin{minipage}[t]{0.425\linewidth}
                        \textbf{Memória:}
                        \begin{table}[H]
                            \centering
                            \begin{tabular}[]{c|c|c|}
                                Adress & Little Endian & Big Endian\\
                                0 & H & \\
                                1 & E & \\
                                2 & L & !\\
                                3 & L & O\\
                                4 & O & L\\
                                5 & ! & L\\
                                6 &   & E\\
                                7 &   & H\\
                            \end{tabular}
                            \caption{Estrutura de Memoria em RISC-V}
                        \end{table}\noindent
                    \end{minipage}
                \end{center}\noindent
            Note que está diferença não influência para a leitura do código, pois isto influenciará apenas o processamento do processador. RISC-V é padronizado em \texttt{Little Endian}, desta forma será necessário atentar-se quando outros dispositivos sejam \texttt{Big Endian}.

        \subsection{Execução de Funções}
            \paragraph{Definição}Trechos de código que executam uma tarefa específica organizadas separadamente para facilitar seu reuso e legibilidade dentro das convenções utilizadas, abaixo serão listados as principais:
                \begin{enumerate}[noitemsep]
                    \item \textbf{Atribuição de Variável}: Parâmetros utilizados na função serão sequênciados a partir de \texttt{a0}, \texttt{a1}, ...;

                    \item \textbf{Retorno de Variável}: Valores retornados na função serão sequênciados a partir de \texttt{a0}, \texttt{a1}, ...; 
                \end{enumerate}
            Note que quando uma função for executada variáveis atualmente armazenadas em registradores podem ser perdidas, pois durante a execução da função estes locais de memória podem ser acessados e reescritos. Desta maneira recomenda-se salvar as variáveis necessárias na pilha, deslocando o espaço de memória quando não for mais necessário.

        \subsection{Pilha}
            \paragraph{Definição}Espaço da memória reservado para armazenar elementos momentaneamente cujo controle partirá do usuário. Nesta arquitetura o registrador \texttt{sp} sempre apontará para o último elemento da Pilha.

            \paragraph{Organização de Memória}Processadores terão suas memórias organizadas com objetivo de otimizar sua utilização, desta maneira será divida nos seguintes espaços:
                \begin{enumerate}[rightmargin = \leftmargin, noitemsep]
                    \item \textbf{Pilha}: Inicia-se no fim da memória e cresce para baixo;
                    \item \textbf{Heap}: Inicia-se no começo da memória e cresce para cima;
                \end{enumerate}
                \begin{table}[H]
                    \centering
                    \begin{tabular}[]{|c|}\hline
                        $\downarrow$ Pilha\\
                        \\
                        $\vdots$\\
                        \\
                        $\uparrow$ Heap\\
                        Dados\\
                        Programa\\\hline
                    \end{tabular}
                    \caption{Estrutura de Memoria em RISC-V}
                \end{table}\noindent
            Note que desta forma há diferentes combinações possíveis entre as memórias, sendo apenas limitadas pela outra. Assim, o \textbf{Heap} poderá crescer até encontrar-se com a \textbf{Pilha} e vice-versa, possibilitando flexibilixar e maximizar a utilização da memória disponível. Caso deseja-se alocar espaço na \textbf{Pilha} recomenda-se utilizar a seguinte abordagem:
                \begin{center}
                    \begin{minipage}[t]{0.425\linewidth}
                        \textbf{Início:}
                        \begin{scriptsize}
                            \myStyleAssembly
                            \begin{lstlisting}
    addi sp, sp, -8
    sw   ra, 0(sp)
    sw   s0, 4(sp)
                            \end{lstlisting}
                        \end{scriptsize}
                    \end{minipage}
                        \hspace{12.5mm}
                    \begin{minipage}[t]{0.425\linewidth}
                        \textbf{Final:}
                        \begin{scriptsize}
                            \myStyleAssembly
                            \begin{lstlisting}
    sw   s0, 4(sp)
    sw   ra, 0(sp)
    addi sp, sp, 8
                            \end{lstlisting}
                        \end{scriptsize}
                    \end{minipage}
                \end{center}\noindent
\newpage

        \subsection{Programas}
            \begin{enumerate}[rightmargin = \leftmargin]
                \item \textbf{Menor Valor Vetor}:
                    \begin{scriptsize}
                        \myStyleAssembly
                        \lstinputlisting{examples/lab07/mc404_smaVec.s}
                    \end{scriptsize}
\newpage
                \item \textbf{Soma Vetores}:
                    \begin{scriptsize}
                        \myStyleAssembly
                        \lstinputlisting{examples/lab07/mc404_sumVec.s}
                    \end{scriptsize}
            \end{enumerate}
\end{document}