\documentclass{article}

\usepackage[a4paper, hmargin={20mm, 20mm}, vmargin={25mm, 30mm}]{geometry}
\usepackage[utf8]{inputenc}
\usepackage[english, main=portuguese]{babel}

\usepackage[hidelinks]{hyperref}
\usepackage{bookmark}
\usepackage{cancel}
\usepackage{comment}

\usepackage{array}
\usepackage{indentfirst}
\usepackage{multicol}
\setlength{\multicolsep}{2pt}% 50% of original values
\usepackage{subfiles}

\usepackage{titlesec}

\usepackage{amsmath}
\usepackage{amssymb}
\usepackage{systeme}
\usepackage{float}
\usepackage{enumitem}
\usepackage[thinc]{esdiff} %parcial derivatives
\restylefloat{table}

\usepackage{graphicx}
\usepackage{subcaption}
\graphicspath{ {./images/} }

% Pacote para a definição de novas cores
\usepackage{xcolor}
% Definindo novas cores
\definecolor{darkgreen}{rgb}{0.0, 0.42, 0.24}
\definecolor{darkpurple}{rgb}{0.74, 0.2, 0.64}
\definecolor{darkblue}{rgb}{0.0, 0.28, 0.67}

%Configurando espaços entre paragrafos
%\setlength{\parskip}{0.5em}

\usepackage{chngcntr}
\counterwithin{figure}{section}
\counterwithin{table}{section}
\counterwithin{equation}{section}

%Configurando pacote de Gráficos plots
\usepackage{pgfplots}
\usepackage{tikz}

%Configurando pacote de circuitos
\usepackage{circuitikz}

\tikzset{
    mux2by1/.style={
        muxdemux, muxdemux def={
            Lh=4,
            Rh=2,
            w=2,
            NL=2,
            NB=1,
            NR=1,
            NT=0,
            inset Lh=0,
            inset Rh=0,
            inset w=0,
            square pins=0
        }
    }
}

\tikzset{
    demux2by1/.style={
        muxdemux, muxdemux def={
            Lh=2,
            Rh=4,
            w=2,
            NL=1,
            NB=1,
            NR=2,
            NT=0,
            inset Lh=0,
            inset Rh=0,
            inset w=0,
            square pins=0
        }
    }
}



%Configurando multiple files
\usepackage{filecontents}

%Configurando quotes
\usepackage{csquotes}


%Configurando layout para mostrar códigos
\usepackage{listings}
%Setting default style for code
\lstdefinestyle{myStyle}{
    language=Bash,                            % the language of the code
    basicstyle=\ttfamily\small,               % the size of the fonts that are used for the code
    keywordstyle=\color{darkpurple}\bfseries, %
    stringstyle=\color{darkblue},             %
    commentstyle=\color{darkgreen},           %
    morecomment=[s][\color{blue}]{/**}{*/},   %
    extendedchars=true,                       %
    showtabs=false,                           % show tabs within strings adding particular underscores
    showspaces=false,                         % show spaces adding particular underscores
    showstringspaces=false,                   % underline spaces within strings
    numbers=left,                             % where to put the line-numbers
    numberstyle=\tiny\color{gray},            % the style that is used for the line-numbers
    stepnumber=1,                             % the step between two line-numbers. If it's 1, each line will be numbered
    numbersep=5pt,                            % how far the line-numbers are from the code
    frame=single,                             % adds a frame around the code
    rulecolor=\color{black},                  % if not set, the frame-color may be changed on line-breaks within not-black text
    breaklines=true,                          % sets automatic line breaking
    backgroundcolor=\color{white},            % choose the background color
    breakatwhitespace=true,                   % sets if automatic breaks should only happen at whitespace
    breakautoindent=false,                    %
    captionpos=b,                             % sets the caption-position to bottom
    xleftmargin=0pt,                          %
    tabsize=2,                                % sets default tabsize to 2 spaces
}

%setting bash setup
\newcommand{\myStyleBash}{
\lstset{
    language=Bash,                            % the language of the code
    basicstyle=\ttfamily\small,               % the size of the fonts that are used for the code
    keywordstyle=\color{darkpurple}\bfseries, %
    stringstyle=\color{darkblue},             %
    commentstyle=\color{darkgreen},           %
    morecomment=[s][\color{blue}]{/**}{*/},   %
    extendedchars=true,                       %
    showtabs=false,                           % show tabs within strings adding particular underscores
    showspaces=false,                         % show spaces adding particular underscores
    showstringspaces=false,                   % underline spaces within strings
    numbers=left,                             % where to put the line-numbers
    numberstyle=\tiny\color{gray},            % the style that is used for the line-numbers
    stepnumber=1,                             % the step between two line-numbers. If it's 1, each line will be numbered
    numbersep=5pt,                            % how far the line-numbers are from the code
    frame=single,                             % adds a frame around the code
    rulecolor=\color{black},                  % if not set, the frame-color may be changed on line-breaks within not-black text
    breaklines=true,                          % sets automatic line breaking
    backgroundcolor=\color{white},            % choose the background color
    breakatwhitespace=true,                   % sets if automatic breaks should only happen at whitespace
    breakautoindent=false,                    %
    captionpos=b,                             % sets the caption-position to bottom
    xleftmargin=0pt,                          %
    tabsize=2,                                % sets default tabsize to 2 spaces
}}
%setting VHDL setup
\newcommand{\myStyleVHDL}{
\lstset{
    language=VHDL,                            % the language of the code
    basicstyle=\ttfamily\small,               % the size of the fonts that are used for the code
    keywordstyle=\color{darkpurple}\bfseries, %
    stringstyle=\color{darkblue},             %
    commentstyle=\color{darkgreen},           %
    morecomment=[s][\color{blue}]{/**}{*/},   %
    extendedchars=true,                       %
    showtabs=false,                           % show tabs within strings adding particular underscores
    showspaces=false,                         % show spaces adding particular underscores
    showstringspaces=false,                   % underline spaces within strings
    numbers=left,                             % where to put the line-numbers
    numberstyle=\tiny\color{gray},            % the style that is used for the line-numbers
    stepnumber=1,                             % the step between two line-numbers. If it's 1, each line will be numbered
    numbersep=5pt,                            % how far the line-numbers are from the code
    frame=single,                             % adds a frame around the code
    rulecolor=\color{black},                  % if not set, the frame-color may be changed on line-breaks within not-black text
    breaklines=true,                          % sets automatic line breaking
    backgroundcolor=\color{white},            % choose the background color
    breakatwhitespace=true,                   % sets if automatic breaks should only happen at whitespace
    breakautoindent=false,                    %
    captionpos=b,                             % sets the caption-position to bottom
    xleftmargin=0pt,                          %
    tabsize=2,                                % sets default tabsize to 2 spaces
}}

\newcommand{\myStyleAssembler}{
\lstset{
    language=[x86masm]Assembler,                       % the language of the code
    basicstyle=\ttfamily\small,               % the size of the fonts that are used for the code
    keywordstyle=\color{darkpurple}\bfseries, %
    stringstyle=\color{darkblue},             %
    commentstyle=\color{darkgreen},           %
    morecomment=[s][\color{blue}]{/**}{*/},   %
    extendedchars=true,                       %
    showtabs=false,                           % show tabs within strings adding particular underscores
    showspaces=false,                         % show spaces adding particular underscores
    showstringspaces=false,                   % underline spaces within strings
    numbers=left,                             % where to put the line-numbers
    numberstyle=\tiny\color{gray},            % the style that is used for the line-numbers
    stepnumber=1,                             % the step between two line-numbers. If it's 1, each line will be numbered
    numbersep=5pt,                            % how far the line-numbers are from the code
    frame=single,                             % adds a frame around the code
    rulecolor=\color{black},                  % if not set, the frame-color may be changed on line-breaks within not-black text
    breaklines=true,                          % sets automatic line breaking
    backgroundcolor=\color{white},            % choose the background color
    breakatwhitespace=true,                   % sets if automatic breaks should only happen at whitespace
    breakautoindent=false,                    %
    captionpos=b,                             % sets the caption-position to bottom
    xleftmargin=0pt,                          %
    tabsize=2,                                % sets default tabsize to 2 spaces
}}

%\titleformat{<command>}[<shape>]{<format>}{<label>}{<sep>}{<before-code>}[<after-code>]
\titleformat
{\section} %comand
[block]  %shape
{\normalfont\LARGE} %format
{\thesection. } %label
{0mm} %sep
{} %before-code
[{\titlerule[0.1mm]}] %after-code

\titlespacing*{\section}{0mm}{0mm}{15mm}

\titleformat
{\subsection} %comand
[block]  %shape
{\normalfont\Large} %format
{\thesubsection. } %label
{0mm} %sep
{} %before-code
[] %after-code

\titlespacing*{\subsection}{0mm}{5mm}{2.5mm}


\begin{document}
    \begin{titlepage}
        \begin{center}
            \rule{450pt}{0.5pt}\\[4mm]
            {\Huge MC404 - Organição Básica de Computadores}\\
            \rule{450pt}{0.5pt}\\[2mm]
            {\Large Resumo Teórico}\\[200mm]
            \today\\
            \rule{250pt}{0.5pt}\\
            {\large Guilherme Nunes Trofino}\\
            {\large 217276}\\
        \end{center}
    \end{titlepage}
\newpage

    \tableofcontents
\newpage

    \paragraph{Instruções}Comandos necessários para que o simulador seja aberto a partir do terminal: 
        \begin{scriptsize}
            \myStyleBash
            \begin{lstlisting}
    cd classes/mc404/nand2tetris/tools/
    chmod 755 HardwareSimulator.sh
    ./HardwareSimulator.sh 
            \end{lstlisting}
        \end{scriptsize}
\newpage

    \section{Projeto 1}
        \subsection{Not}
            \paragraph{Definição}[Funcionamento]

            \paragraph{Tabela Verdade}Esta porta lógica possuirá a seguinte tabela verdade:
                \begin{table}[H]
                    \centering  
                    \begin{tabular}[]{c|c}\hline
                        in & out\\\hline
                        0  & 1\\
                        1  & 0\\\hline
                    \end{tabular}
                    \caption{Tabela Verdade Not}
                \end{table}

            \paragraph{Representação}Esta porta lógica pode ser expressa pelo seguinte circuito:
                \begin{figure}[H]
                    \centering
                    \begin{circuitikz}
                        \ctikzset{component text=left}
                        \draw
                        (0,0) node[not port] (myPort) {not}
                        (myPort.in 1) node [anchor = east] {in}
                        (myPort.out)  node [anchor = west] {out};
                    \end{circuitikz} 
                    \caption{Porta Lógica Not}
                \end{figure} \noindent

            \paragraph{Implementação}Esta porta lógica pode ser implementada de acordo com o seguinte código:
                \begin{scriptsize}
                    \myStyleVHDL
                    \lstinputlisting{./nand2tetris/projects/01/Not.hdl}
                \end{scriptsize}
\newpage

        \subsection{And}
            \paragraph{Definição}[Funcionamento]

            \paragraph{Tabela Verdade}Esta porta lógica possuirá a seguinte tabela verdade:
                \begin{table}[H]
                    \centering
                    \begin{tabular}[]{cc|c}\hline
                        a & b & out\\\hline
                        0 & 0 & 0\\
                        0 & 1 & 0\\
                        1 & 0 & 0\\
                        1 & 1 & 1\\\hline
                    \end{tabular}
                    \caption{Tabela Verdade And}
                \end{table}

            \paragraph{Representação}Esta porta lógica pode ser expressa pelo seguinte circuito:
                \begin{figure}[H]
                    \centering
                    \begin{circuitikz}
                        \ctikzset{component text=left}
                        \draw
                        (0,0) node[and port] (myPort) {and}
                        (myPort.in 1) node [anchor = east] {a}
                        (myPort.in 2) node [anchor = east] {b}
                        (myPort.out)  node [anchor = west] {out};
                    \end{circuitikz} 
                    \caption{Porta Lógica And}
                \end{figure} \noindent

            \paragraph{Implementação}Esta porta lógica pode ser implementada de acordo com o seguinte código:
                \begin{scriptsize}
                    \myStyleVHDL
                    \lstinputlisting{./nand2tetris/projects/01/And.hdl}
                \end{scriptsize}
\newpage

        \subsection{Or}
            \paragraph{Definição}[Funcionamento]

            \paragraph{Tabela Verdade}Esta porta lógica possuirá a seguinte tabela verdade:
                \begin{table}[H]
                    \centering
                    \begin{tabular}[]{cc|c}\hline
                        a & b & out\\\hline
                        0 & 0 & 0\\
                        0 & 1 & 1\\
                        1 & 0 & 1\\
                        1 & 1 & 1\\\hline
                    \end{tabular}
                    \caption{Tabela Verdade Or}
                \end{table}

            \paragraph{Representação}Esta porta lógica pode ser expressa pelo seguinte circuito:
                \begin{figure}[H]
                    \centering
                    \begin{circuitikz}
                        \ctikzset{component text=left}
                        \draw
                        (0,0) node[or port] (myPort) {or}
                        (myPort.in 1) node [anchor = east] {a}
                        (myPort.in 2) node [anchor = east] {b}
                        (myPort.out)  node [anchor = west] {out};
                    \end{circuitikz} 
                    \caption{Porta Lógica Or}
                \end{figure} \noindent

            \paragraph{Implementação}Esta porta lógica pode ser implementada de acordo com o seguinte código:
                \begin{scriptsize}
                    \myStyleVHDL
                    \lstinputlisting{./nand2tetris/projects/01/Or.hdl}
                \end{scriptsize}
\newpage

        \subsection{Xor}
            \paragraph{Definição}[Funcionamento]

            \paragraph{Tabela Verdade}Esta porta lógica possuirá a seguinte tabela verdade:
                \begin{table}[H]
                    \centering
                    \begin{tabular}[]{cc|c}\hline
                        a & b & out\\\hline
                        0 & 0 & 0\\
                        0 & 1 & 1\\
                        1 & 0 & 1\\
                        1 & 1 & 0\\\hline
                    \end{tabular}
                    \caption{Tabela Verdade Xor}
                \end{table}

            \paragraph{Representação}Esta porta lógica pode ser expressa pelo seguinte circuito:
                \begin{figure}[H]
                    \centering
                    \begin{circuitikz}
                        \ctikzset{component text=left}
                        \draw
                        (0,0) node[xor port] (myPort) {xor}
                        (myPort.in 1) node [anchor = east] {a}
                        (myPort.in 2) node [anchor = east] {b}
                        (myPort.out)  node [anchor = west] {out};
                    \end{circuitikz} 
                    \caption{Porta Lógica Xor}
                \end{figure} \noindent

            \paragraph{Implementação}Esta porta lógica pode ser implementada de acordo com o seguinte código:
                \begin{scriptsize}
                    \myStyleVHDL
                    \lstinputlisting{./nand2tetris/projects/01/Xor.hdl}
                \end{scriptsize}
\newpage

        \subsection{Mux}
            \paragraph{Definição}[Funcionamento]

            \paragraph{Tabela Verdade}Esta porta lógica possuirá a seguinte tabela verdade:
                \begin{table}[H]
                    \centering
                    \begin{tabular}[]{ccc|c}\hline
                        s & a & b & out\\\hline
                        0 & \textcolor{red}{0} & \textcolor{blue}{0} & \textcolor{red}{0}\\
                        0 & \textcolor{red}{0} & \textcolor{blue}{1} & \textcolor{red}{0}\\
                        0 & \textcolor{red}{1} & \textcolor{blue}{0} & \textcolor{red}{1}\\
                        0 & \textcolor{red}{1} & \textcolor{blue}{1} & \textcolor{red}{1}\\
                        1 & \textcolor{red}{0} & \textcolor{blue}{0} & \textcolor{blue}{0}\\
                        1 & \textcolor{red}{0} & \textcolor{blue}{1} & \textcolor{blue}{1}\\
                        1 & \textcolor{red}{1} & \textcolor{blue}{0} & \textcolor{blue}{0}\\
                        1 & \textcolor{red}{1} & \textcolor{blue}{1} & \textcolor{blue}{1}\\
                    \end{tabular}
                    \caption{Tabela Verdade Mux}
                \end{table}

            \paragraph{Representação}Esta porta lógica pode ser expressa pelo seguinte circuito:
                \begin{figure}[H]
                    \centering
                    \begin{circuitikz}
                        \ctikzset{component text=left}
                        \draw
                        (0,0) node[mux2by1] (myPort) {mux}
                        (myPort.lpin 1) node [anchor = east] {a}
                        (myPort.lpin 2) node [anchor = east] {b}
                        (myPort.bpin 1) node [anchor = north] {s}
                        (myPort.rpin 1)  node [anchor = west] {out};
                    \end{circuitikz} 
                    \caption{Porta Lógica Mux}
                \end{figure} \noindent

            \paragraph{Implementação}Esta porta lógica pode ser implementada de acordo com o seguinte código:
                \begin{scriptsize}
                    \myStyleVHDL
                    \lstinputlisting{./nand2tetris/projects/01/Mux.hdl}
                \end{scriptsize}
\newpage

        \subsection{DMux}
            \paragraph{Definição}[Funcionamento]

            \paragraph{Tabela Verdade}Esta porta lógica possuirá a seguinte tabela verdade:
                \begin{table}[H]
                    \centering
                    \begin{tabular}[]{cc|c}\hline
                        a & b & out\\\hline
                        0 & 1 & 1\\
                        0 & 1 & 1\\
                        0 & 1 & 1\\
                        0 & 1 & 1\\\hline
                    \end{tabular}
                    \caption{Tabela Verdade DMux}
                \end{table}

            \paragraph{Representação}Esta porta lógica pode ser expressa pelo seguinte circuito:
                \begin{figure}[H]
                    \centering
                    \begin{circuitikz}
                        \ctikzset{component text=left}
                        \draw
                        (0,0) node[demux2by1] (myPort) {dmux}
                        (myPort.rpin 1) node [anchor = west] {a}
                        (myPort.rpin 2) node [anchor = west] {b}
                        (myPort.bpin 1) node [anchor = north] {s}
                        (myPort.lpin 1)  node [anchor = east] {in};
                    \end{circuitikz} 
                    \caption{Porta Lógica DMux}
                \end{figure} \noindent

            \paragraph{Implementação}Esta porta lógica pode ser implementada de acordo com o seguinte código:
                \begin{scriptsize}
                    \myStyleVHDL
                    \lstinputlisting{./nand2tetris/projects/01/DMux.hdl}
                \end{scriptsize}
\newpage

        \subsection{Not16}
            \paragraph{Definição}[Funcionamento]

            \paragraph{Tabela Verdade}Esta porta lógica possuirá a seguinte tabela verdade:
                \begin{table}[H]
                    \centering
                    \begin{tabular}[]{c|c}\hline
                        in[i] & out[i]\\\hline
                        0     & 1\\
                        1     & 0\\\hline
                    \end{tabular}
                    \caption{Tabela Verdade Not16}
                \end{table}

            \paragraph{Representação}Esta porta lógica pode ser expressa pelo seguinte circuito:
                \begin{figure}[H]
                    \centering
                    \begin{circuitikz}
                        \ctikzset{component text=left}
                        \draw
                        (0,0) node[not port] (myPort) {not16}
                        (myPort.in 1) node [anchor = east] {in[0..15]}
                        (myPort.out)  node [anchor = west] {out[0..15]};
                    \end{circuitikz} 
                    \caption{Porta Lógica Not16}
                \end{figure} \noindent

            \paragraph{Implementação}Esta porta lógica pode ser implementada de acordo com o seguinte código:
                \begin{scriptsize}
                    \myStyleVHDL
                    \lstinputlisting{./nand2tetris/projects/01/Not16.hdl}
                \end{scriptsize}
\newpage

    \subsection{AND16}
        \paragraph{Definição}[Funcionamento]

        \paragraph{Tabela Verdade}Esta porta lógica possuirá a seguinte tabela verdade:
            \begin{table}[H]
                \centering
                \begin{tabular}[]{cc|c}\hline
                    a & b & out\\\hline
                    0 & 1 & 1\\
                    0 & 1 & 1\\
                    0 & 1 & 1\\
                    0 & 1 & 1\\\hline
                \end{tabular}
                \caption{Tabela Verdade AND16}
            \end{table}

        \paragraph{Representação}Esta porta lógica pode ser expressa pelo seguinte circuito:
            \begin{figure}[H]
                \centering
                \begin{circuitikz}
                    \ctikzset{component text=left}
                    \draw
                    (0,0) node[and port] (myPort) {AND16}
                    (myPort.in 1) node [anchor = east] {a}
                    (myPort.in 2) node [anchor = east] {b}
                    (myPort.out)  node [anchor = west] {out};
                \end{circuitikz} 
                \caption{Porta Lógica AND16}
            \end{figure} \noindent

        \paragraph{Implementação}Esta porta lógica pode ser implementada de acordo com o seguinte código:
            \begin{scriptsize}
                \myStyleVHDL
                \lstinputlisting{./nand2tetris/projects/01/And16.hdl}
            \end{scriptsize}
\newpage

        \subsection{OR16}
            \paragraph{Definição}[Funcionamento]

            \paragraph{Tabela Verdade}Esta porta lógica possuirá a seguinte tabela verdade:
                \begin{table}[H]
                    \centering
                    \begin{tabular}[]{cc|c}\hline
                        a & b & out\\\hline
                        0 & 1 & 1\\
                        0 & 1 & 1\\
                        0 & 1 & 1\\
                        0 & 1 & 1\\\hline
                    \end{tabular}
                    \caption{Tabela Verdade OR16}
                \end{table}

            \paragraph{Representação}Esta porta lógica pode ser expressa pelo seguinte circuito:
                \begin{figure}[H]
                    \centering
                    \begin{circuitikz}
                        \ctikzset{component text=left}
                        \draw
                        (0,0) node[or port] (myPort) {OR16}
                        (myPort.in 1) node [anchor = east] {a}
                        (myPort.in 2) node [anchor = east] {b}
                        (myPort.out)  node [anchor = west] {out};
                    \end{circuitikz} 
                    \caption{Porta Lógica OR16}
                \end{figure} \noindent

            \paragraph{Implementação}Esta porta lógica pode ser implementada de acordo com o seguinte código:
                \begin{scriptsize}
                    \myStyleVHDL
                    \lstinputlisting{./nand2tetris/projects/01/Or16.hdl}
                \end{scriptsize}
\newpage

    \subsection{MUX16}
        \paragraph{Definição}[Funcionamento]

        \paragraph{Tabela Verdade}Esta porta lógica possuirá a seguinte tabela verdade:
            \begin{table}[H]
                \centering
                \begin{tabular}[]{cc|c}\hline
                    a & b & out\\\hline
                    0 & 1 & 1\\
                    0 & 1 & 1\\
                    0 & 1 & 1\\
                    0 & 1 & 1\\\hline
                \end{tabular}
                \caption{Tabela Verdade MUX16}
            \end{table}

        \paragraph{Representação}Esta porta lógica pode ser expressa pelo seguinte circuito:
            \begin{figure}[H]
                \centering
                \begin{circuitikz}
                    \ctikzset{component text=left}
                    \draw
                    (0,0) node[muxdemux] (myPort) {MUX16};
                    % (myPort.in 1) node [anchor = east] {a}
                    % (myPort.in 2) node [anchor = east] {b}
                    % (myPort.out)  node [anchor = west] {out};
                \end{circuitikz} 
                \caption{Porta Lógica MUX16}
            \end{figure} \noindent

        \paragraph{Implementação}Esta porta lógica pode ser implementada de acordo com o seguinte código:
            \begin{scriptsize}
                \myStyleVHDL
                \lstinputlisting{./nand2tetris/projects/01/Mux16.hdl}
            \end{scriptsize}
\newpage

        \subsection{OR8WAY}
            \paragraph{Definição}[Funcionamento]

            \paragraph{Tabela Verdade}Esta porta lógica possuirá a seguinte tabela verdade:
                \begin{table}[H]
                    \centering
                    \begin{tabular}[]{cc|c}\hline
                        a & b & out\\\hline
                        0 & 1 & 1\\
                        0 & 1 & 1\\
                        0 & 1 & 1\\
                        0 & 1 & 1\\\hline
                    \end{tabular}
                    \caption{Tabela Verdade OR8WAY}
                \end{table}

            \paragraph{Representação}Esta porta lógica pode ser expressa pelo seguinte circuito:
                \begin{figure}[H]
                    \centering
                    \begin{circuitikz}
                        \ctikzset{component text=left}
                        \draw
                        (0,0) node[or port] (myPort) {OR8WAY}
                        (myPort.in 1) node [anchor = east] {a}
                        (myPort.in 2) node [anchor = east] {b}
                        (myPort.out)  node [anchor = west] {out};
                    \end{circuitikz} 
                    \caption{Porta Lógica OR8WAY}
                \end{figure} \noindent

            \paragraph{Implementação}Esta porta lógica pode ser implementada de acordo com o seguinte código:
                \begin{scriptsize}
                    \myStyleVHDL
                    \lstinputlisting{./nand2tetris/projects/01/Or8Way.hdl}
                \end{scriptsize}
\newpage

        \subsection{MUX4WAY16}
            \paragraph{Definição}[Funcionamento]

            \paragraph{Tabela Verdade}Esta porta lógica possuirá a seguinte tabela verdade:
                \begin{table}[H]
                    \centering
                    \begin{tabular}[]{cc|c}\hline
                        a & b & out\\\hline
                        0 & 1 & 1\\
                        0 & 1 & 1\\
                        0 & 1 & 1\\
                        0 & 1 & 1\\\hline
                    \end{tabular}
                    \caption{Tabela Verdade MUX4WAY16}
                \end{table}

            \paragraph{Representação}Esta porta lógica pode ser expressa pelo seguinte circuito:
                \begin{figure}[H]
                    \centering
                    \begin{circuitikz}
                        \ctikzset{component text=left}
                        \draw
                        (0,0) node[muxdemux] (myPort) {MUX4WAY16};
                        % (myPort.in 1) node [anchor = east] {a}
                        % (myPort.in 2) node [anchor = east] {b}
                        % (myPort.out)  node [anchor = west] {out};
                    \end{circuitikz} 
                    \caption{Porta Lógica MUX4WAY16}
                \end{figure} \noindent

            \paragraph{Implementação}Esta porta lógica pode ser implementada de acordo com o seguinte código:
                \begin{scriptsize}
                    \myStyleVHDL
                    \lstinputlisting{./nand2tetris/projects/01/Mux4Way16.hdl}
                \end{scriptsize}
\newpage

        \subsection{MUX8WAY16}
            \paragraph{Definição}[Funcionamento]

            \paragraph{Tabela Verdade}Esta porta lógica possuirá a seguinte tabela verdade:
                \begin{table}[H]
                    \centering
                    \begin{tabular}[]{cc|c}\hline
                        a & b & out\\\hline
                        0 & 1 & 1\\
                        0 & 1 & 1\\
                        0 & 1 & 1\\
                        0 & 1 & 1\\\hline
                    \end{tabular}
                    \caption{Tabela Verdade MUX8WAY16}
                \end{table}

            \paragraph{Representação}Esta porta lógica pode ser expressa pelo seguinte circuito:
                \begin{figure}[H]
                    \centering
                    \begin{circuitikz}
                        \ctikzset{component text=left}
                        \draw
                        (0,0) node[muxdemux] (myPort) {MUX8WAY16};
                        % (myPort.in 1) node [anchor = east] {a}
                        % (myPort.in 2) node [anchor = east] {b}
                        % (myPort.out)  node [anchor = west] {out};
                    \end{circuitikz} 
                    \caption{Porta Lógica MUX8WAY16}
                \end{figure} \noindent

            \paragraph{Implementação}Esta porta lógica pode ser implementada de acordo com o seguinte código:
                \begin{scriptsize}
                    \myStyleVHDL
                    \lstinputlisting{./nand2tetris/projects/01/Mux8Way16.hdl}
                \end{scriptsize}
\newpage

        \subsection{DMUX4WAY}
            \paragraph{Definição}[Funcionamento]

            \paragraph{Tabela Verdade}Esta porta lógica possuirá a seguinte tabela verdade:
                \begin{table}[H]
                    \centering
                    \begin{tabular}[]{cc|c}\hline
                        a & b & out\\\hline
                        0 & 1 & 1\\
                        0 & 1 & 1\\
                        0 & 1 & 1\\
                        0 & 1 & 1\\\hline
                    \end{tabular}
                    \caption{Tabela Verdade DMUX4WAY}
                \end{table}

            \paragraph{Representação}Esta porta lógica pode ser expressa pelo seguinte circuito:
                \begin{figure}[H]
                    \centering
                    \begin{circuitikz}
                        \ctikzset{component text=left}
                        \draw
                        (0,0) node[demux] (myPort) {DMUX4WAY};
                        % (myPort.in 1) node [anchor = east] {a}
                        % (myPort.in 2) node [anchor = east] {b}
                        % (myPort.out)  node [anchor = west] {out};
                    \end{circuitikz} 
                    \caption{Porta Lógica DMUX4WAY}
                \end{figure} \noindent

            \paragraph{Implementação}Esta porta lógica pode ser implementada de acordo com o seguinte código:
                \begin{scriptsize}
                    \myStyleVHDL
                    \lstinputlisting{./nand2tetris/projects/01/DMux4Way.hdl}
                \end{scriptsize}
\newpage

        \subsection{DMUX8WAY}
            \paragraph{Definição}[Funcionamento]

            \paragraph{Tabela Verdade}Esta porta lógica possuirá a seguinte tabela verdade:
                \begin{table}[H]
                    \centering
                    \begin{tabular}[]{cc|c}\hline
                        a & b & out\\\hline
                        0 & 1 & 1\\
                        0 & 1 & 1\\
                        0 & 1 & 1\\
                        0 & 1 & 1\\\hline
                    \end{tabular}
                    \caption{Tabela Verdade DMUX8WAY}
                \end{table}

            \paragraph{Representação}Esta porta lógica pode ser expressa pelo seguinte circuito:
                \begin{figure}[H]
                    \centering
                    \begin{circuitikz}
                        \ctikzset{component text=left}
                        \draw
                        (0,0) node[demux] (myPort) {DMUX8WAY};
                        % (myPort.in 1) node [anchor = east] {a}
                        % (myPort.in 2) node [anchor = east] {b}
                        % (myPort.out)  node [anchor = west] {out};
                    \end{circuitikz} 
                    \caption{Porta Lógica DMUX8WAY}
                \end{figure} \noindent

            \paragraph{Implementação}Esta porta lógica pode ser implementada de acordo com o seguinte código:
                \begin{scriptsize}
                    \myStyleVHDL
                    \lstinputlisting{./nand2tetris/projects/01/DMux8Way.hdl}
                \end{scriptsize}
\newpage

    \section{Projeto 2}
        \subsection{HalfAdder}
            \paragraph{Definição}[Funcionamento]

            \paragraph{Tabela Verdade}Esta porta lógica possuirá a seguinte tabela verdade:
                \begin{table}[H]
                    \centering
                    \begin{tabular}[]{cc|cc}\hline
                        a & b & sum & out\\\hline
                        0 & 0 & 0   & 0\\
                        0 & 1 & 1   & 0\\
                        1 & 0 & 1   & 0\\
                        1 & 1 & 0   & 1\\\hline
                    \end{tabular}
                    \caption{Tabela Verdade HalfAdder}
                \end{table}

            \paragraph{Representação}Esta porta lógica pode ser expressa pelo seguinte circuito:
                \begin{figure}[H]
                    \centering
                    \begin{circuitikz}
                        \ctikzset{component text=left}
                        \draw;
                        % (0,0) node[demux] (myPort) {HalfAdder};
                        % (myPort.in 1) node [anchor = east] {a}
                        % (myPort.in 2) node [anchor = east] {b}
                        % (myPort.out)  node [anchor = west] {out};
                    \end{circuitikz} 
                    \caption{Porta Lógica HalfAdder}
                \end{figure} \noindent

            \paragraph{Implementação}Esta porta lógica pode ser implementada de acordo com o seguinte código:
                \begin{scriptsize}
                    \myStyleVHDL
                    \lstinputlisting{./nand2tetris/projects/02/HalfAdder.hdl}
                \end{scriptsize}
\newpage
    
        \subsection{FullAdder}
            \paragraph{Definição}[Funcionamento]

            \paragraph{Tabela Verdade}Esta porta lógica possuirá a seguinte tabela verdade:
                \begin{table}[H]
                    \centering
                    \begin{tabular}[]{cc|c}\hline
                        a & b & out\\\hline
                        0 & 1 & 1\\
                        0 & 1 & 1\\
                        0 & 1 & 1\\
                        0 & 1 & 1\\\hline
                    \end{tabular}
                    \caption{Tabela Verdade FullAdder}
                \end{table}

            \paragraph{Representação}Esta porta lógica pode ser expressa pelo seguinte circuito:
                \begin{figure}[H]
                    \centering
                    \begin{circuitikz}
                        \ctikzset{component text=left}
                        \draw;
                        % (0,0) node[demux] (myPort) {HalfAdder};
                        % (myPort.in 1) node [anchor = east] {a}
                        % (myPort.in 2) node [anchor = east] {b}
                        % (myPort.out)  node [anchor = west] {out};
                    \end{circuitikz} 
                    \caption{Porta Lógica FullAdder}
                \end{figure} \noindent

            \paragraph{Implementação}Esta porta lógica pode ser implementada de acordo com o seguinte código:
                \begin{scriptsize}
                    \myStyleVHDL
                    \lstinputlisting{./nand2tetris/projects/02/FullAdder.hdl}
                \end{scriptsize}
\newpage

        \subsection{Add16}
            \paragraph{Definição}[Funcionamento]

            \paragraph{Tabela Verdade}Esta porta lógica possuirá a seguinte tabela verdade:
                \begin{table}[H]
                    \centering
                    \begin{tabular}[]{cc|c}\hline
                        a & b & out\\\hline
                        0 & 1 & 1\\
                        0 & 1 & 1\\
                        0 & 1 & 1\\
                        0 & 1 & 1\\\hline
                    \end{tabular}
                    \caption{Tabela Verdade Add16}
                \end{table}

            \paragraph{Representação}Esta porta lógica pode ser expressa pelo seguinte circuito:
                \begin{figure}[H]
                    \centering
                    \begin{circuitikz}
                        \ctikzset{component text=left}
                        \draw;
                        % (0,0) node[demux] (myPort) {HalfAdder};
                        % (myPort.in 1) node [anchor = east] {a}
                        % (myPort.in 2) node [anchor = east] {b}
                        % (myPort.out)  node [anchor = west] {out};
                    \end{circuitikz} 
                    \caption{Porta Lógica Add16}
                \end{figure} \noindent

            \paragraph{Implementação}Esta porta lógica pode ser implementada de acordo com o seguinte código:
                \begin{scriptsize}
                    \myStyleVHDL
                    \lstinputlisting{./nand2tetris/projects/02/Add16.hdl}
                \end{scriptsize}
\newpage

        \subsection{Inc16}
            \paragraph{Definição}[Funcionamento]

            \paragraph{Tabela Verdade}Esta porta lógica possuirá a seguinte tabela verdade:
                \begin{table}[H]
                    \centering
                    \begin{tabular}[]{cc|c}\hline
                        a & b & out\\\hline
                        0 & 1 & 1\\
                        0 & 1 & 1\\
                        0 & 1 & 1\\
                        0 & 1 & 1\\\hline
                    \end{tabular}
                    \caption{Tabela Verdade Inc16}
                \end{table}

            \paragraph{Representação}Esta porta lógica pode ser expressa pelo seguinte circuito:
                \begin{figure}[H]
                    \centering
                    \begin{circuitikz}
                        \ctikzset{component text=left}
                        \draw;
                        % (0,0) node[demux] (myPort) {HalfAdder};
                        % (myPort.in 1) node [anchor = east] {a}
                        % (myPort.in 2) node [anchor = east] {b}
                        % (myPort.out)  node [anchor = west] {out};
                    \end{circuitikz} 
                    \caption{Porta Lógica Inc16}
                \end{figure} \noindent

            \paragraph{Implementação}Esta porta lógica pode ser implementada de acordo com o seguinte código:
                \begin{scriptsize}
                    \myStyleVHDL
                    \lstinputlisting{./nand2tetris/projects/02/Inc16.hdl}
                \end{scriptsize}
\newpage

        \subsection{ALU}
            \paragraph{Definição}[Funcionamento]

            \paragraph{Tabela Verdade}Esta porta lógica possuirá a seguinte tabela verdade:
                \begin{table}[H]
                    \centering
                    \begin{tabular}[]{cc|c}\hline
                        a & b & out\\\hline
                        0 & 1 & 1\\
                        0 & 1 & 1\\
                        0 & 1 & 1\\
                        0 & 1 & 1\\\hline
                    \end{tabular}
                    \caption{Tabela Verdade ALU}
                \end{table}

            \paragraph{Representação}Esta porta lógica pode ser expressa pelo seguinte circuito:
                \begin{figure}[H]
                    \centering
                    \begin{circuitikz}
                        \ctikzset{component text=left}
                        \draw;
                        % (0,0) node[demux] (myPort) {HalfAdder};
                        % (myPort.in 1) node [anchor = east] {a}
                        % (myPort.in 2) node [anchor = east] {b}
                        % (myPort.out)  node [anchor = west] {out};
                    \end{circuitikz} 
                    \caption{Porta Lógica ALU}
                \end{figure} \noindent

            \paragraph{Implementação}Esta porta lógica pode ser implementada de acordo com o seguinte código:
                \begin{scriptsize}
                    \myStyleVHDL
                    \lstinputlisting{./nand2tetris/projects/02/ALU.hdl}
                \end{scriptsize}
\newpage

    \section{Projeto 3}
        \subsection{Bit}
            \paragraph{Definição}[Funcionamento]

            \paragraph{Tabela Verdade}Esta porta lógica possuirá a seguinte tabela verdade:
                \begin{table}[H]
                    \centering
                    \begin{tabular}[]{cc|c}\hline
                        a & b & out\\\hline
                        0 & 1 & 1\\
                        0 & 1 & 1\\
                        0 & 1 & 1\\
                        0 & 1 & 1\\\hline
                    \end{tabular}
                    \caption{Tabela Verdade Bit}
                \end{table}

            \paragraph{Representação}Esta porta lógica pode ser expressa pelo seguinte circuito:
                \begin{figure}[H]
                    \centering
                    \begin{circuitikz}
                        \ctikzset{component text=left}
                        \draw;
                        % (0,0) node[demux] (myPort) {HalfAdder};
                        % (myPort.in 1) node [anchor = east] {a}
                        % (myPort.in 2) node [anchor = east] {b}
                        % (myPort.out)  node [anchor = west] {out};
                    \end{circuitikz} 
                    \caption{Porta Lógica Bit}
                \end{figure} \noindent

            \paragraph{Implementação}Esta porta lógica pode ser implementada de acordo com o seguinte código:
                \begin{scriptsize}
                    \myStyleVHDL
                    \lstinputlisting{./nand2tetris/projects/03/a/Bit.hdl}
                \end{scriptsize}
\newpage

        \subsection{Register}
            \paragraph{Definição}[Funcionamento]

            \paragraph{Tabela Verdade}Esta porta lógica possuirá a seguinte tabela verdade:
                \begin{table}[H]
                    \centering
                    \begin{tabular}[]{cc|c}\hline
                        a & b & out\\\hline
                        0 & 1 & 1\\
                        0 & 1 & 1\\
                        0 & 1 & 1\\
                        0 & 1 & 1\\\hline
                    \end{tabular}
                    \caption{Tabela Verdade Register}
                \end{table}

            \paragraph{Representação}Esta porta lógica pode ser expressa pelo seguinte circuito:
                \begin{figure}[H]
                    \centering
                    \begin{circuitikz}
                        \ctikzset{component text=left}
                        \draw;
                        % (0,0) node[demux] (myPort) {HalfAdder};
                        % (myPort.in 1) node [anchor = east] {a}
                        % (myPort.in 2) node [anchor = east] {b}
                        % (myPort.out)  node [anchor = west] {out};
                    \end{circuitikz} 
                    \caption{Porta Lógica Register}
                \end{figure} \noindent

            \paragraph{Implementação}Esta porta lógica pode ser implementada de acordo com o seguinte código:
                \begin{scriptsize}
                    \myStyleVHDL
                    \lstinputlisting{./nand2tetris/projects/03/a/Register.hdl}
                \end{scriptsize}
\newpage

    \subsection{RAM8}
        \paragraph{Definição}[Funcionamento]

        \paragraph{Tabela Verdade}Esta porta lógica possuirá a seguinte tabela verdade:
            \begin{table}[H]
                \centering
                \begin{tabular}[]{cc|c}\hline
                    a & b & out\\\hline
                    0 & 1 & 1\\
                    0 & 1 & 1\\
                    0 & 1 & 1\\
                    0 & 1 & 1\\\hline
                \end{tabular}
                \caption{Tabela Verdade RAM8}
            \end{table}

        \paragraph{Representação}Esta porta lógica pode ser expressa pelo seguinte circuito:
            \begin{figure}[H]
                \centering
                \begin{circuitikz}
                    \ctikzset{component text=left}
                    \draw;
                    % (0,0) node[demux] (myPort) {HalfAdder};
                    % (myPort.in 1) node [anchor = east] {a}
                    % (myPort.in 2) node [anchor = east] {b}
                    % (myPort.out)  node [anchor = west] {out};
                \end{circuitikz} 
                \caption{Porta Lógica RAM8}
            \end{figure} \noindent

        \paragraph{Implementação}Esta porta lógica pode ser implementada de acordo com o seguinte código:
            \begin{scriptsize}
                \myStyleVHDL
                \lstinputlisting{./nand2tetris/projects/03/a/RAM8.hdl}
            \end{scriptsize}
\newpage

    \subsection{RAM64}
        \paragraph{Definição}[Funcionamento]

        \paragraph{Tabela Verdade}Esta porta lógica possuirá a seguinte tabela verdade:
            \begin{table}[H]
                \centering
                \begin{tabular}[]{cc|c}\hline
                    a & b & out\\\hline
                    0 & 1 & 1\\
                    0 & 1 & 1\\
                    0 & 1 & 1\\
                    0 & 1 & 1\\\hline
                \end{tabular}
                \caption{Tabela Verdade RAM64}
            \end{table}

        \paragraph{Representação}Esta porta lógica pode ser expressa pelo seguinte circuito:
            \begin{figure}[H]
                \centering
                \begin{circuitikz}
                    \ctikzset{component text=left}
                    \draw;
                    % (0,0) node[demux] (myPort) {HalfAdder};
                    % (myPort.in 1) node [anchor = east] {a}
                    % (myPort.in 2) node [anchor = east] {b}
                    % (myPort.out)  node [anchor = west] {out};
                \end{circuitikz} 
                \caption{Porta Lógica RAM64}
            \end{figure} \noindent

        \paragraph{Implementação}Esta porta lógica pode ser implementada de acordo com o seguinte código:
            \begin{scriptsize}
                \myStyleVHDL
                \lstinputlisting{./nand2tetris/projects/03/a/RAM64.hdl}
            \end{scriptsize}
\newpage

    \subsection{RAM512}
        \paragraph{Definição}[Funcionamento]

        \paragraph{Tabela Verdade}Esta porta lógica possuirá a seguinte tabela verdade:
            \begin{table}[H]
                \centering
                \begin{tabular}[]{cc|c}\hline
                    a & b & out\\\hline
                    0 & 1 & 1\\
                    0 & 1 & 1\\
                    0 & 1 & 1\\
                    0 & 1 & 1\\\hline
                \end{tabular}
                \caption{Tabela Verdade RAM512}
            \end{table}

        \paragraph{Representação}Esta porta lógica pode ser expressa pelo seguinte circuito:
            \begin{figure}[H]
                \centering
                \begin{circuitikz}
                    \ctikzset{component text=left}
                    \draw;
                    % (0,0) node[demux] (myPort) {HalfAdder};
                    % (myPort.in 1) node [anchor = east] {a}
                    % (myPort.in 2) node [anchor = east] {b}
                    % (myPort.out)  node [anchor = west] {out};
                \end{circuitikz} 
                \caption{Porta Lógica RAM512}
            \end{figure} \noindent

        \paragraph{Implementação}Esta porta lógica pode ser implementada de acordo com o seguinte código:
            \begin{scriptsize}
                \myStyleVHDL
                \lstinputlisting{./nand2tetris/projects/03/b/RAM512.hdl}
            \end{scriptsize}
\newpage

    \subsection{RAM4K}
        \paragraph{Definição}[Funcionamento]

        \paragraph{Tabela Verdade}Esta porta lógica possuirá a seguinte tabela verdade:
            \begin{table}[H]
                \centering
                \begin{tabular}[]{cc|c}\hline
                    a & b & out\\\hline
                    0 & 1 & 1\\
                    0 & 1 & 1\\
                    0 & 1 & 1\\
                    0 & 1 & 1\\\hline
                \end{tabular}
                \caption{Tabela Verdade RAM4K}
            \end{table}

        \paragraph{Representação}Esta porta lógica pode ser expressa pelo seguinte circuito:
            \begin{figure}[H]
                \centering
                \begin{circuitikz}
                    \ctikzset{component text=left}
                    \draw;
                    % (0,0) node[demux] (myPort) {HalfAdder};
                    % (myPort.in 1) node [anchor = east] {a}
                    % (myPort.in 2) node [anchor = east] {b}
                    % (myPort.out)  node [anchor = west] {out};
                \end{circuitikz} 
                \caption{Porta Lógica RAM4K}
            \end{figure} \noindent

        \paragraph{Implementação}Esta porta lógica pode ser implementada de acordo com o seguinte código:
            \begin{scriptsize}
                \myStyleVHDL
                \lstinputlisting{./nand2tetris/projects/03/b/RAM4K.hdl}
            \end{scriptsize}
\newpage

    \subsection{RAM16K}
        \paragraph{Definição}[Funcionamento]

        \paragraph{Tabela Verdade}Esta porta lógica possuirá a seguinte tabela verdade:
            \begin{table}[H]
                \centering
                \begin{tabular}[]{cc|c}\hline
                    a & b & out\\\hline
                    0 & 1 & 1\\
                    0 & 1 & 1\\
                    0 & 1 & 1\\
                    0 & 1 & 1\\\hline
                \end{tabular}
                \caption{Tabela Verdade RAM16K}
            \end{table}

        \paragraph{Representação}Esta porta lógica pode ser expressa pelo seguinte circuito:
            \begin{figure}[H]
                \centering
                \begin{circuitikz}
                    \ctikzset{component text=left}
                    \draw;
                    % (0,0) node[demux] (myPort) {HalfAdder};
                    % (myPort.in 1) node [anchor = east] {a}
                    % (myPort.in 2) node [anchor = east] {b}
                    % (myPort.out)  node [anchor = west] {out};
                \end{circuitikz} 
                \caption{Porta Lógica RAM16K}
            \end{figure} \noindent

        \paragraph{Implementação}Esta porta lógica pode ser implementada de acordo com o seguinte código:
            \begin{scriptsize}
                \myStyleVHDL
                \lstinputlisting{./nand2tetris/projects/03/b/RAM16K.hdl}
            \end{scriptsize}
\newpage

    \subsection{PC}
        \paragraph{Definição}[Funcionamento]

        \paragraph{Tabela Verdade}Esta porta lógica possuirá a seguinte tabela verdade:
            \begin{table}[H]
                \centering
                \begin{tabular}[]{cc|c}\hline
                    a & b & out\\\hline
                    0 & 1 & 1\\
                    0 & 1 & 1\\
                    0 & 1 & 1\\
                    0 & 1 & 1\\\hline
                \end{tabular}
                \caption{Tabela Verdade PC}
            \end{table}

        \paragraph{Representação}Esta porta lógica pode ser expressa pelo seguinte circuito:
            \begin{figure}[H]
                \centering
                \begin{circuitikz}
                    \ctikzset{component text=left}
                    \draw;
                    % (0,0) node[demux] (myPort) {HalfAdder};
                    % (myPort.in 1) node [anchor = east] {a}
                    % (myPort.in 2) node [anchor = east] {b}
                    % (myPort.out)  node [anchor = west] {out};
                \end{circuitikz} 
                \caption{Porta Lógica PC}
            \end{figure} \noindent

        \paragraph{Implementação}Esta porta lógica pode ser implementada de acordo com o seguinte código:
            \begin{scriptsize}
                \myStyleVHDL
                \lstinputlisting{./nand2tetris/projects/03/a/PC.hdl}
            \end{scriptsize}
\newpage

    \section{Projeto 4}
        \subsection{Linguagem de Máquina Hack}
            \paragraph{Definição}Linguagem básica utilizada para manipular as portas lógicas implementadas anteriormente, consistindo de comandos binários com palavras de 16-bit. Há duas possíveis instruções suportadas por esta linguagem, descritas a seguir:
                \begin{enumerate}[noitemsep]
                    \item \textbf{A Instructions:} Registrador A assume o valor de \texttt{value};
                        \begin{scriptsize}
                            \myStyleAssembler
                            \begin{lstlisting}
    @value
                            \end{lstlisting}
                        \end{scriptsize}
                    \item \textbf{C Instructions:} Realiza a operação \texttt{comp}, armazena o resultado em \texttt{dest} e ;
                        \begin{scriptsize}
                            \myStyleAssembler
                            \begin{lstlisting}
    dest = comp ; jump
                            \end{lstlisting}
                        \end{scriptsize}
                    Onde:
                        \begin{enumerate}[noitemsep]
                            \item \texttt{comp}: Representa todas as possíveis operações que a \texttt{ALU} poderá realizar, expressas a seguir:
                                \begin{table}[H]
                                    \centering
                                    \begin{tabular}[]{l|rrrr}\hline
                                        Operação     & Resultado & & &\\\hline
                                        Tornar Zero  & 0          &            &       &\\
                                        Tornar Um    & 1          &            &       &\\
                                        Tornar $-$Um & $-$1       &            &       &\\
                                        Manter       & D          &  A         &  M    &\\
                                        Negar        &!D          & !A         & !M    &\\
                                        Oposto       &-D          & $-$A       & $-$M  &\\
                                        Incrementar  & D+1        & A+1        & M+1   &\\
                                        Decrementar  & D$-$1      & A$-$1      & M$-$1 &\\
                                        Somar        & D+A        & D+M        &       &\\
                                        Subtrair     & D$-$A      & D$-$M      & A$-$D & M$-$D\\
                                        \texttt{AND} & D\&A       & D\&M       &       &\\
                                        \texttt{OR}  & D $\mid$ A & D $\mid$ M &       &\\\hline
                                    \end{tabular}
                                    \caption{Operações \texttt{ALU}}
                                \end{table}
                            \item \texttt{dest}: Representa o destino do resultado da operação realizada, expressa a seguir:
                                \begin{table}[H]
                                    \centering
                                    \begin{tabular}[]{l|l}\hline
                                        Operação      & Resultado\\\hline
                                        \texttt{null} & Descarta Resultado\\
                                        \texttt{M}    & Armazena na Memória\\
                                        \texttt{D}    & Armazena no Registrador D\\
                                        \texttt{MD}   & Armazena na Memnória e no Registrador D\\
                                        \texttt{A}    & Armazena em A\\
                                        \texttt{AM}   & Armazena em A e na Memória\\
                                        \texttt{AD}   & Armazena em A e no Registrador D\\
                                        \texttt{AMD}  & Armazena em A, na Memória e no Registrador D\\
                                    \end{tabular}
                                    \caption{Destinos de \texttt{C}}
                                \end{table}
                        \end{enumerate}
                \end{enumerate}
\end{document}