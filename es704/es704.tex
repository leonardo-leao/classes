\documentclass{article}
\usepackage{tpack}

\title{ES704 - Instrumentação Básica}
\author{Guilherme Nunes Trofino}
\authorRA{217276}
\project{Resumo Teórico}

\begin{document}
    \maketitle
\newpage

    \tableofcontents
\newpage

\section{Introdução}
    \paragraph{Apresentação}Neste documento será descrito as informações necessárias para compreensão e solução de exercícios relacionados a disciplina \thetitle. Note que este documento são notas realizadas por \theauthor, em \today.

    \subsection{Medição}
        \paragraph{Definição}Atribuição do valor, ou tendência de valores, à variável de interesse, normalmente proviniente a um sistema que se deja analisar.

    \subsection{Calibração}
        \paragraph{Definição}Determinar matematicamente a relação entre a entrada e sistema de medição, possuindo \textbf{dimensões} que devem estar de acordo com as \textbf{Normas Técnicas}. Classificadas de acordo com mostrado a seguir:

        \subsubsection{Calibração Estática}
            \paragraph{Definição}Aplicar uma entrada conhecida para medir a resposta do sistema, representada em um gráfico como mostrado a seguir:

            \paragraph{Curva de Calibração}Relação entre a entrada e saída de dados definida por $y = f(x)$.

            \paragraph{Sensibilidade Estática}Representa a proporção entre a entrada e a saída nas regiões de baixa e alta sensibilidade como expressado pela seguinte equação:
                \begin{equation}
                    \boxed{
                        K = \diff{f(x)}{x}\Bigr|_{\substack{x = x_{0}}}
                    }\label{eq:staticSensitibility}
                \end{equation}

            \paragraph{Faixa Dinâmica}Intervalo no qual a curva de calibração é válida, pois há dados para suportar as hipóteses. Fora deste intervalo, a resposta será uma \textbf{Extrapolação}.
                \begin{equation}
                    \boxed{r_{i} = x_{\text{max}} - x_{\text{min}}}
                    \qquad
                    \boxed{r_{o} = y_{\text{max}} - y_{\text{min}}}
                \end{equation}

            \paragraph{Resolução}Menor incremento que pode ser detectado pelo sistema de medição. Geralmente determinado pela escala indicada pelos equipamentos de medição ou pelas restrições do sistema obtido pela seguinte equação:
                \begin{equation}
                    \boxed{\Delta R = \frac{\Delta E}{K}}
                \end{equation}
            Onde:
                \begin{enumerate}[rightmargin = \leftmargin]
                    \item \textbf{Resolução Equipamento:} $\Delta E$, Presente no equipamento de medição;
                    \item \textbf{Sensibilidade Estática:} $K$ dada pela equação \ref{eq:staticSensitibility};
                \end{enumerate}

            \paragraph{Coeficiente de Determinação}Grau de compatibilidade entre os valores amostrados, $y$, e os valores do \textbf{Curve Fitting}, $y_{C}$, relacionados pela seguinte equação:
                \begin{equation}
                    \boxed{
                        R^{2} = 1 - \frac{\sum(y - y_{C})^{2}}{\sum(y - \bar{y})^{2}}
                    }
                \end{equation}
            Quanto mais próximo de 1 melhor será o ajuste da curva.

        \subsubsection{Calibração Dinâmica}
            \paragraph{Definição}Aplicar uma entrada conhecida para medir a resposta do sistema com a saída sempre seguindo a forma de onda da entrada.

            \paragraph{Constante de Tempo}Variável denotada pela constante $\tau$ para representar o tempo necessário para que o sistema atinga $63,2\%$ de seu valor final. Comumente, define-se que $5\tau$ representa o tempo necessário para que sistema se estabilize após um estímulo.

    \subsection{Erro}
        \paragraph{Definição}Diferença entre os valores medidos $y$ e seus correspondentes valores reais $y'$ de uma variável como expresso pela seguinte equação:
            \begin{equation}
                \boxed{
                    e = |y - y'|
                }
            \end{equation}

        \subsubsection{Erro Aleatório}
            \paragraph{Definição}Problemas que afetam a \textbf{Precisão} do sistema, pois causam o espalhamento dos dados em relação ao valor real.

        \subsubsection{Erro Sistemático}
            \paragraph{Definição}Problemas que afetam a \textbf{Exatidão} do sistema, pois causam a deflexão do valor medido em relação ao valor real.

        \subsubsection{Erro Experimental}
            \paragraph{Definição}Diferentes problemas visíveis durante a análise dos dados causados por métodos inadequados de experimentação, entre os principais estão:
                \begin{enumerate}[rightmargin = \leftmargin]
                    \item \textbf{Histerese:} Diferença na resposta do sistema exitado por entradas crescentes, \texttt{upscale} e decrescentes, \texttt{downscale};

                    \item \textbf{Linearidade:} Diferença na resposta em relação a curva de calibração linear;

                    \item \textbf{Sensibilidade:} Diferença no valor do ganho do sistema ;

                    \item \textbf{Retorno ao Zero:} Diferença na resposta do sistema para uma entrada nula;
                \end{enumerate}

    \section{Análise de Sinais}
        \paragraph{Definição}Estudo de informações físicas sobre a variável medida, transmitida entre um processo e o sistema de medição entre os eságios do sistema de medição classificados em:
            \begin{enumerate}
                \item \textbf{Classes:}
                    \begin{enumerate}[noitemsep, rightmargin = \leftmargin]
                        \item \texttt{Analógico:} Contínuo em Magnitude e Tempo;
                        \item \texttt{Discreto:} Contínuo em Magnitude e Discretizado no Tempo;
                        \item \texttt{Digital:} Quantizado em Magnitude e Discretizado no Tempo;
                    \end{enumerate}

                \item \textbf{Magnitude:}
                    \begin{enumerate}[noitemsep, rightmargin = \leftmargin]
                        \item \texttt{Dinâmico:} Varia no tempo;
                        \item \texttt{Estático:} Não varia no tempo;
                    \end{enumerate}

                \item \textbf{Periodicidade:}
                    \begin{enumerate}[noitemsep, rightmargin = \leftmargin]
                        \item \texttt{Periódico:} Apresenta repetição;
                        \item \texttt{Aperiódico:} Não apresenta repetição;
                    \end{enumerate}

                \item \textbf{Previsibilidade:}
                    \begin{enumerate}[noitemsep, rightmargin = \leftmargin]
                        \item \texttt{Determinístico:} Expressa por uma função matemática;
                        \item \texttt{Não-Determinístico:} Não expressa por uma função matemática;
                    \end{enumerate}
            \end{enumerate}

        \subsection{Redução de Dados}
            \paragraph{Definição}Análise dos dados sobre dois parâmetros listados abaixo:

            \subsubsection{Valor Médio}
                \paragraph{Definição}Caraceriza a componente DC, contínua, do sinal através da seguinte equação:
                    \begin{equation}
                        \boxed{\bar{y} = \frac{1}{N} \sum_{n=1}^{N} y(n)}
                    \end{equation}

            \subsubsection{Valor Root Mean Square}
                \paragraph{Definição}Caraceriza a componente AC, alternada, do sinal através da seguinte equação:
                    \begin{equation}
                        \boxed{
                            y_{\text{RMS}} = \sqrt{\frac{1}{N} \sum_{n=1}^{N} y^{2}(n)}
                        }
                    \end{equation}

        \subsection{Análise em Frequência}
            \paragraph{Definição}Sinais apresentaram informações relevantes sobre suas frequências de funcionamento e dessa forma será necessário analisá-lo com algumas ferramentas descritas na sequência.

            \subsubsection{Série de Fourier}
                \paragraph{Definição}Dado um sinal periódico $y(t) = y(t + T)$ com período $T = \frac{1}{f} = \frac{2\pi}{\omega}$ então este poderá ser ser representado através de uma \textbf{Série de Fourier} pela seguinte equação:
                    \begin{equation}
                        \boxed{
                            y(t) = 
                            A_{0} + 
                            \sum_{n=1}^{\infty} (A_{n}\cos(n\omega t) + B_{n}\sin(n\omega t))
                        }
                        \qquad
                        \text{onde:}
                        \qquad
                        \boxed{
                            \begin{cases}
                                A_{0} = \frac{1}{T} \int_{-\frac{T}{2}}^{+\frac{T}{2}} y(t)\text{d}t\\[2.5mm]
                                A_{n} = \frac{2}{T} \int_{-\frac{T}{2}}^{+\frac{T}{2}} y(t)\cos(n\omega t)\text{d}t\\[2.5mm]
                                B_{n} = \frac{2}{T} \int_{-\frac{T}{2}}^{+\frac{T}{2}} y(t)\sin(n\omega t)\text{d}t\\
                            \end{cases}
                        }
                    \end{equation}
                Onde:
                    \begin{enumerate}[noitemsep]
                        \item Se $y(t)$ é par então $B_{n} = 0$;
                        \item Se $y(t)$ é ímpar então $A_{n} = 0$;
                    \end{enumerate}

            \subsubsection{Transformada de Fourier}
                \paragraph{Definição}Extensão da séri de Fourier para sinais periódicos e aperiódicos descrito a seguir:
                    \begin{multicols}{2}
                        \raggedcolumns
                        \paragraph{Transformada Direta:}
                            \begin{equation}
                                \boxed{
                                    Y(\omega) = \mathcal{F} \{ y(t) \} := 
                                    \int_{-\infty}^{+\infty} y(t) \; e^{-j\omega t} \; \text{d}t
                                }
                            \end{equation}
        
                        \columnbreak
        
                        \paragraph{Transformada Inversa:}
                            \begin{equation}
                                \boxed{
                                    y(t) = \mathcal{F}^{-1}\{ Y(\omega) \} := 
                                    \int_{-\infty}^{+\infty} Y(\omega) \; e^{j\omega t} \; \text{d} \omega
                                }
                            \end{equation}
                    \end{multicols}\noindent
                Note que o resultado da \textbf{Transformada de Fourier} será um número complexo da forma:
                    \begin{equation*}
                        \boxed{
                            Y(\omega) = Y_{\text{R}}(\omega) + j Y_{\text{I}}(\omega)
                        }
                    \end{equation*}
                Onde:
                    \begin{enumerate}[noitemsep]
                        \item Representa a parcela \texttt{Real} como $Y_{\text{R}}(\omega)$;

                        \item Representa a parcela \texttt{Imaginária} como $Y_{\text{I}}(\omega)$;
                    \end{enumerate}
                Assim como presente na teoria de Circuitos Elétricos, números complexos podem ser representados através de \textbf{Fasores} como represenado pela seguinte equação:
                    \begin{equation}
                        \boxed{
                            Y(\omega) = |Y(\omega)| \phase{\phi(\omega)^{\circ}}
                        }
                    \end{equation}
                Onde:
                    \begin{enumerate}[noitemsep]
                        \item \textbf{Espectro de Magnitude:} Ou também nomeado como \textbf{Amplitude} pela seguinte equação:
                            \begin{equation*}
                                \boxed{
                                    |Y(\omega)| = \sqrt{Y_{\text{R}}^{2}(\omega) + Y_{\text{I}}^{2}(\omega)}
                                }
                            \end{equation*}

                        \item \textbf{Espectro de Fase:} Ou também nomeado como \textbf{Fase} pela seguinte equação:
                            \begin{equation*}
                                \boxed{
                                    \phi(\omega) = \tan^{-1}
                                    \left(
                                        \frac{Y_{\text{I}}}{Y_{\text{R}}}
                                    \right)
                                }
                            \end{equation*}
                    \end{enumerate}

        \subsubsection{Transformada de Fourier Discreta}
            \paragraph{Definição}Seja um sinal discreto $y(n\Delta t) \approx y(n)$, onde $\Delta t$ é o incremento temporal então a \textbf{Transformada de Fourier} será dada pela seguinte equação:
                \begin{equation}
                    \boxed{
                        Y(k\Delta f) \approx Y(k) = \frac{2}{N} \sum_{n=0}^{N-1} y(n) e^{-j2\pi \frac{k}{N}}
                    }
                \end{equation}
            Onde:
                \begin{enumerate}[noitemsep]
                    \item \textbf{Incremento Espectral:} $\Delta f = \frac{1}{M\Delta t} = \frac{\Delta\omega}{2\pi}$;
                \end{enumerate}
            Note que esta transformação será calculada computacionalmente pelo algoritmo \texttt{Fast Fourier Transform} onde $M$ deve ser utilizada como uma potência de 2. A fim de evitar vazameno espectral, também nomeado \textbf{Leakage}, aplica-se a \texttt{Função de Janelamento} antes da \texttt{FFT}.
\end{document}