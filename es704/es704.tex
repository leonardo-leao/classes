\documentclass{article}
\usepackage{tpack}

\title{ES704 - Instrumentação Básica}
\author{Guilherme Nunes Trofino}
\authorRA{217276}
\project{Resumo Teórico}

\begin{document}
    \maketitle
\newpage

    \tableofcontents
\newpage

    \section{Introdução}
        \paragraph{Apresentação}Neste documento será descrito as informações necessárias para compreensão e solução de exercícios relacionados a disciplina \thetitle. Note que este documento são notas realizadas por \theauthor, em \today.

    \subsection{Medição}
        \paragraph{Definição}Atribuição do valor, ou tendência de valores, à variável de interesse, normalmente proviniente a um sistema que se deja analisar.

    \subsection{Calibração}
        \paragraph{Definição}Determinar matematicamente a relação entre a entrada e sistema de medição, possuindo \textbf{dimensões} que devem estar de acordo com as \textbf{Normas Técnicas}. Classificadas de acordo com mostrado a seguir:

        \subsubsection{Calibração Estática}
            \paragraph{Definição}Aplicar uma entrada conhecida para medir a resposta do sistema, representada em um gráfico como mostrado a seguir:

            \paragraph{Curva de Calibração}Relação entre a entrada e saída de dados definida por $y = f(x)$.

            \paragraph{Sensibilidade Estática}Representa a proporção entre a entrada e a saída nas regiões de baixa e alta sensibilidade como expressado pela seguinte equação:
                \begin{equation}
                    \boxed{
                        K = \diff{f(x)}{x}\Bigr|_{\substack{x = x_{0}}}
                    }
                \end{equation}

            \paragraph{Faixa Dinâmica}Intervalo no qual a curva de calibração é válida, pois há dados para suportar as hipóteses. Fora deste intervalo, a resposta será uma \textbf{Extrapolação}.
                \begin{equation}
                    \boxed{r_{i} = x_{\text{max}} - x_{\text{min}}}
                    \qquad
                    \boxed{r_{o} = y_{\text{max}} - y_{\text{min}}}
                \end{equation}

            \paragraph{Resolução}Menor incremento que pode ser detectado pelo sistema de medição. Geralmente determinado pela escala indicada pelos equipamentos de medição ou pelas restrições do sistema.

\end{document}