\documentclass{article}
\usepackage{tpack}

\title{ES570 - Transferência de Calor}
\author{Guilherme Nunes Trofino}
\authorRA{217276}
\project{Resumo Teórico}

\begin{document}
    \maketitle
\newpage

    \tableofcontents
\newpage

    \section{Introdução}
        \paragraph{Apresentação}Neste documento será descrito as informações necessárias para compreensão e solução de exercícios relacionados a disciplina \thetitle. Note que este documento são notas realizadas por \theauthor, em \today.

        \paragraph{Definição}Transferência de Calor é a energia térmica em trânsito devido a uma \texttt{Diferença de Temperatura} no espaço que ocorre nos seguintes processos:

        \subsection{Condução}
            \paragraph{Definição}Energia transferida de partículas mais energéticas para menos energéticas de uma substância devido às interações entre partículas em repouso.

        \subsubsection{Lei de Fourier}
            \paragraph{Definição}Considera-se que o fluxo de energia causado pela transferência de calor, \textbf{Fluxo Térmico}, através do espaço por unidade de tempo será dado por:
                \begin{equation}
                    \boxed{
                        \boldsymbol{q}'' = - K \nabla \boldsymbol{T} 
                        \quad 
                        \left[\frac{\text{W}}{\text{m\textsuperscript{2}}}\right]
                    }\label{eq.Fourier}
                \end{equation}
            Onde:
                \begin{enumerate}[rightmargin = \leftmargin]
                    \item \textbf{Fluxo Térmico:} $\boldsymbol{q}''$ em $\left[\frac{\text{W}}{\text{m\textsuperscript{2}}}\right]$, perpedicular a direção de transferência;
                        \begin{enumerate}[noitemsep]
                            \item \texttt{Unidimensional:} Caso trate-se de uma dimensão está equação será simplificada:
                                \begin{equation*}
                                    \boxed{
                                        q_{x}'' = - K \diff{T}{x}
                                    }
                                \end{equation*}
                            Note que qualquer direção cartesiana poderia ser considerada para a equação.
                        \end{enumerate}
                    Considera-se durante análise de que o meio em que se dá a condução será \textbf{Isotérmico}, implicando que a \texttt{Condutividade Térmica} seja independente das direções empregadas. Além disso, a \textbf{Direção Normal} será normal a isotérmica na direção decrescente de temperatura.

                    \item \textbf{Condutividade Térmica:} $K$ em $\left[\frac{\text{W}}{\text{m K}}\right]$;
                        \begin{enumerate}[noitemsep, rightmargin = \leftmargin]
                            \item \texttt{Convenção:} Trata-se de um fluxo da maior para a menor temperatura dessa forma há o sinal negativo;
                        \end{enumerate}

                    \item \textbf{Gradiente de Temperatura:} $\nabla \boldsymbol{T}$ em $\left[\frac{\text{K}}{\text{m}}\right]$;
                        \begin{enumerate}[noitemsep]
                            \item \texttt{Coordenadas Cartesianas:}
                                \begin{equation}
                                    \boxed{
                                        \nabla \boldsymbol{T} = 
                                        \left(
                                            \boldsymbol{i}\diffp{T}{x} + 
                                            \boldsymbol{j}\diffp{T}{y} + 
                                            \boldsymbol{k}\diffp{T}{z}
                                        \right)
                                    }
                                \end{equation}
                        \end{enumerate}

                        \begin{enumerate}[noitemsep]
                            \item \texttt{Coordenadas Cilíndricas:}
                                \begin{equation}
                                    \boxed{
                                        \nabla \boldsymbol{T} = 
                                        \left(
                                            \boldsymbol{i}\diffp{T}{r} + 
                                            \boldsymbol{j}\frac{1}{r}\diffp{T}{\phi} + 
                                            \boldsymbol{k}\diffp{T}{z}
                                        \right)
                                    }
                                \end{equation}
                        \end{enumerate}

                        \begin{enumerate}[noitemsep]
                            \item \texttt{Coordenadas Esféricas:}
                                \begin{equation}
                                    \boxed{
                                        \nabla \boldsymbol{T} = 
                                        \left(
                                            \boldsymbol{i}\diffp{T}{r} + 
                                            \boldsymbol{j}\frac{1}{r}\diffp{T}{\phi} + 
                                            \boldsymbol{k}\frac{1}{r\sin(\theta)}\diffp{T}{\theta}
                                        \right)
                                    }
                                \end{equation}
                        \end{enumerate}
                \end{enumerate}
            Esta lei é deduzida experimentalmente e portanto não há dedução formal, os resultados são obtidos a partir da observação.

        \subsection{Convecção}
            \paragraph{Definição}Energia transferida pelo fluxo de partículas seja um Movimento Aleatório ou um Movimento Global do fluído através de dois mecanismos:
                \begin{enumerate}[noitemsep]
                    \item \textbf{Convecção Forçada:} Escoamento causado por meios externos; 
                    \item \textbf{Convecção Natural:} Escoamento induzido por diferenças de densidade; 
                \end{enumerate}
            Quantifica-se este fluxo de energia, independente do mecanismo, através da seguinte equação:
                \begin{equation}
                    \boxed{
                        q_{\text{conv}}'' = \frac{q}{A} = h (T_{S} - T_{\infty})
                        \quad 
                        \left[\frac{\text{W}}{\text{m\textsuperscript{2}}}\right]
                    }
                \end{equation}
            Onde:
                \begin{enumerate}[noitemsep, rightmargin = \leftmargin]
                    \item \textbf{Fluxo Térmico:} $q_{\text{conv}}''$ em $\left[\frac{\text{W}}{\text{m\textsuperscript{2}}}\right]$;

                    \item \textbf{Coeficiente de Transferêcia:} $h$ em $\left[\frac{\text{W}}{\text{m\textsuperscript{2} K}}\right]$;
                \end{enumerate}
            Nota-se que este coeficiente apresentará os seguintes valores típicos:
                \begin{table}[H]
                    \centering
                    \begin{tabular}{lll}\hline
                        Situação          &          & h\;$\left[\frac{\text{W}}{\text{m\textsuperscript{2} K}}\right]$\\\hline
                        Convecção Natural & Gases    & 2 - 25\\
                                          & Líquidos & 50 - 1000\\
                        Convecção Forçada & Gases    & 25 - 250\\
                                          & Líquidos & 100 - 20000\\
                        Mudança de Fase   &          & 2500 - 100000\\\hline
                    \end{tabular}
                    \caption{Coeficiente de Transferência Térmica por Convecção}
                    \label{table:convectionConstant}
                \end{table}\noindent


        \subsection{Radiação}
            \paragraph{Definição}Energia transferida, não necessariamente demandando um meio material, pela matéria que se encontra a uma temperatura não nula. 

        \subsubsection{Radiação Emitida}
            \paragraph{Definição}Quantifica-se o fluxo de energia saindo de um corpo por radiação através da seguinte equação:
                \begin{equation}
                    \boxed{
                        E = \epsilon\;\sigma\;T_{S}^{4}
                        \quad
                        \left[\frac{\text{W}}{\text{m\textsuperscript{2}}}\right]
                    }
                \end{equation}
            Onde:
                \begin{enumerate}[noitemsep]
                    \item \textbf{Permissividade:} $0 \leq \epsilon \leq 1$ adimensional;
                    \item \textbf{Constante de Stefan-Boltzmann:} $\sigma = 5,67\times 10^{-8}$ em $\left[\frac{\text{W}}{\text{m\textsuperscript{2} K\textsuperscript{4}}}\right]$;
                    \item \textbf{Temperatura Absoluta:} $T_{S}$ em $[K]$;
                \end{enumerate}

        \subsubsection{Radiação Recebida}
            \paragraph{Definição}Quantifica-se o fluxo de energia recebida, também nomeada como \textbf{Irradiação}, por um corpo por radiação através da seguinte equação:
                \begin{equation}
                    \boxed{
                        G = G_{\text{abs}} + G_{\text{unknown}} + G_{\text{ref}}
                    }
                \end{equation}
            Onde:
                \begin{enumerate}[noitemsep]
                    \item \textbf{Radiação Recebida:} $G$ em ;
                    \item \textbf{Radiação Absorvida:} $G_{\text{abs}}$ em ;
                    \item \textbf{Radiação :} $G_{\text{unknown}}$ em ;
                    \item \textbf{Radiação Refletida:} $G_{\text{ref}}$ em ;
                \end{enumerate}

        \subsubsection{Radiação Absorvida}
            \paragraph{Definição}Quantifica-se o fluxo de energia entrando de um corpo por radiação através da seguinte equação:
                \begin{equation}
                    \boxed{
                        G_{\text{abs}} = \alpha G
                        \quad
                        \left[\frac{\text{W}}{\text{m\textsuperscript{2}}}\right]
                    }
                \end{equation}
            Onde:
                \begin{enumerate}[noitemsep]
                    \item \textbf{Absortividade:} $0 \leq \alpha \leq 1$ adimensional;
                    \item \textbf{Irradiação:} $G$ em $\left[\frac{\text{W}}{\text{m\textsuperscript{2}}}\right]$;
                \end{enumerate}

        \subsubsection{Reservatório Térmico}
            \paragraph{Definição}Caso haja uma superfície reduzida com temperatura $T_{S}$ cercada por outra envolvente muito aumentada com temperatura $T_{\text{sur}}$ então, caso $\epsilon = \alpha$, o fluxo de energia causado pela radiação será dado pela seguinte equação:
                \begin{equation}
                    \boxed{
                        q_{\text{rad}}'' = \frac{q}{A} = \epsilon\sigma(T_{S}^{4} - T_{\text{sur}}^{4}) 
                        \quad 
                        \left[\frac{\text{W}}{\text{m\textsuperscript{2}}}\right]
                    }
                \end{equation}
            Note que a equação acima pode ser resescrita como demonstrado a seguir:
                \begin{equation}
                    \boxed{
                        q_{\text{rad}} = h_{r} A (T_{S} - T_{\text{sur}}) 
                        \quad 
                        [\text{W}]
                    }
                    \qquad
                    \boxed{
                        h_{r} = \epsilon \sigma (T_{S} + T_{\text{sur}})(T_{S}^{2} + T_{\text{sur}}^{2})
                    }
                \end{equation}
            Desta forma, quando houver troca de calor simultaneamente na superfície por convecção e por radiação o fluxo de energia será dado pela seguinte equação:
                \begin{equation}
                    \boxed{
                        q = q_{\text{conv}} + q_{\text{rad}} = 
                        h A (T_{S} - T_{\text{sur}}) + \epsilon\sigma(T_{S}^{4} - T_{\text{sur}}^{4}) 
                        \quad 
                        [\text{W}]
                    }
                \end{equation}

        \subsection{Conservação de Energia}
            \paragraph{Definição}
                \begin{equation}
                    \boxed{
                        \dot{E}_{\text{acu}} = 
                        \diff{E_{\text{corpo}}}{t} = 
                        \dot{E}_{\text{in}} - 
                        \dot{E}_{\text{out}} + 
                        \dot{E}_{\text{ger}}
                        \quad
                        \left[
                            \text{W} = \frac{\text{Kg m\textsuperscript{2}}}{s\textsuperscript{3}}
                        \right]
                    }\label{eq:energyConservation}
                \end{equation}
            Onde:
                \begin{enumerate}[noitemsep]
                    \item \textbf{Energia Gerada:} $\dot{E}_{\text{ger}}$ em $[\text{W}]$ obtida pela seguinte equação:
                        \begin{equation}
                            \boxed{
                                \dot{E}_{\text{ger}} = \dot{q} \text{d}x\; \text{d}y\; \text{d}z
                            }
                        \end{equation}

                    \item \textbf{Energia Acumulada:} $\dot{E}_{\text{acu}}$ em $[\text{W}]$ obtida pela seguinte equação:
                        \begin{equation}
                            \boxed{
                                \dot{E}_{\text{acu}} = p c_{\text{p}} \diffp{T}{t} \text{d}x\; \text{d}y\; \text{d}z
                            }
                        \end{equation}
                \end{enumerate}

        \subsubsection{Balanço de Energia em Superfície}
            \paragraph{Definição}Considera-se que superfícies não apresentam massa e portanto apresentam a seguinte equação para conservação de massa:
                \begin{equation}
                    \boxed{
                        \dot{E}_{\text{in}} - \dot{E}_{\text{out}} = 0
                    }
                \end{equation}

    \section{Condução}
        \subsection{Equação Difusão Térmica}
            \paragraph{Definição}Equação que permite analisar a distribuição de temperatura sobre uma superfície. Primeiramente define-se um \textbf{Volume de Controle Diferencial} uma região infinitesimal do espaço analisado como definido na seguinte figura:
                \begin{figure}[H]
                    \begin{subfigure}[t]{0.45\textwidth}
                        \centering
                        \includegraphics[height = 4cm]{es570_im01.png}
                        \caption{Coordenadas Cartesianas}
                    \end{subfigure}
                    \begin{subfigure}[t]{0.45\textwidth}
                        \centering
                        \includegraphics[height = 4cm]{es570_im02.png}
                        \caption{Coordenadas Cilíndricas}
                    \end{subfigure}
                    \caption{Volume de Controle Diferencial}
                \end{figure} \noindent
            Na sequência substitui-se as variáveis definidas na equação \ref{eq:energyConservation}, obtendo a seguinte equação em coordenadas cartesianas:
                \begin{equation}
                    \boxed{
                        \diffp{}{x}\left(k\diffp{T}{x}\right) + 
                        \diffp{}{y}\left(k\diffp{T}{y}\right) + 
                        \diffp{}{z}\left(k\diffp{T}{z}\right) + 
                        \dot{q} = 
                        p\;c_{p}\;\diffp{T}{t}
                    }\label{eq.DifusionCartesian}
                \end{equation}
            Alternativamente a mesma equação em coordenadas cilíndricas será:
                \begin{equation}
                    \boxed{
                        \frac{1}{r}\diffp{}{r}\left(kr\diffp{T}{r}\right) + 
                        \frac{1}{r^2}\diffp{}{\phi}\left(k\diffp{T}{\phi}\right) + 
                        \diffp{}{z}\left(k\diffp{T}{z}\right) + 
                        \dot{q} = 
                        p\;c_{p}\;\diffp{T}{t}
                    }\label{eq.DifusionCilinder}
                \end{equation}

    \section{Condução em Regime Permanente e Unidimensional}
        \paragraph{Apresentação}Descrição de sistemas assumindo as seguintes considerações durante a análise:
            \begin{enumerate}[noitemsep]
                \item Unidimensional;
                \item Regime Permanente;
                \item Geração de Calor Nula;
                \item Condutividade Térmica Constante;
                \item Temperaturas Conhecidas nas Fronteiras;
            \end{enumerate}

        \subsection{Sistemas Planares}
            \paragraph{Definição}
                \begin{proof}
                    Nestas condições a Equação de Difusão Térmica na forma Cartesiana dada por \ref{eq.DifusionCartesian} será simplificada à:
                        \begin{align*}
                            \diffp{}{x}\left(k\diffp{T}{x}\right) &= 0 & \text{Considerando (4)}\\
                            \Aboxed{\diffp[2]{T}{x} &= 0}\\
                            \diff{T}{x} &= c_{1}\\
                            \Aboxed{T(x) &= c_{1}x + c_{2}}\\
                        \end{align*}
                    Desta forma, considera-se as condições iniciais trazidas por (5) como:
                        \begin{equation*}
                            T(x) = c_{1}x + c_{2} = 
                            \begin{cases}
                                T(0) = T_{S1} = c_{1}0 + c_{2}\\
                                T(x) = T_{S2} = c_{1}L + c_{2}\\
                            \end{cases}
                            \quad
                            \text{obtendo}
                            \quad
                            \boxed{
                                T(x) = \frac{(T_{S2} - T_{S1})}{L}x + T_{S1}
                            }
                        \end{equation*}
                    Finalmente, aplica-se a Equação de Fourier dada por \ref{eq.Fourier} com a condição (1) obtendo:
                        \begin{equation*}
                            \boxed{q_{x} = -\frac{K A}{L}(T_{S2} - T_{S1})}
                        \end{equation*}
                \end{proof}

        \subsubsection{Resistência Térmica}
            \paragraph{Definição}Represenação da dificuldade para o fluxo de calor ao longo de um material expressado pela seguinte equação:
                \begin{equation}
                    \boxed{
                        R_{\text{eq}} = \frac{T_{1} - T_{2}}{q} = 
                        \begin{cases}
                            R_{\text{cnd}} = \frac{L}{KA}     & \text{Condução};\\[2.5mm]
                            R_{\text{cnv}} = \frac{1}{hA}     & \text{Convecção};\\[2.5mm]
                            R_{\text{rad}} = \frac{1}{h_{r}A} & \text{Radiação};\\
                        \end{cases}
                    }
                \end{equation}

        \subsubsection{Coeficiente Global de Transferência de Calor}
            \paragraph{Definição}Obtido pela seguinte equação:
                \begin{align}
                    q_{x} &= U A \Delta T & \text{onde $\Delta T = T_{\infty,1} - T_{\infty,2}$}\\\nonumber
                    \Aboxed{
                        U     &= \frac{1}{R_{\text{eq}A}} 
                               = \frac{
                                   1
                                }{
                                    \frac{1}{h_{1}} + 
                                    \frac{L_{A}}{K_{A}} + 
                                    \frac{L_{B}}{K_{B}} + 
                                    \frac{L_{C}}{K_{C}} + 
                                    \frac{1}{h_{2}}
                                }
                    }
                \end{align}

            \paragraph{Definição}Considera-se
                \begin{figure}[H]
                    \centering
                    \includegraphics[height = 6cm]{es570_im04.png}
                    \caption{Coordenadas Cartesianas}
                \end{figure} \noindent
                \begin{equation}
                    \boxed{
                            q_{x} = \frac{
                            T_{\infty,1} - T_{\infty,2}
                        }{
                            \frac{1}{h_{1}A} + 
                            \frac{1}{KA} + 
                            \frac{1}{h_{2}A}
                        }
                    }
                \end{equation}

            \paragraph{Definição}Considera-se
                \begin{figure}[H]
                    \centering
                    \includegraphics[height = 6cm]{es570_im05.png}
                    \caption{Coordenadas Cartesianas}
                \end{figure} \noindent
                \begin{equation}
                    \boxed{
                            q_{x} = \frac{
                            T_{\infty,1} - T_{\infty,2}
                        }{
                            \frac{1}{h_{1}A} + 
                            \frac{L_{A}}{K_{A}A} + 
                            \frac{L_{B}}{K_{B}A} + 
                            \frac{L_{C}}{K_{C}A} + 
                            \frac{1}{h_{2}A}
                        }
                    }
                \end{equation}

        \subsubsection{Resistência de Contato}
            \paragraph{Definição}Superfícies em contato obtido pela seguinte equação:
                \begin{equation}
                    \boxed{
                        R_{\text{cnt}}'' = \frac{T_{A} - T_{B}}{q_{x}''}
                    }
                \end{equation}

        \subsection{Sistemas Radiais}
            \paragraph{Definição}
                \begin{proof}
                    Nestas condições a Equação de Difusão Térmica na forma Cartesiana dada por \ref{eq.DifusionCartesian} será simplificada à:
                        \begin{align*}
                            \diffp{}{r}\left(Kr\diffp{T}{r}\right) &= 0 & \text{Considerando (4)}\\
                            \Aboxed{\diffp{}{r}\left(r\diffp{T}{r}\right) &= 0}\\
                            r\diff{T}{r} &= c_{1}\\
                            \Aboxed{T(r) &= c_{1}\ln(r) + c_{2}}\\
                        \end{align*}
                    Desta forma, considera-se as condições iniciais trazidas por (5) como:
                        \begin{equation*}
                            T(r) = c_{1}\ln(r) + c_{2} = 
                            \begin{cases}
                                T(0) = T_{i} = c_{1}\ln(T_{i}) + c_{2}\\
                                T(r) = T_{e} = c_{1}\ln(T_{e}) + c_{2}\\
                            \end{cases}
                            \quad
                            \text{obtendo}
                            \quad
                            \boxed{
                                T(r) = 
                                \left(\frac{T_{i} - T_{e}}{\ln\left(\frac{r_{i}}{r_{e}}\right)}\right)
                                \ln\left(\frac{r}{r_{e}}\right) + T_{e}
                            }
                        \end{equation*}
                    Finalmente, aplica-se a Equação de Fourier dada por \ref{eq.Fourier} com a condição (1) obtendo:
                        \begin{equation*}
                            \boxed{q_{r} = -\frac{2\pi KL(T_{e} - T_{i})}{\ln\left(\frac{r_{e}}{r_{i}}\right)}}
                        \end{equation*}
                \end{proof}

        \subsubsection{Resistência Térmica}
            \paragraph{Definição}Represenação da dificuldade para o fluxo de calor ao longo de um material expressado pela seguinte equação:
                \begin{equation}
                    \boxed{
                        R_{\text{eq}} = \frac{T_{1} - T_{2}}{q} = 
                        \begin{cases}
                            R_{\text{cnd}} = \frac{\ln\left(\frac{r_{e}}{r_{i}}\right)}{2\pi KL} & \text{Condução};\\[2.5mm]
                            R_{\text{cnv}} =                                                         & \text{Convecção};\\[2.5mm]
                            R_{\text{rad}} =                                                         & \text{Radiação};\\
                        \end{cases}
                    }
                \end{equation}

        \subsubsection{Raio Crítico de Isolamento}
            \paragraph{Definição}Isolamento ideal para superfícies cilíndricas ou cascas esféricas que causará maior discipação de calor obtido pela seguinte equação:
                \begin{equation}
                    \boxed{
                        r_{C} = \frac{K}{h}
                    }
                \end{equation}

        \subsection{Sistemas Esféricos}
            \paragraph{Definição}
                \begin{proof}
                    Nestas condições a Equação de Difusão Térmica na forma Cartesiana dada por \ref{eq.DifusionCartesian} será simplificada à:
                        \begin{align*}
                            q_{r} &= -K(4\pi r^{2})\diff{T}{r} & \text{Separação de Equações}\\
                            \Aboxed{
                                \frac{q_{r}}{4\pi} \int_{r_{1}}^{r_{2}} \frac{\text{d}r}{r^{2}} &= 
                                - K \int_{T_{S1}}^{T_{S2}}\text{d}T
                            }\\
                            \frac{q_{r}}{4\pi} \left[\frac{-1}{r_{2}} - \frac{-1}{r_{1}}\right] &= -K(T_{S2} - T_{S1})\\
                            \Aboxed{q_{r} &= -4\pi K \frac{T_{S2} - T_{S1}}{\frac{1}{r_{1}} - \frac{1}{r_{2}}}}\\
                        \end{align*}
                \end{proof}

        \subsubsection{Resistência Térmica}
            \paragraph{Definição}Represenação da dificuldade para o fluxo de calor ao longo de um material expressado pela seguinte equação:
                \begin{equation}
                    \boxed{
                        R_{\text{eq}} = \frac{T_{1} - T_{2}}{q} = 
                        \begin{cases}
                            R_{\text{cnd}} = \frac{1}{4\pi K}\left(\frac{1}{r_{1}} - \frac{1}{r_{2}}\right) & \text{Condução};\\[2.5mm]
                            R_{\text{cnv}} =                                                                & \text{Convecção};\\[2.5mm]
                            R_{\text{rad}} =                                                                & \text{Radiação};\\
                        \end{cases}
                    }
                \end{equation}

    \section{Geração de Calor em Sólido}
        \paragraph{Apresentação}Calor originário de processos internos ao corpo análise através dos seguintes processos:
            \begin{enumerate}[noitemsep]
                \item Radiação;
                \item Reações Químicas;
                \item Reações Nucleares;
                \item Resistência de Fios;
            \end{enumerate}
        Obtido ela seguintes equação:
            \begin{equation}
                \boxed{
                    \dot{q} = 
                    \frac{\dot{E}}{V}
                    \quad
                    \left[
                        \frac{\text{W}}{\text{m}^{3}}
                    \right]
                }
            \end{equation}

        \subsubsection{Parede Plana não Isolada}
            \paragraph{Definição}
                \begin{enumerate}[noitemsep]
                    \item Unidimensional;
                    \item Regime Permanente;
                    \item Condutividade Térmica Constante;
                    \item Temperaturas Conhecidas nas Fronteiras;
                \end{enumerate}
                \begin{proof}
                    Nestas condições a Equação de Difusão Térmica na forma Cartesiana dada por \ref{eq.DifusionCartesian} será simplificada à:
                        \begin{align*}
                            \diffp{}{x}\left(K\diffp{T}{x}\right) + \dot{q} &= 0 & \text{Considerando (3)}\\
                            \Aboxed{\diffp[2]{T}{x}                         &= -\frac{\dot{q}}{K}}\\
                            \diff{T}{x}  &= -\frac{\dot{q}}{K} x + c_{1}\\
                            \Aboxed{T(x) &= -\frac{\dot{q}}{K}\frac{x^{2}}{2} + c_{1}x + c_{2}}\\
                        \end{align*}
                    Desta forma, considera-se as condições iniciais trazidas por (5) como:
                        \begin{equation*}
                            T(x) = -\frac{\dot{q}}{K}\frac{x^{2}}{2} + c_{1}x + c_{2} = 
                            \begin{cases}
                                T(0) = T_{S1} = -\frac{\dot{q}}{K} + c_{1}0 + c_{2}\\
                                T(L) = T_{S2} = -\frac{\dot{q}}{K}\frac{L^{2}}{2} + c_{1}L + c_{2}\\
                            \end{cases}
                        \end{equation*}
                    Finalmente, obtêm-se:
                        \begin{equation}
                            \boxed{
                                T(x) = -\frac{\dot{q}}{2K}(x^{2} - Lx) + \frac{(T_{S2} - T_{S1})}{L}x + T_{S1}
                            }
                        \end{equation}
                \end{proof}

        \subsubsection{Parede Plana Semi-Isolada}
            \paragraph{Definição}
                \begin{enumerate}[noitemsep]
                    \item Unidimensional;
                    \item Regime Permanente;
                    \item Condutividade Térmica Constante;
                    \item Temperaturas Conhecidas nas Fronteiras;
                \end{enumerate}
                \begin{proof}
                    Nestas condições a Equação de Difusão Térmica na forma Cartesiana dada por \ref{eq.DifusionCartesian} será simplificada à:
                        \begin{align*}
                            \diffp{}{x}\left(K\diffp{T}{x}\right) + \dot{q} &= 0 & \text{Considerando (3)}\\
                            \Aboxed{\diffp[2]{T}{x}                         &= -\frac{\dot{q}}{K}}\\
                        \end{align*}
                    Desta forma, considera-se as condições iniciais trazidas por (5) como:
                        \begin{equation*}
                            \begin{cases}
                                T(0) = -K\diffp{T}{x} = 0\\[2.5mm]
                                T(L) = -K\diffp{T}{x} = h(T_{2} - T_{\infty})\\
                            \end{cases}
                        \end{equation*}
                    Finalmente, obtêm-se:
                        \begin{equation}
                            \boxed{
                                T(x) = T_{\infty} + \dot{q} 
                                \left[
                                    \frac{L}{h} + \frac{L^{2} - x^{2}}{2K}
                                \right]
                            }
                        \end{equation}
                \end{proof}

        \subsection{Transferência de Calor em Superfícies}
            \begin{definition}
                Considera-se que uma superfície com área não constante apresentará a seguinte equação:
                    \begin{equation}
                        \boxed{
                            \diff{}{x} \left[A_{T}\diff{T}{x}\right]_{\substack{x}} -
                            \frac{h}{K}\diff{A_{S}}{x}[T(x) - T_{\infty}] = 0
                        }
                    \end{equation}
                Onde:
                    \begin{enumerate}
                        \item 
                    \end{enumerate}
            \end{definition}
            \begin{definition}
                Considera-se que uma superfície com área constante apresentará a seguinte equação:
                    \begin{equation}
                        \boxed{
                            \diff[2]{T}{x} -
                            \frac{h}{K}\frac{P}{A_{T}}[T(x) - T_{\infty}] = 0
                        }
                    \end{equation}
                Apresentando a seguinte \textbf{Solução Geral}:
                    \begin{equation}
                        \boxed{
                            T(x) - T_{\infty} = 
                            c_{1}\text{e}^{mx} + c_{2}\text{e}^{-mx}
                        }
                    \end{equation}
                Onde:
                    \begin{equation*}
                        \boxed{m := \sqrt{\frac{h P}{K A_{T}}}}
                        \quad
                        \text{e}
                        \quad
                        \boxed{\theta(x) := T(x) - T_{\infty}}
                    \end{equation*}
            \end{definition}
            \begin{proof}
            area se transforma em perimetro
            \end{proof}

        \subsubsection{Aleta Infinita}
            \begin{definition}
                Considera-se uma aleta muito longa como infinita implicando que sua \textbf{Distribuição de Temperatura}:
                    \begin{equation}
                        T(x) - T_{\infty} = 
                        (T_{B} - T_{\infty})\text{e}^{-mx}
                    \end{equation}
            \end{definition}
\end{document}