\documentclass{article}
\usepackage{tpack}

\title{ES570 - Transferência de Calor}
\author{Guilherme Nunes Trofino}
\authorRA{217276}
\project{Resumo Teórico}

\begin{document}
    \maketitle
\newpage

    \tableofcontents
\newpage

    \section{Introdução}
        \paragraph{Apresentação}Neste documento será descrito as informações necessárias para compreensão e solução de exercícios relacionados a disciplina \thetitle. Note que este documento são notas realizadas por \theauthor, em \today.

        \paragraph{Definição}Transferência de Calor é a energia térmica em trânsito devido a uma \texttt{Diferença de Temperatura} no espaço que ocorre nos seguintes processos:

        \subsection{Condução}
            \paragraph{Definição}Energia transferida de partículas mais energéticas para menos energéticas de uma substância devido às interações entre partículas em repouso.

        \subsubsection{Lei de Fourier}
            \paragraph{Definição}Energia transferida por unidade de tempo será dado por:
                \begin{equation}
                    \boxed{
                        q_{x}^{''} = - K \nabla T 
                        \quad 
                        \left[\frac{\text{W}}{\text{m\textsuperscript{2}}}\right]
                    }
                \end{equation}
            Onde:
                \begin{enumerate}[noitemsep, rightmargin = \leftmargin]
                    \item \textbf{Fluxo Térmico:} $q_{x}''$ em $\left[\frac{\text{W}}{\text{m\textsuperscript{2}}}\right]$, perpedicular a direção de transferência;

                    \item \textbf{Condutividade Térmica:} $K$ em $\left[\frac{\text{W}}{\text{m K}}\right]$;

                    \item \textbf{Gradiente de Temperatura:} $\nabla T$ em $\left[\frac{\text{K}}{\text{m}}\right]$;
                        \begin{enumerate}[noitemsep]
                            \item \texttt{Coordenadas Cartesianas:}
                                \begin{equation}
                                    \boxed{
                                        \nabla T = 
                                        \left(
                                            \hat{i}\diffp{T}{x} + 
                                            \hat{j}\diffp{T}{y} + 
                                            \hat{k}\diffp{T}{z}
                                        \right)
                                    }
                                \end{equation}
                        \end{enumerate}

                        \begin{enumerate}[noitemsep]
                            \item \texttt{Coordenadas Cilíndricas:}
                                \begin{equation}
                                    \boxed{
                                        \nabla T = 
                                        \left(
                                            \hat{i}\diffp{T}{r} + 
                                            \hat{j}\frac{1}{r}\diffp{T}{\phi} + 
                                            \hat{k}\diffp{T}{z}
                                        \right)
                                    }
                                \end{equation}
                        \end{enumerate}

                        \begin{enumerate}[noitemsep]
                            \item \texttt{Coordenadas Esféricas:}
                                \begin{equation}
                                    \boxed{
                                        \nabla T = 
                                        \left(
                                            \hat{i}\diffp{T}{r} + 
                                            \hat{j}\frac{1}{r}\diffp{T}{\phi} + 
                                            \hat{k}\frac{1}{r\sin(\theta)}\diffp{T}{\theta}
                                        \right)
                                    }
                                \end{equation}
                        \end{enumerate}
                \end{enumerate}

        \subsection{Convecção}
            \paragraph{Definição}Energia transferida pelo fluxo de partículas seja um Movimento Aleatório ou um Movimento Global do fluído através de dois mecanismos:
                \begin{enumerate}[noitemsep]
                    \item \textbf{Convecção Forçada:} Escoamento causado por meios externos; 
                    \item \textbf{Convecção Natural:} Escoamento induzido por diferenças de densidade; 
                \end{enumerate}
            Quantifica-se este fluxo de energia, independente do mecanismo, através da seguinte equação:
                \begin{equation}
                    \boxed{
                        q_{\text{conv}}'' = \frac{q}{A} = h (T_{S} - T_{\infty})
                        \quad 
                        \left[\frac{\text{W}}{\text{m\textsuperscript{2}}}\right]
                    }
                \end{equation}
            Onde:
                \begin{enumerate}[noitemsep, rightmargin = \leftmargin]
                    \item \textbf{Fluxo Térmico:} $q_{\text{conv}}''$ em $\left[\frac{\text{W}}{\text{m\textsuperscript{2}}}\right]$;

                    \item \textbf{Coeficiente de Transferêcia:} $h$ em $\left[\frac{\text{W}}{\text{m\textsuperscript{2} K}}\right]$;
                \end{enumerate}
            Nota-se que este coeficiente apresentará os seguintes valores típicos:
                \begin{table}[H]
                    \centering
                    \begin{tabular}{lll}\hline
                        Situação          &          & h\;$\left[\frac{\text{W}}{\text{m\textsuperscript{2} K}}\right]$\\\hline
                        Convecção Natural & Gases    & 2 - 25\\
                                          & Líquidos & 50 - 1000\\
                        Convecção Forçada & Gases    & 25 - 250\\
                                          & Líquidos & 100 - 20000\\
                        Mudança de Fase   &          & 2500 - 100000\\\hline
                    \end{tabular}
                    \caption{Coeficiente de Transferência Térmica por Convecção}
                    \label{table:convectionConstant}
                \end{table}\noindent


        \subsection{Radiação}
            \paragraph{Definição}Energia transferida, não necessariamente demandando um meio material, pela matéria que se encontra a uma temperatura não nula. 

        \subsubsection{Radiação Emitida}
            \paragraph{Definição}Quantifica-se o fluxo de energia saindo de um corpo por radiação através da seguinte equação:
                \begin{equation}
                    \boxed{
                        E = \epsilon\;\sigma\;T_{S}^{4}
                        \quad
                        \left[\frac{\text{W}}{\text{m\textsuperscript{2}}}\right]
                    }
                \end{equation}
            Onde:
                \begin{enumerate}[noitemsep]
                    \item \textbf{Permissividade:} $0 \leq \epsilon \leq 1$ adimensional;
                    \item \textbf{Constante de Stefan-Boltzmann:} $\sigma = 5,67\times 10^{-8}$ em $\left[\frac{\text{W}}{\text{m\textsuperscript{2} K\textsuperscript{4}}}\right]$;
                    \item \textbf{Temperatura Absoluta:} $T_{S}$ em $[K]$;
                \end{enumerate}

        \subsubsection{Radiação Recebida}
            \paragraph{Definição}Quantifica-se o fluxo de energia recebida, também nomeada como \textbf{Irradiação}, por um corpo por radiação através da seguinte equação:
                \begin{equation}
                    \boxed{
                        G = G_{\text{abs}} + G_{\text{unknown}} + G_{\text{ref}}
                    }
                \end{equation}
            Onde:
                \begin{enumerate}[noitemsep]
                    \item \textbf{Radiação Recebida:} $G$ em ;
                    \item \textbf{Radiação Absorvida:} $G_{\text{abs}}$ em ;
                    \item \textbf{Radiação :} $G_{\text{unknown}}$ em ;
                    \item \textbf{Radiação Refletida:} $G_{\text{ref}}$ em ;
                \end{enumerate}

        \subsubsection{Radiação Absorvida}
            \paragraph{Definição}Quantifica-se o fluxo de energia entrando de um corpo por radiação através da seguinte equação:
                \begin{equation}
                    \boxed{
                        G_{\text{abs}} = \alpha G
                        \quad
                        \left[\frac{\text{W}}{\text{m\textsuperscript{2}}}\right]
                    }
                \end{equation}
            Onde:
                \begin{enumerate}[noitemsep]
                    \item \textbf{Absortividade:} $0 \leq \alpha \leq 1$ adimensional;
                    \item \textbf{Irradiação:} $G$ em $\left[\frac{\text{W}}{\text{m\textsuperscript{2}}}\right]$;
                \end{enumerate}

        \subsubsection{Reservatório Térmico}
            \paragraph{Definição}Caso haja uma superfície reduzida com temperatura $T_{S}$ cercada por outra envolvente muito aumentada com temperatura $T_{\text{sur}}$ então, caso $\epsilon = \alpha$, o fluxo de energia causado pela radiação será dado pela seguinte equação:
                \begin{equation}
                    \boxed{
                        q_{\text{rad}}'' = \frac{q}{A} = \epsilon\sigma(T_{S}^{4} - T_{\text{sur}}^{4}) 
                        \quad 
                        \left[\frac{\text{W}}{\text{m\textsuperscript{2}}}\right]
                    }
                \end{equation}
            Note que a equação acima pode ser resescrita como demonstrado a seguir:
                \begin{equation}
                    \boxed{
                        q_{\text{rad}} = h_{r} A (T_{S} - T_{\text{sur}}) 
                        \quad 
                        [\text{W}]
                    }
                    \qquad
                    \boxed{
                        h_{r} = \epsilon \sigma (T_{S} + T_{\text{sur}})(T_{S}^{2} + T_{\text{sur}}^{2})
                    }
                \end{equation}
            Desta forma, quando houver troca de calor simultaneamente na superfície por convecção e por radiação o fluxo de energia será dado pela seguinte equação:
                \begin{equation}
                    \boxed{
                        q = q_{\text{conv}} + q_{\text{rad}} = 
                        h A (T_{S} - T_{\text{sur}}) + \epsilon\sigma(T_{S}^{4} - T_{\text{sur}}^{4}) 
                        \quad 
                        [\text{W}]
                    }
                \end{equation}

        \subsection{Conservação de Energia}
            \paragraph{Definição}
                \begin{equation}
                    \boxed{
                        \dot{E}_{\text{acu}} = 
                        \diff{E_{\text{corpo}}}{t} = 
                        \dot{E}_{\text{in}} - 
                        \dot{E}_{\text{out}} + 
                        \dot{E}_{\text{ger}}
                        \quad
                        \left[
                            \text{W} = \frac{\text{Kg m\textsuperscript{2}}}{s\textsuperscript{3}}
                        \right]
                    }
                \end{equation}
            Onde:
                \begin{enumerate}[noitemsep]
                    \item \textbf{Energia Gerada:} $\dot{E}_{\text{ger}}$ em $[\text{W}]$ obtida pela seguinte equação:
                        \begin{equation}
                            \boxed{
                                \dot{E}_{\text{ger}} = \dot{q} \text{d}x\; \text{d}y\; \text{d}z
                            }
                        \end{equation}

                    \item \textbf{Energia Acumulada:} $\dot{E}_{\text{acu}}$ em $[\text{W}]$ obtida pela seguinte equação:
                        \begin{equation}
                            \boxed{
                                \dot{E}_{\text{acu}} = p c_{\text{p}} \diffp{T}{t} \text{d}x\; \text{d}y\; \text{d}z
                            }
                        \end{equation}
                \end{enumerate}

        \subsubsection{Balanço de Energia em Superfície}
            \paragraph{Definição}Considera-se que superfícies não apresentam massa e portanto apresentam a seguinte equação para conservação de massa:
                \begin{equation}
                    \boxed{
                        \dot{E}_{\text{in}} - \dot{E}_{\text{out}} = 0
                    }
                \end{equation}

        \subsubsection{Equação Difusão Térmica}
            \paragraph{Definição}Equação que permite analisar a distribuição de temperatura sobre uma superfície. Primeiramente define-se um \textbf{Volume de Controle Diferencial} uma região infinitesimal do espaço analisado como definido na seguinte figura:
                \begin{figure}[H]
                    \begin{subfigure}[t]{0.45\textwidth}
                        \centering
                        \includegraphics[height = 4cm]{es570_im01.png}
                        \caption{Coordenadas Cartesianas}
                    \end{subfigure}
                    \begin{subfigure}[t]{0.45\textwidth}
                        \centering
                        \includegraphics[height = 4cm]{es570_im02.png}
                        \caption{Coordenadas Cilíndricas}
                    \end{subfigure}
                    \caption{Volume de Controle Diferencial}
                \end{figure} \noindent
            Na sequência substitui-se as variáveis definidas na equação \ref{eq:energyConservation}, obtendo a seguinte equação em coordenadas cartesianas:
                \begin{equation}
                    \boxed{
                        \diffp{}{x}\left(k\diffp{T}{x}\right) + 
                        \diffp{}{y}\left(k\diffp{T}{y}\right) + 
                        \diffp{}{z}\left(k\diffp{T}{z}\right) + 
                        \dot{q} = 
                        p\;c_{p}\;\diffp{T}{t}
                    }
                \end{equation}
            Alternativamente a mesma equação em coordenadas cilíndricas será:
                \begin{equation}
                    \boxed{
                        \frac{1}{r}\diffp{}{r}\left(kr\diffp{T}{r}\right) + 
                        \frac{1}{r^2}\diffp{}{\phi}\left(k\diffp{T}{\phi}\right) + 
                        \diffp{}{z}\left(k\diffp{T}{z}\right) + 
                        \dot{q} = 
                        p\;c_{p}\;\diffp{T}{t}
                    }
                \end{equation}
\end{document}