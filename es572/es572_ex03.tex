\documentclass{article}
\usepackage{tpack}



\titleformat
    {\subsection}           % Part
    [block]                 % Part Shape
    {\normalfont\Large}     % Font Size
    {}                      % Label Numbering
    {0mm}                   % Part Separation
    {}                      % Code Before
    []                      % Code After

    \titlespacing*{\subsection}{0mm}{5mm}{2.5mm}


\title{ES572 - Circuitos Lógicos}
\author{Guilherme Nunes Trofino}
\authorRA{217276}
\project{Atividade Teórica}


\begin{document}
    \maketitle
\newpage

    \section{Atividade Teórica}
        \paragraph{Apresentação}Resolução das questões de Circuitos Lógicos por Guilherme Nunes Trofino, 217276, sobre \textbf{Sistemas de Numeração} e \textbf{Portas Lógicas}.

        \subsection{Questão 1}
            \begin{exercise}
                Babilônios desenvolveram os números sexagesimais, base 60, há mais de 4000 mil anos. Quantos bis de informação são descritos por um dígito sexagesimal? Converta $2021_{(10)}$.
            \end{exercise}
            \begin{resolution}
                Nota-se que a quantidade de bits de informação na base sexagesimal será obtido pela seguinte equação:
                    \begin{equation*}
                        \boxed{\log_{2}(60) \approx 5.9069}
                    \end{equation*}
                Na sequência nota-se que $2021_{(10)}$ poderá ser convertido através da \textbf{Divisão Sucessiva} como demonstrado abaixo:
                    \begin{figure}[H]
                        \centering
                        \basetenconversiontable{2021}{60}
                    \end{figure}
                Note que será necessário converter os valores numéricos para seus equivalentes alfabéticos, obtendo assim:
                    \begin{equation*}
                        \boxed{2021_{(10)} = Xf_{(60)}}
                    \end{equation*}
            \end{resolution}
\newpage

        \subsection{Questão 2}
            \begin{exercise}
                Realize as seguintes conversões:
                    \begin{table}[H]
                        \centering  
                        \begin{tabular}[]{crl}\hline
                            (1) & $0110111_{(2)}$   & $?_{(10)}$\\
                            (2) & $101101010_{(2)}$ & $?_{(16)}$\\
                            (3) & $C9_{(16)}$       & $?_{(8)}$\\
                            (4) & $A7_{(16)}$       & $?_{(10)}$\\
                            (5) & $743_{(10)}$      & $?_{(16)}$\\
                            (6) & $221_{(10)}$      & $?_{(2)}$\\\hline
                        \end{tabular}
                    \end{table}
            \end{exercise}
            \begin{resolution}
                As seguintes conversões devem ser realizadas:
                \begin{enumerate}[rightmargin = \leftmargin]
                    \item Nota-se que $0110111_{(2)}$ poderá ser convertido através de \textbf{Complemento}:
                        \begin{figure}[H]
                            \centering
                            \todecimal[2]{0110111}
                        \end{figure}

                    \item Nota-se que $101101010_{(2)}$ poderá ser convertido através de \textbf{Agrupamento}:
                        \begin{equation*}
                            \underbrace{\text{\textcolor{red}{000}1}}_{
                                \text{1}
                            }
                            \quad
                            \underbrace{\text{0110}}_{
                                \text{6}
                            }
                            \quad
                            \underbrace{\text{1010}}_{
                                \text{A}
                            } = 
                            \boxed{16A_{(16)}}
                        \end{equation*}

                    \item Nota-se que $C9_{(16)}$ poderá ser convertido através de \textbf{Agrupamento}:
                        \begin{equation*}
                            \underbrace{\text{C9}}_{
                                \underbrace{\text{\textcolor{red}{0}11}}_{
                                    \text{3}
                                }
                                \quad
                                \underbrace{\text{001}}_{
                                    \text{1}
                                }
                                \quad
                                \underbrace{\text{001}}_{
                                    \text{1}
                                }
                            } = 
                            \boxed{311_{(8)}}
                        \end{equation*}

                    \item Nota-se que $A7_{(16)}$ poderá ser convertido através de \textbf{Complemento}:
                        \begin{figure}[H]
                            \centering
                            \todecimal[16]{A7}
                        \end{figure}

                    \item Nota-se que $743_{(10)}$ poderá ser convertido através da \textbf{Divisão Sucessiva}:
                        \begin{figure}[H]
                            \centering
                            \basetenconversiontable{743}{16}
                        \end{figure}
                    Note que será necessário converter os valores numéricos para seus equivalentes alfabéticos, obtendo assim:
                        \begin{equation*}
                            \boxed{743_{(10)} = 2E7_{(16)}}
                        \end{equation*}

                    \item Nota-se que $221_{(10)}$ poderá ser convertido através da \textbf{Divisão Sucessiva}:
                        \begin{figure}[H]
                            \centering
                            \basetenconversiontable{221}{2}
                        \end{figure}
                \end{enumerate}
            \end{resolution}
\newpage

        \subsection{Questão 3}
            \begin{exercise}
                Determine os valores de um número de 12 bits nas seguintes configurações:
                    \begin{enumerate}[noitemsep]
                        \item Número sem sinal;
                        \item Número em complemento de 1;
                        \item Número em complemento de 2;
                        \item Número em sinal-magnitude;
                    \end{enumerate}
            \end{exercise}
            \begin{resolution}
                Nota-se que os seguintes intervalos seriam possíveis:
                    \begin{enumerate}
                        \item (0, $2^{n}-1$)
                        \item (,)
                        \item ($-2^{n}$, $2^{n-1}-1$)
                        \item ($-2^{n-1}$, $2^{n-1}$)
                    \end{enumerate}
            \end{resolution}
\newpage

        \subsection{Questão 4}
            \begin{exercise}
                Considere o número de 8 bits $11010011_{(2)}$ e represente-o através das seguintes codificações:
                    \begin{enumerate}[noitemsep]
                        \item Número sem sinal;
                        \item Número em sinal-magnitude;
                        \item Número em complemento de 1;
                        \item Número em complemento de 2;
                    \end{enumerate}
                Repita considerando o número de 9 bits $011010011_{(2)}$.
            \end{exercise}
            \begin{resolution}

            \end{resolution}
\newpage

        \subsection{Questão 5}
            \begin{exercise}
                Considere os números abaixo para binário de 8 bits utilizando complemento de 2 e some-os. Verifique se os números são corretos e, caso contrário, indique quais \textbf{flags} devem ser ativas.
                \begin{enumerate}[noitemsep]
                    \item 63 e 17;
                    \item 27 e -39;
                    \item -44 e -28;
                    \item -102 e - 95;
                \end{enumerate}
            \end{exercise}
            \begin{resolution}

            \end{resolution}
\newpage

        \subsection{Questão 6}
            \begin{exercise}
                Converta cada número decimal em código BCD8421 e de Gray com o menor número de bits possível:
                    \begin{enumerate}[noitemsep]
                        \item 28;
                        \item 71;
                        \item 145;
                    \end{enumerate}
            \end{exercise}
            \begin{resolution}

            \end{resolution}
\newpage

        \subsection{Questão 7}
            \begin{exercise}
                Em qual base numérica $b$ a expressão $32_{(b)} + 4_{(b)} = 40_{(b)}$ está correta?
            \end{exercise}
            \begin{resolution}

            \end{resolution}
\newpage

        \subsection{Questão 8}
            \begin{exercise}
                Determine as portas lógicas com defeito analisando o diagrama de tempo abaixo:
                    \begin{figure}[H]
                        \centering
                        \includegraphics[height = 1.75cm]{es572_ex03_im01.png}
                    \end{figure} \noindent
            \end{exercise}
            \begin{resolution}

            \end{resolution}
\end{document}