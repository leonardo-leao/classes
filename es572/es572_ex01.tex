\documentclass{article}
\usepackage{tpack}


\begin{document}
    \section{Exercício 01}
        \paragraph{Apresentação}Resolução das questões de Circuitos Lógicos por Guilherme Nunes Trofino, 217276.

        \subsection{Questão 1}
            \begin{exercise}
                Você recebe um número binário de 4 bits desconhecido, e foi informado de que a representação binária contém apenas um par de bits vizinhos repetidos. Qual a quantidade de informação que você recebeu?
            \end{exercise}
            \begin{resolution}
                Há no total $2^{4} = 16$ números possíveis, sendo que apenas $6$ os seguintes atendem as condições estabelecidas como mostrados a seguir:
                    \begin{table}[H]
                        \centering
                        \begin{tabular}[]{cc}\hline
                            1101\textsubscript{2} & 0010\textsubscript{2}\\
                            0110\textsubscript{2} & 1001\textsubscript{2}\\
                            1011\textsubscript{2} & 0100\textsubscript{2}\\\hline
                        \end{tabular}
                    \end{table}\noindent
                Sabe-se que a \textbf{Quantidade de Informação Recebida} será obtida pela seguinte equação:
                    \begin{equation*}
                        I(x) = \log_{2}\left(\frac{1}{p_{x}}\right)
                        \quad\text{com}\quad
                        p_{x} = \frac{6}{16}
                    \end{equation*}
                Logo:
                    \begin{equation*}
                        I(x) = \log_{2}\left(\frac{16}{6}\right) \approx \boxed{1.41504 \text{ bits}}
                    \end{equation*}
            \end{resolution}

            \begin{exercise}
                Você recebe uma informação adicional dizendo que o número também é par. Qual a quantidade de informação adicional que você recebeu?
            \end{exercise}
            \begin{resolution}
                Agora há no total as 6 combinações já conhecidas, sendo que apenas 3 delas são pares como mostradas a seguir:
                    \begin{table}[H]
                        \centering
                        \begin{tabular}[]{c}\hline
                            0110\textsubscript{2}\\
                            0010\textsubscript{2}\\
                            0100\textsubscript{2}\\\hline
                        \end{tabular}
                    \end{table}\noindent
                Desta forma a \textbf{Quantidade de Informação Recebida} será obtida pela seguinte equação:
                    \begin{equation*}
                        I(x) = \log_{2}\left(\frac{1}{p_{x}}\right)
                        \quad\text{com}\quad
                        p_{x} = \frac{3}{16}
                    \end{equation*}
                Logo:
                    \begin{equation*}
                        I(x) = \log_{2}\left(\frac{16}{3}\right) \approx 2.41504 \text{ bits}
                    \end{equation*}
                Assim obtendo aproximadamente $\boxed{1 \text{bit}}$ adicional de informação.
            \end{resolution}

        \newpage\subsection{Questão 2}
            \begin{exercise}
                X é um número binário de 8 bits desconhecido. Você é avisado que a distância de Hamming entre X e 00101101\textsubscript{2} é dois. Quantos bits de informação sobre X você recebeu?
            \end{exercise}
            \begin{resolution}
                Há no total $2^{8} = 256$ números possíveis, sendo que apenas $56$ atendem as condições visto que a partir do código informado é necessário selecionar \texttt{2 bits}.
                    \begin{table}[H]
                        \centering
                        \begin{tabular}[]{c}\hline
                            00101101\textsubscript{2}\\\hline
                        \end{tabular}
                    \end{table}\noindent
                Inicialmente há 8 posições a serem escolhidas, na sequência há 7 posições disponíveis e a ordem não importa visto que a distância de Hamming deve ser identicamente 2. Desta forma a \textbf{Quantidade de Informação Recebida} será obtida pela seguinte equação:
                    \begin{equation*}
                        I(x) = \log_{2}\left(\frac{1}{p_{x}}\right)
                        \quad\text{com}\quad
                        p_{x} = \frac{28}{256}
                    \end{equation*}
                Logo:
                    \begin{equation*}
                        I(x) = \log_{2}\left(\frac{256}{28}\right) \approx \boxed{3.19265 \text{ bits}}
                    \end{equation*}
            \end{resolution}

        \newpage\subsection{Questão 3}
            \begin{exercise}
                Para cada uma das distribuições de probabilidade dos símbolos, construa uma árvore a partir do altoritmo de Huffman nessas distribuições de probabilidade.
                    \begin{enumerate}[noitemsep]
                        \item $p(A) = 0.31$, $p(B) = 0.2$, $p(C) = 0.2$, $p(D) = 0.19$ e $p(E) = 0.1$;

                        \item $p(A) = 0.6$, $p(B) = 0.1$, $p(C) = 0.1$, $p(D) = 0.1$ e $p(E) = 0.1$;

                        \item $p(A) = 0.29$, $p(B) = 0.22$, $p(C) = 0.18$, $p(D) = 0.17$ e $p(E) = 0.14$;
                    \end{enumerate}
            \end{exercise}

            \begin{resolution}
                Considera-se a seguintes árvores:
                \begin{figure}[H]
                    \centering
                    \begin{forest}
                        for tree={
                            s sep = 10mm,   % Horizontal Distance
                            l = 0mm,        % Vertical Distance
                            where n children={0}{ev}{iv},
                            l+=8mm,
                            if n=1{
                                edge label={
                                    node [midway, left] {0}
                                }
                            }{
                                edge label={
                                    node [midway, right] {1}
                                }
                            }
                        }
                        [1
                            [0.4
                                [B] [C]
                            ]
                            [0.6
                                [A]
                                [0.29
                                    [D] [E]
                                ]
                            ]
                        ]
                    \end{forest}
                    \caption{Árvore 1}
                \end{figure}
                \begin{figure}[H]
                    \centering
                    \begin{forest}
                        for tree={
                            s sep = 10mm,   % Horizontal Distance
                            l = 0mm,        % Vertical Distance
                            where n children={0}{ev}{iv},
                            l+=8mm,
                            if n=1{
                                edge label={
                                    node [midway, left] {0}
                                }
                            }{
                                edge label={
                                    node [midway, right] {1}
                                }
                            }
                        }
                        [1
                            [A]
                            [0.4
                                [0.2
                                    [B] [C]
                                ]
                                [0.2
                                    [D] [E]
                                ]
                            ]
                        ]
                    \end{forest}
                    \caption{Árvore 2}
                \end{figure}
                \begin{figure}[H]
                    \centering
                    \begin{forest}
                        for tree={
                            s sep = 10mm,   % Horizontal Distance
                            l = 0mm,        % Vertical Distance
                            where n children={0}{ev}{iv},
                            l+=8mm,
                            if n=1{
                                edge label={
                                    node [midway, left] {0}
                                }
                            }{
                                edge label={
                                    node [midway, right] {1}
                                }
                            }
                        }
                        [1
                            [0.4
                                [B] [C]
                            ]
                            [0.6
                                [A]
                                [0.31
                                    [D] [E]
                                ]
                            ]
                        ]
                    \end{forest}
                    \caption{Árvore 3}
                \end{figure} \noindent
            \end{resolution}

        \newpage\subsection{Questão 4}
            \begin{exercise}
                Considere a codificação para dois símbolos $A = 1100110011$ e $B = 0110100000$. Qual é a distância de Hammming entre elas? Quantos bits de erro podem ser detectados? Quantos bits de erro podem ser corrigidos?
            \end{exercise}
            \begin{resolution}
                Nota-se que há 5 bits de diferença como ilustrado pelos caractéres em \textcolor{red}{vermelho}:
                    \begin{table}[H]
                        \centering
                        \begin{tabular}[]{cc}\hline
                            \textcolor{red}{1}1\textcolor{red}{0}01\textcolor{red}{1}00\textcolor{red}{11}\textsubscript{2}\\
                            \textcolor{red}{0}1\textcolor{red}{1}01\textcolor{red}{0}00\textcolor{red}{00}\textsubscript{2}\\\hline
                        \end{tabular}
                    \end{table}\noindent
                Desta forma, a distância de Hamming, $H$, a quantidade de erros que podem ser detectados, $E_{D}$, e a quantidade de erros que podem ser corrigidos, $E_{C}$, são obtidos pelas seguintes equações:
                    \begin{equation*}
                        \boxed{H = 5}
                        \qquad
                        H =  E_{D} + 1 \rightarrow \boxed{E_{D} = 4}
                        \qquad
                        H = 2E_{C} + 1 \rightarrow \boxed{E_{C} = 2}
                    \end{equation*}
            \end{resolution}

        \newpage\subsection{Questão 5}
            \begin{exercise}
                Codifique a mensagem 01101000101\textsubscript{2} através do código de Hamming (15, 11) estendido. Decodifique-o após de inserir 0, 1 e 2 erros aleatórios, respectivamente. Verifique que as propriedades de correção do código são verificadas.
            \end{exercise}
            \begin{resolution}
                Considere a seguinte codificação em paridade par:
                    \begin{table}[H]
                        \centering
                        \begin{tabular}{|c|c|c|c|}\hline
                            \mycell{0}{0}  & \mycell{0}{1}  & \mycell{0}{2}  & \mycell{0}{3}\\\hline
                            \mycell{0}{4}  & \mycell{1}{5}  & \mycell{1}{6}  & \mycell{0}{7}\\\hline
                            \mycell{1}{8}  & \mycell{1}{9}  & \mycell{0}{10} & \mycell{0}{11}\\\hline
                            \mycell{0}{12} & \mycell{1}{13} & \mycell{0}{14} & \mycell{1}{15}\\\hline
                        \end{tabular}
                    \end{table}\noindent
                Quando há 0 erros será trivial, basta analisar os bits de paridade da codificação. Considere que os erros implementados estarão representados com a cor \textcolor{red}{vermelha}, desta forma o seguinte procedimento será aplicado para 1 erro:
                    \begin{table}[H]
                        \centering
                        \begin{tabular}{|c|c|c|c|}\hline
                            \cellcolor{red!40}\mycell{0}{0}  & \mycell{0}{1}                   & \mycell{0}{2}  & \mycell{0}{3}\\\hline
                            \mycell{0}{4}                    & \mycell{\textcolor{red}{0}}{5}  & \mycell{1}{6}  & \mycell{0}{7}\\\hline
                            \mycell{1}{8}                    & \mycell{1}{9}                   & \mycell{0}{10} & \mycell{0}{11}\\\hline
                            \mycell{0}{12}                   & \mycell{1}{13}                  & \mycell{0}{14} & \mycell{1}{15}\\\hline
                        \end{tabular}
                        \quad
                        \begin{tabular}{|c|c|c|c|}\hline
                            \mycell{0}{0}  & \cellcolor{red!40}\mycell{0}{1}                    & \mycell{0}{2}  & \cellcolor{gray!50}\mycell{0}{3}\\\hline
                            \mycell{0}{4}  & \cellcolor{gray!50}\mycell{\textcolor{red}{0}}{5}  & \mycell{1}{6}  & \cellcolor{gray!50}\mycell{0}{7}\\\hline
                            \mycell{1}{8}  & \cellcolor{gray!50}\mycell{1}{9}                   & \mycell{0}{10} & \cellcolor{gray!50}\mycell{0}{11}\\\hline
                            \mycell{0}{12} & \cellcolor{gray!50}\mycell{1}{13}                  & \mycell{0}{14} & \cellcolor{gray!50}\mycell{1}{15}\\\hline
                        \end{tabular}
                        \quad
                        \begin{tabular}{|c|c|c|c|}\hline
                            \mycell{0}{0}                     & \mycell{0}{1}                                      &  \mycell{0}{2}                    & \mycell{0}{3}\\\hline
                            \cellcolor{red!40}\mycell{0}{4}   & \cellcolor{gray!50}\mycell{\textcolor{red}{0}}{5}  & \cellcolor{gray!50}\mycell{1}{6}  & \cellcolor{gray!50}\mycell{0}{7}\\\hline
                            \mycell{1}{8}                     & \mycell{1}{9}                                      & \mycell{0}{10}                    & \mycell{0}{11}\\\hline
                            \cellcolor{gray!50}\mycell{0}{12} & \cellcolor{gray!50}\mycell{1}{13}                  & \cellcolor{gray!50}\mycell{0}{14} & \cellcolor{gray!50}\mycell{1}{15}\\\hline
                        \end{tabular}
                    \end{table}\noindent
                Nota-se que o \texttt{bit de paridade de conjunto} está errado, pois há uma quantidade ímpar de números uns no conjunto de dados. Na sequência nota-se que há um erro na segunda ou quarta coluna e um erro na segunda ou quarta linha da matriz. Assim, conclui-se que há um erro na segunda coluna com a segunda linha. Agora será apresentado o procedimento para 2 erros:
                    \begin{table}[H]
                        \centering
                        \begin{tabular}{|c|c|c|c|}\hline
                            \cellcolor{red!40}\mycell{\textcolor{red}{1}}{0}  & \mycell{0}{1}                   & \mycell{0}{2}  & \mycell{0}{3}\\\hline
                            \mycell{0}{4}                                     & \mycell{\textcolor{red}{0}}{5}  & \mycell{1}{6}  & \mycell{0}{7}\\\hline
                            \mycell{1}{8}                                     & \mycell{1}{9}                   & \mycell{0}{10} & \mycell{0}{11}\\\hline
                            \mycell{0}{12}                                    & \mycell{1}{13}                  & \mycell{0}{14} & \mycell{1}{15}\\\hline
                        \end{tabular}
                        \quad
                        \begin{tabular}{|c|c|c|c|}\hline
                            \mycell{\textcolor{red}{1}}{0}  & \cellcolor{red!40}\mycell{0}{1}                    & \mycell{0}{2}  & \cellcolor{gray!50}\mycell{0}{3}\\\hline
                            \mycell{0}{4}                   & \cellcolor{gray!50}\mycell{\textcolor{red}{0}}{5}  & \mycell{1}{6}  & \cellcolor{gray!50}\mycell{0}{7}\\\hline
                            \mycell{1}{8}                   & \cellcolor{gray!50}\mycell{1}{9}                   & \mycell{0}{10} & \cellcolor{gray!50}\mycell{0}{11}\\\hline
                            \mycell{0}{12}                  & \cellcolor{gray!50}\mycell{1}{13}                  & \mycell{0}{14} & \cellcolor{gray!50}\mycell{1}{15}\\\hline
                        \end{tabular}
                        \quad
                        \begin{tabular}{|c|c|c|c|}\hline
                            \mycell{\textcolor{red}{1}}{0}    & \mycell{0}{1}                                      &  \mycell{0}{2}                    & \mycell{0}{3}\\\hline
                            \cellcolor{red!40}\mycell{0}{4}   & \cellcolor{gray!50}\mycell{\textcolor{red}{0}}{5}  & \cellcolor{gray!50}\mycell{1}{6}  & \cellcolor{gray!50}\mycell{0}{7}\\\hline
                            \mycell{1}{8}                     & \mycell{1}{9}                                      & \mycell{0}{10}                    & \mycell{0}{11}\\\hline
                            \cellcolor{gray!50}\mycell{0}{12} & \cellcolor{gray!50}\mycell{1}{13}                  & \cellcolor{gray!50}\mycell{0}{14} & \cellcolor{gray!50}\mycell{1}{15}\\\hline
                        \end{tabular}
                    \end{table}\noindent
                Nota-se que o \texttt{bit de paridade de conjunto} está correto, pois há uma quantidade ímpar de números uns no conjunto de dados. Como o restante da análise anterior se mantém neste caso há 2 erros que não podem ser corrigidos, apenas detectados.
            \end{resolution}
\end{document}