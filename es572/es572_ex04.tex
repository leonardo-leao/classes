\documentclass{article}
\usepackage{tpack}



\titleformat
    {\subsection}           % Part
    [block]                 % Part Shape
    {\normalfont\Large}     % Font Size
    {}                      % Label Numbering
    {0mm}                   % Part Separation
    {}                      % Code Before
    []                      % Code After

    \titlespacing*{\subsection}{0mm}{5mm}{2.5mm}

    \input{kvmacros}

\title{ES572 - Circuitos Lógicos}
\author{Guilherme Nunes Trofino}
\authorRA{217276}
\project{Atividade Teórica}


\begin{document}
    \maketitle
\newpage

    \section{Atividade Teórica}
        \paragraph{Apresentação}Resolução das questões de Circuitos Lógicos por Guilherme Nunes Trofino, 217276, sobre \textbf{Simplificação de Expressões Booleanas}.

        \subsection{Questão 1}
            \begin{exercise}
                Use o mapa de Karnaugh para reduzir cada expressão para uma forma SoP mínima:
                \begin{enumerate}[label=(\alph*), rightmargin = \leftmargin]
                    \item $AB + \overline{A}D + B\overline{C}D + \overline{B}D$
                    \item $AB \oplus (C+D)$
                    \item $ABD + A\bar{B}\bar{C} + \bar{A}BC + \bar{A}CD + \bar{A}\bar{B}\bar{D}$
                    \item $\bar{A}B\bar{C}D + \bar{A}BCD + AB\bar{C}D + ABCD$
                    \item $BD + A\bar{B}D + \bar{A}B\bar{C} + \bar{A}CD + \bar{A}\bar{B}\bar{D}$
                \end{enumerate}
            \end{exercise}
            \begin{resolution}
                Nota-se:
                \begin{enumerate}[label=(\alph*), rightmargin = \leftmargin]
                    \item $f(A,B,C,D) = AB + \overline{A}D + B\overline{C}D + \overline{B}D$
                    \begin{multicols}{2}
                        \centering
                        \begin{table}[H]
                            \centering
                            \begin{tabular}[]{cccc|cr}
                                $A$&$B$&$C$&$D$&$f(A,B,C,D)$&Binário\\\hline
                                0&0&0&0&   &0\\
                                0&0&0&1& 1 &1\\
                                0&0&1&0&   &2\\
                                0&0&1&1& 1 &3\\
                                0&1&0&0&   &4\\
                                0&1&0&1& 1 &5\\
                                0&1&1&0&   &6\\
                                0&1&1&1& 1 &7\\
                                1&0&0&0&   &8\\
                                1&0&0&1& 1 &9\\
                                1&0&1&0&   &10\\
                                1&0&1&1& 1 &11\\
                                1&1&0&0& 1 &12\\
                                1&1&0&1& 1 &13\\
                                1&1&1&0& 1 &14\\
                                1&1&1&1& 1 &15\\\hline
                            \end{tabular}
                        \end{table}
                        \columnbreak
                        \begin{figure}[H]
                            \centering
                            \begin{karnaugh-map}[4][4][1][$C\;D$][$A\;B$]
                                \minterms{1,3,5,7,9,11,12,13,14,15}
                                \autoterms[0]
                                \implicant{1}{11}
                                \implicant{12}{14}
                            \end{karnaugh-map}
                        \end{figure}
                    \end{multicols}
                    Desta forma tem-se a seguinte simplificação:
                        \begin{equation}
                            \boxed{
                                f(A,B,C,D) = AB + D
                            }
                        \end{equation}
\newpage

                    \item $f(A,B,C,D) = AB \oplus (C+D) = AB\overline{(C+D)} + \overline{AB}(C+D) = AB\overline{C}\overline{D} + \overline{AB}C + \overline{AB}D$
                    \begin{multicols}{2}
                        \centering
                        \begin{table}[H]
                            \centering
                            \begin{tabular}[]{cccc|cr}
                                $A$&$B$&$C$&$D$&$f(A,B,C,D)$&Binário\\\hline
                                0&0&0&0&   &0\\
                                0&0&0&1& 1 &1\\
                                0&0&1&0& 1 &2\\
                                0&0&1&1& 1 &3\\
                                0&1&0&0&   &4\\
                                0&1&0&1&   &5\\
                                0&1&1&0&   &6\\
                                0&1&1&1&   &7\\
                                1&0&0&0&   &8\\
                                1&0&0&1&   &9\\
                                1&0&1&0&   &10\\
                                1&0&1&1&   &11\\
                                1&1&0&0& 1 &12\\
                                1&1&0&1&   &13\\
                                1&1&1&0&   &14\\
                                1&1&1&1&   &15\\\hline
                            \end{tabular}
                        \end{table}
                        \columnbreak
                        \begin{figure}[H]
                            \centering
                            \begin{karnaugh-map}[4][4][1][$C\;D$][$A\;B$]
                                \minterms{1,2,3,12}
                                \autoterms[0]
                                \implicant{12}{12}
                                \implicant{1}{3}
                                \implicant{3}{2}
                            \end{karnaugh-map}
                        \end{figure}
                    \end{multicols}
                    Desta forma tem-se a seguinte simplificação:
                        \begin{equation}
                            \boxed{
                                f(A,B,C,D) = AB\overline{CD} + \overline{AB}C + \overline{AB}D
                            }
                        \end{equation}

                    \item $f(A,B,C,D) = ABD + A\overline{BC} + \overline{A}BC + \overline{A}CD + \overline{ABD}$
                    \begin{multicols}{2}
                        \centering
                        \begin{table}[H]
                            \centering
                            \begin{tabular}[]{cccc|cr}
                                $A$&$B$&$C$&$D$&$f(A,B,C,D)$&Binário\\\hline
                                0&0&0&0& 1 &0\\
                                0&0&0&1&   &1\\
                                0&0&1&0& 1 &2\\
                                0&0&1&1& 1 &3\\
                                0&1&0&0&   &4\\
                                0&1&0&1&   &5\\
                                0&1&1&0& 1 &6\\
                                0&1&1&1& 1 &7\\
                                1&0&0&0& 1 &8\\
                                1&0&0&1& 1 &9\\
                                1&0&1&0&   &10\\
                                1&0&1&1&   &11\\
                                1&1&0&0&   &12\\
                                1&1&0&1& 1 &13\\
                                1&1&1&0&   &14\\
                                1&1&1&1& 1 &15\\\hline
                            \end{tabular}
                        \end{table}
                        \columnbreak
                        \begin{figure}[H]
                            \centering
                            \begin{karnaugh-map}[4][4][1][$C\;D$][$A\;B$]
                                \minterms{0,2,3,6,7,8,9,13,15}
                                \autoterms[0]
                                \implicantedge{0}{0}{8}{8}
                                \implicant{3}{6}
                                \implicant{8}{9}
                                \implicant{13}{15}
                                \implicant{7}{6}
                            \end{karnaugh-map}
                        \end{figure}
                    \end{multicols}
                    Desta forma tem-se a seguinte simplificação:
                        \begin{equation}
                            \boxed{
                                f(A,B,C,D) = \bar{A}C + ABD + A\overline{BC} + \overline{BCD}
                            }
                        \end{equation}
\newpage

                    \item $f(A,B,C,D) = \bar{A}B\bar{C}D + \bar{A}BCD + AB\bar{C}D + ABCD$
                    \begin{multicols}{2}
                        \centering
                        \begin{table}[H]
                            \centering
                            \begin{tabular}[]{cccc|cr}
                                $A$&$B$&$C$&$D$&$f(A,B,C,D)$&Binário\\\hline
                                0&0&0&0&   &0\\
                                0&0&0&1&   &1\\
                                0&0&1&0&   &2\\
                                0&0&1&1&   &3\\
                                0&1&0&0&   &4\\
                                0&1&0&1& 1 &5\\
                                0&1&1&0&   &6\\
                                0&1&1&1& 1 &7\\
                                1&0&0&0&   &8\\
                                1&0&0&1&   &9\\
                                1&0&1&0&   &10\\
                                1&0&1&1&   &11\\
                                1&1&0&0&   &12\\
                                1&1&0&1& 1 &13\\
                                1&1&1&0&   &14\\
                                1&1&1&1& 1 &15\\\hline
                            \end{tabular}
                        \end{table}
                        \columnbreak
                        \begin{figure}[H]
                            \centering
                            \begin{karnaugh-map}[4][4][1][$C\;D$][$A\;B$]
                                \minterms{5,7,13,15}
                                \autoterms[0]
                                \implicant{5}{15}
                            \end{karnaugh-map}
                        \end{figure}
                    \end{multicols}
                    Desta forma tem-se a seguinte simplificação:
                        \begin{equation}
                            \boxed{
                                f(A,B,C,D) = BD
                            }
                        \end{equation}
\newpage

                    \item $f(A,B,C,D) = BD + A\bar{B}D + \bar{A}B\bar{C} + \bar{A}CD + \overline{ABD}$
                    \begin{multicols}{2}
                        \centering
                        \begin{table}[H]
                            \centering
                            \begin{tabular}[]{cccc|cr}
                                $A$&$B$&$C$&$D$&$f(A,B,C,D)$&Binário\\\hline
                                0&0&0&0& 1 &0\\
                                0&0&0&1&   &1\\
                                0&0&1&0& 1 &2\\
                                0&0&1&1& 1 &3\\
                                0&1&0&0& 1 &4\\
                                0&1&0&1& 1 &5\\
                                0&1&1&0&   &6\\
                                0&1&1&1& 1 &7\\
                                1&0&0&0&   &8\\
                                1&0&0&1& 1 &9\\
                                1&0&1&0&   &10\\
                                1&0&1&1& 1 &11\\
                                1&1&0&0&   &12\\
                                1&1&0&1& 1 &13\\
                                1&1&1&0&   &14\\
                                1&1&1&1& 1 &15\\\hline
                            \end{tabular}
                        \end{table}
                        \columnbreak
                        \begin{figure}[H]
                            \centering
                            \begin{karnaugh-map}[4][4][1][$C\;D$][$A\;B$]
                                \minterms{0,2,3,4,5,7,9,11,13,15}
                                \autoterms[0]
                                \implicant{3}{11}
                                \implicant{13}{11}
                                \implicant{4}{5}
                                \implicantedge{0}{0}{2}{2}
                            \end{karnaugh-map}
                        \end{figure}
                    \end{multicols}
                    Desta forma tem-se a seguinte simplificação:
                        \begin{equation}
                            \boxed{
                                f(A,B,C,D) = AD + \overline{CD} + \overline{ABD} + \bar{A}B\bar{C}
                            }
                        \end{equation}
                \end{enumerate}
            \end{resolution}
\newpage

        \subsection{Questão 2}
            \begin{exercise}
                Determine a expressão SoP de custo mínimo para a função com quatro entradas expressa pela seguinte equação:
                    \begin{equation*}
                        f(A,B,C,D) = \sum M(6,7,8) + \sum D(10,11,12,13,14,15)
                    \end{equation*}
                Onde $M(i) = M_{i}$ é o i-ésimo \texttt{Mintermo} e $D(i) = D_{i}$ representa os estados de \texttt{don't care}.
            \end{exercise}
            \begin{resolution}
                Nota-se:
                \begin{figure}[H]
                    \centering
                    \begin{karnaugh-map}[4][4][1][$C\;D$][$A\;B$]
                        \minterms{6,7,8}
                        \indeterminants{10,11,12,13,14,15}
                        \autoterms[0]
                        \implicant{7}{14}
                        \implicantedge{12}{8}{14}{10}
                    \end{karnaugh-map}
                \end{figure}\noindent
                Desta forma tem-se a seguinte simplicação:
                    \begin{equation}
                        \boxed{
                            f(A,B,C,D) = BC + A\bar{D}
                        }
                    \end{equation}
            \end{resolution}
\newpage

        \subsection{Questão 3}
            \begin{exercise}
                Determine a expressão SoP de custo mínimo para a função com quatro entradas expressa pela seguinte equação:
                    \begin{equation*}
                        f(A,B,C,D) = \sum M(0,2,4,8,10,11,15) + \sum D(1,6,14)
                    \end{equation*}
                Onde $M(i) = M_{i}$ é o i-ésimo \texttt{Mintermo} e $D(i) = D_{i}$ representa os estados de \texttt{don't care}.
            \end{exercise}
            \begin{resolution}
                Nota-se
                \begin{figure}[H]
                    \centering
                    \begin{karnaugh-map}[4][4][1][$C\;D$][$A\;B$]
                        \minterms{0,2,4,8,10,11,15}
                        \indeterminants{1,6,14}
                        \autoterms[0]
                        \implicant{15}{10}
                        \implicantedge{0}{4}{2}{6}
                        \implicantcorner[0]
                    \end{karnaugh-map}
                \end{figure}\noindent
                Desta forma tem-se a seguinte simplicação:
                    \begin{equation}
                        \boxed{
                            f(A,B,C,D) = AC + \bar{B}\bar{D} + \bar{A}\bar{D}
                        }
                    \end{equation}
            \end{resolution}
\newpage

        \subsection{Questão 4}
            \begin{exercise}
                Determine a expressão SoP de custo mínimo para a função com quatro entradas expressa pela seguinte equação:
                    \begin{equation*}
                        G(X,Y,Z) = \sum M(2,7) + \sum D(0,3,5)
                    \end{equation*}
                Onde $M(i) = M_{i}$ é o i-ésimo \texttt{Mintermo} e $D(i) = D_{i}$ representa os estados de \texttt{don't care}.
            \end{exercise}
            \begin{resolution}
                Nota-se
                \begin{figure}[H]
                    \centering
                    \begin{karnaugh-map}[2][4][1][$Z$][$X\;Y$]
                        \minterms{2,7}
                        \indeterminants{0,3,5}
                        \autoterms[0]
                        \implicant{0}{2}
                        \implicant{7}{5}
                    \end{karnaugh-map}
                \end{figure}\noindent
                Desta forma tem-se a seguinte simplicação:
                    \begin{equation}
                        \boxed{
                            G(X,Y,Z) = \bar{X}\bar{Z} + XZ
                        }
                    \end{equation}
                Assim, toma-se $A = Z + \bar{X}Y\bar{Z}$, $B = XYZ + \bar{X}Y\bar{Z}$, $C = \bar{X}Y + \bar{X}\bar{Z}$, $D = Y(\bar{X} + Z)$, $E = X\oplus \bar{Z}$ e $F = \bar{X}\bar{Z} + YZ + \bar{Y}\bar{Z}$:
                \begin{table}[H]
                    \centering\begin{tabular}{ccc|cccccc|c|c}
                        X&Y&Z&$A$&$B$&$C$&$D$&$E$&$F$&$G$& Decimal\\\hline
                        0&0&0&   &   & 1 &   & 1 & 1 & x & 0\\
                        0&0&1& \textcolor{red}{1} &   &   &   &   &   & 0 & 1\\
                        0&1&0& 1 & 1 & 1 & 1 & 1 & 1 & 1 & 2\\
                        0&1&1& 1 &   & 1 & 1 &   & 1 & x & 3\\
                        1&0&0&   &   &   &   &   & \textcolor{red}{1} & 0 & 4\\
                        1&0&1& 1 &   &   &   & 1 &   & x & 5\\
                        1&1&0&   &   &   &   &   &   & 0 & 6\\
                        1&1&1& 1 & 1 & \textcolor{red}{0} & 1 & 1 & 1 & 1 & 7\\\hline
                    \end{tabular}
                \end{table}\noindent
                \begin{enumerate}[noitemsep]
                    \item $A$ é inválido pois $\overline{XY}Z$ deveria ser 0;
                    \item $B$ é válido;
                    \item $C$ é inválido pois $XYZ$ deveria ser 1;
                    \item $D$ é válido;
                    \item $E$ é válido;
                    \item $F$ é inválido pois $X\overline{YZ}$ deveria ser 0;
                \end{enumerate}
            \end{resolution}
\newpage

        \subsection{Questão 5}
            \begin{exercise}
                Determine a expressão SoP de custo mínimo para a função com quatro entradas expressa pela seguinte equação:
                    \begin{equation*}
                        f(A,B,C,D) = \sum M(2,3,4,6,7) + \sum D(0,1,14,15)
                    \end{equation*}
                Onde $M(i) = M_{i}$ é o i-ésimo \texttt{Mintermo} e $D(i) = D_{i}$ representa os estados de \texttt{don't care}.
            \end{exercise}
            \begin{resolution}
                Nota-se:
                \begin{figure}[H]
                    \centering
                    \begin{karnaugh-map}[4][4][1][$C\;D$][$A\;B$]
                        \minterms{2,3,4,6,7}
                        \indeterminants{0,1,14,15}
                        \autoterms[0]
                    \end{karnaugh-map}
                \end{figure}
                \paragraph{Método de Quine-McCluskey}Inicialmente será necessário obter os Implicantes Primos da função apresenta agrupando \texttt{mintermos} pela quantidade de bits 1 presentes como ilustrado na seguinte tabela:
                    \begin{table}[H]
                        \centering\begin{tabular}{c cccc c}
                            Bits 1             & A & B & C & D             & Decimal\\\hline
                            0                  & \multicolumn{4}{c}{vazio} & \\\hline
                            \multirow{2}{*}{1} & 0 & 0 & 1 & 0             & 2\\
                                               & 0 & 1 & 0 & 0             & 4\\\hline
                            \multirow{2}{*}{2} & 0 & 0 & 1 & 1             & 3\\
                                               & 0 & 1 & 1 & 0             & 6\\\hline
                            3                  & 0 & 1 & 1 & 1             & 7\\\hline
                            4                  & \multicolumn{4}{c}{vazio} & \\\hline
                        \end{tabular}
                    \end{table}
                Na sequência será necessário agrupar os termos que possuem distância de Hamming igual a 1 entre cada um dos grupos na tabela anterior.
                \\\\
                Nota-se que estes seriam vizinhos na representação do \texttt{Mapa de Karnaugh} e portanto poderiam ser agrupados. Nestes agrupamentos a propriedade $AB + A\bar{B} = A$ deve ser utilizada para simplificar os termos vizinhos a \texttt{don't care} como ilustrado nas seguintes tabelas:
                    \begin{table}[H]
                        \centering\begin{tabular}{c cccc c}
                            \texttt{B}         & A & B & C & D & \texttt{D}\\\hline
                            \multirow{2}{*}{1} & 0 & 0 & 1 & 0 & 2\\
                                               & 0 & 1 & 0 & 0 & 4\\\hline
                            \multirow{2}{*}{2} & 0 & 0 & 1 & 1 & 3\\
                                               & 0 & 1 & 1 & 0 & 6\\\hline
                            3                  & 0 & 1 & 1 & 1 & 7\\\hline
                        \end{tabular}
                        \qquad
                        \centering\begin{tabular}{c cccc c}
                            \texttt{B}         & A & B & C & D & \texttt{D}\\\hline
                            \multirow{3}{*}{1} & 0 & 0 & 1 & x & 2,3\\
                                               & 0 & x & 1 & 0 & 2,6\\
                                               & 0 & 1 & x & 0 & 4,6\\\hline
                            \multirow{2}{*}{2} & 0 & x & 1 & 1 & 3,7\\
                                               & 0 & 1 & 1 & x & 6,7\\\hline
                        \end{tabular}
                        \qquad
                        \centering\begin{tabular}{c cccc c}
                            \texttt{B}         & A & B & C & D & \texttt{D}\\\hline
                            \multirow{2}{*}{1} & 0 & x & 1 & x & 2,3,6,7\\
                                               & 0 & 1 & x & 0 & 4,6\\\hline
                        \end{tabular}
                    \end{table}\noindent
                Desta forma:
                    \begin{equation}
                        \boxed{f(A,B,C,D) = \bar{A}C + \bar{A}B\bar{D}}
                    \end{equation}
            \end{resolution}
\end{document}