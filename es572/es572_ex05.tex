\documentclass{article}
\usepackage{tpack}



\titleformat
{\subsection}           % Part
[block]                 % Part Shape
{\normalfont\Large}     % Font Size
{}                      % Label Numbering
{0mm}                   % Part Separation
{}                      % Code Before
[]                      % Code After

\titlespacing*{\subsection}{0mm}{5mm}{2.5mm}


\title{ES572 - Circuitos Lógicos}
\author{Guilherme Nunes Trofino}
\authorRA{217276}
\project{Atividade Teórica}


\begin{document}
\maketitle
\newpage

\section{Atividade Teórica}
\paragraph{Apresentação}Resolução das questões de Circuitos Lógicos por Guilherme Nunes Trofino, 217276, sobre \textbf{Análise e Projeto de Circuitos Sequências}.

\subsection{Questão 1}
    \begin{exercise}
        Determine a sequência de saída ($Q_3 \; Q_2 \; Q_1 \; Q_0$) do circuito para todas as possíveis condições iniciais.
        \begin{figure}[H]
            \centering
            \includegraphics[width = 10cm]{images/ex05_q1.png}
            \caption{Contador de Johnson de 4 bits}
        \end{figure}
        Observe que a sequência iniciada em 0000 é conhecida como \href{http://www.fem.unicamp.br/~grace/circuitos_sequenciais.pdf}{\textbf{Sequência de Johnson}}.
    \end{exercise}
    \begin{resolution}
        Nota-se que trata-se de um Contador de Johnson implementado com \href{https://en.wikipedia.org/wiki/Flip-flop_(electronics)}{\textbf{Flip-Flop's D}} cujo comportamento será cíclico a partir de quaisquer valores iniciais como representado abaixo em variáveis:
        \begin{center}
            \begin{minipage}{0.45\linewidth}
            \begin{table}[H]
                \centering\begin{tabular}[]{c|cccc}\hline
                \multirow{2}{*}{clock} & \multicolumn{4}{c}{Estados}\\
                                        & $Q_0$ & $Q_1$ & $Q_2$ & $Q_3$\\\hline
                0 & 0 & 0 & 0 & 0\\
                1 & 1 & 0 & 0 & 0\\
                2 & 1 & 1 & 0 & 0\\
                3 & 1 & 1 & 1 & 0\\
                4 & 1 & 1 & 1 & 1\\
                5 & 0 & 1 & 1 & 1\\
                6 & 0 & 0 & 1 & 1\\
                7 & 0 & 0 & 0 & 1\\\hline
            \end{tabular}\caption{Condição (0000)}
            \end{table}
        \end{minipage}
        \begin{minipage}{0.45\linewidth}
            \begin{table}[H]
                \centering\begin{tabular}[]{c|cccc}\hline
                \multirow{2}{*}{clock} & \multicolumn{4}{c}{Estados}\\
                                        & $Q_0$ & $Q_1$ & $Q_2$ & $Q_3$\\\hline
                0 & $A$            & $B$            & $C$            & $D$\\
                1 & $\overline{D}$ & $A$            & $B$            & $C$\\
                2 & $\overline{C}$ & $\overline{D}$ & $A$            & $B$\\
                3 & $\overline{B}$ & $\overline{C}$ & $\overline{D}$ & $A$\\
                4 & $\overline{A}$ & $\overline{B}$ & $\overline{C}$ & $\overline{D}$\\
                5 & $D$            & $\overline{A}$ & $\overline{B}$ & $\overline{C}$\\
                6 & $C$            & $D$            & $\overline{A}$ & $\overline{B}$\\
                7 & $B$            & $C$            & $D$            & $\overline{A}$\\\hline
            \end{tabular}\caption{Condição (ABCD)}
            \end{table}
        \end{minipage}
        \end{center}
    \end{resolution}

\newpage
\subsection{Questão 2}
    \begin{exercise}
        Reimplemente o circuito abaixo utilizando apenas FF do tipo T.
        \begin{figure}[H]
            \centering
            \includegraphics[width = 10cm]{images/ex05_q2.png}
            \caption{Circuito Inicial}
        \end{figure}
        Observe que o circuito não tem uma saída explícita. Neste caso, consideramos a própria representação do estado ($\; Q_2 \; Q_1 \; Q_0$) como sendo a saída de interesse.
    \end{exercise}
    \begin{resolution}
        Inicialmente será necessário analisar quais são as entradas e saídas do circuito e o funcionamento do \href{https://en.wikipedia.org/wiki/Flip-flop_(electronics)}{\textbf{Flip-Flop's JK}} como representado na seguinte tabela:
        \begin{center}
            \begin{minipage}{0.3\linewidth}
            \begin{table}[H]
            \centering\begin{tabular}[]{cl}
                Entradas                & U    \\\hline
                \multirow{3}{*}{Saídas} & $Q_0$\\
                                        & $Q_1$\\
                                        & $Q_2$\\\hline
            \end{tabular}\caption{Inputs Outputs}
            \end{table}
        \end{minipage}
        \begin{minipage}{0.3\linewidth}
            \begin{table}[H]
            \centering\begin{tabular}[]{cc|c}
                J & K & $Q^\star$\\\hline
                0 & 0 & $Q$\\
                0 & 1 & 0\\
                1 & 0 & 1\\
                1 & 1 & $\overline{Q}$\\\hline
            \end{tabular}\caption{FF JK}
            \end{table}
        \end{minipage}
        \begin{minipage}{0.3\linewidth}
            \begin{table}[H]
            \centering\begin{tabular}[]{c|c}
                T & $Q^\star$\\\hline
                0 & $Q$\\
                1 & $\overline{Q}$\\\hline
            \end{tabular}\caption{FF T}
            \end{table}
        \end{minipage}
        \end{center}
        Na sequência será necessário descrever o comportamento de cada entrada e saída de acordo com uma equação obtida pela combinação entre as portas lógicas e circuitos sequenciais como mostradas abaixo:
        \begin{table}[H]
            \centering\begin{tabular}[]{cl}\hline
                \multirow{2}{*}{FF JK0} & $Q_0^\star = \overline{Q}_0$\\
                                        & $J_0 = K_0 = 1$\\\hline
                FF JK1                  & $J_1 = K_1 = (U \text{ and } Q_0) \text{ or } (\text{not } U \text{ and } \text{not } Q_0)$\\\hline
                FF JK2                  & $J_2 = K_2 = (U \text{ and } Q_0 \text{ and } Q_1) \text{ or } (\text{not } U \text{ and } \text{not } Q_0 \text{ and } \text{not } Q_1)$\\\hline
            \end{tabular}\caption{Equações FFJK}
        \end{table}
        Em seguida será necessário elaborar uma tabela verdade das equações obtidas anteriormente:
        \begin{table}[H]
            \centering\begin{tabular}[]{cccc|cc|cc|cc|ccc|ccc}
                \multicolumn{4}{c|}{input}& \multicolumn{9}{c|}{FF JK}                                & \multicolumn{3}{c}{FF T}\\
                $U$&$Q_0$&$Q_1$&$Q_2$     &$J_0$&$K_0$&$J_1$&$K_1$&$J_2$&$K_2$&$Q_0^\star$&$Q_1^\star$&$Q_2^\star$&$T_0$&$T_1$&$T_2$ \\\hline
                0  & 0   & 0   & 0        & 1   & 1   & 1   & 1   & 1   & 1   & 1         & 1         & 1         & 1   & 1   & 1\\
                0  & 0   & 0   & 1        & 1   & 1   & 1   & 1   & 1   & 1   & 1         & 1         & 0         & 1   & 1   & 1\\
                0  & 0   & 1   & 0        & 1   & 1   & 1   & 1   & 0   & 0   & 1         & 0         & 0         & 1   & 1   & 0\\
                0  & 0   & 1   & 1        & 1   & 1   & 1   & 1   & 0   & 0   & 1         & 0         & 1         & 1   & 1   & 0\\
                0  & 1   & 0   & 0        & 1   & 1   & 0   & 0   & 0   & 0   & 0         & 0         & 0         & 1   & 0   & 0\\
                0  & 1   & 0   & 1        & 1   & 1   & 0   & 0   & 0   & 0   & 0         & 0         & 1         & 1   & 0   & 0\\
                0  & 1   & 1   & 0        & 1   & 1   & 0   & 0   & 0   & 0   & 0         & 1         & 0         & 1   & 0   & 0\\
                0  & 1   & 1   & 1        & 1   & 1   & 0   & 0   & 0   & 0   & 0         & 1         & 1         & 1   & 0   & 0\\
                1  & 0   & 0   & 0        & 1   & 1   & 0   & 0   & 0   & 0   & 1         & 0         & 0         & 1   & 0   & 0\\
                1  & 0   & 0   & 1        & 1   & 1   & 0   & 0   & 0   & 0   & 1         & 0         & 1         & 1   & 0   & 0\\
                1  & 0   & 1   & 0        & 1   & 1   & 0   & 0   & 0   & 0   & 1         & 1         & 0         & 1   & 0   & 0\\
                1  & 0   & 1   & 1        & 1   & 1   & 0   & 0   & 0   & 0   & 1         & 1         & 1         & 1   & 0   & 0\\
                1  & 1   & 0   & 0        & 1   & 1   & 1   & 1   & 0   & 0   & 0         & 1         & 0         & 1   & 1   & 0\\
                1  & 1   & 0   & 1        & 1   & 1   & 1   & 1   & 0   & 0   & 0         & 1         & 1         & 1   & 1   & 0\\
                1  & 1   & 1   & 0        & 1   & 1   & 1   & 1   & 1   & 1   & 0         & 0         & 1         & 1   & 1   & 1\\
                1  & 1   & 1   & 1        & 1   & 1   & 1   & 1   & 1   & 1   & 0         & 0         & 0         & 1   & 1   & 1\\\hline
            \end{tabular}\caption{Tabela de Simplificação}
        \end{table}\noindent
        Nota-se que os valores de $T_0$, $T_1$ e $T_2$ são exatamente iguais as entradas $J$ e $K$ dos respectivos FF. Desta forma, ambos apresentará as mesmas equações e portanto basta substituir os FFJK pelos FFT.\\\\
        Isso era esperado dado que o comportamento de um FFJK, quando as entradas são iguais, é idêntico a um FFT, obtendo o seguinte circuito:
        \begin{figure}[H]
            \centering
            \includegraphics[width = 10cm]{images/ex05_q21.png}
            \caption{Circuito Final}
        \end{figure}
    \end{resolution}

\newpage
\subsection{Questão 3}
    \begin{exercise}
        Reimplemente o circuito a seguir utilizando apenas FF do tipo JK.
        \begin{figure}[H]
            \centering
            \includegraphics[width = 10cm]{images/ex05_q3.png}
            \caption{Circuito Inicial}
        \end{figure}
    \end{exercise}
    \begin{resolution}
        Considera-se as seguintes equações de FF's:
        \begin{center}
        \begin{minipage}{0.3\linewidth}
            \begin{table}[H]
            \centering\begin{tabular}[]{cc|c}
                J & K & $Q^\star$\\\hline
                0 & 0 & $Q$\\
                0 & 1 & 0\\
                1 & 0 & 1\\
                1 & 1 & $\overline{Q}$\\\hline
            \end{tabular}\caption{FF JK}
            \end{table}
        \end{minipage}
        \begin{minipage}{0.3\linewidth}
            \begin{table}[H]
                \centering\begin{tabular}[]{cc|c}
                    J & K & $Q^\star$\\\hline
                    0 & 0 & 0\\
                    0 & 1 & 0\\
                    1 & 0 & 1\\
                    1 & 1 & 1\\\hline
                \end{tabular}\caption{FF JK $Q=0$}
                \end{table}
        \end{minipage}
        \begin{minipage}{0.3\linewidth}
            \begin{table}[H]
                \centering\begin{tabular}[]{cc|c}
                    J & K & $Q^\star$\\\hline
                    0 & 0 & 1\\
                    0 & 1 & 0\\
                    1 & 0 & 1\\
                    1 & 1 & 0\\\hline
                \end{tabular}\caption{FF JK $Q=1$}
                \end{table}
        \end{minipage}
        \end{center}
        Procedimento:
        \begin{enumerate}[rightmargin = \leftmargin, noitemsep]
            \item \textbf{Equações:}
            \begin{table}[H]
                \centering\begin{tabular}[]{cl}\hline
                    \multirow{2}{*}{FF D} & $Q_1^\star = D_1$\\
                                          & $D_1 = W \text{ and }(\overline{Q}_1 \text{ or } Q_2)$\\\hline
                    \multirow{2}{*}{FF T} & $Q_2^\star = T_2 \text{ xor } Q_2$\\
                                          & $T_2 = (\overline{W} \text{ and } Q_2) \text{ or } (Q_1 \text{ and } W \text{ and }\overline{Q}_2)$\\\hline
                    output                & $Z = Q_1 \text{ and } Q_2$\\\hline
                \end{tabular}\caption{Equações FFD e FFT}
            \end{table}
            \item \textbf{Tabela:}
            \begin{table}[H]
                \centering
                \begin{tabular}[]{ccc|c|c|cc|cc|cc}
                    \multicolumn{3}{c|}{input}& \multicolumn{4}{c|}{FFs DT}              & \multicolumn{4}{c}{FFs JK}\\
                    $W$ & $Q_1$ & $Q_2$       & $D_1$ & $T_2$ & $Q_1^\star$ & $Q_2^\star$&$J_1$&$K_1$&$J_2$&$K_2$\\\hline
                     0  &  0    &  0          & 0     & 0     & 0           & 0          & 0   & x   & 0   & x \\
                     0  &  0    &  1          & 0     & 1     & 0           & 0          & 0   & x   & x   & 1 \\
                     0  &  1    &  0          & 0     & 0     & 0           & 0          & x   & 1   & 0   & x \\
                     0  &  1    &  1          & 0     & 1     & 0           & 0          & x   & 1   & x   & 1 \\
                     1  &  0    &  0          & 1     & 0     & 1           & 0          & 1   & x   & 0   & x \\
                     1  &  0    &  1          & 1     & 0     & 1           & 1          & 1   & x   & x   & 0 \\
                     1  &  1    &  0          & 0     & 1     & 0           & 1          & x   & 1   & 1   & x \\
                     1  &  1    &  1          & 1     & 0     & 1           & 1          & x   & 0   & x   & 0 \\\hline
                \end{tabular}\caption{Tabela de Simplificação}
            \end{table}\noindent
            \item \textbf{Simplificação}
            \begin{center}
                \begin{minipage}{0.45\linewidth}
                \begin{figure}[H]
                    \centering
                    \begin{karnaugh-map}[2][4][1][$Q_2$][$W\;Q_1$]
                        \minterms{4,5}
                        \indeterminants{2,3,6,7}
                        \autoterms[0]
                        \implicant{6}{5}
                    \end{karnaugh-map}\caption{Karnaugh $J_1 = W$}
                \end{figure}\noindent
                \end{minipage}
                \begin{minipage}{0.45\linewidth}
                \begin{figure}[H]
                    \centering
                    \begin{karnaugh-map}[2][4][1][$Q_2$][$W\;Q_1$]
                        \minterms{2,3,6}
                        \indeterminants{0,1,4,5}
                        \autoterms[0]
                        \implicant{0}{3}
                        \implicant{0}{4}
                    \end{karnaugh-map}\caption{Karnaugh $K_1 = W \text{ or } \overline{Q}_2$}
                \end{figure}\noindent
                \end{minipage}
            \end{center}
            \begin{center}
                \begin{minipage}{0.45\linewidth}
                \begin{figure}[H]
                    \centering
                    \begin{karnaugh-map}[2][4][1][$Q_2$][$W\;Q_1$]
                        \minterms{6}
                        \indeterminants{1,3,5,7}
                        \autoterms[0]
                        \implicant{6}{7}
                    \end{karnaugh-map}\caption{Karnaugh $J_2 = \overline{W} \text{ and } \overline{Q}_1$}
                \end{figure}\noindent
                \end{minipage}
                \begin{minipage}{0.45\linewidth}
                \begin{figure}[H]
                    \centering
                    \begin{karnaugh-map}[2][4][1][$Q_2$][$W\;Q_1$]
                        \minterms{1,3}
                        \indeterminants{0,2,4,6}
                        \autoterms[0]
                        \implicant{0}{3}
                    \end{karnaugh-map}\caption{Karnaugh $K_2 = \overline{W}$}
                \end{figure}\noindent
                \end{minipage}
            \end{center}
            \item \textbf{Representação}
            \begin{figure}[H]
                \centering
                \includegraphics[width = 10cm]{images/ex05_q3circuit.png}
                \caption{Circuito Final}
            \end{figure}
        \end{enumerate}
    \end{resolution}

\newpage
\subsection{Questão 4}
    \begin{exercise}
        Projete um sistema utilizando apenas FF-D para a tabela de estados e de cada atribuição a seguir.
        \begin{figure}[H]
            \centering
            \includegraphics[width = 10cm]{images/ex05_q4.png}
            \caption{Tabela Verdade}
        \end{figure}
        Mostre as equações simplificadas para D1, D2 e z. Esboce o diagrama de blocos da parte b.
    \end{exercise}
    \begin{resolution}
        Inicialmente será necessário representar as transições de estado em uma tabela:
        \begin{center}
            \begin{minipage}{0.45\linewidth}
            \begin{table}[H]
                \centering\begin{tabular}[]{c|ccc|c|ccc}
                    \multicolumn{4}{c|}{input}      & \multicolumn{4}{c}{output}\\
                                      &$D_1$&$D_2$&$x$ &   &$D_1^\star$&$D_2^\star$&$z$\\\hline
                    \multirow{2}{*}{A}& 0   & 0   &  0 & C & 1         & 0         &  1 \\
                                      & 0   & 0   &  1 & B & 0         & 1         &  1 \\
                    \multirow{2}{*}{B}& 0   & 1   &  0 & D & 1         & 1         &  0 \\
                                      & 0   & 1   &  1 & D & 1         & 1         &  0 \\
                    \multirow{2}{*}{C}& 1   & 0   &  0 & A & 0         & 0         &  0 \\
                                      & 1   & 0   &  1 & D & 1         & 1         &  0 \\
                    \multirow{2}{*}{D}& 1   & 1   &  0 & C & 1         & 0         &  0 \\
                                      & 1   & 1   &  1 & B & 0         & 1         &  0 \\\hline
                \end{tabular}\caption{Configuração a.}
            \end{table}\noindent
            \end{minipage}
            \begin{minipage}{0.45\linewidth}
                \begin{table}[H]
                    \centering\begin{tabular}[]{c|ccc|c|ccc}
                        \multicolumn{4}{c|}{input}         & \multicolumn{4}{c}{output}\\
                                          &$D_1$&$D_2$&$x$ &   &$D_1^\star$&$D_2^\star$&$z$\\\hline
                        \multirow{2}{*}{A}& 0   & 0   &  0 & C & 0         & 1         & 1 \\
                                          & 0   & 0   &  1 & B & 1         & 1         & 1 \\
                        \multirow{2}{*}{B}& 1   & 1   &  0 & D & 1         & 0         & 0 \\
                                          & 1   & 1   &  1 & D & 1         & 0         & 0 \\
                        \multirow{2}{*}{C}& 1   & 0   &  0 & A & 0         & 0         & 0 \\
                                          & 1   & 0   &  1 & D & 1         & 0         & 0 \\
                        \multirow{2}{*}{D}& 1   & 0   &  0 & C & 0         & 1         & 0 \\
                                          & 1   & 0   &  1 & B & 1         & 1         & 0 \\\hline
                    \end{tabular}\caption{Configuração b.}
                \end{table}\noindent
            \end{minipage}
        \end{center}
        Na sequência será necessário realizar a simplicação das equações através de Karnaugh:
        \begin{center}
            \begin{minipage}{0.45\linewidth}
            \begin{figure}[H]
                \centering
                \begin{karnaugh-map}[2][4][1][$x$][$D_1\;D_2$]
                    \minterms{0,2,3,5,6}
                    \autoterms[0]
                    \implicant{0}{2}
                    \implicant{2}{3}
                    \implicant{2}{6}
                    \implicant{5}{5}
                \end{karnaugh-map}\caption{Karnaugh $D_1^\star$ a.}
            \end{figure}\noindent
            \end{minipage}
            \begin{minipage}{0.45\linewidth}
            \begin{figure}[H]
                \centering
                \begin{karnaugh-map}[2][4][1][$x$][$D_1\;D_2$]
                    \minterms{1,2,3,5,7}
                    \autoterms[0]
                    \implicant{1}{5}
                    \implicant{2}{3}
                \end{karnaugh-map}\caption{Karnaugh $D_1^\star$ b.}
            \end{figure}\noindent
            \end{minipage}
        \end{center}
        \begin{center}
            \begin{minipage}{0.45\linewidth}
            \begin{figure}[H]
                \centering
                \begin{karnaugh-map}[2][4][1][$x$][$D_1\;D_2$]
                    \minterms{1,2,3,5,7}
                    \autoterms[0]
                    \implicant{1}{5}
                    \implicant{2}{3}
                \end{karnaugh-map}\caption{Karnaugh $D_2^\star$ a.}
            \end{figure}\noindent
            \end{minipage}
            \begin{minipage}{0.45\linewidth}
            \begin{figure}[H]
                \centering
                \begin{karnaugh-map}[2][4][1][$x$][$D_1\;D_2$]
                    \minterms{0,1,6,7}
                    \autoterms[0]
                    \implicant{0}{1}
                    \implicant{6}{7}
                \end{karnaugh-map}\caption{Karnaugh $D_2^\star$ b.}
            \end{figure}\noindent
            \end{minipage}
        \end{center}
        
        \begin{table}[H]
            \centering\begin{tabular}[]{cl}\hline
                \multirow{2}{*}{a.} & $D_1^\star = (D_2 \text{ and } \overline{x}) \text{ or } (\overline{D}_1 \text{ and } \overline{x}) \text{ or } (\overline{D}_1 \text{ and } D_2) \text{ or } (D_1 \text{ and } \overline{D}_2 \text{ and } x)$\\
                                    & $D_2^\star = (\overline{D}_1 \text{ and } D_2) \text{ or } x$\\\hline
                \multirow{2}{*}{b.} & $D_1^\star = (\overline{D}_1 \text{ and } D_2) \text{ or } x$\\
                                    & $D_2^\star = (\overline{D}_1 \text{ and } \overline{D}_2) \text{ or } (D_1 \text{ and } D_2)$\\\hline
                z                   & $z         = (\overline{D}_1 \text{ and } \overline{D}_2)$\\\hline
            \end{tabular}\caption{Equações FFD e FFT}
        \end{table}
        Obtendo assim, para o caso b. o seguinte diagrama:
        \begin{figure}[H]
            \centering
            \includegraphics[width = 10cm]{images/ex05_q4circuit.png}
            \caption{Circuito Final}
        \end{figure}
        Nota-se que a escolha da codificação dos sinais influenciará a implementação das portas lógicos, visto que o caso b. é mais simples de implementar do que o caso a. .
    \end{resolution}

\newpage
\subsection{Questão 5}
    \begin{exercise}
        Projete um contador que faça a sequência 0, 6, 3, 7, 5, 2 e repita, percorrida em ordem direta caso a entrada externa seja $x=0$ e em ordem reversa caso $x=1$, usando FF-D.\\\\
        Mostre o diagrama de estados, incluindo o que ocorre caso o sistema inicialize em 1 ou 4.
    \end{exercise}
    \begin{resolution}
        Primeiramente será necessário realizar análise dos estados:
        \begin{table}[H]
            \centering\begin{tabular}[]{c|cccc|c|ccc}
                \multicolumn{5}{c|}{input}               & \multicolumn{4}{c}{output}            \\
                                  &$D_2$&$D_1$&$D_0$&$x$ &   &$D_2^\star$&$D_1^\star$&$D_0^\star$\\\hline
                \multirow{2}{*}{0}& 0   & 0   & 0   &  0 & 6 & 0         & 0         & 1         \\
                                  & 0   & 0   & 0   &  1 & 2 & 0         & 1         & 0         \\
                \multirow{2}{*}{6}& 0   & 0   & 1   &  0 & 3 & 0         & 1         & 1         \\
                                  & 0   & 0   & 1   &  1 & 0 & 0         & 0         & 0         \\
                \multirow{2}{*}{2}& 0   & 1   & 0   &  0 & 0 & 0         & 0         & 0         \\
                                  & 0   & 1   & 0   &  1 & 5 & 1         & 0         & 1         \\
                \multirow{2}{*}{3}& 0   & 1   & 1   &  0 & 7 & 1         & 0         & 0         \\
                                  & 0   & 1   & 1   &  1 & 6 & 0         & 0         & 1         \\
                \multirow{2}{*}{7}& 1   & 0   & 0   &  0 & 5 & 1         & 0         & 1         \\
                                  & 1   & 0   & 0   &  1 & 3 & 0         & 1         & 1         \\
                \multirow{2}{*}{5}& 1   & 0   & 1   &  0 & 2 & 0         & 1         & 0         \\
                                  & 1   & 0   & 1   &  1 & 7 & 1         & 0         & 0         \\
                \multirow{2}{*}{1}& 1   & 1   & 0   &  0 & 0 & 0         & 0         & 0         \\
                                  & 1   & 1   & 0   &  1 & 0 & 0         & 0         & 0         \\
                \multirow{2}{*}{4}& 1   & 1   & 1   &  0 & 5 & 0         & 0         & 0         \\
                                  & 1   & 1   & 1   &  1 & 3 & 0         & 0         & 0         \\\hline
            \end{tabular}\caption{Tabela de transições de Estados}
        \end{table}\noindent
        Note que considerou-se que caso o sistema seja inicializado em 1 ou 4 estes, independente da entrada, irão para o estado 0 onde podem dar sequência ao ciclo. Na sequência aplica-se Karnaugh as expressões para simplificá-las:
        \begin{figure}[H]
            \centering
            \begin{karnaugh-map}[4][4][1][$D_0\;x$][$D_2\;D_1$]
                \minterms{5,6,8,11}
                \autoterms[0]
                \implicant{5}{5}
                \implicant{8}{8}
                \implicant{6}{6}
                \implicant{11}{11}
            \end{karnaugh-map}\caption{Karnaugh $D_2^\star = \bar{D}_2 D_1 \bar{D}_0 x + \bar{D}_2 D_1 D_0 \bar{x} + D_2 \bar{D}_1 \bar{D}_0 \bar{x} + D_2 \bar{D}_1 D_0 x$}
        \end{figure}\noindent
        \begin{figure}[H]
            \centering
            \begin{karnaugh-map}[4][4][1][$D_0\;x$][$D_2\;D_1$]
                \minterms{1,2,9,10}
                \autoterms[0]
                \implicantedge{1}{1}{9}{9}
                \implicantedge{2}{2}{10}{10}
            \end{karnaugh-map}\caption{Karnaugh $D_1^\star = \bar{D}_1 \bar{D}_0 x + \bar{D}_1 D_0 \bar{x}$}
        \end{figure}\noindent
        \begin{figure}[H]
            \centering
            \begin{karnaugh-map}[4][4][1][$D_0\;x$][$D_2\;D_1$]
                \minterms{0,2,5,7,8,9}
                \autoterms[0]
                \implicantedge{0}{0}{2}{2}
                \implicant{8}{9}
                \implicant{5}{7}
            \end{karnaugh-map}\caption{Karnaugh $D_0^\star = D_2 \bar{D}_1 \bar{D}_0 + \bar{D}_2 D_1 x + \bar{D}_2 \bar{D}_1 \bar{x}$}
        \end{figure}\noindent
        Representação:
        \begin{figure}[H]
            \centering
            \includegraphics[width = 10cm]{images/ex05_q5circuit.png}
            \caption{Circuito Final}
        \end{figure}
    \end{resolution}

\newpage
\subsection{Questão 6}
    \begin{exercise}
        Construa diagramas de estados para uma máquina de Mealy (com 4 estados) que produza a saída 1 se e apenas se após a entrada exibir a palavra 1010. No primeiro diagrama, assuma que sobreposição é permitida, e no segundo, que deve ser desconsiderada. Por exemplo:
        \begin{figure}[H]
            \centering
            \includegraphics[width = 7.5cm]{images/ex05_q6.png}
            \caption{Circuito Inicial}
        \end{figure}
    \end{exercise}
    \begin{resolution}
        Representação:
        \begin{figure}[H]
            \centering
            \includegraphics[width = 10cm]{images/ex05_q6circuit.png}
            \caption{Circuito Final}
        \end{figure}
    \end{resolution}
\end{document}