\documentclass{article}
\usepackage{tpack}



\titleformat
    {\subsection}           % Part
    [block]                 % Part Shape
    {\normalfont\Large}     % Font Size
    {}                      % Label Numbering
    {0mm}                   % Part Separation
    {}                      % Code Before
    []                      % Code After

    \titlespacing*{\subsection}{0mm}{5mm}{2.5mm}


\title{ES572 - Circuitos Lógicos}
\author{Guilherme Nunes Trofino}
\authorRA{217276}
\project{Atividade Teórica}


\begin{document}
    \maketitle
\newpage

    \section{Atividade Teórica}
        \paragraph{Apresentação}Resolução das questões de Circuitos Lógicos por Guilherme Nunes Trofino, 217276, sobre \textbf{Abstração Digital} e \textbf{Dispositivos Eletrônicos}.

        \subsection{Questão 1}
            \begin{exercise}
                Você recebe as \texttt{Curvas Características de Transferência} de dispositivos de uma entrada e uma saída, para serem utilizados em uma nova família de dispositivos lógicos:
                    \begin{figure}[H]
                        \centering
                        \includegraphics[height = 6cm]{es572_ex02_im01.png}
                    \end{figure} \noindent
                Obtenha um conjunto único de valores de ($V_{OL}$, $V_{IL}$, $V_{OH}$, $V_{IH}$) adequado para serem usados nestes dispositivos. Maximize a \textbf{Imunidade ao Ruído}, definida como a menor entre as duas margens de ruído.
            \end{exercise}
            \begin{resolution}
                1
            \end{resolution}

\newpage
        \subsection{Questão 2}
            \begin{exercise}
                Uma família de circuitos lógicos combinacionais possui as seguintes especificações:
                    \begin{enumerate}
                        \item Saída '0' será garantidamente representada por uma tensão de $0.4 \pm 0.1$ volts;
                        \item Saída '1' será garantidamente representada por uma tensão de $4.6 \pm 0.2$ volts;
                        \item Tensão de Threshold de $2.5 \pm 0.2$ volts com:
                            \begin{enumerate}[noitemsep]
                                \item $V_{TH}-0.5$ volts são garantidamente interpretadas como '0';
                                \item $V_{TH}+0.5$ volts são garantidamente interpretadas como '1';
                            \end{enumerate}
                    \end{enumerate}
                Forneça valores adequados para ($V_{OL}$, $V_{IL}$, $V_{OH}$, $V_{IH}$). Forneça também as duas margens de ruído e a imunidade do ruído desta família de dispositivos.
            \end{exercise}
            \begin{resolution}
                1
            \end{resolution}

\newpage
        \subsection{Questão 3}
            \begin{exercise}
                Você recebe as \texttt{Curvas Características de Transferência} de um inversor NMOS como mostrado abaixo:
                    \begin{figure}[H]
                        \centering
                        \includegraphics[height = 6cm]{es572_ex02_im02.png}
                    \end{figure} \noindent
                Considere as seguintes combinações entre ($V_{OL}$, $V_{IL}$, $V_{OH}$, $V_{IH}$) fornecida:
                    \begin{table}[H]
                        \centering  
                        \begin{tabular}[]{cccc}\hline
                            $V_{OL}$ & $V_{IL}$ & $V_{IH}$ & $V_{OH}$\\\hline
                            0.1      & 0.4      & 4.6      & 4.9\\
                            0.6      & 0.9      & 4.1      & 4.4\\
                            1.1      & 1.4      & 3.6      & 3.9\\\hline
                        \end{tabular}
                    \end{table}
                Verifique se as regras estáticas estão satisfeitas. Em caso negativo, detalhe o motivo. Em caso positivo informe a \textbf{Imunidade ao Ruído}.
            \end{exercise}

            \begin{resolution}
                1
            \end{resolution}

\newpage
        \subsection{Questão 4}
            \begin{exercise}
                Construa a rede \texttt{pull-down} correspondente à rede de \texttt{pull-up} do circuito CMOS apresentado:
                    \begin{figure}[H]
                        \centering
                        \begin{circuitikz}
                            \ctikzset{component text=left}
                            \draw
                            (-0.5,0) node[pmos] (myFET_A1) {}
                            ( 0.5,0) node[pmos, xscale=-1] (myFET_B1) {}
                
                            (myFET_A1.S)-- (-0.5,1)
                                        -- (+0.5,1)
                                        -- (myFET_B1.S)
                                
                            (myFET_A1.D)-- (-0.5,-1)
                                        -- (+0.5,-1)
                                        -- (myFET_B1.D)

                            (0, 1) to[short, o-] ++(0,+0.5)
                            (0,-1) to[short, o-] ++(0,-0.5)

                            (-0.5,3) node[pmos] (myFET_A2) {}
                            ( 0.5,3) node[pmos, xscale=-1] (myFET_B2) {}

                            (myFET_A2.S)-- (-0.5,4)
                                        -- (+0.5,4)
                                        -- (myFET_B2.S)

                            (myFET_A2.D)-- (-0.5,2)
                                        -- (+0.5,2)
                                        -- (myFET_B2.D)

                            (0,4) to[short, o-] ++(0,+0.5)
                            (0,2) to[short, o-] ++(0,-0.5)

                            (myFET_A1.G) node[left] {$A$}
                            (myFET_B1.G) node[right]{$B$}

                            (myFET_A2.G) node[left] {$\bar{A}$}
                            (myFET_B2.G) node[right]{$\bar{B}$};
                        \end{circuitikz} 
                    \end{figure} \noindent
                Apresente a \textbf{Tabela Verdade} deste circuito.
            \end{exercise}
            \begin{resolution}
                1
            \end{resolution}

\newpage
        \subsection{Questão 5}
            \begin{exercise}
                Uma única porta CMOS, que consiste de uma saída conectada a uma única rede \texttt{pull-up} construída por PMOS e uma única rede de \text{pull-down} construída por NMOS, calcula $F(A,B,C,D)$.
                    \begin{enumerate}[noitemsep]
                        \item $F(0,0,1,0)$
                        \item $F(1,1,1,0)$
                        \item $F(1,1,1,1)$
                    \end{enumerate}
                Observa-se que $F(1,0,1,0)=1$ então sobre as combinações acima, responda com \texttt{0}, \texttt{1} ou \texttt{não é possível saber}.
            \end{exercise}
            \begin{resolution}
                1
            \end{resolution}

\newpage
        \subsection{Questão 6}
            \begin{exercise}
                Considere as funções $F(A,B,C)$ e $G(A,B,C)$ apresentadas na tabela a seguir:
                    \begin{table}[H]
                        \centering  
                        \begin{tabular}[]{ccc|cc}\hline
                            A & B & C & F & G\\\hline
                            0 & 0 & 0 & 1 & 1\\
                            0 & 0 & 1 & 1 & 1\\
                            0 & 1 & 0 & 0 & 1\\
                            0 & 1 & 1 & 1 & 0\\
                            1 & 0 & 0 & 1 & 1\\
                            1 & 0 & 1 & 0 & 0\\
                            1 & 1 & 0 & 0 & 1\\
                            1 & 1 & 1 & 1 & 0\\\hline
                        \end{tabular}
                    \end{table}
                Apresente o esquemático do circuito de uma única rede de \texttt{pull-up} por PMOS e uma única rede de \texttt{pull-down} por NMOS se puder ser implementado. Caso contrário, indique que não é possível.
            \end{exercise}
            \begin{resolution}
                1
            \end{resolution}

\newpage
        \subsection{Questão 7}
            \begin{exercise}
                Considere o seguinte circuito:
                    \begin{figure}[H]
                        \centering
                        \includegraphics[height = 6cm]{es572_ex02_im03.png}
                    \end{figure} \noindent
                Considere os seguintes atrasados para cada uma das portas lógicas apresentadas:
                    \begin{table}[H]
                        \centering  
                        \begin{tabular}[]{c|cc}\hline
                            Porta & $t_{CD}$ & $t_{PD}$\\\hline
                            INV   & 0.1 ns   & 1.0 ns\\
                            NAND2 & 0.2 ns   & 1.5 ns\\
                            NAND3 & 0.3 ns   & 1.8 ns\\
                            XOR2  & 0.6 ns   & 2.5 ns\\\hline
                        \end{tabular}
                    \end{table}
                Calcule o atraso de propagação e o atraso de contaminação do circuito completo.
            \end{exercise}
            \begin{resolution}
                1
            \end{resolution}
\end{document}