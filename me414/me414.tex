\documentclass{article}

\usepackage[a4paper, hmargin={20mm, 20mm}, vmargin={25mm, 30mm}]{geometry}
\usepackage[utf8]{inputenc}
\usepackage[english, main=portuguese]{babel}

\usepackage[hidelinks]{hyperref}
\usepackage{bookmark}
\usepackage{cancel}
\usepackage{comment}

\usepackage{array}
\usepackage{indentfirst}
\usepackage{multicol}
\setlength{\multicolsep}{2pt}% 50% of original values
\usepackage{subfiles}

\usepackage{titlesec}

\usepackage{amsmath}
\usepackage{amssymb}
\usepackage{systeme}
\usepackage{float}
\usepackage{enumitem}
\usepackage[thinc]{esdiff} %parcial derivatives
\restylefloat{table}

\usepackage{graphicx}
\usepackage{subcaption}
\graphicspath{ {./images/} }

% Pacote para a definição de novas cores
\usepackage{xcolor}
% Definindo novas cores
\definecolor{darkgreen}{rgb}{0.0, 0.42, 0.24}
\definecolor{darkpurple}{rgb}{0.74, 0.2, 0.64}
\definecolor{darkblue}{rgb}{0.0, 0.28, 0.67}

% Configurando espaços entre paragrafos
%\setlength{\parskip}{0.5em}


% Configurando pacote de circuitos
\usepackage{circuitikz}

%Configurando pacote de Gráficos
\usepackage{tikz}

% Configurando layout para mostrar códigos
\usepackage{listings}

\newcommand{\myStyle}{
\lstset{
    language=Octave,                            % the language of the code
    basicstyle=\ttfamily\small,               % the size of the fonts that are used for the code
    keywordstyle=\color{darkpurple}\bfseries, %
    stringstyle=\color{darkblue},             %
    commentstyle=\color{darkgreen},           %
    morecomment=[s][\color{blue}]{/**}{*/},   %
    extendedchars=true,                       %
    showtabs=false,                           % show tabs within strings adding particular underscores
    showspaces=false,                         % show spaces adding particular underscores
    showstringspaces=false,                   % underline spaces within strings
    numbers=left,                             % where to put the line-numbers
    numberstyle=\tiny\color{gray},            % the style that is used for the line-numbers
    stepnumber=1,                             % the step between two line-numbers. If it's 1, each line will be numbered
    numbersep=5pt,                            % how far the line-numbers are from the code
    frame=single,                             % adds a frame around the code
    rulecolor=\color{black},                  % if not set, the frame-color may be changed on line-breaks within not-black text
    breaklines=true,                          % sets automatic line breaking
    backgroundcolor=\color{white},            % choose the background color
    breakatwhitespace=true,                   % sets if automatic breaks should only happen at whitespace
    breakautoindent=false,                    %
    captionpos=b,                             % sets the caption-position to bottom
    xleftmargin=0pt,                          %
    tabsize=2,                                % sets default tabsize to 2 spaces
}}

%\titleformat{<command>}[<shape>]{<format>}{<label>}{<sep>}{<before-code>}[<after-code>]
\titleformat
{\section} %comand
[block]  %shape
{\normalfont\LARGE} %format
{\thesection. } %label
{0mm} %sep
{} %before-code
[{\titlerule[0.1mm]}] %after-code

\titlespacing*{\section}{0mm}{0mm}{15mm}

\titleformat
{\subsection} %comand
[block]  %shape
{\normalfont\Large} %format
{\thesubsection. } %label
{0mm} %sep
{} %before-code
[] %after-code

\titlespacing*{\subsection}{0mm}{5mm}{2.5mm}


\begin{document}
    \begin{titlepage}
        \begin{center}
            \rule{450pt}{0.5pt}\\[4mm]
            {\Huge ME414 - Estatística para Experimentalistas}\\
            \rule{450pt}{0.5pt}\\[2mm]
            {\Large Resumo Teórico}\\[200mm]
            \today\\
            \rule{250pt}{0.5pt}\\
            {\large Guilherme Nunes Trofino}\\
            {\large 217276}\\
        \end{center}
    \end{titlepage}
\newpage

    \tableofcontents
\newpage

    \section{Introdução}
        \subsection{Estrutura dos Dados}
            \paragraph{Definição}Organição das condições de um elemento estudado de acordo com sua estrutura e atributos, classificados por sua semelhança como mostrados a seguir:
                \begin{enumerate}[noitemsep]
                    \item \textbf{Quantitativos:} Números;
                        \begin{enumerate}[noitemsep]
                            \item \texttt{Contínuos:} Valores possíveis em um intervalo, aberto ou fechado, real;
                            \item \texttt{Discretos:} Valores possíveis em um intervalo, aberto ou fechado, natural;
                        \end{enumerate}
                    \item \textbf{Qualitativos:} Categorias;
                        \begin{enumerate}[noitemsep]
                            \item \texttt{Nominais:} Não possue ordenação;
                            \item \texttt{Ordinais:} Possue ordenação;
                        \end{enumerate}
                \end{enumerate}

        \subsection{Análise Descritiva}
            \paragraph{Definição}Métodos para resumir e sintetizar dados obtidos de uma amostra, podendo ser resumidos através de diferentes métricas, adequadas as variáveis analisadas, como mostradas a seguir:
                \begin{enumerate}[noitemsep]
                    \item \textbf{Métricas Quantitativas:} Resultados numéricos consolidados, como:
                        \begin{enumerate}[noitemsep]
                            \item \texttt{Média:} 
                            \item \texttt{Mediana:} 
                            \item \texttt{Desvio Padrão:} 
                        \end{enumerate}
                    \item \textbf{Métricas Visuais:} Resultados visualmente consolidados, como:
                        \begin{enumerate}[noitemsep]
                            \item \texttt{Gráficos:} 
                            \item \texttt{Figuras:} 
                        \end{enumerate}
                \end{enumerate}

        \subsection{Análise Descritiva Univariada}
            \paragraph{Descrição}Análise individual de cada variável presente no estudo, buscando classificá-la e obter métricas adequadas para resumi-lá. Recomenda-se as seguintes métricas para cada categoria analisada:
                \begin{enumerate}[noitemsep]
                    \item \textbf{Quantitativos:} ;
                        \begin{enumerate}[noitemsep]
                            \item \texttt{Contínuos:} Recomenda-se agrupar valores, quando muitos, em intervalos;
                                \begin{enumerate}[noitemsep]
                                    \item Gráfico de Histograma;
                                    \item Gráfico de Boxplot;
                                \end{enumerate}
                            \item \texttt{Discretos:} Recomenda-se agrupar valores, quando muitos, em intervalos;
                                \begin{enumerate}[noitemsep]
                                    \item Tabela de Frequências: Absolutas e Relativas;
                                    \item Gráfico de Barras;
                                \end{enumerate}
                        \end{enumerate}
                    \item \textbf{Qualitativos:} ;
                        \begin{enumerate}[noitemsep]
                            \item \texttt{Nominais:} Ordem Arbitraria; 
                                \begin{enumerate}[noitemsep]
                                    \item Tabela de Frequências: Absolutas e Relativas;
                                    \item Gráfico de Barras;
                                    \item Gráfico de Setores;
                                \end{enumerate}
                            \item \texttt{Ordinais:} Ordem Sequencial;
                                \begin{enumerate}[noitemsep]
                                    \item Tabela de Frequências: Absolutas e Relativas;
                                    \item Gráfico de Barras;
                                    \item Gráfico de Setores;
                                \end{enumerate}
                        \end{enumerate}
                \end{enumerate}
\newpage

    \section{Ferramentas de Análise}
        \subsection{Histograma}
            \paragraph{Definição}
\end{document}