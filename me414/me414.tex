\documentclass{article}

\usepackage[a4paper, hmargin={20mm, 20mm}, vmargin={25mm, 30mm}]{geometry}
\usepackage[utf8]{inputenc}
\usepackage[english, main=portuguese]{babel}

\usepackage[hidelinks]{hyperref}
\usepackage{bookmark}
\usepackage{cancel}
\usepackage{comment}

\usepackage{array}
\usepackage{indentfirst}
\usepackage{multicol}
\setlength{\multicolsep}{2pt}% 50% of original values
\usepackage{subfiles}

\usepackage{titlesec}

\usepackage{amsmath}
\usepackage{amssymb}
\usepackage{systeme}
\usepackage{float}
\usepackage{enumitem}
\usepackage[thinc]{esdiff} %parcial derivatives
\restylefloat{table}

\usepackage{graphicx}
\usepackage{subcaption}
\graphicspath{ {./images/} }

% Pacote para a definição de novas cores
\usepackage{xcolor}
% Definindo novas cores
\definecolor{darkgreen}{rgb}{0.0, 0.42, 0.24}
\definecolor{darkpurple}{rgb}{0.74, 0.2, 0.64}
\definecolor{darkblue}{rgb}{0.0, 0.28, 0.67}

% Configurando espaços entre paragrafos
%\setlength{\parskip}{0.5em}

%Configurando numeração de objetos; figuras, equações e etc., em ambientes; section, subsection e etc..
\usepackage{chngcntr}
\counterwithin{figure}{section}
\counterwithin{equation}{subsection}

%Configurando pacote de Gráficos plots
\usepackage{pgfplots}
\usepackage{tikz}
\usepgfplotslibrary{statistics}

%Configurando pacote de circuitos
\usepackage{circuitikz}

%Configurando layout para mostrar códigos
\usepackage{listings}

%Configurando multiple files
\usepackage{filecontents}

%Configurando quotes
\usepackage{csquotes}

\newcommand{\myStyle}{
\lstset{
    language=Octave,                            % the language of the code
    basicstyle=\ttfamily\small,               % the size of the fonts that are used for the code
    keywordstyle=\color{darkpurple}\bfseries, %
    stringstyle=\color{darkblue},             %
    commentstyle=\color{darkgreen},           %
    morecomment=[s][\color{blue}]{/**}{*/},   %
    extendedchars=true,                       %
    showtabs=false,                           % show tabs within strings adding particular underscores
    showspaces=false,                         % show spaces adding particular underscores
    showstringspaces=false,                   % underline spaces within strings
    numbers=left,                             % where to put the line-numbers
    numberstyle=\tiny\color{gray},            % the style that is used for the line-numbers
    stepnumber=1,                             % the step between two line-numbers. If it's 1, each line will be numbered
    numbersep=5pt,                            % how far the line-numbers are from the code
    frame=single,                             % adds a frame around the code
    rulecolor=\color{black},                  % if not set, the frame-color may be changed on line-breaks within not-black text
    breaklines=true,                          % sets automatic line breaking
    backgroundcolor=\color{white},            % choose the background color
    breakatwhitespace=true,                   % sets if automatic breaks should only happen at whitespace
    breakautoindent=false,                    %
    captionpos=b,                             % sets the caption-position to bottom
    xleftmargin=0pt,                          %
    tabsize=2,                                % sets default tabsize to 2 spaces
}}

%\titleformat{<command>}[<shape>]{<format>}{<label>}{<sep>}{<before-code>}[<after-code>]
\titleformat
{\section} %comand
[block]  %shape
{\normalfont\LARGE} %format
{\thesection. } %label
{0mm} %sep
{} %before-code
[{\titlerule[0.1mm]}] %after-code

\titlespacing*{\section}{0mm}{0mm}{15mm}

\titleformat
{\subsection} %comand
[block]  %shape
{\normalfont\Large} %format
{\thesubsection. } %label
{0mm} %sep
{} %before-code
[] %after-code

\titlespacing*{\subsection}{0mm}{5mm}{2.5mm}


\begin{document}
    \begin{titlepage}
        \begin{center}
            \rule{450pt}{0.5pt}\\[4mm]
            {\Huge ME414 - Estatística para Experimentalistas}\\
            \rule{450pt}{0.5pt}\\[2mm]
            {\Large Resumo Teórico}\\[200mm]
            \today\\
            \rule{250pt}{0.5pt}\\
            {\large Guilherme Nunes Trofino}\\
            {\large 217276}\\
        \end{center}
    \end{titlepage}
\newpage

    \tableofcontents
\newpage

    \section{Análise Descritiva}
        \paragraph{Definição}Métodos para resumir e sintetizar dados obtidos de uma amostra, podendo ser resumidos através de diferentes métricas, adequadas as variáveis analisadas.

        \subsection{Estrutura dos Dados}
            \paragraph{Definição}Organição das condições de um elemento estudado de acordo com sua estrutura e atributos, classificados por sua semelhança como mostrados a seguir:
                \begin{enumerate}[noitemsep]
                    \item \textbf{Quantitativos:} Números;
                        \begin{enumerate}[noitemsep]
                            \item \texttt{Contínuos:} Valores possíveis em um intervalo, aberto ou fechado, real;
                            \item \texttt{Discretos:} Valores possíveis em um intervalo, aberto ou fechado, natural;
                        \end{enumerate}
                    \item \textbf{Qualitativos:} Categorias;
                        \begin{enumerate}[noitemsep]
                            \item \texttt{Nominais:} Não possue ordenação;
                            \item \texttt{Ordinais:} Possue ordenação;
                        \end{enumerate}
                \end{enumerate}

        \subsection{Média}
            \paragraph{Definição}Medida de posição central em que se $x_{1}, \cdots, x_{n}$ são $n$ observações, a média aritmética será dada pela seguinte equação:
                \begin{equation}
                    \boxed{
                        \overline{x} = \frac{1}{n} \sum_{i = 1}^{n} x_{i}
                    }
                \end{equation}

        \subsection{Mediana}
            \paragraph{Definição}Medida de posição central em que divide o conjunto de dados em dois grupos, cada um com 50\% das observações, sendo dada pela seguinte equação:
                \begin{equation}
                    \boxed{
                        Q_{2} =
                        \begin{cases}
                            x_{(\frac{n+1}{2})},  & \text{se $n$ é ímpar};\vspace{2.5mm}\\
                            \frac{x_{(\frac{n}{2})} + x_{(\frac{n}{2} + 1)}}{2},  & \text{se $n$ é par};\\
                        \end{cases}
                    }
                \end{equation}

        \subsection{Moda}
            \paragraph{Definição}Métrica com maior número de ocorrências na amostra;

        \subsection{Variância}
            \paragraph{Definição}Medida de dispersão em que a média dos desvios ao quadrado será dada pela seguinte equação:
                \begin{equation}
                    \boxed{
                        s^{2} = \frac{1}{n-1} \sum_{i=1}^{n} (x_{i} - \overline{x})^{2}
                    }
                \end{equation}
                \begin{enumerate}[rightmargin=\leftmargin, noitemsep]
                    \item \textbf{Coeficiente de Variação:} Normaliza o desvio padrão com relação a média do conjunto de dados de tal forma que os desvios padrões de diferentes amostras possam ser comparadas, será dado pela seguinte equação:
                        \begin{equation}
                            \boxed{
                                C_{V} = \frac{s}{\overline{x}}
                            }
                        \end{equation}
                \end{enumerate}

        \subsection{Desvio Padrão}
            \paragraph{Definição}Medida de dispersão em que a raiz quadrada da variância será dada pela seguinte equação:
            \begin{equation}
                \boxed{
                    s = \sqrt{\frac{1}{n-1} \sum_{i=1}^{n} (x_{i} - \overline{x})^{2}}
                }
            \end{equation}

        \subsection{Coeficiente de Correlação}
            \paragraph{Definição}Quantificação da relação linear entre duas variáveis quantitativas, avaliando como cada uma delas estará deslocada proporcionalmente da média da amostra com $n$ observações. Formalmente descrita pela seguinte equação:
                \begin{equation}
                    \boxed{
                        -1 \le
                        C_{C}(x,y) = \frac{1}{n-1} \sum_{i=1}^{n} 
                        \left(\frac{x_{i}-\overline{x}}{s_{x}}\right)
                        \cdot
                        \left(\frac{y_{i}-\overline{y}}{s_{y}}\right)
                        \le 1
                    }
                \end{equation}
            Onde:
                \begin{enumerate}[rightmargin = \leftmargin, noitemsep]
                    \item \textbf{Desvio Padronizado:} Define-se que a operação realizada representa quão distante uma observação $x_{i}$, ou $y_{i}$, está afastada de sua respectiva média $\overline{x}$, ou $\overline{y}$, sendo descrita pela seguinte equação:
                        \begin{equation}
                            \boxed{
                                z_{x_{i}} = \frac{x_{i}-\overline{x}}{s_{x}}
                            }
                            \hspace{20mm}
                            \boxed{
                                z_{y_{i}} = \frac{y_{i}-\overline{y}}{s_{y}}
                            }
                        \end{equation}
                \end{enumerate}
            Assim classifica-se a relação entre duas variáveis de acordo com o coeficiente de correlação:
                \begin{enumerate}[rightmargin = \leftmargin, noitemsep]
                    \item \textbf{Positivamente Associadas:} Quando as variáveis apresentam associação linear crescente, ou seja, apresentam um coeficiente de correlação:
                        \[\boxed{0 < C_{C}(x,y) \le 1}\]
                    \item \textbf{Negativamente Associadas:} Quando as variáveis apresentam associação linear decrescente, ou seja, apresentam um coeficiente de correlação:
                        \[\boxed{-1 \le C_{C}(x,y) < 0}\]
                    \item \textbf{Não Associadas:} Quando as variáveis não apresentam associação linear, ou seja, apresentam um coeficiente de correlação:
                        \[\boxed{C_{C}(x,y) = 0}\]
                \end{enumerate}

        \subsection{Análise Descritiva Univariada}
            \paragraph{Descrição}Análise individual de cada variável presente no estudo, buscando classificá-la e obter métricas adequadas para resumi-lá. Recomenda-se as seguintes métricas para cada categoria analisada:
                \begin{enumerate}[noitemsep]
                    \item \textbf{Quantitativos:} ;
                        \begin{enumerate}[noitemsep]
                            \item \texttt{Contínuos:} Recomenda-se agrupar valores, quando muitos, em intervalos;
                                \begin{enumerate}[noitemsep]
                                    \item Gráfico de Histograma;
                                    \item Gráfico de Ramo-e-Folhas;
                                \end{enumerate}
                            \item \texttt{Discretos:} Recomenda-se agrupar valores, quando muitos, em intervalos;
                                \begin{enumerate}[noitemsep]
                                    \item Tabela de Frequências: Absolutas e Relativas;
                                    \item Gráfico de Barras;
                                \end{enumerate}
                        \end{enumerate}
                    \item \textbf{Qualitativos:} ;
                        \begin{enumerate}[noitemsep]
                            \item \texttt{Nominais:} Ordem Arbitraria; 
                                \begin{enumerate}[noitemsep]
                                    \item Tabela de Frequências: Absolutas e Relativas;
                                    \item Gráfico de Barras;
                                    \item Gráfico de Setores;
                                \end{enumerate}
                            \item \texttt{Ordinais:} Ordem Sequencial;
                                \begin{enumerate}[noitemsep]
                                    \item Tabela de Frequências: Absolutas e Relativas;
                                    \item Gráfico de Barras;
                                    \item Gráfico de Setores;
                                \end{enumerate}
                        \end{enumerate}
                \end{enumerate}

        \subsection{Análise Descritiva Bivariada}
            \paragraph{Definição}Análise conjunta de dois conjuntos de variáveis presente no estudo, buscando classificá-las, relacioná-las e obter métricas para resumi-las. Recomenda-se as seguintes métrica para cada categoria analisada:
                \begin{enumerate}[noitemsep]
                    \item \textbf{Qualitativas:} Comparação dos dados independentes em relações proporcionais aos diferentes subconjuntos dos dados;
                        \begin{enumerate}[noitemsep]
                            \item \texttt{Porcentagens Relativas};
                            \item \texttt{Gráficos de Barras};
                        \end{enumerate}
                    \item \textbf{Quantitativas:} Compação relacional entre as variáveis procurando explorar como uma influencia a outra;
                        \begin{enumerate}[noitemsep]
                            \item \texttt{Coeficiente de Correlação:} Gráfico relacionando as diferente variáveis em diferentes eixos;
                            \item \texttt{Gráfico de Dispersão:} Medida que representa a associação linear entre variáveis quantitativas;
                        \end{enumerate}
                \end{enumerate}
\newpage

    \section{Ferramentas Gráficas}
        \subsection{Histograma}
            \paragraph{Definição}Representação gráfica de uma variável contínua que agrupada os valores em intervalos, abertos em uma extremidade e fechados na outra, regulares de acordo com as frequências dos dados, construídos como mostrado no exemplo a seguir:

                \begin{table}[H]
                    \centering
                    \begin{tabular}[]{cccccccc}\hline
                        114 & 122 & 103 & 118 &  99 & 105 & 134 & 125\\
                        117 & 106 & 109 & 104 & 111 & 127 & 133 & 111\\
                        117 & 103 & 120 &  98 & 100 & 130 & 141 & 119\\
                        128 & 106 & 109 & 115 & 113 & 121 & 100 & 130\\\hline
                    \end{tabular}
                    \caption{Distribuição de Dados Inicial}\label{table:data}
                \end{table}

                \begin{enumerate}
                    \item \textbf{Etapa:} Organize os dados em ordem crescente;
                        \begin{table}[H]
                            \centering
                            \begin{tabular}[]{cccccccc}\hline
                                98 &  99 & 100 & 100 & 103 & 103 & 104 & 105\\
                                106 & 106 & 109 & 109 & 111 & 111 & 113 & 114\\
                                115 & 117 & 117 & 118 & 119 & 120 & 121 & 122\\
                                125 & 127 & 128 & 130 & 130 & 133 & 134 & 141\\\hline
                            \end{tabular}
                            \caption{Distribuição de Dados Ordenados}\label{table:dataShort}
                        \end{table}
                    \item \textbf{Etapa:} Defina intervalos disjuntos, cada ocorrência em um único intervalo aberto à esquerda e fechado à direita;
                    \item \textbf{Etapa:} Construa uma tabela de frequências;
                        \begin{table}[H]
                            \centering
                            \begin{tabular}[]{c|c}\hline
                                Intervalo  & Frequência\\\hline
                                ( 95, 100] & 4\\
                                (100, 105] & 4\\
                                (105, 110] & 4\\
                                (110, 115] & 5\\
                                (115, 120] & 5\\\hline
                            \end{tabular}
                            \hspace{10mm}
                            \begin{tabular}[]{c|c}\hline
                                Intervalo  & Frequência\\\hline
                                (120, 125] & 3\\
                                (125, 130] & 4\\
                                (130, 135] & 2\\
                                (135, 140] & 0\\
                                (140, 145] & 1\\\hline
                            \end{tabular}
                            \caption{Intervalos e Frequências}\label{table:dataFinal}
                        \end{table}
                    \item \textbf{Etapa:} Desenhe o gráfico com a frequência na ordenada e os intervalos na abscissas;
                \end{enumerate}

                \begin{figure}[H]
                    \centering
                    \begin{tikzpicture}
                        \begin{axis}[
                                xmin = 90,
                                xmax = 150,
                                ymin = 0, 
                                ymax = 8,
                                minor y tick num = 6,
                                minor x tick num = 6,
                                area style,
                                width = 12cm,
                                height = 6cm,
                                ylabel = {Frequência},
                                xlabel = {Quantidade},
                            ]
                            \addplot+[
                                fill = white,
                                draw = black,
                                ybar interval,
                                mark = no] plot coordinates {
                                ( 95, 4) 
                                (100, 4) 
                                (105, 4) 
                                (110, 5) 
                                (115, 5) 
                                (120, 3)
                                (125, 4)
                                (130, 2)
                                (135, 0)
                                (140, 1)
                                (145, 0)
                            };
                        \end{axis}
                    \end{tikzpicture}
                \end{figure} \noindent
        Estes gráficos possibilitam expressar dados de maneira eficiente, pois conclusões visuais auxiliam na análise dos dados apresentados. Há diferentes classificações de Histogramas, sendo apresentados abaixo:
            \begin{figure}[H]
                \centering
                \begin{subfigure}[t]{0.3\textwidth}
                    \centering
                    \begin{tikzpicture}
                        \begin{axis}[
                                xmin = 0,
                                xmax = 11,
                                ymin = 0, 
                                ymax = 10,
                                minor x tick num = 4,
                                minor y tick num = 1,
                                area style,
                                width  = 5cm,
                                height = 5cm,
                                %ylabel = {Frequência},
                                %xlabel = {Quantidade},
                            ]
                            \addplot+[
                                fill = white,
                                draw = black,
                                ybar interval,
                                mark = no
                            ] 
                                plot coordinates {
                                (1,5) 
                                (2,7) 
                                (3,9) 
                                (4,7) 
                                (5,5) 
                                (6,4)
                                (7,3)
                                (8,2)
                                (9,1)
                                (10,1)
                            };
                        \end{axis}
                    \end{tikzpicture}
                    \caption{Assimétrico à Esquerda}
                \end{subfigure}
                \begin{subfigure}[t]{0.3\textwidth}
                    \centering
                    \begin{tikzpicture}
                        \begin{axis}[
                                xmin = 0,
                                xmax = 11,
                                ymin = 0, 
                                ymax = 10,
                                minor x tick num = 4,
                                minor y tick num = 1,
                                area style,
                                width  = 5cm,
                                height = 5cm,
                                %ylabel = {Frequência},
                                %xlabel = {Quantidade},
                            ]
                            \addplot+[
                                fill = white,
                                draw = black,
                                ybar interval,
                                mark = no
                            ] 
                                plot coordinates {
                                (1,1) 
                                (2,3) 
                                (3,5) 
                                (4,7) 
                                (5,9) 
                                (6,7)
                                (7,5)
                                (8,3)
                                (9,1)
                                (10,1)
                            };
                        \end{axis}
                    \end{tikzpicture}
                    \caption{Perfeitamente Simétrico}
                \end{subfigure}
                \begin{subfigure}[t]{0.3\textwidth}
                    \centering
                    \begin{tikzpicture}
                        \begin{axis}[
                                xmin = 0,
                                xmax = 11,
                                ymin = 0, 
                                ymax = 10,
                                minor x tick num = 4,
                                minor y tick num = 1,
                                area style,
                                width  = 5cm,
                                height = 5cm,
                                %ylabel = {Frequência},
                                %xlabel = {Quantidade},
                            ]
                            \addplot+[
                                fill = white,
                                draw = black,
                                ybar interval,
                                mark = no
                            ]
                                plot coordinates {
                                (1,1) 
                                (2,2) 
                                (3,3) 
                                (4,4) 
                                (5,5) 
                                (6,7)
                                (7,9)
                                (8,7)
                                (9,5)
                                (10,3)
                            };
                        \end{axis}
                    \end{tikzpicture}
                    \caption{Assimétrico à Direita}
                \end{subfigure}
                \caption{Simétria de Distribuição}
            \end{figure} \noindent

        \subsection{Ramo-e-Folhas}
            \paragraph{Definição}Representação Gráfica de uma contínua que agrupada os valores separando-os em duas partes: ramo, colocado a esquerda, e folhas, colocado a direita; mantendo as informações, construídos como mostrado no exemplo a seguir:
                \begin{table}[H]
                    \centering
                    \begin{tabular}[]{cccccccc}\hline
                        98 &  99 & 100 & 100 & 103 & 103 & 104 & 105\\
                        106 & 106 & 109 & 109 & 111 & 111 & 113 & 114\\
                        115 & 117 & 117 & 118 & 119 & 120 & 121 & 122\\
                        125 & 127 & 128 & 130 & 130 & 133 & 134 & 141\\\hline
                    \end{tabular}
                    \caption{Distribuição de Dados Ordenados}\label{table:dataLeafs}
                \end{table} \noindent
            Em sequência separa-se os valores entre as dezenas e unidades, obtendo a tabela a seguir:
                \begin{table}[H]
                    \centering
                    \begin{tabular}[]{r|l}\hline
                        Ramo & Folha\\\hline
                         9   & 8 9\\
                        10   & 0 0 3 3 4 5 6 6 9 9\\
                        11   & 1 1 3 4 5 7 7 8 9\\
                        12   & 0 1 2 5 7 8\\
                        13   & 0 0 3 4\\
                        14   & 1\\\hline
                    \end{tabular}
                    \caption{Distribuição de Ramos-e-Folhas}\label{table:leafs}
                \end{table}

        \subsection{Quartis}
            \paragraph{Definição}Representação gráfica que divide o conjunto de dados em 4 partes iguais: primeiro quartil, $Q_{1}$; segundo quartil, $Q_{2}$; e terceiro quartil, $Q_{3}$. Obtidos através do seguinte procedimento:
                \begin{enumerate}[noitemsep]
                    \item \textbf{Ordenação:} Organizar os dados do conjunto amostral;
                    \item \textbf{Segundo Quartil:} Obter a mediana deste conjunto, $Q_{2}$;
                    \item \textbf{Primeiro Quartil:} Obter a mediana do conjunto inferior a $Q_{2}$, $Q_{1}$;
                    \item \textbf{Terceiro Quartil:} Obter a mediana do conjunto superior a $Q_{2}$, $Q_{3}$;
                \end{enumerate}

        \subsection{Boxplot}
            \paragraph{Definição}Representação gráfica que permite resumir visualmente o esquema de 5 números, possibilitando analisar posição, dispersão, assimetria e outliers. Obtidos como repressentado abaixo:
                \begin{figure}[H]
                    \centering
                    \begin{tikzpicture}
                        \begin{axis}
                            [
                                xmin = 0,
                                xmax = 5,
                                ymin = 0, 
                                ymax = 2,
                                minor x tick num = 3,
                                minor y tick num = 1,
                                %area style,
                                width  = 5cm,
                                height = 5cm,
                                %ylabel = {Frequência},
                                %xlabel = {Quantidade},
                            ]
                          \addplot+[
                              color = black,
                          boxplot prepared={
                            median=2,
                            upper quartile= 3.5,
                            lower quartile= 1.5,
                            upper whisker = 4.5,
                            lower whisker = 0.5
                          },
                          ] coordinates {};
                        \end{axis}
                    \end{tikzpicture}
                    \caption{Representação de Boxplot}
                \end{figure} \noindent
            Neste diagrama temos, medidos da esquerda para direito, as seguintes informações sobre os dados apresentados:
                \begin{enumerate}[rightmargin = \leftmargin, noitemsep]
                    \item \textbf{Limite Inferior:} Representa o valor mínimo da amostra, caso não hajam outliers, ou, caso hajam outliers, será obtido pela seguinte equação:
                        \begin{equation}
                            \boxed{
                                Q_{1} - 1.5 \times IQ
                            }
                        \end{equation}
                    \item $Q_{1}$, \textbf{Primeiro Quartil};
                    \item $Q_{2}$, \textbf{Segundo Quartil};
                    \item $Q_{3}$, \textbf{Terceiro Quartil};
                    \item $IQ$, \textbf{Intervalo Interquartílico:} Intervalo que compreende 50\% dos dados da amostra, será obtido pela seguinte equação:
                        \begin{equation}
                            \boxed{
                                IQ = Q_{3} - Q_{1}
                            }
                        \end{equation}
                    \item \textbf{Limite Superior:} Representa o valor máximo da amostra, caso não hajam outliers, ou, caso hajam outliers, será obtido pela seguinte equação:
                        \begin{equation}
                            \boxed{
                                Q_{3} + 1.5 \times IQ
                            }
                        \end{equation}

                \end{enumerate}
\newpage

    \section{Análise Combinatória}
        \subsection{Regra da Adição}
            \paragraph{Definição}Suponha que hajam dois possíveis procedimentos, $P_{1}$ e $P_{2}$, com diferentes formas de serem realizados, $n_{1}$ e $n_{2}$, para executar uma mesma tarefa, então esta possuirá $n_{1} + n_{2}$ formas de ser executada.

        \subsection{Regra da Multiplicação}
            \paragraph{Definição}Suponha que hajam dois procedimentos necessários, $P_{1}$ e $P_{2}$, com diferentes formas de serem realizados, $n_{1}$ e $n_{2}$, para executar uma mesma tarefa, então esta possuirá $n_{1} \times n_{2}$ formas de ser executada.

        \subsection{Permutação}
            \paragraph{Definição}Suponha que tenhamos uma coleção $\Omega = \{ \omega_{1}, ..., \omega_{n} \}$ de $n$ finitos objetos que devem ser dispostos em sequência, então:
                \begin{displayquote}[][]
                    Haverá $n$ opções disponíveis para a primeira escolha, na sequência, haverá $n-1$ opções disponíveis para a segunda escolha e assim suscetivamente até que reste 1 objeto da coleção. Isso ocorre, pois a \textbf{Permutação} permite trocar posições objetos de uma coleção para formação de uma sequência.
                \end{displayquote}
                \begin{equation}
                    \boxed{
                        n! = n \times (n - 1) \times \cdots \times 1
                    }
                \end{equation}
            Quando houverem $r$ elementos com $n_{1}, ..., n_{r}$ repetições dentro uma coleção, respectivamente, então as organizações possíveis se reduzem:
                \begin{equation}
                    \boxed{
                        \binom{n}{n_{1}, \cdots, n_{r}} = \frac{n!}{n_{1}! \times \cdots \times n_{r}!}
                    }
                \end{equation}

        \subsection{Arranjo}
            \paragraph{Definição}Suponha que tenhamos uma coleção $\Omega = \{ \omega_{1}, ..., \omega_{n} \}$ de $n$ finitos objetos que devem ser separadas em $k$ grupos, então:
                \begin{displayquote}[][]
                    Haverá $n$ opções disponíveis para a primeira escolha, na sequência, haverá $n-1$ opções disponíveis para a segunda escolha e assim suscetivamente até que reste $n-k$ objetos da coleção. Isso ocorre, pois o \textbf{Arranjo} permite separar elementos de uma coleção em grupos considerando sua ordem.
                \end{displayquote}
                \begin{equation}
                    \boxed{
                        A(n, k) = \frac{n!}{(n - k)!} = n \times (n - 1) \times \cdots \times (n - k + 1)
                    }
                \end{equation}

        \subsection{Combinação}
            \paragraph{Definição}Suponha que tenhamos uma coleção $\Omega = \{ \omega_{1}, ..., \omega_{n} \}$ de $n$ finitos objetos que devem ser separadas em $k$ grupos desconsiderando repetições, então:
                \begin{displayquote}[][]
                    Haverá $n$ opções disponíveis para a primeira escolha, na sequência, haverá $n-1$ opções disponíveis para a segunda escolha e assim suscetivamente até que reste $n-k$ objetos da coleção. Isso ocorre, pois a \textbf{Combinação} permite separar elementos de uma coleção em grupos desconsiderando sua ordem.
                \end{displayquote}
                \begin{equation}
                    \boxed{
                        C(n, k) = \binom{n}{k} = \frac{n!}{k! \cdot (n - k)!}
                    }
                \end{equation}

        \subsection{Amostragem}
            \paragraph{Definição}Seleção de elementos de um conjunto amostral que, de acordo com a seleção, terá melhores ou piores chances de representar o conjunto em sua completude, sendo classificados como descrito a seguir:
                \begin{enumerate}[noitemsep]
                    \item \textbf{Amostragem Aleatória Simples com Reposição:} 
                        \begin{equation}
                            \boxed{
                                N^{n}
                            }
                        \end{equation}
                    \item \textbf{Amostragem Aleatória Simples sem Reposição:} 
                        \begin{enumerate}[noitemsep]
                            \item \texttt{Caso Não Ordenado:} 
                                \begin{equation}
                                    \boxed{
                                        \binom{N}{n} = \frac{N!}{(N-n)! \cdot n!}
                                    }
                                \end{equation}
                            \item \texttt{Caso Ordenado:} 
                                \begin{equation}
                                    \boxed{
                                        \frac{N!}{(N - n)!}
                                    }
                                \end{equation}
                        \end{enumerate}
                \end{enumerate}

\newpage

    \section{Probabilidade}
        \subsection{Experimento}
            \paragraph{Definição}Processo que produza uma observação ou resultado que poderá ser reproduzido quantas vezes forem necessárias, sendo classificados como mostrado a seguir:
                \begin{enumerate}[noitemsep]
                    \item \textbf{Experimento Determinístico:} Aquele cujo o resultado obtido será conhecido;
                    \item \textbf{Experimento Aleatório:} Aquele cujo o resultado obtido será desconhecido;
                \end{enumerate}

        \subsection{Lei dos Grandes Números}
            \paragraph{Definição}Segundo Jacob Bernoulli temos como mostrado a seguir:
                \begin{displayquote}[][]
                    Se um evento de probabilidade p é observado repetidamente em ocasiões independentes, a proporção da frequência observada deste evento em relação ao número total de repetições converge em direção a p à medida que o número de repetições se torna arbitrariamente grande.
                \end{displayquote}

        \subsection{Espaço Amostral}
            \paragraph{Definição}Conjunto que agrega todas os possíveis resultados do experimento realidade, sendo denotado como mostrado a seguir:
                \begin{equation}
                    \boxed{
                        \Omega = \{\omega_{1}, ... , \omega_{n} \}
                    }
                \end{equation}
            Onde:
                \begin{enumerate}[rightmargin = \leftmargin, noitemsep]
                    \item \textbf{Elemento:} Acontecimento registrado, $\omega_{i}$;

                    \item \textbf{Espaço Amostral:} Conjunto de todos os eventos, $\Omega$;

                    \item \textbf{Evento:} Subconjunto do espaço amostra denotado por letras, A, B, ...;
                        \begin{enumerate}[rightmargin = \leftmargin]
                            \item \textbf{Interseção de Eventos:} Elementos que pertencem simultaneamente ao evento $A$ e ao evento $B$, $A \cap B$;

                            \item \textbf{União de Eventos:} Elementos que pertencem ao evento $A$ ou ao evento $B$, $A \cup B$;
                                \begin{enumerate}[rightmargin = \leftmargin, noitemsep]
                                    \item \texttt{Eventos Disjuntos:} Nenhum elemento do evento $A$ pertence ao evento $B$ e vice-versa, $A \cap B = \emptyset$
                                \end{enumerate}

                            \item \textbf{Evento Complementar:} Se o evento $A$ e $B$ possuem $A\cap B = \emptyset$ e $A \cup B = \Omega$, então são complementares, $A = B^{C}$.
                        \end{enumerate}
                \end{enumerate}
            Nota-se que cada evento ou elemento possuirá uma probabilidade, uma chance, de ocorrer.

        \subsection{Probabilidade Elemento}
            \paragraph{Definição}Seja $\omega_{i}$ um elemento amostral em $\Omega$, então a probabilidade do elemento $i$, $P(\omega_{i})$, ocorrer será:
                \begin{equation}
                    \boxed{
                        \sum_{i=1}^{n} P(\omega_{i}) = 1
                    }
                    \hspace{10mm}
                    \boxed{
                        0 \le P(\omega_{i}) \le 1
                        \hspace{2.5mm}
                        \begin{cases}
                            P(\omega_{i}) = 1, & \text{Elemento Certo};\\
                            P(\omega_{i}) = 0, & \text{Elemento Impossível};
                        \end{cases}
                    }
                    \end{equation}
            Onde:
                \begin{enumerate}[rightmargin = \leftmargin, noitemsep]
                    \item \textbf{Equiprobabilidade:} Todos os elementos do espaço amostral possuem a mesma chance de ocorrer, formalmente descrito de acordo com a seguinte equação:
                        \begin{equation}
                            \boxed{
                                P(\omega_{i}) = \frac{1}{n},
                                \hspace{5mm}
                                \forall i = 1, ..., n
                            }
                        \end{equation}
                \end{enumerate}

        \subsection{Probabilidade Evento}
            \paragraph{Definição}Seja $A = \{ \omega_{1}, ..., \omega_{m} \}$ um evento em $\Omega$ com $m \le n$ elementos amostrais, então a probabilidade do evento $A$, $P(A)$, ocorrer será:
                \begin{equation}
                    \boxed{
                        P(A) = \frac{m}{n}
                    }
                \end{equation}

        \subsection{Probabilidade União}
            \paragraph{Definição}Sejam $A$ e $B$ eventos de um espaço amostral $\Omega$, então a probabilidade da união dos eventos $A$ e $B$ ocorrer será:
                \begin{equation}
                    \boxed{
                        P(A \cup B) = P(A) + P(B) - P(A \cap B)
                    }
                \end{equation}

        \subsection{Probabilidade Condicional}
            \paragraph{Definição}Sejam $A$ e $B$ eventos, então a probabilidade de que $B$ ocorra dado que $A$ ocorreu será:
                \begin{equation}
                    \boxed{
                        P(B|A) = \frac{P(A\cap B)}{P(A)}
                    }
                \end{equation}
            Onde:
                \begin{enumerate}[rightmargin = \leftmargin, noitemsep]
                    \item \textbf{Independência:} Quando informação sobre $A$ não influência a probabilidade do evento $B$, implicando:
                        \begin{equation}
                            \boxed{
                                P(A\cap B) = P(A) \cdot P(B)
                            }
                        \end{equation} 
                \end{enumerate}

        \subsection{Teorema de Bayes}
            \paragraph{Definição}Seja $\{B_{1}, ... , B_{n}\}$ uma partição de eventos de $\Omega$ e $A$ mutuamente exclusivos cujo a união destes é $\Omega$, então a probabilidade de $A$ pode ser descrita pela seguinte expressão:
                \[
                    P(A) = \sum_{i=1}^{n} P(A\cap B_{i})
                \]
            Onde, aplicando probabilidade condicional, poderá ser reescrito como:
                \[
                    P(A) = \sum_{i=1}^{n} P(A|B_{i})\cdot P(B_{i})
                \]
            Assim, a probabilidade de um destes eventos da partição de $A$, $B_{i}$, ocorrer será dada pela seguinte equação:
                \begin{equation}
                    \boxed{
                        P(B_{i} | A) = \frac{P(A | B_{i}) \cdot P(B_{i})}{\sum\limits_{i=1}^{n} \left(P(A | B_{i}) \cdot P(B_{i})\right)}
                    }
                \end{equation}
\newpage

    \section{Variáveis Aleatórias Discretas}
        \paragraph{Definição}Experimento aleatório que resultada em uma quantidade numérica como resultado, ou seja, o evento em si não será prioridade. Caso uma função $X$ seja utilizada para relacionar os elementos do espaço amostral a um conjunto enumerável de pontos da reta real então está será uma \textbf{Variável Aleatória Discreta}. Assim defini-se a probabilidade desta variável ocorrer como:
            \begin{equation}
                \boxed{
                    \sum_{i=1}^{n} P(X = x_{i}) = 1
                }
                \qquad
                \boxed{
                    0 \le P(X = x_{i}) \le 1
                }
            \end{equation}

        \subsection{Função de Distribuição Acumulada}
            \paragraph{Definição}Representação do somatório dos valores da probabilidade de uma variável aleatória $X$ até um certo intervalos $\{x_{1}, ..., x_{n}\}$, como representado na seguinte equação:
                \begin{equation}
                    \boxed{
                        F(x) = P(X \le x), \quad x\in\mathbb{R}
                    }
                \end{equation}
            Pode-se representar essa variável, seguindo a seguinte notação:
                \begin{equation}
                    \begin{cases}
                        F(x_{1}) &= P(X = x_{1})\\
                        \vdots   & \vdots\\
                        F(x_{i}) &= P(X = x_{1}) + \cdots + P(X = x_{i})\\
                        \vdots   & \vdots\\
                        F(x_{n}) &= P(X = x_{1}) + \cdots + P(X = x_{n})\\
                    \end{cases}
                    \qquad
                    \boxed{
                        F(x) =
                        \begin{cases}
                            F(x_{1}), & x \le x_{1} \\
                            \vdots    & \\
                            F(x_{i}), & x_{i-1} \le x \le x_{i}\\
                            \vdots    & \\
                            F(x_{n}), & x_{n-1} \le x \le x_{n}\\
                        \end{cases}
                    }
                \end{equation}

        \subsection{Esperança}
            \paragraph{Definição}Conhecido em inglês como \textbf{Expected Value}, representa qual seria o valor ponderado entre as probabilidades de todos os possíveis valores de $X$, podendo resultar em um valor impossível. Formalmente representada pela seguinte equação:
                \begin{equation}
                    \boxed{
                        E(X) = \mu = 
                        \sum_{i=1}^{n} x_{i} \cdot P(X = x_{i})
                    }
                \end{equation}
            Obedecendo as seguintes propriedades:
                \begin{enumerate}[rightmargin = \leftmargin, noitemsep]
                    \item Se $X$ é uma variável aleatória e constantes $a$ e $b$, então:
                        \begin{equation}
                            \boxed{
                                E(a X + b) = a E(X) + b
                            }
                        \end{equation}
                    \item Se $X_{1}$, ..., $X_{n}$ são variáveis aleatórias:
                        \begin{equation}
                            \boxed{
                                E \left( \sum_{i=1}^{n} X_{i} \right) = \sum_{i=1}^{n} E(X_{i})
                            }
                        \end{equation}
                \end{enumerate}

        \subsection{Mediana}
            \paragraph{Definição}Representa o valor de uma variável aleatória que a divide em dois conjuntos de igual probabilidade de ocorre, ou seja, atende a seguinte equação:
                \begin{equation}
                    \boxed{
                        P(X\ge Md) \ge \frac{1}{2}
                    }
                    \quad
                    \text{e}
                    \quad
                    \boxed{
                        P(X\le Md) \ge \frac{1}{2}
                    }
                \end{equation}

        \subsection{Moda}
            \paragraph{Definição}Representa o valor da variável $X$ com maior probabilidade de ocorrer, podendo ser mais do que um. Formalmente descrito pela seguinte equação:
                \begin{equation}
                    \boxed{
                        P(X = Mo) = \max\{ p_{1}, ..., p_{n} \}
                    }
                \end{equation}

        \subsection{Variância}
            \paragraph{Definição}Quantificação do quão distantes os valores de uma variável aleatória $X$ estão de sua esperança, novamente uma métrica numérica que desconsidera o evento em si. Formalmente representada pela seguinte equação:
                \begin{equation}
                    \boxed{
                        V(x) = \sigma^{2} = 
                        E[(X - E(X))^{2}] =
                        E(X^{2}) - (E(X))^{2}
                    }
                \end{equation}
            Obedecendo as seguintes propriedades:
                \begin{enumerate}[rightmargin = \leftmargin, noitemsep]
                    \item Se $X$ é uma variável aleatória e constantes $a$ e $b$, então:
                        \begin{equation}
                            \boxed{
                                V(a X + b) = a^{2} V(X)
                            }
                        \end{equation}
                    \item Se $X_{1}$, ..., $X_{n}$ são variáveis aleatórias:
                        \begin{equation}
                            \boxed{
                                V \left( \sum_{i=1}^{n} X_{i} \right) = \sum_{i=1}^{n} V(X_{i})
                            }
                        \end{equation}
                \end{enumerate}

        \subsection{Variável Aleatória Discreta Uniforme}
            \paragraph{Definição}Quando uma variável aleatória discreta assume valores $x_{1}$, ... , $x_{k}$ possuam a mesma probabilidade de ocorrer, sendo descrita pela seguinte equação:
                \begin{equation}
                    \boxed{
                        P(X = x_{i}) = p(x_{i}) = \frac{1}{k},
                        \qquad
                        \forall 1 \le i \le k
                    }
                \end{equation}

        \subsection{Modelo de Bernoulli}
            \paragraph{Definição}Modelagem de experimentos aleatórios em que hajam apenas 2 resultados possíveis: \textbf{Sucesso} ou \textbf{Fracasso}. Denota-se assim, considerando que $p$ seja a probabilidade de sucesso,  a seguinte equação:
                \begin{equation}
                    P(X=x) = 
                    \begin{cases}
                        p,   & \text{\textbf{Sucesso} se $x=1$}\\
                        1-p, & \text{\textbf{Fracasso} se $x=0$}\\
                    \end{cases}
                    \qquad
                    \boxed{
                        P(X=x)=p^{x}(1 - p)^{1-x}
                    }
                \end{equation}
            Assim tem-se os seguintes resultados:
                \begin{equation}
                    \boxed{
                        E(X) = p
                    }
                    \qquad
                    \boxed{
                        V(X) = p(1 - p)
                    }
                \end{equation}

        \subsection{Modelo Binomial}
            \paragraph{Definição}Modelo de experimentos de repetição de $n$ ensaios $X_{i}$ de Bernoulli independentes com mesma probabilidade $p$ de sucesso. Assim a probabilidade de se observar $x$ será dada pela seguinte equação:
                \begin{equation}
                    \boxed{
                        P(X = x) = \binom{n}{x} p^{x}(1-p)^{n-x}
                    }
                \end{equation}
            Assim tem-se os seguintes resultados:
                \begin{equation}
                    \boxed{
                        E(X) = np
                    }
                    \qquad
                    \boxed{
                        V(X) = np(1-p)
                    }
                \end{equation}

        \subsection{Modelo Geométrico}
            \paragraph{Definição}Modelo de ensaios de Bernoulli independentes com mesma probabilidade $p$ de sucesso que sejam repetidos até que o primeiro sucesso seja obtido. Assim a probabilidade de se observar $x$ será dada pela seguinte equação:
                \begin{equation}
                    \boxed{
                        P(X = x) = (1-p)^{x-1}p
                    }
                \end{equation}
            Assim tem-se os seguintes resultados:
                \begin{equation}
                    \boxed{
                        E(X) = \frac{1}{p}
                    }
                    \qquad
                    \boxed{
                        V(X) = \frac{1-p}{p^{2}}
                    }
                \end{equation}

        \subsection{Modelo Hipergeométrico}
            \paragraph{Definição}Modelo em que a população de $N$ objetos é divida em duas características, $r$ tem a caraterística $A$ e $N-r$ tem a caraterística $B$, e realiza-se $n$ extrações sem reposição. Assim a probabilidade de que neste grupo com $n$ elementos possua $x$ elementos da característica $A$ será dada pela seguinte equação:
                \begin{equation}
                    \boxed{
                        P(X = x) = \frac{\binom{r}{x}\binom{N-r}{n-x}}{\binom{N}{n}}
                    }
                \end{equation}
            Assim tem-se os seguintes resultados:
                \begin{equation}
                    \boxed{
                        E(X) = \frac{nr}{N}
                    }
                    \qquad
                    \boxed{
                        V(X) = \frac{nr}{N} \left(1 - \frac{r}{N}\right)\frac{(N-n)}{(N-1)}
                    }
                \end{equation}

        \subsection{Modelo de Poisson}
            \paragraph{Definição}Aproximação para distribuição binomial apresenta $n\to\infty$ e $p\to 0$, geralmente considera-se os seguintes critérios, $np\le 7$ ou $n\ge 20$ e $p\le 0.05$. Assim a probabilidade desta aproximação será dada pela seguinte equação:
                \begin{equation}
                    \boxed{
                        P(X = x) = \binom{n}{x} p^{x}(1-p)^{n-x} \approx
                        \frac{e^{-np}(np)^{x}}{x!}
                    }
                \end{equation}
            Assim tem-se os seguintes resultados:
                \begin{equation}
                    \boxed{
                        E(X) = V(X) = np = \lambda
                    }
                \end{equation}
\newpage

    \section{Variáveis Aleatórias Contínuas}
        \paragraph{Definição}Experimento aleatório que resultada em uma quantidade numérica como resultado, ou seja, o evento em si não será prioridade. Caso uma função $X$ seja utilizada para relacionar os elementos do espaço amostral da reta real então está será uma \textbf{Variável Aleatória Contínua}. Assim defini-se a probabilidade desta variável ocorrer como:
            \begin{equation}
                \boxed{
                    \int_{-\infty}^{+\infty} f_{X}(x)dx = 1
                }
                \qquad
                \boxed{
                    P(a \le X \le b) = \int_{a}^{b} f_{X}(x) dx \ge 0, \quad \forall x\in\mathbb{R}
                }
            \end{equation}
        Onde:
            \begin{equation}
                \boxed{
                    P(a \le X \le b) =
                    P(a \le X < b) =
                    P(a < X \le b) =
                    P(a < X < b)
                }
            \end{equation}

        \subsection{Função de Distribuição Acumulada}
            \paragraph{Definição}Representação a integral da função de probabilidade de uma variável aleatória $X$ até um certo intervalos $\{x_{1}, ..., x_{n}\}$, como representado na seguinte equação:
                \begin{equation}
                    \boxed{
                        F_{X}(x) = P(X \le x) = 
                        \int_{-\infty}^{x} f_{X}(u)du
                    }
                \end{equation}

        \subsection{Esperança}
            \paragraph{Definição}Conhecido em inglês como \textbf{Expected Value}, representa qual seria o valor ponderado entre as probabilidades de todos os possíveis valores de $X$, podendo resultar em um valor impossível. Formalmente representada pela seguinte equação:
                \begin{equation}
                    \boxed{
                        E(X) = \mu = 
                        \int_{-\infty}^{+\infty} x f_{X}(x) dx
                    }
                \end{equation}
            Obedecendo as seguintes propriedades:
                \begin{enumerate}[rightmargin = \leftmargin, noitemsep]
                    \item Se $X$ é uma variável aleatória, então o $k$-ésimo momento será dado por:
                        \begin{equation}
                            \boxed{
                                E(X^{k}) = 
                                \int_{-\infty}^{+\infty} x^{k} f_{X}(x) dx
                            }
                        \end{equation}
                    \item Se $X$ é uma variável aleatória e constantes $a$ e $b$, então:
                        \begin{equation}
                            \boxed{
                                E(a X + b) = a E(X) + b
                            }
                        \end{equation}
                    \item Se $X_{1}$, ..., $X_{n}$ são variáveis aleatórias:
                        \begin{equation}
                            \boxed{
                                E \left( \sum_{i=1}^{n} X_{i} \right) = \sum_{i=1}^{n} E(X_{i})
                            }
                        \end{equation}
                \end{enumerate}

        \subsection{Variância}
            \paragraph{Definição}Quantificação do quão distantes os valores de uma variável aleatória $X$ estão de sua esperança, novamente uma métrica numérica que desconsidera o evento em si. Formalmente representada pela seguinte equação:
                \begin{equation}
                    \boxed{
                        V(X) = \sigma^{2} = 
                        E[(X - E(X))^{2}] =
                        E(X^{2}) - (E(X))^{2}
                    }
                \end{equation}
            Obedecendo as seguintes propriedades:
                \begin{enumerate}[rightmargin = \leftmargin, noitemsep]
                    \item Se $X$ é uma variável aleatória e constantes $a$ e $b$, então:
                        \begin{equation}
                            \boxed{
                                V(a X + b) = a^{2} V(X)
                            }
                        \end{equation}
                    \item Se $X_{1}$, ..., $X_{n}$ são variáveis aleatórias:
                        \begin{equation}
                            \boxed{
                                V \left( \sum_{i=1}^{n} X_{i} \right) = \sum_{i=1}^{n} V(X_{i})
                            }
                        \end{equation}
                \end{enumerate}

        \subsection{Variável Aleatória Contínua Uniforme}
            \paragraph{Definição}Quando uma variável aleatória contínua assume valores $x_{1}$, ... , $x_{k}$ possuam a mesma probabilidade de ocorrer, sendo descrita pela seguinte equação:
                \begin{equation}
                    \boxed{
                        f_{X}(x) = 
                        \begin{cases}
                            \frac{1}{b-a}, & a \le x \le b,\\
                            0,             & x < a \text{ ou } x > b\\
                        \end{cases}
                    }
                \end{equation}
            Assim tem-se os seguintes resultados:
                \begin{equation}
                    \boxed{
                        E(X) = \frac{b+a}{2}
                    }
                    \qquad
                    \boxed{
                        V(X) = \frac{(b-a)^{2}}{12}
                    }
                \end{equation}

        \subsection{Modelo Exponencial}
            \paragraph{Definição}Quando uma variável aleatória, com média $\lambda$, contínua assume uma distribuição exponencial dos valores, sendo descrita pela seguinte equação:
                \begin{equation}
                    \boxed{
                        f_{X}(x) = 
                        \begin{cases}
                            \lambda e^{-\lambda x}, & x \ge 0\\
                            0,                      & x < 0\\
                        \end{cases}
                    }
                    \qquad
                    \boxed{
                        F_{X}(x) = 
                        \begin{cases}
                            1 - e^{-\lambda x}, & x \ge 0\\
                            0,                  & x < 0\\
                        \end{cases}
                    }
                \end{equation}
            Assim tem-se os seguintes resultados:
                \begin{equation}
                    \boxed{
                        E(X) = \frac{1}{\lambda}
                    }
                    \qquad
                    \boxed{
                        V(X) = \frac{1}{\lambda^2}
                    }
                \end{equation}

        \subsection{Modelo Normal}
            \paragraph{Definição}Quando uma variável aleatória contínua assume uma distribuição normal dos valores, isto é, dependendo de sua média, $\mu$, e variância, $\sigma^{2}$, sendo descrita pela seguinte equação:
                \begin{equation}
                    \boxed{
                        f_{X}(x) = 
                        \frac{1}{\sqrt{2\pi \sigma^{2}}} \cdot 
                        e^{-\frac{(x -\mu)^{2}}{2\sigma^{2}}}
                    }
                \end{equation}
            Assim tem-se os seguintes resultados:
                \begin{equation}
                    \boxed{
                        E(X) = \mu
                    }
                    \qquad
                    \boxed{
                        V(X) = \sigma^{2}
                    }
                \end{equation}

        \subsection{Modelo Normal Padrão}
            \paragraph{Definição}Quando uma variável aleatória contínua assume uma distribuição normal padronizada dos valores, isto é, dependendo de sua média, $\mu$, e variância, $\sigma^{2}$, normalizadas, sendo descrita pela seguinte equação:
                \begin{equation}
                    \boxed{
                        Z(X) = \frac{X - \mu}{\sigma}
                    }
                    \qquad
                    \boxed{
                        f_{X}(z) = 
                        \frac{1}{\sqrt{2\pi}} \cdot 
                        e^{-\frac{z^{2}}{2}}
                    }
                \end{equation}
            Assim tem-se os seguintes resultados:
                \begin{equation}
                    \boxed{
                        E(X) = \mu
                    }
                    \qquad
                    \boxed{
                        V(X) = \sigma^{2}
                    }
                \end{equation}
\newpage

    \section{Variável Aleatória Amostral}
        \paragraph{Definição}Descrição do comportamento de uma população através da aproximação das características de uma amostra retirada aleatoriamente, podendo-se aplicar diferentes modelagens para sua descrição das quais as principais são listadas a seguir:
            \begin{enumerate}[rightmargin = \leftmargin]
                \item \textbf{Modelo Discreto:} Variáveis fixadas em valores pré-estabelecidos, possuindo:
                    \begin{enumerate}[noitemsep]
                        \item \texttt{Média Populacional:} $p$;
                        \item \texttt{Variância Populacional:} $p(1-p)$;
                    \end{enumerate}

                \item \textbf{Modelo Aleatório:} Variáveis aleatórias contidas em uma distribuição de Bernoulli, possuindo:
                    \begin{enumerate}[noitemsep]
                        \item \texttt{Média Populacional:} $p$;
                        \item \texttt{Variância Populacional:} $p(1-p)$;
                    \end{enumerate}
            \end{enumerate}
        Note que independente do modelo aplicado quando uma amostra $n$ for retirada as propriedades desta são obtidas através das seguintes equações:
            \begin{equation}
                \boxed{
                    E(\hat{p}) = p
                }
                \qquad
                \boxed{
                    V(\hat{p}) = \frac{p(1-p)}{n}
                }
            \end{equation}
        Assim, a medida que $n$ aumenta estas variáveis poderão ser aproximadas pela seguinte expressão:
            \begin{equation}
                \boxed{
                    \hat{p} \approx N\left(p, \frac{p(1-p)}{n}\right)
                }
            \end{equation}
\end{document}