\documentclass{article}
\usepackage{tpack}

\title{ES572 - Circuitos Lógicos}
\author{Guilherme Nunes Trofino\\217276}
\project{Resumo Teórico}

\newcommand{\myStyle}{
\lstset{
    language=Octave,                            % the language of the code
    basicstyle=\ttfamily\small,               % the size of the fonts that are used for the code
    keywordstyle=\color{darkpurple}\bfseries, %
    stringstyle=\color{darkblue},             %
    commentstyle=\color{darkgreen},           %
    morecomment=[s][\color{blue}]{/**}{*/},   %
    extendedchars=true,                       %
    showtabs=false,                           % show tabs within strings adding particular underscores
    showspaces=false,                         % show spaces adding particular underscores
    showstringspaces=false,                   % underline spaces within strings
    numbers=left,                             % where to put the line-numbers
    numberstyle=\tiny\color{gray},            % the style that is used for the line-numbers
    stepnumber=1,                             % the step between two line-numbers. If it's 1, each line will be numbered
    numbersep=5pt,                            % how far the line-numbers are from the code
    frame=single,                             % adds a frame around the code
    rulecolor=\color{black},                  % if not set, the frame-color may be changed on line-breaks within not-black text
    breaklines=true,                          % sets automatic line breaking
    backgroundcolor=\color{white},            % choose the background color
    breakatwhitespace=true,                   % sets if automatic breaks should only happen at whitespace
    breakautoindent=false,                    %
    captionpos=b,                             % sets the caption-position to bottom
    xleftmargin=0pt,                          %
    tabsize=2,                                % sets default tabsize to 2 spaces
}}

%\titleformat{<command>}[<shape>]{<format>}{<label>}{<sep>}{<before-code>}[<after-code>]
\titleformat
{\section} %comand
[block]  %shape
{\normalfont\LARGE} %format
{\thesection. } %label
{0mm} %sep
{} %before-code
[{\titlerule[0.1mm]}] %after-code

\titlespacing*{\section}{0mm}{0mm}{15mm}

\titleformat
{\subsection} %comand
[block]  %shape
{\normalfont\Large} %format
{\thesubsection. } %label
{0mm} %sep
{} %before-code
[] %after-code

\titlespacing*{\subsection}{0mm}{5mm}{2.5mm}


\newcommand{\nameSubject}{EA611 - Circuitos II}
\newcommand{\nameAuthor}{Guilherme Nunes Trofino}
\begin{document}
    \maketitle
\newpage

    \tableofcontents
    \addtocontents{toc}{\protect\thispagestyle{empty}}
\newpage


    \section{Introdução}
        \paragraph{Apresentação}Neste documento será descrito as informações necessárias para compreensão e solução de exercícios relacionados a disciplina \nameSubject. Note que este documento são notas realizadas por \nameAuthor, em \today.

        \subsection{Transformada de Laplace}
            \paragraph{Definição}Conversão de uma equação diferencial em equação algébrica e uma convolução em multiplicação. Formalmente descrita pelas seguintes equações:

            \begin{multicols}{2}
                \raggedcolumns
                \paragraph{Forma Bilateral:}
                    \begin{equation}
                        \boxed{
                            F(s) = \mathcal{B}\{ f(t) \} \coloneqq \int_{-\infty}^{+\infty} f(t) \; e^{-st} \; \text{d} t
                        }
                    \end{equation}

                \columnbreak

                \paragraph{Forma Unilateral:}
                    \begin{equation}
                        \boxed{
                            F(s) = \mathcal{L}\{ f(t) \} \coloneqq \int_{0}^{+\infty} f(t) \; e^{-st} \; \text{d} t
                        }
                    \end{equation}
            \end{multicols}\noindent
            Note que a forma \texttt{Unilateral} será um caso particular da \texttt{Bilateral}. Além disso, no estudo de circuitos elétricos será conveniente a adoção do domínio dos complexos para análise. Assim $s = \sigma + \omega\text{j}$ onde $\text{j}$ será a \textbf{Unidade Imaginária}, evitando confusão com \textbf{Corrente Elétrica} causada pela notação matemática canónica.

            \paragraph{Transformações}A seguir encontram-se as principais transformações pela definição \texttt{Unilateral} necessárias:

                \begin{table}[H]
                    \centering
                    \begingroup
                    % \setlength{\tabcolsep}{5mm}
                    \renewcommand{\arraystretch}{1.25}
                    \begin{tabular}[]{lcc}
                                         & $f(t)$      & $\mathcal{L}\{ f(t) \}$\\\hline
                        Degrau Unitário  & $u(t)$      & $\frac{1}{s}$\\
                        Impulso Unitário & $\delta(t)$ & $1$\\
                                         & $t^{n}$     & $\frac{n!}{s^{n+1}}$\\
                                         & $e^{-at}$   & $\frac{1}{s+a}$\\
                                         & $te^{-at}$  & $\frac{1}{(s+a)^{2}}$\\
                                         & $\sin(at)$  & $\frac{a}{(s^2+a^2)}$\\
                                         & $\cos(at)$  & $\frac{s}{(s^2+a^2)}$\\
                        Seno Hiperbólico    & $\sinh(at)$  & $\frac{a}{(s^2-a^2)}$\\
                        Cosseno Hiperbólico & $\cosh(at)$  & $\frac{s}{(s^2-a^2)}$\\
                                         & $e^{at}\;\sin(bt)$  & $\frac{b}{(s-a)^2+b^2}$\\
                                         & $e^{at}\;\cos(bt)$  & $\frac{s-a}{(s-a)^2+b^2}$\\
                        Convolução       & $\int_{0}^{t} f(\varphi)\;g(t - \varphi) \text{d}\varphi$ & $F(s)\cdot G(s)$\\
                        Integral         & $\int_{0}^{t} f(\varphi)\;u(t - \varphi) \text{d}\varphi$ & $\frac{F(s)}{s}$\\
                        Derivada         & $\diff{f(\varphi)}{\varphi} $ & $s\cdot F(s)$\\\hline
                    \end{tabular}
                    \endgroup
                    \caption{Tabela de Transformadas de Laplace}\label{table:Laplace}
                \end{table} \noindent
            Conside que as funções \textbf{Trigonométricas Hiperbólicas} são definidas pelas equações abaixo:
                \begin{equation}
                    \boxed{
                        \sinh(x) = \frac{e^{x} - e^{-x}}{2}
                    }
                    \qquad
                    \boxed{
                        \cosh(x) = \frac{e^{x} + e^{-x}}{2}
                    }
                \end{equation}
            \newpage

                    \begin{multicols}{2}
                        \raggedcolumns
                        \paragraph{Degrau Unitário}Representação de descontinuidade unitária. Formalmente descrita pela seguinte equação:
                        \begin{equation}
                            \boxed{
                                u(x - a) = 
                                \begin{cases}
                                    0, & x < a;\\
                                    \frac{1}{2}, & x = a;\\
                                    1, & x > a;\\
                                \end{cases}
                            }
                        \end{equation}

                        \columnbreak

                        \paragraph{Impulso Unitário}Distribuição infinita no ponto zero e nula no restante da reta. Formalmente descrita pela seguinte equação:
                            \begin{equation}
                                \boxed{
                                    \delta(x) = 
                                    \begin{cases}
                                        0, & x \neq 0;\\
                                        \infty, & x = 0;\\
                                    \end{cases}
                                    }
                                \end{equation}
                        Obedecendo:
                            \begin{equation*}
                                \int_{-\infty}^{+\infty} \delta(x) \; \text{d}x = 1
                            \end{equation*}
                \end{multicols}

        \subsection{Transformada de Componentes}
            \paragraph{Definição}Substituir as equações que descrevem cada componente empregado em um circuito através de seu equivalente em \textbf{Laplace} simplificará os cálculos como representado abaixo:
                \begin{multicols*}{2}
                    \raggedcolumns
                    \paragraph{Capacitor}Genericamente considera-se a seguinte equação para descrever o comportamento do componente:
                        \begin{equation*}
                            v_{C}(t) = \frac{1}{C}\;\int_{0}^{t} i_{C}(t)\;\text{d}t + v_{C}(0)
                        \end{equation*}
                    Aplica-se a \textbf{Transformada de Laplace}, obtendo a seguinte equação:
                        \begin{equation}
                            \boxed{
                                V_{C}(s) = \frac{1}{sC}\;I_{C}(s) + \frac{v_{C}(0)}{s}
                            }
                        \end{equation}

                    \paragraph{Indutor}Genericamente considera-se a seguinte equação para descrever o comportamento do componente:
                        \begin{equation*}
                            v_{L}(t) = L\;\diff{i_{L}(t)}{t}
                        \end{equation*}
                    Aplica-se a \textbf{Transformada de Laplace}, obtendo a seguinte equação:
                        \begin{equation}
                            \boxed{
                                V_{L}(s) = sL\;I_{L}(s) - L\;I_{L}(0)
                            }
                        \end{equation}

                    \paragraph{Resistor}Genericamente considera-se a seguinte equação para descrever o comportamento do componente:
                        \begin{equation*}
                            v_{R}(t) = R\;i_{R}(t)
                        \end{equation*}
                    Aplica-se a \textbf{Transformada de Laplace}, obtendo a seguinte equação:
                        \begin{equation}
                            \boxed{
                                V_{R}(s) = R\;I_{R}(s)
                            }
                        \end{equation}
                    \columnbreak
                    \paragraph{Função de Rede}aaaaa
                \end{multicols*}

\end{document}