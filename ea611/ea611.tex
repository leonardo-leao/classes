\documentclass{article}
\usepackage{tpack}

\title{EA611 - Circuitos II}
\author{Guilherme Nunes Trofino}
\authorRA{217276}
\project{Resumo Teórico}

\begin{document}
\maketitle
\newpage

\tableofcontents
\newpage


\section{Introdução}
    \paragraph{Apresentação}Neste documento será descrito as informações necessárias para compreensão e solução de exercícios relacionados a disciplina \title. Note que este documento são notas realizadas por \author, em \today.

    \subsection{Transformada de Laplace}
        \paragraph{Definição}Conversão de uma equação diferencial em equação algébrica e uma convolução em multiplicação. Formalmente descrita pelas seguintes equações:

        \begin{multicols}{2}
            \raggedcolumns
            \paragraph{Forma Bilateral:}
                \begin{equation}
                    \boxed{
                        F(s) = \mathcal{B} \{ f(t) \} := \int_{-\infty}^{+\infty} f(t) \; e^{-st} \; \text{d}t
                    }
                \end{equation}

            \columnbreak

            \paragraph{Forma Unilateral:}
                \begin{equation}
                    \boxed{
                        F(s) = \mathcal{L}\{ f(t) \} := \int_{0}^{+\infty} f(t) \; e^{-st} \; \text{d} t
                    }
                \end{equation}
        \end{multicols}\noindent
        Note que a forma \texttt{Unilateral} será um caso particular da \texttt{Bilateral}. Além disso, no estudo de circuitos elétricos será conveniente a adoção do domínio dos complexos para análise. Assim $s = \sigma + \omega\text{j}$ onde $\text{j}$ será a \textbf{Unidade Imaginária}, evitando confusão com \textbf{Corrente Elétrica} causada pela notação matemática canónica.

        \paragraph{Transformações}A seguir encontram-se as principais transformações pela definição \texttt{Unilateral} necessárias:
            \begin{table}[H]
                \centering
                \begingroup
                % \setlength{\tabcolsep}{5mm}
                \renewcommand{\arraystretch}{1.25}
                \begin{tabular}[]{lcc}
                                        & $f(t)$      & $\mathcal{L}\{ f(t) \}$\\\hline
                    Degrau Unitário  & $u(t)$      & $\frac{1}{s}$\\
                    Impulso Unitário & $\delta(t)$ & $1$\\
                                        & $t^{n}$     & $\frac{n!}{s^{n+1}}$\\
                                        & $e^{-at}$   & $\frac{1}{s+a}$\\
                                        & $\frac{t^{n-1}e^{-at}}{(n-1)!}$  & $\frac{1}{(s+a)^{n}}$\\
                                        & $\sin(at + b)$  & $\frac{s\sin(b) + a\cos(b)}{(s^2+a^2)}$\\
                                        & $\cos(at + b)$  & $\frac{s\cos(b) + a\sin(b)}{(s^2+a^2)}$\\
                    Seno Hiperbólico    & $\sinh(at)$  & $\frac{a}{(s^2-a^2)}$\\
                    Cosseno Hiperbólico & $\cosh(at)$  & $\frac{s}{(s^2-a^2)}$\\
                                        & $e^{at}\;\sin(bt)$  & $\frac{b}{(s-a)^2+b^2}$\\
                                        & $e^{at}\;\cos(bt)$  & $\frac{s-a}{(s-a)^2+b^2}$\\
                    Convolução       & $\int_{0}^{t} f(\varphi)\;g(t - \varphi) \text{d}\varphi$ & $F(s)\cdot G(s)$\\
                    Integral         & $\int_{0}^{t} f(\varphi)\;u(t - \varphi) \text{d}\varphi$ & $\frac{F(s)}{s}$\\
                    Derivada         & $\diff{f(\varphi)}{\varphi}$ & $s\cdot F(s)$\\
                    Frequência       & $e^{-at}f(t)$          & $F(s+a)$\\
                    Temporal         & $f(t-\tau)\mu(t-\tau)$ & $e^{-s\tau}F(s)$\\\hline
                \end{tabular}
                \endgroup
                \caption{Tabela de Transformadas de Laplace}\label{table:Laplace}
            \end{table} \noindent
        Conside que as funções \textbf{Trigonométricas Hiperbólicas} são definidas pelas equações abaixo:
            \begin{equation}
                \boxed{
                    \sinh(ax) = \frac{e^{ax} - e^{-ax}}{2}
                }
                \qquad
                \boxed{
                    \cosh(ax) = \frac{e^{ax} + e^{-ax}}{2}
                }
            \end{equation}
            % \begin{exercise}
            %     Calcule via Definição a transformada de Laplace:
            %         \begin{enumerate}[noitemsep, label={(\alph*)}]
            %             \item $f(t) = (2t + 3t^2)\;u(t)$ \footnote{$F(s) =  \frac{2}{s^2} + \frac{6}{s^3}$}
            %             \item $g(t) = \cosh(kt)\;u(t)$   \footnote{$F(s) =  \frac{s}{s^2 - k^2}$}
            %         \end{enumerate}
            % \end{exercise}

        \subsubsection{Degrau Unitário}
            \paragraph{Definição}Representação de descontinuidade unitária, normalmente utilizada para representar mudanças instantâneas em sistemas. Formalmente descrita pela seguinte equação:
                \begin{equation}
                    \boxed{
                        u(x - a) = 
                        \begin{cases}
                            0, & x < a;\\
                            \frac{1}{2}, & x = a;\\
                            1, & x > a;\\
                        \end{cases}
                    }
                \end{equation}

        \subsubsection{Impulso Unitário}
            \paragraph{Definição}Distribuição infinita no ponto zero e nula no restante da reta. Formalmente descrita pela seguinte equação:
                \begin{equation}
                    \boxed{
                        \delta(x) = 
                        \begin{cases}
                            0, & x \neq 0;\\
                            \infty, & x = 0;\\
                        \end{cases}
                        }
                    \end{equation}
            Obedecendo:
                \begin{equation*}
                    \int_{-\infty}^{+\infty} \delta(x) \; \text{d}x = 1
                    \quad\text{e}\quad
                    \boxed{
                        \int_{a}^{b} f(t) \delta(t - \tau)\;\text{d}t = 
                        \begin{cases}
                            f(\tau);    & \text{se } \tau\in[a,b]\\
                            0;          & \text{se } \tau\notin[a,b]\\
                        \end{cases}
                    }
                \end{equation*}
            Aplica-se o impulso unitário para se extrair uma \textbf{Amostra} do valor de uma função em um determinado ponto.

        \subsubsection{Transformada da Deriva}
            \paragraph{Definição}Quando aplicada em uma derivada de ordem $n$ será necessário utilizar da recursão e integração por partes, obtendo a seguinte equação geral:
                \begin{equation}
                    \boxed{
                        \mathcal{L}\left\{\diff[n]{f(\varphi)}{\varphi}\right\} = 
                        s^{n}\cdot F(s) - 
                        s^{n-1} \cdot f(0) - 
                        s^{n-2} \cdot f'(0) - \dots - 
                        s \cdot f^{n-2}(0) - 
                        f^{n-1}(0)
                    }
                \end{equation}

    \subsection{Transformada de Componentes}
        \paragraph{Definição}Substituir as equações que descrevem cada componente empregado em um circuito através de seu equivalente em \textbf{Laplace} simplificará os cálculos e poderá integrar suas condições iniciais na análise. Nesta transformação o circuito resultante será puramente resistivo e obedecerá às \textbf{Leis de Kirchhoff}.
            \begin{table}[H]
                \centering
                \begingroup
                % \setlength{\tabcolsep}{5mm}
                \renewcommand{\arraystretch}{1.25}
                \begin{tabular}[]{lll}
                                & Equação Geral                                                         & Equação Laplace\\\hline
                    Resistor  & $v_{R}(t) = R\;i_{R}(t)$                                              & $V_{R}(s) = R\;I_{R}(s)$\\[2.5mm]
                    Capacitor & $v_{C}(t) = \frac{1}{C}\;\int_{0}^{t} i_{C}(t)\;\text{d}t + v_{C}(0)$ & $V_{C}(s) = \frac{1}{sC}\;I_{C}(s) + \frac{v_{C}(0)}{s}$\\[2.5mm]
                    Indutor   & $v_{L}(t) = L\;\diff{i_{L}(t)}{t}$                                    & $V_{L}(s) = sL\;I_{L}(s) - L\;I_{L}(0)$\\[2.5mm]\hline
                \end{tabular}
                \endgroup
                \caption{Transformadas de Laplace de Componentes}\label{table:LaplaceComponents}
            \end{table} \noindent

    \subsection{Função de Rede}
        \paragraph{Definição}Simplificação dos circuitos de tal forma que análise seja facilitada pela utilização de suas entradas e de suas saídas sempre presupondo que as condições iniciais nulas obtidas pelas seguintes equações:
            \begin{equation}
                \boxed{H(s) = \frac{I(s)}{V(s)}}
            \end{equation}
        Onde:
            \begin{enumerate}[noitemsep]
                \item $V(s)$, \textbf{Entrada:} Tensão de Entrada;
                \item $I(s)$, \textbf{Saída:} Saída de Corrente;
            \end{enumerate}

\section{Circuitos Periódicos}
    \paragraph{Definição}Circuitos que são submetidos a sinais de tensão senoidais normalmente presentes em sistemas de potência elétrica com corrente alternada que será representada pela seguinte equação:
        \begin{equation}
            \boxed{v(t) = V_{M}\cos(\omega t + \varphi)}
        \end{equation}
    Onde:
        \begin{enumerate}
            \item \textbf{Amplitude:} $V_{M}$, Representa a \texttt{Tensão Máxima} do sinal;

            \item \textbf{Ciclo:} Características da equação:
                \begin{enumerate}[noitemsep, rightmargin = \leftmargin]
                    \item \texttt{Frequência:} $\omega$, Representa a quantidade de \texttt{Oscilações} por intervalo de tempo;

                    \item \texttt{Período:} $T$, Representa o tempo para realizar uma \texttt{Oscilação} do sinal;
                \end{enumerate}

            \item \textbf{Fase:} Indica a defasagem do sinal representada por $\varphi$;
        \end{enumerate}

    \subsection{Fasores}
        \paragraph{Definição}Conversão de equações periódicas em \textbf{Equações Fasoriais}, isto é, equações que envolvam números complexos para representar um comportamento periódico como descrito pela seguinte equação:
            \begin{figure}[H]
                \centering
                \begin{tikzpicture}
                    \begin{axis}[
                        xmin = 0, xmax = 5.25, % Axis Coordenates
                        ymin = 0, ymax = 5.25, % Axis Coordenates
                        xlabel = {$\mathbb{R}$},  % Axis Labels
                        ylabel = {$\mathbb{I}$},  % Axis Labels
                        xtick = {4.25},  % Axis Ticks Potions
                        ytick = {4.25},  % Axis Ticks Potions
                        xticklabels = {$V_{R}$}, % Axis Ticks Labels
                        yticklabels = {$V_{I}$}, % Axis Ticks Labels
                        x label style = {at={(axis cs:{5.5,0})}, anchor=west},
                        y label style = {at={(axis cs:{0,5.5})}, anchor=south, rotate=-90},
                        width  = 5.0cm,
                        height = 5.0cm,
                        axis lines = left,
                    ]
                    \draw [red!60, -{Latex[round]}] (1.5,0) arc [
                        start angle = 0,
                        end angle   = 45,
                        radius      = 1.5
                    ] node [midway, right] {$\theta^{\circ}$};
            
                    \draw[-{Latex[round]}] (axis cs:{0,0}) -- (axis cs:{4.25,4.25}) node [above right] {$V_{M}$};
            
                    \draw[gray, dashed] (axis cs:{4.25,0}) -- (axis cs:{4.25,4.25});
                    \draw[gray, dashed] (axis cs:{0,4.25}) -- (axis cs:{4.25,4.25});
                    \end{axis}
                \end{tikzpicture}
                \caption{Represenação Fasores}
            \end{figure}\noindent
            \begin{equation}
                \boxed{
                    V(t) := V_{M}\phase{\theta^{\circ}}
                }
            \end{equation}
        Onde:
            \begin{enumerate}
                \item \textbf{Módulo:} $V_{M}$, Representa a \texttt{Tensão Máxima} do sinal obtido pela seguinte equação:
                    \begin{equation*}
                        \boxed{
                            V_{M} = \sqrt{
                                V_{\text{I}}^{2} + V_{\text{R}}^{2}
                            }
                        }
                    \end{equation*}
                Onde:
                    \begin{enumerate}[noitemsep, rightmargin = \leftmargin]
                        \item \texttt{Parcela Imaginária:} $V_{\text{I}}$;

                        \item \texttt{Parcela Real:} $V_{\text{R}}$;
                    \end{enumerate}


                \item \textbf{Ciclo:} Características da equação:
                    \begin{enumerate}[noitemsep, rightmargin = \leftmargin]
                        \item \texttt{Frequência:} $\omega$, Constante a todos os componentes do circuito;
                    \end{enumerate}

                \item \textbf{Fase:} Indica a defasagem do sinal representada por $\theta$ obtido pela seguinte equação:
                    \begin{equation*}
                        \boxed{
                            \theta^{\circ} = \tan^{-1}
                            \left(
                                \frac{V_{\text{I}}}{V_{\text{R}}}
                            \right)
                        }
                        \quad
                        \text{onde}
                        \quad
                        \begin{cases}
                            \theta < 0, & \text{\texttt{Atrasado}, Fase Capacitiva}\\
                            \theta > 0, & \text{\texttt{Adiantado}, Fase Indutiva}
                        \end{cases}
                    \end{equation*}
            \end{enumerate}

        \begin{multicols}{2}
            \raggedcolumns
            \subsubsection{Multiplicação}
                \paragraph{Definição}Seja $V_{1}(t) = V_{1} \phase{\theta_{1}^{\circ}}$ e $V_{2}(t) = V_{2} \phase{\theta_{2}^{\circ}}$ então a multiplicação será obtida pela seguinte equação:
                    \begin{equation}
                        \boxed{
                            V_{1}(t) \cdot V_{2}(t) = 
                            (V_{1} \cdot V_{2})\phase{(\theta_{1} + \theta_{2})^{\circ}}
                        }
                    \end{equation}

            \columnbreak

            \subsubsection{Divisão}
                \paragraph{Definição}Seja $V_{1}(t) = V_{1} \phase{\theta_{1}^{\circ}}$ e $V_{2}(t) = V_{2} \phase{\theta_{2}^{\circ}}$ então a multiplicação será obtida pela seguinte equação:
                    \begin{equation}
                        \boxed{
                            \frac{V_{1}(t)}{V_{2}(t)} = 
                            \frac{V_{1}}{V_{2}} \phase{(\theta_{1} - \theta_{2})^{\circ}}
                        }
                    \end{equation}
        \end{multicols}

    \subsection{Transformada de Componentes}
        \paragraph{Definição}Substituir as equações que descrevem cada componente empregada em um \textbf{Circuito Periódico} através apenas de sua amplitude e sua fase. Nesta transformação o circuito resultante será puramente resistivo com componentes reais e imaginárias e obedecerá às \textbf{Leis de Kirchhoff}.
            \begin{table}[H]
                \centering
                \begingroup
                % \setlength{\tabcolsep}{5mm}
                \renewcommand{\arraystretch}{1.25}
                \begin{tabular}[]{lll}
                                & Equação Periódica                  & Equação Fasorial\\\hline
                    Resistor  & $v_{R}(t) = R\;i_{R}(t)$           & $V_{R}(t) = R\;I_{R}(t)$\\[2.5mm]
                    Capacitor & $i_{C}(t) = C\;\diff{v_{C}(t)}{t}$ & $V_{C}(t) = \frac{1}{j\omega C}\;I_{C}(t)$\\[2.5mm]
                    Indutor   & $v_{L}(t) = L\;\diff{i_{L}(t)}{t}$ & $V_{L}(s) = j\omega L\;I_{L}(t)$\\[2.5mm]\hline
                \end{tabular}
                \endgroup
                \caption{Transformadas de Fasorial de Componentes}\label{table:FasorialComponents}
            \end{table} \noindent

    \subsection{Potência}
        \paragraph{Definição}Seja um sistema com uma fonte $v(t) = V_{M}\cos(\omega t)$ e $i(t) = I_{M}\cos(\omega t)$ então a potência instantânea será dada pela seguinte equação:
            \begin{equation}
                p(t) = 
                R\;i(t)^{2} = 
                R\;I_{M}^{2}\cos^{2}(\omega t) = 
                \boxed{
                    \frac{R I_{M}^{2}}{2}(1 + \cos(2\omega t))
                }
            \end{equation}
        Nota-se que a potência instantânea oscila com o dobro da frequência em torno de um valor constante.

        \subsubsection{Potência Média}
            \paragraph{Definição}Potência fornecida durante um ciclo $T$ de oscilação da equação periódica obtida pela seguinte equação:
                \begin{equation}
                    \bar{p} = 
                    \frac{1}{T} \int_{t}^{t+T} p(t)\;\text{d}t = 
                    \boxed{
                        \frac{V_{M}\;I_{M}}{2} \cos(\theta)
                    }
                \end{equation}
            Onde:
                \begin{enumerate}
                    \item \textbf{Fase da Carga:} $\theta$, Representando o impedância geral pelo componente analísado:
                        \begin{equation*}
                            \theta = 
                            \begin{cases}
                                -90^{\circ}, & \text{se carga Capacitiva}\\
                                    0^{\circ}, & \text{se carga Resistiva}\\
                                +90^{\circ}, & \text{se carga Indutiva}\\
                            \end{cases}
                        \end{equation*}
                \end{enumerate}
            Desta forma define-se como valores eficazes de sinais periódicas os valores necessários em sinais contínuos para que haja a mesma entrega de potência média em um resistor obtidos pela seguinte equação:
                \begin{equation}
                    \boxed{V_{\text{ef}} = \frac{V_{M}}{\sqrt{2}}}
                    \qquad
                    \boxed{I_{\text{ef}} = \frac{I_{M}}{\sqrt{2}}}
                \end{equation}

        \subsubsection{Potência Complexa}
            \paragraph{Definição}Potência total consumida por um componente qualquer em um circuito fasorial poderá ser complexa e poderá ser obtida pela seguinte equação:
                \begin{equation}
                    \boxed{
                        S = V_{\text{ef}} \cdot \bar{I}_{\text{ef}} = P + jQ
                    }
                \end{equation}
            Onde:
                \begin{enumerate}
                    \item \textbf{Potência Ativa:} $P$, parte real da potência;
                        \begin{enumerate}[noitemsep]
                            \item \texttt{Carga Resistiva:} Apresenta apenas potência real;
                        \end{enumerate}

                    \item \textbf{Potência Reativa:} $Q$, parte complexa da potência;
                        \begin{enumerate}[noitemsep]
                            \item \texttt{Carga Capacitiva:} Apresenta apenas potência imaginária negaiva;
                            \item \texttt{Carga Indutiva:} Apresenta apenas potência imaginária positiva;
                        \end{enumerate}
                \end{enumerate}

        \subsubsection{Potência Aparente}
            \paragraph{Definição}Considera-se que o produto entre os valores eficazes de corrente e tensão representa o \textbf{Potência Aparente} sobre aquele componente como mostrado na equação a seguir:
                \begin{equation}
                    \boxed{p_{\text{ap}} = V_{\text{ef}}\;I_{\text{ef}}}
                \end{equation}

        \subsubsection{Fator de Potência}
            \paragraph{Definição}Relação entre a potência média e a potência aparente como mostrado na equação a seguir:
                \begin{equation}
                    \boxed{
                        f_{p} = \frac{\bar{p}}{V_{\text{ef}}\;I_{\text{ef}}} = \cos(\theta)
                    }
                \end{equation}
            Legislação brasileira exige que as cargas nas indústrias tenham um fator de potência mínimo para que atender as demandas de potência elétrica. Desta forma, pode ser necessário ajustar a impedância da carga $Z = R + jI$ com a inserção de uma carga paralela $Z_{i} = j I_{i}$ como mostrado pela seguinte equação:
                \begin{equation}
                    \boxed{
                        I_{i} = \frac{R^2 + I^2}{R\;\tan(\cos^{-1}(f_{p})) - I}
                    }
                    \quad
                    \text{onde}
                    \quad
                    \begin{cases}
                        I_{i} < 0, & \text{Carga Capacitiva}\\
                        I_{i} > 0, & \text{Carga Indutiva}
                    \end{cases}
                \end{equation}
                \begin{example}
                    Seja $Z = 100 + j100$ com $\omega = 100$ Hz e deseja-se um \textbf{Fator de Potência} $f_{p} = 0.95$. Primeiramente tem-se:
                        \begin{align*}
                            Z = 100 + j100 &= 141.4\phase{45^{\circ}}\\
                            \Aboxed{
                                f_{p} &= \cos(45^{\circ}) = 0.707
                            } & \text{Fator de Potência Inicial}
                        \end{align*}
                    Nota-se que $0.707 < 0.95$ logo será necessário inserir uma carga paralela:
                        \begin{equation*}
                            I_{i} = \frac{100^2 + 100^2}{100\cdot\tan(\cos^{-1}(0.95)) - 100} = -297.92 \Omega
                        \end{equation*}
                    Nota-se que $I_{i} < 0$ trata-se de uma carga Capacitiva desta forma, tem-se: 
                        \begin{equation*}
                            I_{i} = -\frac{1}{\omega\;C}
                            \qquad
                            \boxed{
                                C = - \frac{1}{\omega\;I_{i}} = 33,6 \mu F
                            }
                        \end{equation*}
                \end{example}

\section{Circuitos Trifásicos}
    \begin{definition}
        Utilização de estruturas com três fases para produção e transmissão de energia elétrica, contornando assim as vibrações causadas pela natureza oscilatória dos sinais.
    \end{definition}

    \subsection{Sistemas Monofásicos}
        \begin{definition}
            Dispositivos alimentados por fontes de tensão com a mesma fase usualmente representados pela seguinte figura:
                \begin{figure}[H]
                    \centering
                    \begin{circuitikz}
                        \ctikzset{component text=left}[american]
                        \draw
                        (-2,-2) coordinate (pb)
                                to[sV=$v_{BN}$]++(0,2) coordinate (nn)
                                to[sV=$v_{AN}$]++(0,2) coordinate (pa)
                                
                        (pa)node[left] {$a$}
                        (nn)node[left] {$n$}
                        (pb)node[left] {$b$}
                
                        ( 2,-2) coordinate (pB)
                                to[R=$Z_{A}$] ++(0,2) coordinate (nN)
                                to[R=$Z_{B}$] ++(0,2) coordinate (pA)
                
                        (pA)node[right] {$A$}
                        (nN)node[right] {$N$}
                        (pB)node[right] {$B$}
                
                        (pa)to[short, f=$I_{aA}$] (pA)
                        (nn)to[short, f=$I_{nN}$, *-*] (nN)
                        (pb)to[short, f=$I_{bB}$] (pB);
                    \end{circuitikz} 
                    \caption{Representação Circuito Monofásico}
                \end{figure} \noindent
            Onde:
                \begin{equation}
                    \boxed{v_{AN}(t)=V_{A}\phase{\omega^{\circ}}\text{ V}_{ef}}
                    \qquad
                    \boxed{v_{BN}(t)=V_{B}\phase{\omega^{\circ}}\text{ V}_{ef}}
                \end{equation}
            Nota-se que, usualmente, as tensões apresentadas serão eficazes.
        \end{definition}\noindent
        \begin{theorem}
            Quando estes sistemas estiverem conectados a cargas de mesma impedância considera-se que estas são \textbf{Carga Equilibradas} como representado abaixo:
                \begin{figure}[H]
                    \centering
                    \begin{circuitikz}
                        \ctikzset{component text=left}[american]
                        \draw
                        (-2,-2) coordinate (pb)
                                to[sV=$v_{BN}$]++(0,2) coordinate (nn)
                                to[sV=$v_{AN}$]++(0,2) coordinate (pa)
                                
                        (pa)node[left] {$a$}
                        (nn)node[left] {$n$}
                        (pb)node[left] {$b$}
                
                        ( 2,-2) coordinate (pB)
                                to[R=$Z$] ++(0,2) coordinate (nN)
                                to[R=$Z$] ++(0,2) coordinate (pA)
                
                        (pA)node[right] {$A$}
                        (nN)node[right] {$N$}
                        (pB)node[right] {$B$}
                
                        (pa)to[short, f=$I_{aA}$] (pA)
                        (nn)to[short, f=$I_{nN}$, *-*] (nN)
                        (pb)to[short, f=$I_{bB}$] (pB);
                    \end{circuitikz}
                    \caption{Representação Circuito Monofásico Equilibrado}
                \end{figure} \noindent
            Neste caso, não haverá corrente no ramo neutro, implicando $\boxed{I_{nN} = 0}$.
        \end{theorem}

    \subsection{Sistemas Trifásicos}
        \begin{definition}
            Dispositivos alimentados por fontes de tensão com $120^{\circ}$ de variação entre fases consecutivas usualmente rerpresentados pela seguinte figura:
                \begin{figure}[H]
                    \centering
                    \begin{circuitikz}
                        \ctikzset{component text=left}[american]
                        \draw
                
                        (0, 0)  coordinate (neutro)
                
                        (neutro)to[sV_=$v_{AN}$] (  90:2) coordinate (phaseA)
                        (neutro)to[sV_=$v_{BN}$] ( -30:2) coordinate (phaseB)
                        (neutro)to[sV_=$v_{CN}$] (-150:2) coordinate (phaseC)
                
                        (neutro)to[short, *-o] ++(2.5,0)
                                node[right] {$N$}
                
                        (phaseA)to[short, -o] ++(3,0)
                                node[right] {$A$}
                
                        (phaseB)to[short, -o] ++(1.25,0)
                                node[right] {$B$}
                
                        (phaseC)to[short] ++(0,-1)
                                to[short, -o] ++(4.75,0)
                                node[right] {$C$};
                    \end{circuitikz}
                    \caption{Representação Circuito Trifásico}
                \end{figure} \noindent
            Onde:
                \begin{equation}
                    \boxed{v_{AN}(t) = V_{P}\phase{   0^{\circ}}\text{ V}_{ef}}
                    \qquad
                    \boxed{v_{BN}(t) = V_{P}\phase{-120^{\circ}}\text{ V}_{ef}}
                    \qquad
                    \boxed{v_{CN}(t) = V_{P}\phase{+120^{\circ}}\text{ V}_{ef}}
                \end{equation}
            Nota-se que, usualmente, as tensões apresentadas serão eficazes.
        \end{definition}

    \subsubsection{Tensão de Linha}
        \begin{definition}
            Considera-se que o primeiro nó como \textbf{referência} da Tensão de Linha zero, o segundo nó como fase \textbf{menos} $120^{\circ}$ e o terceiro nó como fase \textbf{mais} $120^{\circ}$ como representado no seguinte diagrama fasorial:
                \begin{figure}[H]
                    \centering
                    \begin{tikzpicture}
                        \begin{axis}[
                            xmin = -8.25, xmax = 8.25, % Axis Coordenates
                            ymin = -8.25, ymax = 8.25, % Axis Coordenates
                            xlabel = {$\mathbb{R}$},  % Axis Labels
                            ylabel = {$\mathbb{I}$},  % Axis Labels
                            ticks = none,
                            xtick = {},  % Axis Ticks Potions
                            ytick = {},  % Axis Ticks Potions
                            xticklabels = {}, % Axis Ticks Labels
                            yticklabels = {}, % Axis Ticks Labels
                            x label style = {at={(axis cs:{8,0})}, anchor=east},
                            y label style = {at={(axis cs:{0,8})}, anchor=north, rotate=-90},
                            width  = 8.0cm,
                            height = 8.0cm,
                            axis lines = center,
                        ]
                        \draw [black, -{Latex[round]}] (1.5,0) arc [
                            start angle = 0,
                            end angle   = 120,
                            radius      = 1.5
                        ];
                        \node[text width = 1cm] at (2,2) {$+120^{\circ}$};
                
                        \draw [black, -{Latex[round]}] (1.5,0) arc [
                            start angle = 0,
                            end angle   = -120,
                            radius      = 1.5
                        ];
                        \node[text width = 1cm] at (2,-2) {$-120^{\circ}$};
                
                        \draw[blue!60, -{Latex[round]}] (0,0) -- (   0:6) node [below]      {$v_{AN}$};
                        \draw[blue!60, -{Latex[round]}] (0,0) -- (-120:6) node [below left] {$v_{BN}$};
                        \draw[blue!60, -{Latex[round]}] (0,0) -- (+120:6) node [above left] {$v_{CN}$};
                        \end{axis}
                    \end{tikzpicture}
                    \caption{Represenação Fasorial Tensões de Linha}
                \end{figure}\noindent
            Considera-se que a alimentação estará \textbf{Equilibrada} se todas as tensões de linha apresentarem o mesmo pico.
        \end{definition}

    \subsubsection{Tensão entre Fases}
        \begin{definition}
            Representa a diferença entre as Tensões de Fases representado geometricamente pelo seguinte diagrama fasorial:
                \begin{figure}[H]
                    \centering
                    \begin{tikzpicture}
                        \begin{axis}[
                            xmin = -8.25, xmax = 8.25, % Axis Coordenates
                            ymin = -8.25, ymax = 8.25, % Axis Coordenates
                            xlabel = {$\mathbb{R}$},  % Axis Labels
                            ylabel = {$\mathbb{I}$},  % Axis Labels
                            ticks = none,
                            xtick = {},  % Axis Ticks Potions
                            ytick = {},  % Axis Ticks Potions
                            xticklabels = {}, % Axis Ticks Labels
                            yticklabels = {}, % Axis Ticks Labels
                            x label style = {at={(axis cs:{8,0})}, anchor=east},
                            y label style = {at={(axis cs:{0,8})}, anchor=north, rotate=-90},
                            width  = 8.0cm,
                            height = 8.0cm,
                            axis lines = center,
                        ]
                        \draw [black, -{Latex[round]}] (1.5,0) arc [
                            start angle = 0,
                            end angle   = 30,
                            radius      = 1.5
                        ];
                        \node[text width = 1cm] at (3.25,0.75) {$30^{\circ}$};
                
                        \draw[gray, -{Latex[round]}] (0,0) -- (   0:4) node [black, below]      {$v_{an}$};
                        \draw[gray, -{Latex[round]}] (0,0) -- (-120:4) node [black, below left] {$v_{bn}$};
                        \draw[gray, -{Latex[round]}] (0,0) -- (+120:4) node [black, above left] {$v_{cn}$};
                
                        \draw[blue!60, -{Latex[round]}] (0,0) -- (+ 30:6.93) node [black, above right]{$v_{ab}$};
                        \draw[blue!60, -{Latex[round]}] (0,0) -- (- 90:6.93) node [black, below left] {$v_{bc}$};
                        \draw[blue!60, -{Latex[round]}] (0,0) -- (+150:6.93) node [black, above left] {$v_{ca}$};
                        \end{axis}
                    \end{tikzpicture}
                    \caption{Represenação Fasorial Tensões entre Fases}
                \end{figure}\noindent
            Onde:
                \begin{equation}
                    \boxed{v_{ab}(t) = \sqrt{3}V_{P}\phase{  30^{\circ}}\text{ V}_{ef}}
                    \qquad
                    \boxed{v_{bc}(t) = \sqrt{3}V_{P}\phase{- 90^{\circ}}\text{ V}_{ef}}
                    \qquad
                    \boxed{v_{ca}(t) = \sqrt{3}V_{P}\phase{+150^{\circ}}\text{ V}_{ef}}
                \end{equation}
            Convenciona-se que os subescritos representam a ordem da diferença de tensão. Logo $v_{ab} = v_{an} - v_{bn}$. Nota-se, também, que, usualmente, as tensões apresentadas serão eficazes.
        \end{definition}

    \subsubsection{Conexão Y-Y}
        \begin{definition}
            Organização do sistema em que há ponto neutro entre as cargas como representado pela seguinte figura:
                \begin{figure}[H]
                    \centering
                    \begin{circuitikz}
                        \ctikzset{component text=left}[american]
                        \draw
                        (-3, 0)  coordinate (nn)
                
                        (nn)to[sV_=$v_{AN}$] ++(  90:2) coordinate (pa)
                        (nn)to[sV_=$v_{BN}$] ++( -30:2)
                            to[short]        ++(0,-2)   coordinate (pb)
                        (nn)to[sV_=$v_{CN}$] ++(-150:2)
                            to[short]        ++(0,-1)   coordinate (pc)
                
                        (nn)node[below] {$n$}
                        (pa)node[left] {$a$}
                        (pb)node[below] {$b$}
                        (pc)node[left] {$c$}
                
                
                        ( 3, 0)  coordinate (nN)
                
                        (nN)to[R, l=$Z_{A}$] ++(  90:2) coordinate (pA)
                        (nN)to[R, l=$Z_{B}$] ++( -30:2)
                            to[short]    ++(0,-2)   coordinate (pB)
                        (nN)to[R, l=$Z_{C}$] ++(-150:2)
                            to[short]    ++(0,-1)   coordinate (pC)
                
                        (nN)node[below] {$N$}
                        (pA)node[above] {$A$}
                        (pB)node[below] {$B$}
                        (pC)node[right] {$C$}
                
                        (nn)to[short, *-*, f=$I_{nN}$] (nN)
                        (pa)to[short, f=$I_{aA}$] (pA)
                        (pb)to[short, f_=$I_{bB}$] (pB)
                        (pc)to[short, f_=$I_{cC}$] (pC)
                        ;
                    \end{circuitikz}
                    \caption{Representação Conexão Y-Y}
                \end{figure}
        \end{definition}
        \begin{theorem}
            Quando estes sistemas estiverem conectados a cargas de mesma impedância considera-se que estas são \textbf{Cargas Equilibradas} como representado abaixo:
                \begin{figure}[H]
                    \centering
                    \begin{circuitikz}
                        \ctikzset{component text=left}[american]
                        \draw
                        (-3, 0)  coordinate (nn)
                
                        (nn)to[sV_=$v_{AN}$] ++(  90:2) coordinate (pa)
                        (nn)to[sV_=$v_{BN}$] ++( -30:2)
                            to[short]        ++(0,-2)   coordinate (pb)
                        (nn)to[sV_=$v_{CN}$] ++(-150:2)
                            to[short]        ++(0,-1)   coordinate (pc)
                
                        (nn)node[below] {$n$}
                        (pa)node[left] {$a$}
                        (pb)node[below] {$b$}
                        (pc)node[left] {$c$}
                
                
                        ( 3, 0)  coordinate (nN)
                
                        (nN)to[R, l=$Z$] ++(  90:2) coordinate (pA)
                        (nN)to[R, l=$Z$] ++( -30:2)
                            to[short]    ++(0,-2)   coordinate (pB)
                        (nN)to[R, l=$Z$] ++(-150:2)
                            to[short]    ++(0,-1)   coordinate (pC)
                
                        (nN)node[below] {$N$}
                        (pA)node[above] {$A$}
                        (pB)node[below] {$B$}
                        (pC)node[right] {$C$}
                
                        (nn)to[short, *-*, f=$I_{nN}$] (nN)
                        (pa)to[short, f=$I_{aA}$] (pA)
                        (pb)to[short, f_=$I_{bB}$] (pB)
                        (pc)to[short, f_=$I_{cC}$] (pC)
                        ;
                    \end{circuitikz}
                    \caption{Representação Conexão Y-Y}
                \end{figure}
            Neste caso, não haverá corrente no ramo neutro, implicando $\boxed{I_{nN} = 0}$. Demais correntes podem ser obtidas através das seguintes equações:
                \begin{equation}
                    \boxed{I_{aA} = \frac{v_{an}}{|Z|}\phase{-\theta^{\circ}}\text{ A}_{ef}}
                    \quad
                    \boxed{I_{bB} = \frac{v_{bn}}{|Z|}\phase{(-120-\theta)^{\circ}}\text{ A}_{ef}}
                    \quad
                    \boxed{I_{cC} = \frac{v_{cn}}{|Z|}\phase{(+120-\theta)^{\circ}}\text{ A}_{ef}}
                \end{equation}
            Onde:
                \begin{equation*}
                    \boxed{Z = |Z|\phase{\theta^{\circ}}}
                \end{equation*}
        \end{theorem}
        \begin{definition}
            Nesta configuração a potência média entregue por cada fase será obtida pela seguinte equação:
                \begin{equation}
                    \boxed{P_{p} = V_{p}I_{p}\cos(\theta)}
                \end{equation}
            Desta forma, potência total entregue pelo sistema trifásico será obtida pela seguinte equação:
                \begin{equation}
                    \boxed{P = 3P_{p}}
                \end{equation}
            Nota-se que as componentes oscilatórias se anulam, gerando uma potência total constante.
        \end{definition}
\newpage

    \subsubsection{Conexão Y-$\Delta$}
        \begin{definition}
            Organização do sistema em que não há ponto neutro entre as cargas como representado pela seguinte figura:
                \begin{figure}[H]
                    \centering
                    \begin{circuitikz}
                        \ctikzset{component text=left}[american]
                        \draw
                        (-3, 0)  coordinate (pn)
                
                        (pn)to[sV_=$v_{AN}$] ++(  90:2) coordinate (pa)
                        (pn)to[sV_=$v_{BN}$] ++( -30:2) coordinate (pb)
                        (pn)to[sV_=$v_{CN}$] ++(-150:2)
                            to[short]        ++(0,-1)   coordinate (pc)
                
                        (pn)node[below] {$n$}
                        (pa)node[left] {$a$}
                        (pb)node[below left] {$b$}
                        (pc)node[left] {$c$}
                
                
                        ( 3, 2) coordinate (pN)
                                to[R, l=$Z_{C}$] ++( -60:3.4641) coordinate (pC)
                                to[R, l=$Z_{B}$] ++(-180:3.4641) coordinate (pB)
                                to[R, l=$Z_{A}$] ++( +60:3.4641) coordinate (pA)
                        (pC)to[short] ++(0,-1) coordinate (pC)
                
                        (pA)node[above right] {$A$}
                        (pB)node[below right] {$B$}
                        (pC)node[right] {$C$}
                
                        (pa)to[short, f=$I_{aA}$] (pA)
                        (pb)to[short, f_=$I_{bB}$] (pB)
                        (pc)to[short, f_=$I_{cC}$] (pC)
                        ;
                    \end{circuitikz}
                    \caption{Representação Conexão Y-D}
                \end{figure}
        \end{definition}
        \begin{theorem}
            Quando estes sistemas estiverem conectados a cargas de mesma impedância considera-se que estas são \textbf{Cargas Equilibradas} como representado abaixo:
                \begin{figure}[H]
                    \centering
                    \begin{circuitikz}
                        \ctikzset{component text=left}[american]
                        \draw
                        (-3, 0)  coordinate (pn)
                
                        (pn)to[sV_=$v_{an}$] ++(  90:2) coordinate (pa)
                        (pn)to[sV_=$v_{bn}$] ++( -30:2) coordinate (pb)
                        (pn)to[sV_=$v_{cn}$] ++(-150:2)
                            to[short]        ++(0,-1)   coordinate (pc)
                
                        (pn)node[below] {$n$}
                        (pa)node[above] {$a$}
                        (pb)node[below] {$b$}
                        (pc)node[below] {$c$}
                
                
                        ( 3, 2) coordinate (pN)
                                to[R, l=$Z$, f<=$I_{CA}$, *-*] ++( -60:3.4641) coordinate (pC)
                                to[R, l=$Z$, f<=$I_{BC}$, *-*] ++(-180:3.4641) coordinate (pB)
                                to[R, l=$Z$, f<=$I_{AB}$, *-*] ++( +60:3.4641) coordinate (pA)
                        (pC)to[short] ++(0,-1) coordinate (pC)
                
                        (pA)node[above] {$A$}
                        (pB)node[below] {$B$}
                        (pC)node[below] {$C$}
                
                        (pa)to[short, f=$I_{aA}$] (pA)
                        (pb)to[short, f_=$I_{bB}$] (pB)
                        (pc)to[short, f_=$I_{cC}$] (pC)
                        ;
                    \end{circuitikz}
                    \caption{Representação Conexão Y-D}
                \end{figure}
            Neste caso as cargas estão conectadas entre duas tensões de linha e portanto as tensões entre fases devem ser consideradas para obter as correntes sobre as cargas obtidas através das seguintes equações:
                \begin{equation}
                    \boxed{I_{AB} = I_{Z}\phase( 30-\theta^{\circ})\text{A}_{ef}}
                    \quad
                    \boxed{I_{BC} = I_{Z}\phase(-90-\theta^{\circ})\text{A}_{ef}}
                    \quad
                    \boxed{I_{CA} = I_{Z}\phase(150-\theta^{\circ})\text{A}_{ef}}
                \end{equation}
            Onde:
                \begin{equation*}
                    \boxed{I_{Z} = \frac{V_{p}\sqrt{3}}{|Z|}}
                    \qquad
                    \text{onde:}
                    \quad
                    Z = |Z|\phase{\theta^{\circ}}
                \end{equation*}
            Neste caso as correntes de linha serão obtidas pelas seguintes equações:
                \begin{equation}
                    \boxed{I_{aA} = \sqrt{3}I_{Z}\phase{(-\theta)^{\circ}}}
                    \quad
                    \boxed{I_{bB} = \sqrt{3}I_{Z}\phase{(-120-\theta)^{\circ}}}
                    \quad
                    \boxed{I_{cC} = \sqrt{3}I_{Z}\phase{(+120-\theta)^{\circ}}}
                \end{equation}
        \end{theorem}
\newpage

    \subsubsection{Transformação Y-$\Delta$}
        \begin{definition}
            Configurações são passíveis de transformações perservando equivalência de acordo com as necessidades do circuito como ilustrado pelas equações abaixo:
                \begin{equation}
                    \boxed{Z_{AB} = \frac{Z_{a}\;Z_{b}}{Z_{a} + Z_{b} + Z_{c}}}
                    \quad
                    \boxed{Z_{BC} = \frac{Z_{b}\;Z_{c}}{Z_{a} + Z_{b} + Z_{c}}}
                    \quad
                    \boxed{Z_{CA} = \frac{Z_{c}\;Z_{a}}{Z_{a} + Z_{b} + Z_{c}}}
                \end{equation}
        \end{definition}

    \subsubsection{Transformação $\Delta$-Y}
        \begin{definition}
            Configurações são passíveis de transformações perservando equivalência de acordo com as necessidades do circuito como ilustrado pelas equações abaixo:
                \begin{equation}
                    \boxed{Z_{A} = \frac{Z_{AB}\;Z_{CA}}{Z_{AB} + Z_{BC} + Z_{CA}}}
                    \quad
                    \boxed{Z_{B} = \frac{Z_{BC}\;Z_{AB}}{Z_{AB} + Z_{BC} + Z_{CA}}}
                    \quad
                    \boxed{Z_{C} = \frac{Z_{CA}\;Z_{BC}}{Z_{AB} + Z_{BC} + Z_{CA}}}
                \end{equation}
        \end{definition}

\subsection{Componentes Simétricas}
    \begin{definition}
        Durante análise de Sistemas Trifásicos desequilibrados considera-se componentes em condições de simetria para facilitar análise como mostrado nas seguintes equações:
            \begin{equation}
                \boxed{I_{a} = I_{a^{(+)}} + I_{a^{(-)}} + I_{a^{(0)}}}
                \quad
                \boxed{I_{b} = I_{b^{(+)}} + I_{b^{(-)}} + I_{b^{(0)}}}
                \quad
                \boxed{I_{c} = I_{c^{(+)}} + I_{c^{(-)}} + I_{c^{(0)}}}
            \end{equation}
        Onde:
            \begin{equation*}
                I_{a^{(0)}} = I_{b^{(0)}} = I_{c^{(0)}}
            \end{equation*}
            \begin{align}
                I_{b^{(+)}} &= I_{a^{(+)}}\cdot\text{e}^{-j120^{\circ}} & I_{b^{(-)}} = I_{a^{(-)}}\cdot\text{e}^{+j120^{\circ}}\nonumber\\
                I_{c^{(+)}} &= I_{a^{(+)}}\cdot\text{e}^{+j120^{\circ}} & I_{c^{(-)}} = I_{a^{(-)}}\cdot\text{e}^{-j120^{\circ}}\nonumber
            \end{align}
    \end{definition}

\subsubsection{Matriz de Transformação}
    \begin{definition}
        Simplificação de notação para conversão em Componentes Simétricas como representado pela seguinte equação:
            \begin{equation}
                \boxed{
                    \begin{bmatrix}
                        I_{a}\\
                        I_{b}\\
                        I_{c}
                    \end{bmatrix}
                    = 
                    \underbrace{
                        \begin{bmatrix}
                            1 & 1 & 1\\
                            \alpha^{2} & \alpha     & 1\\
                            \alpha     & \alpha^{2} & 1\\
                        \end{bmatrix}
                    }_{\text{T}}
                    \times 
                    \begin{bmatrix}
                        I_{a_{(+)}}\\
                        I_{a_{(-)}}\\
                        I_{a_{(0)}}
                    \end{bmatrix}
                }
                \qquad
                \text{onde:}
                \quad
                \alpha = \text{e}^{+j120^{\circ}}
            \end{equation}
    \end{definition}

\subsubsection{Matriz de Transformação Inversa}
    \begin{definition}
        Simplificação de notação para conversão em Componentes Simétricas como representado pela seguinte equação:
            \begin{equation}
                \boxed{
                    \text{T}^{-1}
                    = \frac{1}{3}
                    \begin{bmatrix}
                        1 & \alpha     & \alpha^{2}\\
                        1 & \alpha^{2} & \alpha\\
                        1 & 1 & 1\\
                    \end{bmatrix}
                }
                \qquad
                \text{onde:}
                \quad
                \alpha = \text{e}^{+j120^{\circ}}
            \end{equation}
    \end{definition}

\section{Quadripolos}
\begin{definition}
    Dispositivos que apresentam dois pares de terminais que adotam a convenção apresentada na seguinte figura:
    \begin{figure}[H]
        \centering\begin{circuitikz}[american]
            \tikzset{quad/.style={draw, thick, minimum height=20mm, minimum width=10mm}}
            \node[quad] (A) at (0,0) {Quadripolo};
            \draw
                ($(A.north west)!.25!(A.west)$) coordinate (I11)
                ($(A.south west)!.25!(A.west)$) coordinate (I12)
                ($(A.north east)!.25!(A.east)$) coordinate (I21)
                ($(A.south east)!.25!(A.east)$) coordinate (I22)

                (I11)   to[short, f<_=$I_{1}$, -o]  ++(-1.5,0) coordinate (O11)
                (I12)   to[short,-o]                ++(-1.5,0) coordinate (O12)
                (I21)   to[short, f<=$I_{2}$,-o]    ++(+1.5,0) coordinate (O21)
                (I22)   to[short,-o]                ++(+1.5,0) coordinate (O22)
                ;

            \draw (O11) to[open, v=$V_{1}$] (O12);
            \draw (O21) to[open, v=$V_{2}$] (O22);
        \end{circuitikz}
        \caption{Representação Quadripolos}
        \label{im:quadripolos}
    \end{figure}\noindent
    Nesta configuração adota-se as seguintes equações para o \textbf{Modelo de Impedâncias} do quadriopolo:
    \begin{align}
        V_{1} &= Z_{11}I_{1} + Z_{12}I_{2}\\[1.5mm]
        V_{2} &= Z_{21}I_{1} + Z_{22}I_{2}
    \end{align}
    Onde:
    \begin{enumerate}
        \item $Z_{11}$ e $Z_{22}$ são \textbf{Impedâncias de Entrada} obtidas por:
        \begin{equation}
            Z_{11} = \frac{V_1}{I_1}\quad\text{com $I_2 = 0$} \qquad
            Z_{22} = \frac{V_2}{I_2}\quad\text{com $I_1 = 0$}
        \end{equation}

        \item $Z_{12}$ e $Z_{21}$ são \textbf{Impedâncias de Transferência} obtidas por:
        \begin{equation}
            Z_{12} = \frac{V_1}{I_2}\quad\text{com $I_1 = 0$} \qquad
            Z_{21} = \frac{V_2}{I_1}\quad\text{com $I_2 = 0$}
        \end{equation}
    \end{enumerate}
\end{definition}
\subsection{Configuração $Y$}
\begin{theorem}
    Configuração $Y$ do sistema como representado na seguinte figura:
    \begin{figure}[H]
        \centering\begin{circuitikz}[american]
            \tikzset{quad/.style={draw, thick, dashed, minimum height=20mm, minimum width=40mm}}
            \node[quad] (A) at (0,0) {};
            \draw   
                ($(A.north west)!.25!(A.west)$) coordinate (I11)
                ($(A.south west)!.25!(A.west)$) coordinate (I12)
                ($(A.north east)!.25!(A.east)$) coordinate (I21)
                ($(A.south east)!.25!(A.east)$) coordinate (I22)

                (I11)   to[short, f<_=$I_{1}$, -o]  ++(-1.5,0) coordinate (O11)
                (I12)   to[short,-o]                ++(-1.5,0) coordinate (O12)
                (I21)   to[short, f<=$I_{2}$,-o]    ++(+1.5,0) coordinate (O21)
                (I22)   to[short,-o]                ++(+1.5,0) coordinate (O22)

                (I11)   to[R, l =$Z_1$]  ++(+2,0)
                (I21)   to[R, l_=$Z_2$]  ++(-2,0)
                        to[R, l=$Z_3$, *-*] ++(0,-1.5)
                (I22)   to[short] (I12)

                ;

            \draw (O11) to[open, v=$V_{1}$] (O12);
            \draw (O21) to[open, v=$V_{2}$] (O22);
        \end{circuitikz}
        \caption{Representação Quadripolos Y}
        \label{im:quadripolosY}
    \end{figure}\noindent
    Nesta configuração adota-se as seguintes equações para o \textbf{Modelo de Impedâncias} do quadriopolo:
    \begin{align}
        V_{1} &= (Z_1 + Z_3)I_{1} + Z_3I_{2}\\[1.5mm]
        V_{2} &= Z_3I_{1} + (Z_2 + Z_3)I_{2}
    \end{align}
    Onde:
    \begin{enumerate}
        \item $Z_{11}$ e $Z_{22}$ são \textbf{Impedâncias de Entrada} obtidas por:
        \begin{equation}
            Z_{11} = \frac{V_1}{I_1} = Z_1 + Z_3\quad\text{com $I_2 = 0$} \qquad
            Z_{22} = \frac{V_2}{I_2} = Z_2 + Z_3\quad\text{com $I_1 = 0$}
        \end{equation}

        \item $Z_{12}$ e $Z_{21}$ são \textbf{Impedâncias de Transferência} obtidas por:
        \begin{equation}
            Z_{12} = \frac{V_1}{I_2} = Z_3\quad\text{com $I_1 = 0$} \qquad
            Z_{21} = \frac{V_2}{I_1} = Z_3\quad\text{com $I_2 = 0$}
        \end{equation}
    \end{enumerate}
    Nesta configuração adota-se as seguintes equações para o \textbf{Modelo de Admitância} do quadriopolo:
    \begin{align}
        I_{1} &= Y_{11}V_{1} + Y_{12}V_{2}\\[1.5mm]
        I_{2} &= Y_{21}V_{1} + Y_{22}V_{2}
    \end{align}
    Onde:
    \begin{enumerate}
        \item $Y_{11}$ e $Y_{22}$ são \textbf{Admitância de Entrada} obtidas por:
        \begin{equation}
            Y_{11} = \frac{I_1}{V_1}\quad\text{com $V_2 = 0$} \qquad
            Y_{22} = \frac{I_2}{V_2}\quad\text{com $V_1 = 0$}
        \end{equation}

        \item $Y_{12}$ e $Y_{21}$ são \textbf{Admitância de Transferência} obtidas por:
        \begin{equation}
            Y_{12} = \frac{I_1}{V_2}\quad\text{com $V_1 = 0$} \qquad
            Y_{21} = \frac{I_2}{V_1}\quad\text{com $V_2 = 0$}
        \end{equation}
    \end{enumerate}
\end{theorem}

\subsection{Configuração $\Delta$}
\begin{theorem}
    Configuração $\Delta$ do sistema como representado na seguinte figura:
    \begin{figure}[H]
        \centering\begin{circuitikz}[american]
            \tikzset{quad/.style={draw, thick, dashed, minimum height=20mm, minimum width=40mm}}
            \node[quad] (A) at (0,0) {};
            \draw   
                ($(A.north west)!.25!(A.west)$) coordinate (I11)
                ($(A.south west)!.25!(A.west)$) coordinate (I12)
                ($(A.north east)!.25!(A.east)$) coordinate (I21)
                ($(A.south east)!.25!(A.east)$) coordinate (I22)

                (I11)   to[short, f<_=$I_{1}$, -o]  ++(-1.5,0) coordinate (O11)
                (I12)   to[short,-o]                ++(-1.5,0) coordinate (O12)
                (I21)   to[short, f<=$I_{2}$,-o]    ++(+1.5,0) coordinate (O21)
                (I22)   to[short,-o]                ++(+1.5,0) coordinate (O22)

                (I11)   to[R, l =$Y_B$] (I21)
                (I11)   ++(+0.5,0)
                        to[R, l=$Y_A$, *-*] ++(0,-1.5)
                (I21)   ++(-0.5,0)
                        to[R, l_=$Y_C$, *-*] ++(0,-1.5)
                (I22)   to[short] (I12)
                ;

            \draw (O11) to[open, v=$V_{1}$] (O12);
            \draw (O21) to[open, v=$V_{2}$] (O22);
        \end{circuitikz}
        \caption{Representação Quadripolos $\Delta$}
        \label{im:quadripolosD}
    \end{figure}\noindent
    Nesta configuração adota-se as seguintes equações para o \textbf{Modelo de Admitância} do quadriopolo:
    \begin{align}
        I_{1} &= (Y_A + Y_B)V_{1} + Y_BV_{2}\\[1.5mm]
        I_{2} &= Y_BV_{1} + (Y_B + Y_C)V_{2}
    \end{align}
    Onde:
    \begin{enumerate}
        \item $Y_{11}$ e $Y_{22}$ são \textbf{Admitância de Entrada} obtidas por:
        \begin{equation}
            Y_{11} = \frac{I_1}{V_1} = Y_A + Y_B\quad\text{com $V_2 = 0$} \qquad
            Y_{22} = \frac{I_2}{V_2} = Y_B + Y_C\quad\text{com $V_1 = 0$}
        \end{equation}

        \item $Y_{12}$ e $Y_{21}$ são \textbf{Admitância de Transferência} obtidas por:
        \begin{equation}
            Y_{12} = \frac{I_1}{V_2} = -Y_B\quad\text{com $V_1 = 0$} \qquad
            Y_{21} = \frac{I_2}{V_1} = -Y_B\quad\text{com $V_2 = 0$}
        \end{equation}
    \end{enumerate}
\end{theorem}

\subsection{Ganho de Quadripolos}
\begin{definition}
    Relações entre entradas e saídas dos dispositivos como representados pelo seguinte diagrama:
    \begin{figure}[H]
        \centering\begin{circuitikz}[american]
            \tikzset{quad/.style={draw, thick, minimum height=20mm, minimum width=10mm}}
            \node[quad] (A) at (0,0) {Quadripolo};
            \draw
                ($(A.north west)!.25!(A.west)$) coordinate (I11)
                ($(A.south west)!.25!(A.west)$) coordinate (I12)
                ($(A.north east)!.25!(A.east)$) coordinate (I21)
                ($(A.south east)!.25!(A.east)$) coordinate (I22)

                (I11)   to[short, f<_=$I_{1}$, -o]  ++(-1.5,0) coordinate (O11)
                (I12)   to[short,-o]                ++(-1.5,0) coordinate (O12)
                (I21)   to[short, f<=$I_{2}$,-o]    ++(+1.5,0) coordinate (O21)
                (I22)   to[short,-o]                ++(+1.5,0) coordinate (O22)
                ;

            \draw (O11) to[open, v=$V_{1}$] (O12);
            \draw (O21) to[open, v=$V_{2}$] (O22);
        \end{circuitikz}
    \end{figure}\noindent
    Nesta configuração adota-se \textbf{Ganho de Tensão} como GT obtido pela seguinte equação:
    \begin{equation}
        GT = \frac{V_2}{V_1} = \frac{Z_{21}}{Z_{11}}\quad\text{com $I_2 = 0$}
    \end{equation}
    Nesta configuração adota-se \textbf{Ganho de Corrente} como GC obtido pela seguinte equação:
    \begin{equation}
        GT = \frac{I_2}{I_1} = \frac{Y_{21}}{Y_{11}}\quad\text{com $V_2 = 0$}
    \end{equation}
\end{definition}
\subsubsection{Ganho de Quadripolos com Impedância}
\begin{theorem}
    Relações entre entradas e saídas dos dispositivos como representados pelo seguinte diagrama:
    \begin{figure}[H]
        \centering\begin{circuitikz}[american]
            \tikzset{quad/.style={draw, thick, minimum height=20mm, minimum width=10mm}}
            \node[quad] (A) at (0,0) {Quadripolo};
            \draw
                ($(A.north west)!.25!(A.west)$) coordinate (I11)
                ($(A.south west)!.25!(A.west)$) coordinate (I12)
                ($(A.north east)!.25!(A.east)$) coordinate (I21)
                ($(A.south east)!.25!(A.east)$) coordinate (I22)

                (I11)   to[short, f<_=$I_{1}$, -o]  ++(-1,0) coordinate (O11)
                (I12)   to[short,-o]                ++(-1,0) coordinate (O12)
                (I21)   to[short, f<=$I_{2}$,-o]    ++(+1,0) coordinate (O21)
                (I22)   to[short,-o]                ++(+1,0) coordinate (O22)
                ;

                \draw (O11) to[open, v=$V_{1}$] (O12);
                \draw (O21) to[open, v=$V_{2}$] (O22);

                \draw
                (O11)   to[R, l=$Z_G$]   ++(-2,0)
                        to[sV, v_=$V_G$] ++(0,-1.5)
                        to[short]        (O12);

                \draw
                (O21)   to[short] ++(+2,0)
                        to[R, l=$Z_G$]   ++(0,-1.5)
                        to[short]        (O22);
        \end{circuitikz}
    \end{figure}\noindent
    Nesta configuração adota-se \textbf{Ganho de Tensão} como GT obtido pela seguinte equação:
    \begin{equation}
        GT = \frac{V_2}{V_G} = \frac{Z_{21}Z_C}{(Z_{11} + Z_G)(Z_{22} + Z_C) - Z_{12}Z_{21}}\quad\text{com $I_2 = 0$}
    \end{equation}
    Nesta configuração adota-se \textbf{Ganho de Corrente} como GC obtido pela seguinte equação:
    \begin{equation}
        GT = \frac{I_2}{I_1} = \frac{-Z_{21}}{Z_{22} + Z_C}\quad\text{com $V_2 = 0$}
    \end{equation}
\end{theorem}
\end{document}