\documentclass{article}
\usepackage{tpack}

\title{EA611 - Circuitos II}
\author{Guilherme Nunes Trofino}
\authorRA{217276}
\project{Resumo Teórico}

\begin{document}
    \maketitle
\newpage

    \tableofcontents
\newpage


    \section{Introdução}
        \paragraph{Apresentação}Neste documento será descrito as informações necessárias para compreensão e solução de exercícios relacionados a disciplina \title. Note que este documento são notas realizadas por \author, em \today.

        \subsection{Transformada de Laplace}
            \paragraph{Definição}Conversão de uma equação diferencial em equação algébrica e uma convolução em multiplicação. Formalmente descrita pelas seguintes equações:

            \begin{multicols}{2}
                \raggedcolumns
                \paragraph{Forma Bilateral:}
                    \begin{equation}
                        \boxed{
                            F(s) = \mathcal{B} \{ f(t) \} := \int_{-\infty}^{+\infty} f(t) \; e^{-st} \; \text{d}t
                        }
                    \end{equation}

                \columnbreak

                \paragraph{Forma Unilateral:}
                    \begin{equation}
                        \boxed{
                            F(s) = \mathcal{L}\{ f(t) \} := \int_{0}^{+\infty} f(t) \; e^{-st} \; \text{d} t
                        }
                    \end{equation}
            \end{multicols}\noindent
            Note que a forma \texttt{Unilateral} será um caso particular da \texttt{Bilateral}. Além disso, no estudo de circuitos elétricos será conveniente a adoção do domínio dos complexos para análise. Assim $s = \sigma + \omega\text{j}$ onde $\text{j}$ será a \textbf{Unidade Imaginária}, evitando confusão com \textbf{Corrente Elétrica} causada pela notação matemática canónica.

            \paragraph{Transformações}A seguir encontram-se as principais transformações pela definição \texttt{Unilateral} necessárias:
                \begin{table}[H]
                    \centering
                    \begingroup
                    % \setlength{\tabcolsep}{5mm}
                    \renewcommand{\arraystretch}{1.25}
                    \begin{tabular}[]{lcc}
                                         & $f(t)$      & $\mathcal{L}\{ f(t) \}$\\\hline
                        Degrau Unitário  & $u(t)$      & $\frac{1}{s}$\\
                        Impulso Unitário & $\delta(t)$ & $1$\\
                                         & $t^{n}$     & $\frac{n!}{s^{n+1}}$\\
                                         & $e^{-at}$   & $\frac{1}{s+a}$\\
                                         & $te^{-at}$  & $\frac{1}{(s+a)^{2}}$\\
                                         & $\sin(at)$  & $\frac{a}{(s^2+a^2)}$\\
                                         & $\cos(at)$  & $\frac{s}{(s^2+a^2)}$\\
                        Seno Hiperbólico    & $\sinh(at)$  & $\frac{a}{(s^2-a^2)}$\\
                        Cosseno Hiperbólico & $\cosh(at)$  & $\frac{s}{(s^2-a^2)}$\\
                                         & $e^{at}\;\sin(bt)$  & $\frac{b}{(s-a)^2+b^2}$\\
                                         & $e^{at}\;\cos(bt)$  & $\frac{s-a}{(s-a)^2+b^2}$\\
                        Convolução       & $\int_{0}^{t} f(\varphi)\;g(t - \varphi) \text{d}\varphi$ & $F(s)\cdot G(s)$\\
                        Integral         & $\int_{0}^{t} f(\varphi)\;u(t - \varphi) \text{d}\varphi$ & $\frac{F(s)}{s}$\\
                        Derivada         & $\diff{f(\varphi)}{\varphi}$ & $s\cdot F(s)$\\
                        Frequência       & $e^{-at}f(t)$          & $F(s+a)$\\
                        Temporal         & $f(t-\tau)\mu(t-\tau)$ & $e^{-s\tau}F(s)$\\\hline
                    \end{tabular}
                    \endgroup
                    \caption{Tabela de Transformadas de Laplace}\label{table:Laplace}
                \end{table} \noindent
            Conside que as funções \textbf{Trigonométricas Hiperbólicas} são definidas pelas equações abaixo:
                \begin{equation}
                    \boxed{
                        \sinh(ax) = \frac{e^{ax} - e^{-ax}}{2}
                    }
                    \qquad
                    \boxed{
                        \cosh(ax) = \frac{e^{ax} + e^{-ax}}{2}
                    }
                \end{equation}
                % \begin{exercise}
                %     Calcule via Definição a transformada de Laplace:
                %         \begin{enumerate}[noitemsep, label={(\alph*)}]
                %             \item $f(t) = (2t + 3t^2)\;u(t)$ \footnote{$F(s) =  \frac{2}{s^2} + \frac{6}{s^3}$}
                %             \item $g(t) = \cosh(kt)\;u(t)$   \footnote{$F(s) =  \frac{s}{s^2 - k^2}$}
                %         \end{enumerate}
                % \end{exercise}

            \subsubsection{Degrau Unitário}
                \paragraph{Definição}Representação de descontinuidade unitária, normalmente utilizada para representar mudanças instantâneas em sistemas. Formalmente descrita pela seguinte equação:
                    \begin{equation}
                        \boxed{
                            u(x - a) = 
                            \begin{cases}
                                0, & x < a;\\
                                \frac{1}{2}, & x = a;\\
                                1, & x > a;\\
                            \end{cases}
                        }
                    \end{equation}

            \subsubsection{Impulso Unitário}
                \paragraph{Definição}Distribuição infinita no ponto zero e nula no restante da reta. Formalmente descrita pela seguinte equação:
                    \begin{equation}
                        \boxed{
                            \delta(x) = 
                            \begin{cases}
                                0, & x \neq 0;\\
                                \infty, & x = 0;\\
                            \end{cases}
                            }
                        \end{equation}
                Obedecendo:
                    \begin{equation*}
                        \int_{-\infty}^{+\infty} \delta(x) \; \text{d}x = 1
                        \quad\text{e}\quad
                        \boxed{
                            \int_{a}^{b} f(t) \delta(t)\;\text{d}t = 
                            \begin{cases}
                                f(0);   & \text{se } 0\in[a,b]\\
                                0;      & \text{se } 0\notin[a,b]\\
                            \end{cases}
                        }
                    \end{equation*}

            \subsubsection{Transformada da Deriva}
                \paragraph{Definição}Quando aplicada em uma derivada de ordem $n$ será necessário utilizar da recursão e integração por partes, obtendo a seguinte equação geral:
                    \begin{equation}
                        \boxed{
                            \mathcal{L}\left\{\diff[n]{f(\varphi)}{\varphi}\right\} = 
                            s^{n}\cdot F(s) - 
                            s^{n-1} \cdot f(0) - 
                            s^{n-1} \cdot f'(0) - \dots - 
                            s \cdot f^{n-2}(0) - 
                            f^{n-1}(0)
                        }
                    \end{equation}

        \subsection{Transformada de Componentes}
            \paragraph{Definição}Substituir as equações que descrevem cada componente empregado em um circuito através de seu equivalente em \textbf{Laplace} simplificará os cálculos e poderá integrar suas condições iniciais na análise. Nesta transformação o circuito resultante será puramente resistivo e obedecerá às \textbf{Leis de Kirchhoff}.
                \begin{table}[H]
                    \centering
                    \begingroup
                    % \setlength{\tabcolsep}{5mm}
                    \renewcommand{\arraystretch}{1.25}
                    \begin{tabular}[]{lll}
                                  & Equação Geral                                                         & Equação Laplace\\\hline
                        Resistor  & $v_{R}(t) = R\;i_{R}(t)$                                              & $V_{R}(s) = R\;I_{R}(s)$\\[2.5mm]
                        Capacitor & $v_{C}(t) = \frac{1}{C}\;\int_{0}^{t} i_{C}(t)\;\text{d}t + v_{C}(0)$ & $V_{C}(s) = \frac{1}{sC}\;I_{C}(s) + \frac{v_{C}(0)}{s}$\\[2.5mm]
                        Indutor   & $v_{L}(t) = L\;\diff{i_{L}(t)}{t}$                                    & $V_{L}(s) = sL\;I_{L}(s) - L\;I_{L}(0)$\\[2.5mm]\hline
                    \end{tabular}
                    \endgroup
                    \caption{Transformadas de Laplace de Componentes}\label{table:LaplaceComponents}
                \end{table} \noindent

        \subsection{Função de Rede}
            \paragraph{Definição}Simplificação dos circuitos de tal forma que análise seja facilitada pela utilização de suas entradas e de suas saídas sempre presupondo que as condições iniciais nulas obtidas pelas seguintes equações:
                \begin{equation}
                    \boxed{H(s) = \frac{I(s)}{V(s)}}
                \end{equation}
            Onde:
                \begin{enumerate}[noitemsep]
                    \item $V(s)$, \textbf{Entrada:} Tensão de Entrada;
                    \item $I(s)$, \textbf{Saída:} Saída de Corrente;
                \end{enumerate}

    \section{Circuitos Periódicos}
        \paragraph{Definição}Circuitos que são submetidos a sinais de tensão senoidais normalmente presentes em sistemas de potência elétrica com corrente alternada que será representada pela seguinte equação:
            \begin{equation}
                \boxed{v(t) = V_{M}\cos(\omega t + \varphi)}
            \end{equation}
        Onde:
            \begin{enumerate}
                \item \textbf{Amplitude:} $V_{M}$, Representa a \texttt{Tensão Máxima} do sinal;

                \item \textbf{Ciclo:} Características da equação:
                    \begin{enumerate}[noitemsep, rightmargin = \leftmargin]
                        \item \texttt{Frequência:} $\omega$, Representa a quantidade de \texttt{Oscilações} por intervalo de tempo;

                        \item \texttt{Período:} $T$, Representa o tempo para realizar uma \texttt{Oscilação} do sinal;
                    \end{enumerate}

                \item \textbf{Fase:} Indica a defasagem do sinal representada por $\varphi$;
            \end{enumerate}

        \subsection{Fasores}
            \paragraph{Definição}Conversão de equações periódicas em \textbf{Equações Fasoriais}, isto é, equações que envolvam números complexos para representar um comportamento periódico como descrito pela seguinte equação:
                \begin{equation}
                    \boxed{
                        V(t) := V_{M}\phase{\varphi^{\circ}}
                    }
                \end{equation}
            Onde:
                \begin{enumerate}
                    \item \textbf{Amplitude:} $V_{M}$, Representa a \texttt{Tensão Máxima} do sinal obtido pela seguinte equação:
                        \begin{equation*}
                            \boxed{
                                V_{M} = \sqrt{
                                    V_{\text{I}}^{2} + V_{\text{R}}^{2}
                                }
                            }
                        \end{equation*}
                    Onde:
                        \begin{enumerate}[noitemsep, rightmargin = \leftmargin]
                            \item \texttt{Parcela Imaginária:} $V_{\text{I}}$;

                            \item \texttt{Parcela Real:} $V_{\text{R}}$;
                        \end{enumerate}


                    \item \textbf{Ciclo:} Características da equação:
                        \begin{enumerate}[noitemsep, rightmargin = \leftmargin]
                            \item \texttt{Frequência:} $\omega$, Constante a todos os componentes do circuito;
                        \end{enumerate}

                    \item \textbf{Fase:} Indica a defasagem do sinal representada por $\varphi$ obtido pela seguinte equação:
                        \begin{equation*}
                            \boxed{
                                \varphi = \tan^{-1}
                                \left(
                                    \frac{V_{\text{I}}}{V_{\text{R}}}
                                \right)
                            }
                        \end{equation*}
                \end{enumerate}

            \begin{multicols}{2}
                \raggedcolumns
                \subsubsection{Multiplicação}
                    \paragraph{Definição}Seja $V_{1}(t) = V_{1} \phase{\alpha_{1}^{\circ}}$ e $V_{2}(t) = V_{2} \phase{\alpha_{2}^{\circ}}$ então a multiplicação será obtida pela seguinte equação:
                        \begin{equation}
                            \boxed{
                                V_{1}(t) \cdot V_{2}(t) = 
                                (V_{1} \cdot V_{2})\phase{(\alpha_{1} + \alpha_{2})^{\circ}}
                            }
                        \end{equation}

                \columnbreak

                \subsubsection{Divisão}
                    \paragraph{Definição}Seja $V_{1}(t) = V_{1} \phase{\alpha_{1}^{\circ}}$ e $V_{2}(t) = V_{2} \phase{\alpha_{2}^{\circ}}$ então a multiplicação será obtida pela seguinte equação:
                        \begin{equation}
                            \boxed{
                                \frac{V_{1}(t)}{V_{2}(t)} = 
                                \frac{V_{1}}{V_{2}} \phase{(\alpha_{1} - \alpha_{2})^{\circ}}
                            }
                        \end{equation}
            \end{multicols}

        \subsection{Transformada de Componentes}
            \paragraph{Definição}Substituir as equações que descrevem cada componente empregada em um \textbf{Circuito Periódico} através apenas de sua amplitude e sua fase. Nesta transformação o circuito resultante será puramente resistivo com componentes reais e imaginárias e obedecerá às \textbf{Leis de Kirchhoff}.
                \begin{table}[H]
                    \centering
                    \begingroup
                    % \setlength{\tabcolsep}{5mm}
                    \renewcommand{\arraystretch}{1.25}
                    \begin{tabular}[]{lll}
                                  & Equação Periódica                  & Equação Fasorial\\\hline
                        Resistor  & $v_{R}(t) = R\;i_{R}(t)$           & $V_{R}(t) = R\;I_{R}(t)$\\[2.5mm]
                        Capacitor & $i_{C}(t) = C\;\diff{v_{C}(t)}{t}$ & $V_{C}(t) = \frac{1}{j\omega C}\;I_{C}(t)$\\[2.5mm]
                        Indutor   & $v_{L}(t) = L\;\diff{i_{L}(t)}{t}$ & $V_{L}(s) = j\omega L\;I_{L}(t)$\\[2.5mm]\hline
                    \end{tabular}
                    \endgroup
                    \caption{Transformadas de Fasorial de Componentes}\label{table:FasorialComponents}
                \end{table} \noindent

        \subsection{Potência}
            \paragraph{Definição}Seja um sistema com uma fonte $v(t) = V_{M}\cos(\omega t)$ e $i(t) = I_{M}\cos(\omega t)$ então a potência instantânea será dada pela seguinte equação:
                \begin{equation}
                    p(t) = 
                    R\;i(t)^{2} = 
                    R\;I_{M}^{2}\cos^{2}(\omega t) = 
                    \boxed{
                        \frac{R I_{M}^{2}}{2}(1 + \cos(2\omega t))
                    }
                \end{equation}
            Nota-se que a potência instantânea oscila com o dobro da frequência em torno de um valor constante.

            \subsubsection{Potência Média}
                \paragraph{Definição}Potência fornecida durante um ciclo $T$ de oscilação da equação periódica obtida pela seguinte equação:
                    \begin{equation}
                        \bar{p} = 
                        \frac{1}{T} \int_{t}^{t+T} p(t)\;\text{d}t = 
                        \boxed{
                            \frac{V_{M}\;I_{M}}{2} \cos(\theta)
                        }
                    \end{equation}
                Onde:
                    \begin{enumerate}
                        \item \textbf{Fase da Carga:} $\theta$, Representando o impedância geral pelo componente analísado:
                            \begin{equation*}
                                \theta = 
                                \begin{cases}
                                    -90^{\circ}, & \text{se carga Capacitiva}\\
                                      0^{\circ}, & \text{se carga Resistiva}\\
                                    +90^{\circ}, & \text{se carga Indutiva}\\
                                \end{cases}
                            \end{equation*}
                    \end{enumerate}
                Desta forma define-se como valores eficazes de sinais periódicas os valores necessários em sinais contínuos para que haja a mesma entrega de potência média em um resistor obtidos pela seguinte equação:
                    \begin{equation}
                        \boxed{V_{\text{ef}} = \frac{V_{M}}{\sqrt{2}}}
                        \qquad
                        \boxed{I_{\text{ef}} = \frac{I_{M}}{\sqrt{2}}}
                    \end{equation}

            \subsubsection{Potência Complexa}
                \paragraph{Definição}Potência total consumida por um componente qualquer em um circuito fasorial poderá ser complexa e poderá ser obtida pela seguinte equação:
                    \begin{equation}
                        \boxed{
                            S = V_{\text{ef}} \cdot \bar{I}_{\text{ef}} = P + jQ
                        }
                    \end{equation}
                Onde:
                    \begin{enumerate}
                        \item \textbf{Potência Ativa:} $P$ parte real da potência;
                            \begin{enumerate}[noitemsep]
                                \item \texttt{Carga Resistiva:} Apresenta apenas potência real;
                            \end{enumerate}

                        \item \textbf{Potência Reativa:} $Q$ parte complexa da potência;
                            \begin{enumerate}[noitemsep]
                                \item \texttt{Carga Capacitiva:} Apresenta apenas potência imaginária negaiva;
                                \item \texttt{Carga Indutiva:} Apresenta apenas potência imaginária positiva;
                            \end{enumerate}
                    \end{enumerate}

            \subsubsection{Potência Aparente}
                \paragraph{Definição}Considera-se que o produto entre os valores eficazes de corrente e tensão representa o \textbf{Potência Aparente} sobre aquele componente como mostrado na equação a seguir:
                    \begin{equation}
                        \boxed{p_{\text{ap}} = V_{\text{ef}}\;I_{\text{ef}}}
                    \end{equation}

            \subsubsection{Fator de Potência}
                \paragraph{Definição}Relação entre a potência média e a potência aparente como mostrado na equação a seguir:
                    \begin{equation}
                        \boxed{
                            f_{p} = \frac{\bar{p}}{V_{\text{ef}}\;I_{\text{ef}}} = \cos(\theta)
                        }
                    \end{equation}
                Legislação brasileira exige que as cargas nas indústrias tenham um fator de potência mínimo para que atender as demandas de potência elétrica. Desta forma, pode ser necessário ajustar a impedância da carga $Z = R + jI$ com a inserção de uma carga paralela $Z_{i} = j I_{i}$ como mostrado pela seguinte equação:
                    \begin{equation}
                        \boxed{
                            I_{i} = \frac{R^2 + I^2}{R\;\tan(\cos^{-1}(f_{p})) - I}
                        }
                    \end{equation}
                    \begin{example}
                        Seja $Z = 100 + j100$ com $\omega = 100$ e deseja-se um \textbf{Fator de Potência} $f_{p} = 0.95$. Primeiramente tem-se:
                            \begin{align*}
                                Z = 100 + j100 &= 141.4\phase{45^{\circ}}\\
                                \Aboxed{
                                    f_{p} &= \cos(45^{\circ}) = 0.707
                                } & \text{Fator de Potência Inicial}
                            \end{align*}
                        Nota-se que $0.707 < 0.95$ logo será necessário inserir uma carga paralela:
                            \begin{equation*}
                                I_{i} = \frac{100^2 + 100^2}{100\cdot\tan(\cos^{-1}(0.95)) - 100} = -297.92 \Omega
                            \end{equation*}
                        Nota-se que $I_{i} < 0$ trata-se de uma carga Capacitiva, caso $I_{i} > 0$ seria uma carga Indutiva. Desta forma, tem-se: 
                            \begin{equation*}
                                I_{i} = -\frac{1}{\omega\;C}
                                \qquad
                                \boxed{
                                    C = - \frac{1}{\omega\;I_{i}} = 33,6 \mu F
                                }
                            \end{equation*}
                    \end{example}
\end{document}