\documentclass{article}
\usepackage{tpack}


\titleformat
    {\subsection}           % Part
    [block]                 % Part Shape
    {\normalfont\Large}     % Font Size
    {}                      % Label Numbering
    {0mm}                   % Part Separation
    {}                      % Code Before
    []                      % Code After

    \titlespacing*{\subsection}{0mm}{5mm}{2.5mm}


\title{ES601 - Análise Linear de Sistemas}
\author{Guilherme Nunes Trofino}
\authorRA{217276}
\project{Atividade Teórica}


\begin{document}
    \maketitle
\newpage

\section{Atividade Teórica}
    \paragraph{Apresentação}Resolução das questões de Análise Linear de Sistemas por Guilherme Nunes Trofino, 217276, sobre a resposta de um Sistema de 2\textsuperscript{a} Ordem a uma \href{https://en.wikipedia.org/wiki/Sawtooth_wave}{\textbf{Onda Dente de Serra}}.
    \begin{exercise}
        Encontre a resposta em \textbf{Regime Permanente} de um sistema mecânico de 2\textsuperscript{a} ordem composto por Massa-Mola-Amortecedor, utilizando \href{https://en.wikipedia.org/wiki/Fourier_series}{Série de Fourier} e \href{https://en.wikipedia.org/wiki/Frequency_response}{FRF}, Frequency Response Function, sobre as seguintes condições:
        \begin{enumerate}
            \item \textbf{Massa-Mola-Amortecedor:}\begin{table}[H]
                \centering\begin{tabular}{rl}\hline
                    Massa        & $M=1$\\
                    Mola         & $K=10000$\\
                    Amortecedor  & $C=10$\\\hline
                \end{tabular}
                \caption{Condições do Sistema}
                \label{tb:MMA}
            \end{table}
            \item \textbf{Dente de Serra:}\begin{table}[H]
                \centering\begin{tabular}{rl}\hline
                    Amplitude & $A=100$\\
                    Período   & $T=0.1$\\\hline
                \end{tabular}
                \caption{Entrada Dente de Serra}
                \label{tb:DS}
            \end{table}
        \end{enumerate}
    \end{exercise}
\newpage
    \begin{resolution}
        Solução será proposta em Matlab. Inicialmente declara-se as seguintes constantes para função Dente de Serra:
        \begin{scriptsize}
            \myOctave\lstinputlisting[firstline=10, lastline=30]{es601_ondaDenteSerra.m}
        \end{scriptsize}
        Obtêm-se a seguinte onda:
        \begin{figure}[H]
            \centering
            \includegraphics[height = 7cm]{es601_ondaDenteSerraInput.png}
            \caption{Entrada da Função Dente de Serra}
        \end{figure}
\newpage
        Na sequência serão obtidos os coeficientes de Fourier da Série de Fourier através do seguinte algoritmo:
        \begin{scriptsize}
            \myOctave\lstinputlisting[firstline=33, lastline=60]{es601_ondaDenteSerra.m}
        \end{scriptsize}
        Obtêm-se a seguinte onda:
        \begin{figure}[H]
            \centering
            \includegraphics[height = 8cm]{es601_ondaDenteSerraFourierInput.png}
            \caption{Coeficientes de Fourier da Função Dente de Serra}
        \end{figure}
\newpage
        Na sequência define-se a Função Reposta em Frequência através do seguinte algoritmo:
        \begin{scriptsize}
            \myOctave\lstinputlisting[firstline=67, lastline=98]{es601_ondaDenteSerra.m}
        \end{scriptsize}
        Obtêm-se a seguinte onda:
        \begin{figure}[H]
            \centering
            \includegraphics[height = 8cm]{es601_ondaDenteSerraFourierFRF.png}
            \caption{Coeficientes de Fourier da FRF}
        \end{figure}
\newpage
        Na sequência multiplica-se a função os coeficientes do Dente de Serra pelos coeficientes da Função Reposta em Frequência através do seguinte algoritmo:
        \begin{scriptsize}
            \myOctave\lstinputlisting[firstline=101, lastline=110]{es601_ondaDenteSerra.m}
        \end{scriptsize}
        Obtêm-se a seguinte onda:
        \begin{figure}[H]
            \centering
            \includegraphics[height = 8cm]{es601_ondaDenteSerraFourierOutput.png}
            \caption{Coeficientes de Fourier da Saída}
        \end{figure}
\newpage
        Finalmente os valores temporais serão obtidos pela conversão dos valores da Série através do seguinte algoritmo:
        \begin{scriptsize}
            \myOctave\lstinputlisting[firstline=113, lastline=136]{es601_ondaDenteSerra.m}
        \end{scriptsize} 
        Obtêm-se a seguinte onda:
        \begin{figure}[H]
            \centering
            \includegraphics[height = 8cm]{es601_ondaDenteSerraOutput.png}
            \caption{Resposta à Função Dente de Serra}
        \end{figure}
        Nota-se que inicialmente, até aproximadamente $t = 1.5$ s, não há equivalência entre a solução Analítica e o Simulink. Isso se deve ao fato de que na primeira abordagem desconsidera-se o transiente do sistema, enquanto na segunda abordagem considera-se este transiente.
\newpage
    \begin{figure}[H]
        \centering
        \includegraphics[height = 8cm]{es601_ondaDenteSerraOutput1.png}
        \caption{Transiente Resposta da Função Dente de Serra}
    \end{figure}
    \begin{figure}[H]
        \centering
        \includegraphics[height = 8cm]{es601_ondaDenteSerraOutput2.png}
        \caption{Regime Permanente da Resposta à Função Dente de Serra}
    \end{figure}
\newpage
        Implementou-se a Onda Dente de Serra no Simulink através do seguinte diagrama:
        \begin{figure}[H]
            \centering
            \includegraphics[width = \linewidth]{es601_ondaDenteSerraSimulink.png}
            \caption{Diagrama Simulink}
        \end{figure}
        Onde suas variáveis foram declaradas no Matlab para facilitar a manutenção e edição do código.
    \end{resolution}
\end{document}