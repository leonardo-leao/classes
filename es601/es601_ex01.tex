\documentclass{article}
\usepackage{tpack}


\begin{document}
    \section{Exercício 01}
        \paragraph{Apresentação}Resolução das questões de Análise Linear de Sistemas por Guilherme Nunes Trofino, 217276.

        \subsection{Questão 1}
            \begin{exercise}
                Modelar um sistema RC série com tensão aplicada no circuito como entrada e tensão no capacitor como saída. Implementar a equação diferencial como um diagrama de blocos e simular no Simulink para um entrada em degrau de tensão de 10 V e condições iniciais nulas. A resistência é de 1 kOhm e o capacitor de 2000 uF.
            \end{exercise}
            \begin{resolution}
                Primeiramente será necessário elaborar o circuito requisitado:
                    \begin{figure}[H]
                        \centering
                        \begin{circuitikz}[american voltages]
                            \ctikzset{
                                component text=left, 
                                diodes/scale=0.5, 
                                resistors/width=0.25, 
                                resistors/zigs=1
                            }
                            \draw
                            (0,0)   to[sV, l=$v_{\text{in}}$]       ++(0,3) {}
                                    to[R, l_=$R_{1}$, v^=$v_{R}$]   ++(3,0) coordinate (vOUT)
                                    to[C, l_=$C_{1}$, v^=$v_{C}$]   ++(0,-3)
                                    to[short]                       (0,0);
                        \end{circuitikz}
                        \caption{Circuito RC}
                    \end{figure}
                    Note que o degrau de alimentação será represetado como uma fonte variável e analisando pela \textbf{Lei das Malhas} nessa única malha, obtêm-se a seguinte equação:
                    \begin{align}
                        v_{\text{in}}   &= v_{R} + v_{C}\\
                                        &= R_{1}\;i + v_{C}\\
                                        &= R_{1}\;C_{1}\diff{v_{C}}{t} + v_{C}\\
                                        &= R_{1}\;C_{1}\dot{v}_{C} + v_{C}\\
                        \dot{v}_{C}     &= \frac{1}{R_{1}C_{1}}v_{\text{in}} - \frac{1}{R_{1}C_{1}}v_{C}\label{eq:simulink}
                    \end{align}
                Desta forma, a equação \ref{eq:simulink} será representada no Simulink com o seguinte diagrama:
                    \begin{figure}[H]
                        \centering
                        \includegraphics[height = 5cm]{es601_ex01_im01.png}
                        \caption{Diagrama no Simulink}
                    \end{figure}
                Note que nesta configuração apresenta $\tau = R_{1}C_{1} = 2$ s, desta forma a simulação deve possuir ao menos $5\tau = 10$ s de análise para que o transiente seja superado. Consequentemente obtêm-se o seguinte gráfico para representar a tensão sobre o capacitor:
                    \begin{figure}[H]
                        \centering
                        \includegraphics[height = 8cm]{es601_ex01_im02.png}
                        \caption{Gráfico da Simulação no Simulink}
                    \end{figure}
                \end{resolution}

            \begin{exercise}
                Calcular a solução analítica (resolver a equação diferencial ou usar Laplace, como preferir), implementar no MATLAB e comparar o resultado com o do Simulink.
            \end{exercise}
            \begin{resolution}
                a
            \end{resolution}
\end{document}