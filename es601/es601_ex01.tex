\documentclass{article}
\usepackage{tpack}


\titleformat
    {\subsection}           % Part
    [block]                 % Part Shape
    {\normalfont\Large}     % Font Size
    {}                      % Label Numbering
    {0mm}                   % Part Separation
    {}                      % Code Before
    []                      % Code After

    \titlespacing*{\subsection}{0mm}{5mm}{2.5mm}


\title{ES601 - Análise Linear de Sistemas}
\author{Guilherme Nunes Trofino}
\authorRA{217276}
\project{Atividade Teórica}


\begin{document}
    \maketitle
\newpage

    \section{Atividade Teórica}
        \paragraph{Apresentação}Resolução das questões de Análise Linear de Sistemas por Guilherme Nunes Trofino, 217276, sobre \textbf{Introdução ao Matlab e Simulink}.

        \subsection{Questão 1}
            \begin{exercise}
                Modelar um sistema RC série com tensão aplicada no circuito como entrada e tensão no capacitor como saída. Implementar a equação diferencial como um diagrama de blocos e simular no Simulink para um entrada em degrau de tensão de 10 V e condições iniciais nulas. A resistência é de 1 kOhm e o capacitor de 2000 uF.
            \end{exercise}
            \begin{resolution}
                Primeiramente será necessário elaborar o circuito requisitado:
                    \begin{figure}[H]
                        \centering
                        \begin{circuitikz}[american voltages]
                            \ctikzset{
                                component text=left, 
                                diodes/scale=0.5, 
                                resistors/width=0.25, 
                                resistors/zigs=1
                            }
                            \draw
                            (0,0)   to[sV, l=$v_{\text{in}}$]   ++(0,3) {}
                                    to[R, l_=$R$, v^=$v_{R}$]   ++(3,0) coordinate (vOUT)
                                    to[C, l_=$C$, v^=$v_{C}$]   ++(0,-3)
                                    to[short]                   (0,0);
                        \end{circuitikz}
                        \caption{Circuito RC}
                    \end{figure}\noindent
                Note que o degrau de alimentação será represetado como uma fonte variável e analisando pela \textbf{Lei das Malhas} nessa única malha, obtêm-se a seguinte equação:
                    \begin{align}
                        v_{\text{in}}   &= v_{R} + v_{C}\nonumber\\
                                        &= Ri + v_{C}\nonumber\\
                        v_{\text{in}}   &= RC\diff{v_{C}}{t} + v_{C}\label{eq:analytics}\\
                                        &= RC\dot{v}_{C} + v_{C}\nonumber\\
                        \Aboxed{
                            \dot{v}_{C} &= \frac{1}{RC}v_{\text{in}} - \frac{1}{RC}v_{C}\label{eq:simulink}
                        }
                    \end{align}
                Desta forma, a equação \ref{eq:simulink} será representada no Simulink com o seguinte diagrama:
                    \begin{figure}[H]
                        \centering
                        \includegraphics[height = 5cm]{es601_ex01_im01.png}
                        \caption{Diagrama no Simulink}
                    \end{figure}
                Note que nesta configuração apresenta $\tau = RC = 2$ s, desta forma a simulação deve possuir ao menos $5\tau = 10$ s de análise para que o transiente seja superado. Consequentemente obtêm-se o seguinte gráfico para representar a tensão sobre o capacitor:
                    \begin{figure}[H]
                        \centering
                        \includegraphics[height = 12cm]{es601_ex01_im02.png}
                        \caption{Gráfico da Simulação no Simulink}
                    \end{figure}
                \end{resolution}
\newpage
            \begin{exercise}
                Calcular a solução analítica (resolver a equação diferencial ou usar Laplace, como preferir), implementar no MATLAB e comparar o resultado com o do Simulink.
            \end{exercise}
            \begin{resolution}
                Inicialmente parte-se da equação \ref{eq:analytics} que será solucionada pela \textbf{Transformada de Laplace} como representada abaixo:
                    \begin{align}
                        v_{\text{in}}           &= RC\diff{v_{C}}{t} + v_{C}\nonumber\\
                        v_{\text{in}}u(t)       &= RC\diff{v_{C}}{t} + v_{C} & \text{Incluindo \texttt{Função Degrau}}\nonumber\\
                        \Aboxed{
                            \frac{v_{\text{in}}}{s} &= RC(sV_{C} - v_{C0}) + V_{C}
                        } & \text{Aplicando \texttt{Laplace}}
                    \end{align}
                As condições iniciais serão consideradas como não nulas no desenvolvimento para demonstração de um caso geral. Na sequência será necessário isolar a variável desejada, $V_{C}$, procedendo da seguinte forma:
                    \begin{align}
                        V_{C}(RCs + 1)          &= \frac{v_{\text{in}}}{s} + RCv_{C0}\nonumber\\
                        V_{C}(s + \frac{1}{RC}) &= \frac{v_{\text{in}}}{RCs} + v_{C0}\nonumber\\
                        \Aboxed{
                            V_{C}                   &= \frac{v_{\text{in}}}{RCs(s + \frac{1}{RC})} + \frac{v_{C0}}{(s + \frac{1}{RC})}
                        } & \text{Equação Isolada}
                    \end{align}
                Note que será necessário simplificar a equação através de \textbf{Frações Parciais} para que a \textbf{Anti-Transformada de Laplace} possa ser aplicada e solução encontrada. Desta forma tem-se a seguinte equação:
                    \begin{equation*}
                        \frac{v_{\text{in}}}{RCs(s + \frac{1}{RC})} = 
                        \frac{A}{RCs} + \frac{B}{(s + \frac{1}{RC})}
                        \quad
                        \begin{cases}
                            As + BRCs = 0                & \rightarrow \boxed{B =  -v_{\text{in}}}\\[2.5mm]
                            \frac{A}{RC} = v_{\text{in}} & \rightarrow \boxed{A = RCv_{\text{in}}}\\
                        \end{cases}
                    \end{equation*}
                Obtém-se a seguinte equação geral para $V_{C}$:
                    \begin{align}
                        V_{C}       &= \frac{v_{\text{in}}}{s} + \frac{v_{\text{in}}}{s + \frac{1}{RC}} + \frac{v_{C0}}{s + \frac{1}{RC}}\nonumber\\
                        v_{C}(t)    &= v_{\text{in}} - v_{\text{in}}e^{-\frac{1}{RC}t} + v_{C0}e^{-\frac{1}{RC}t} & \text{Aplicando \texttt{Anti-Laplace}}\nonumber\\
                        \Aboxed{
                            v_{C}(t)    &= v_{\text{in}}(1 - e^{-\frac{1}{RC}t}) + v_{C0}e^{-\frac{1}{RC}t}
                        } & \text{Aplicando Condições Iniciais}\\
                        \Aboxed{
                            v_{C}(t)    &= 10(1 - e^{-\frac{1}{2}t})
                        }
                    \end{align}
\newpage
                Equação acima será modelada em Matlab através do seguinte algoritmo:
                    \begin{scriptsize}
                        \myOctave
                        \lstinputlisting{es601_ex01m.m}
                    \end{scriptsize}
\newpage
                Isso trará o seguinte resultado:
                    \begin{figure}[H]
                        \centering
                        \includegraphics[height = 12cm]{es601_ex01_im03.png}
                        \caption{Gráfico Analítico no Matlab}
                    \end{figure}
                Nota-se que o gráfico é, como esperado, semelhante ao obtido através da simulação no Simulink. Desta forma, os métodos de solução são igualmente eficazes para descrever o comportamento da tensão no capacitor.
            \end{resolution}
\end{document}