\documentclass{article}
\usepackage{tpack}


% \usepgfplotslibrary{external} 
% \tikzexternalize


\titleformat
    {\subsection}           % Part
    [block]                 % Part Shape
    {\normalfont\Large}     % Font Size
    {}                      % Label Numbering
    {0mm}                   % Part Separation
    {}                      % Code Before
    []                      % Code After

    \titlespacing*{\subsection}{0mm}{5mm}{2.5mm}


\title{ES601 - Análise Linear de Sistemas}
\author{Guilherme Nunes Trofino}
\authorRA{217276}
\project{Atividade Teórica}


\begin{document}
    \maketitle
\newpage

\section{Atividade Teórica}
    \paragraph{Apresentação}Resolução das questões de Análise Linear de Sistemas por Guilherme Nunes Trofino, 217276, sobre a resposta de um \textbf{Sistema de Estados} de 2\textsuperscript{a} Ordem.
    \begin{exercise}
        Obter para o sistema de Segunda Ordem com $\omega_{n} = 100$ e $\zeta = 0.05$ pede-se:
        \begin{enumerate}
            \item\label{1} Obter a equação do sistema em \textbf{Estados};

            \item\label{2} Calcular a \textbf{Resposta Livre} a uma condição inicial utilizando a exponencial matricial dos seguintes métodos:
            \begin{enumerate}[rightmargin = \leftmargin, noitemsep]
                \item\label{2A} Função \href{https://www.mathworks.com/help/matlab/ref/expm.html}{\texttt{expm()}} do MATLAB;
                \item\label{2B} \href{https://en.wikipedia.org/wiki/Matrix_exponential}{\texttt{Método dos Autovalores}};
                \item\label{2C} \href{https://en.wikipedia.org/wiki/Matrix_exponential}{\texttt{Método de Laplace Inverso}};
                \item\label{2D} \href{https://en.wikipedia.org/wiki/Matrix_exponential}{\texttt{Método de Cayley-Hamilton}};
            \end{enumerate}

            \item\label{3} Implementar em  Simulink e obter a \textbf{Resposta Forçada};

            \item\label{4} Implementar em Estados em MATLAB usando a função \href{https://www.mathworks.com/help/control/ref/ss.html}{\texttt{ss()}}:
            \begin{enumerate}[rightmargin = \leftmargin, noitemsep]
                \item\label{4A} \href{https://www.mathworks.com/help/matlab/ref/ss2tf.html}{\texttt{ss2tf()}};
                \item\label{4B} \href{https://www.mathworks.com/help/signal/ref/tf2ss.html}{\texttt{tf2ss()}};
            \end{enumerate}

            \item\label{5} Obter a Resposta \textbf{Livre} e \textbf{Forçada} com o comando \href{https://www.mathworks.com/help/control/ref/lti.lsim.html}{\texttt{lsim()}};
        \end{enumerate}
    \end{exercise}
\newpage
    \begin{resolution}
    Solução será proposta em Matlab. Inicialmente declara-se as variáveis básicas do problema com base na seguinte equação:
    \begin{equation*}
        \ddot{y}(t) + 2\zeta\omega_{n}\dot{y}(t) + \omega_{n}^{2} y(t) = \omega_{n}^{2} u(t)
    \end{equation*}
    (\ref{1}) Considera-se a seguinte, $x_1(t)$ Espaço e $x_2(t)$ Velocidade, substituição obtêm-se as seguintes matrizes:
    \begin{equation*}
        \text{adotando }\begin{cases}
            x_1(t) = y(t)\\
            x_2(t) = \dot{y}(t)\\
        \end{cases}
        \text{ derivando }\begin{cases}
            \dot{x}_1(t) =  \dot{y}(t) = x_2(t)\\
            \dot{x}_2(t) = \ddot{y}(t) = -2\zeta\omega_{n}\dot{y}(t) - \omega_{n}^{2} y(t) + \omega_{n}^{2} u(t)\\
        \end{cases}
    \end{equation*}
    Na sequência toma-se:
    \begin{equation*}
        \boldsymbol{x} = 
        \begin{bmatrix}
            x_1\\
            x_2
        \end{bmatrix}
        \text{ derivando }
        \dot{\boldsymbol{x}} = 
        \begin{bmatrix}
            x_2\\
            -2\zeta\omega_{n}x_2 - \omega_{n}^{2} x_1 + \omega_{n}^{2} u(t)
        \end{bmatrix} = 
        \underbrace{
        \begin{bmatrix}
            0               & 1\\
            -\omega_{n}^{2} & -2\zeta\omega_{n}
        \end{bmatrix}
        }_{\boldsymbol{A}}
        \boldsymbol{x} + 
        \underbrace{
        \begin{bmatrix}
            0\\
            \omega_{n}^{2}
        \end{bmatrix}
        }_{\boldsymbol{B}}
        u(t)
    \end{equation*}
    Deseja-se visualizar o espaço dessa forma $\boldsymbol{C} = [1\;0]$ e $\boldsymbol{D} = 1$ como implementado a seguir:
    \begin{scriptsize}
        \myOctave\lstinputlisting[firstline=11, lastline=51]{es601_analiseEstadoTotal.m}
    \end{scriptsize}
\newpage
    Todos os sistemas tiveram como entrada um \texttt{Degrau Unitário} a partir do 0 e foram avaliados por um intervalo de 1 segundo. A fim de simplificar a manipulação do código as seguintes variáveis criadas:
    \begin{scriptsize}
        \myOctave\lstinputlisting[firstline=53, lastline=63]{es601_analiseEstadoTotal.m}
    \end{scriptsize}
    Códigos para realização dos \texttt{plot} não foram incluidos neste documento pois são repetitivos e não acrescento ao conteúdo prático.
\newpage
    (\ref{2A}) Implementação Método de Controle:
    \begin{scriptsize}
        \myOctave\lstinputlisting[firstline=66, lastline=83]{es601_analiseEstadoTotal.m}
    \end{scriptsize}
    (\ref{2B}) Implementação de Autovalores e Autovetores:
    \begin{scriptsize}
        \myOctave\lstinputlisting[firstline=90, lastline=107]{es601_analiseEstadoTotal.m}
    \end{scriptsize}
    (\ref{2C}) Implementação de Inversa de Laplace:
    \begin{scriptsize}
        \myOctave\lstinputlisting[firstline=115, lastline=132]{es601_analiseEstadoTotal.m}
    \end{scriptsize}
\newpage
    (\ref{2D}) Implementação de Cayley-Hamilton:
    \begin{scriptsize}
        \myOctave\lstinputlisting[firstline=137, lastline=179]{es601_analiseEstadoTotal.m}
    \end{scriptsize}
    Note que a solução forçada, independente do método, será obtida simbolicamente através da integração da exponencial matricial com relação a \texttt{t} no intervalo determinado por \texttt{[x0 x]}.
\newpage
    Determinou-se um intervalo unitário para analisar o sistema proposto. Caso o sistema demandasse mais tempo para superar o \textbf{Regime Transiente} a variável \texttt{x0} deve ser expandida para que um intervalo suficientemente grande seja considerado para análise.
    \begin{scriptsize}
        \myOctave\lstinputlisting[firstline=183, lastline=210]{es601_analiseEstadoTotal.m}
    \end{scriptsize}
    Note que apenas os valores reais serão considerados através do comando \texttt{real(f)} para evitar warnings desnecessários.\\\\
    Durante a solução numérica haverá algum lixo numérico complexo esperado e indesejado, na ordem de $10^{-15}$, produzido pelas operações realizadas pelo MATLAB. Caso seja necessário, pode-se alinalizar o resíduo numérico como indicativo do funcionamento adequado dos métodos.
\newpage
    Representação do \texttt{plot} do \textbf{Resultado Livre}, condições iniciais não nulas e independente da entrada, causadora do \textbf{Regime Transitório} do sistema:
    \begin{figure}[H]
        \centering
        \includegraphics[height = 5cm]{images/es601_ex6_RLAnalitico.png}
        \caption{Resposta Livre Analítica}
    \end{figure}
    Nota-se que, como esperado, $y(0) = 10$. Representação do \texttt{plot} do \textbf{Resultado Forçado}, condições iniciais nulas e dependente da entrada, causadora do \textbf{Regime Permanente} do sistema:
    \begin{figure}[H]
        \centering
        \includegraphics[height = 5cm]{images/es601_ex6_RFAnalitico.png}
        \caption{Resposta Forçada Analítica}
    \end{figure}
    Nota-se que, como esperado, quando $t\to\infty$ tem-se $y(t) = 1$. Representação do \texttt{plot} do \textbf{Resultado Total}, combinação entre \textbf{Livre} e \textbf{Forçada}:
    \begin{figure}[H]
        \centering
        \includegraphics[height = 5cm]{images/es601_ex6_RTAnalitico.png}
        \caption{Resposta Total Analítica}
    \end{figure}
    Nota-se que, como esperado, que $y(t) = y_\text{Livre}(t) + y_\text{Forçada}(t)$.
\newpage
    (\ref{3}) Utiliza-se o bloco de \texttt{Sistema de Estados} para encontrar cada resposta desejada de acordo com as descrições apresentadas na seguinte figura:
    \begin{figure}[H]
        \centering
        \includegraphics[width = 15cm]{images/es601_ex6_Simulink.png}
        \caption{Diagrama Simulink}
    \end{figure}
    Note que as configurações utilizadas são semelhantes as consideradas para as demais aplicações e condições necessárias.
\newpage
    Representação do \texttt{plot} do \textbf{Resultado Livre}, condições iniciais não nulas e independente da entrada, causadora do \textbf{Regime Transitório} do sistema:
    \begin{figure}[H]
        \centering
        \includegraphics[height = 5cm]{images/es601_ex6_RLSimulink.png}
        \caption{Resposta Livre Simulink}
    \end{figure}
    Nota-se que, como esperado, $y(0) = 10$. Representação do \texttt{plot} do \textbf{Resultado Forçado}, condições iniciais nulas e dependente da entrada, causadora do \textbf{Regime Permanente} do sistema:
    \begin{figure}[H]
        \centering
        \includegraphics[height = 5cm]{images/es601_ex6_RFSimulink.png}
        \caption{Resposta Forçada Simulink}
    \end{figure}
    Nota-se que, como esperado, quando $t\to\infty$ tem-se $y(t) = 1$. Representação do \texttt{plot} do \textbf{Resultado Total}, combinação entre \textbf{Livre} e \textbf{Forçada}:
    \begin{figure}[H]
        \centering
        \includegraphics[height = 5cm]{images/es601_ex6_RTSimulink.png}
        \caption{Resposta Total Simulink}
    \end{figure}
    Nota-se que, como esperado, que $y(t) = y_\text{Livre}(t) + y_\text{Forçada}(t)$.
\newpage
    (\ref{4}) Além das soluções analíticas e simulações do SIMULINK, é possível realizar as operações necessárias através das funções \texttt{ss}, \texttt{ss2tf} e \texttt{tf2ss} como representado abaixo:
    \begin{scriptsize}
        \myOctave\lstinputlisting[firstline=300, lastline=340]{es601_analiseEstadoTotal.m}
    \end{scriptsize}
    Note que apesar das matrizes obtidas após a conversão para \textbf{Função de Transferência} sejam diferentes o \textbf{Sistema de Estados} representado é o mesmo, visto que os \textbf{Autovalores} da matriz \texttt{A0} e \texttt{A1} são os iguais e negativos, garantindo assim a convergência da solução.\\\\
    Todavia, as condições iniciais são diferentes para cada sistema visto que a conversão de uma representação de Estado para outra passa por uma matriz de transformação que altera, portanto, os valores iniciais para a \textbf{Resposta Livre}.\\\\
    Além da função \texttt{lsim}, próxima ao funcionamento do bloco de Sistema de Estado do Simulink, há as funções \texttt{initial(ss)} e \texttt{step(ss)} que são casos particulares de \texttt{lsim} para obtenção das respostas \textbf{Livre} e \textbf{Forçadas} respectivamente.
\newpage
    (\ref{5}) Representação do \texttt{plot} do \textbf{Resultado Livre}, condições iniciais não nulas e independente da entrada, causadora do \textbf{Regime Transitório} do sistema:
    \begin{figure}[H]
        \centering
        \includegraphics[height = 5cm]{images/es601_ex6_RLlsim.png}
        \caption{Resposta Livre \texttt{lsim()}}
    \end{figure}
    Nota-se que, como esperado, $y(0) = 10$. Representação do \texttt{plot} do \textbf{Resultado Forçado}, condições iniciais nulas e dependente da entrada, causadora do \textbf{Regime Permanente} do sistema:
    \begin{figure}[H]
        \centering
        \includegraphics[height = 5cm]{images/es601_ex6_RFlsim.png}
        \caption{Resposta Forçada \texttt{lsim()}}
    \end{figure}
    Nota-se que, como esperado, quando $t\to\infty$ tem-se $y(t) = 1$. Representação do \texttt{plot} do \textbf{Resultado Total}, combinação entre \textbf{Livre} e \textbf{Forçada}:
    \begin{figure}[H]
        \centering
        \includegraphics[height = 5cm]{images/es601_ex6_RTlsim.png}
        \caption{Resposta Total \texttt{lsim()}}
    \end{figure}
    Nota-se que, como esperado, que $y(t) = y_\text{Livre}(t) + y_\text{Forçada}(t)$.
\end{resolution}
\end{document}