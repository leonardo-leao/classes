\documentclass{article}
\usepackage{tpack}

\title{ES601 - Análise Linear de Sistemas}
\author{Guilherme Nunes Trofino}
\authorRA{217276}
\project{Resumo Teórico}

\begin{document}
    \maketitle
\newpage

    \tableofcontents
\newpage

    \section{Introdução}
        \paragraph{Apresentação}Neste documento será descrito as informações necessárias para compreensão e solução de exercícios relacionados a disciplina \thetitle . Note que este documento são notas realizadas por \theauthor , em \today.

        \subsection{Modelagem Mecânica}
            \paragraph{Definição}Modelos básicos para situações usualmente encontradas em sistemas mecânicos simples, descrevendo as equações necessárias para a descrição do movimento.
                \begin{multicols}{2}
                    \raggedcolumns
                    \subsubsection{Mola Ideal}
                        \paragraph{Definição}Dispositivo linear que apresenta uma \texttt{Constante Elástica} constante igual a $k$.\\

                        Assim, haverá uma força $\vec{F}$ exercida pela mola proporcional ao seu deslocamento $x$ com sentido oposto, de acordo com a seguinte equação:
                            \begin{equation}
                                \boxed{
                                    \vec{F} = - k\;\vec{x}
                                }
                            \end{equation}
                        Analogamente, no caso \textbf{Rotacional} um torque $\vec{T}$ causa um deslocamento angular $\theta$. Assim, a seguinte equação será válida:
                            \begin{equation}
                                \boxed{
                                    \vec{T} = - k\;\vec{\theta}
                                }
                            \end{equation}

                    \columnbreak

                    \subsubsection{Amortecedor Ideal}
                        \paragraph{Definição}Dispositivo linear que apresenta uma \texttt{Constante de Amortecimento} constante igual a $c$.\\

                        Assim, haverá uma força $\vec{F}$ exercida pelo amortecedor  proporcional a sua velocidade $\dot{x}$ com sentido oposto, de acordo com a seguinte equação:
                            \begin{equation}
                                \boxed{
                                    \vec{F} = - c\;\vec{\dot{x}}
                                }
                            \end{equation}
                        Analogamente, no caso \textbf{Rotacional} um torque $\vec{T}$ causa um velocidade angular $\dot{\theta}$. Assim, a seguinte equação será válida:
                            \begin{equation}
                                \boxed{
                                    \vec{T} = - k\;\vec{\dot{\theta}}
                                }
                            \end{equation}
                \end{multicols}\noindent
\newpage

\subsection{Transformada de Laplace}
    \paragraph{Definição}Conversão de uma equação diferencial em equação algébrica e uma convolução em multiplicação. Formalmente descrita pelas seguintes equações:
    \begin{multicols}{2}
        \raggedcolumns
        \paragraph{Forma Bilateral:}
        \begin{equation}
            \boxed{
                F(s) = \mathcal{B} \{ f(t) \} := \int_{-\infty}^{+\infty} f(t) \; e^{-st} \; \text{d}t
            }
        \end{equation}
        \columnbreak
        \paragraph{Forma Unilateral:}
        \begin{equation}
            \boxed{
                F(s) = \mathcal{L}\{ f(t) \} := \int_{0}^{+\infty} f(t) \; e^{-st} \; \text{d} t
            }
        \end{equation}
    \end{multicols}\noindent
    Note que a forma \texttt{Unilateral} será um caso particular da \texttt{Bilateral}. Além disso, no estudo de circuitos elétricos será conveniente a adoção do domínio dos complexos para análise. Assim $s = \sigma + \omega\text{j}$ onde $\text{j}$ será a \textbf{Unidade Imaginária}, evitando confusão com \textbf{Corrente Elétrica} causada pela notação matemática canónica.

    \paragraph{Transformações}A seguir encontram-se as principais transformações pela definição \texttt{Unilateral} necessárias:
    \begin{table}[H]
        \centering\begingroup
        % \setlength{\tabcolsep}{5mm}
        \renewcommand{\arraystretch}{1.25}
        \begin{tabular}[]{lcc}
                                & $f(t)$      & $\mathcal{L}\{ f(t) \}$\\\hline
            Degrau Unitário     & $u(t)$      & $\frac{1}{s}$\\
            Impulso Unitário    & $\delta(t)$ & $1$\\
            Rampa Unitário      & $r(t)$      & $\frac{1}{s^2}$\\
                                & $t^{n}$     & $\frac{n!}{s^{n+1}}$\\
                                & $e^{-at}$   & $\frac{1}{s+a}$\\
                                & $\frac{t^{n-1}e^{-at}}{(n-1)!}$  & $\frac{1}{(s+a)^{n}}$\\
                                & $\sin(at + b)$  & $\frac{s\sin(b) + a\cos(b)}{(s^2+a^2)}$\\
                                & $\cos(at + b)$  & $\frac{s\cos(b) + a\sin(b)}{(s^2+a^2)}$\\
            Seno Hiperbólico    & $\sinh(at)$  & $\frac{a}{(s^2-a^2)}$\\
            Cosseno Hiperbólico & $\cosh(at)$  & $\frac{s}{(s^2-a^2)}$\\
                                & $e^{at}\;\sin(bt)$  & $\frac{b}{(s-a)^2+b^2}$\\
                                & $e^{at}\;\cos(bt)$  & $\frac{s-a}{(s-a)^2+b^2}$\\
            Convolução          & $\int_{0}^{t} f(\varphi)\;g(t - \varphi) \text{d}\varphi$ & $F(s)\cdot G(s)$\\
            Integral            & $\int_{0}^{t} f(\varphi)\;u(t - \varphi) \text{d}\varphi$ & $\frac{F(s)}{s}$\\
            Derivada         & $\diff{f(\varphi)}{\varphi}$ & $s\cdot F(s)$\\
            Frequência       & $e^{-at}f(t)$          & $F(s+a)$\\
            Temporal         & $f(t-\tau)\mu(t-\tau)$ & $e^{-s\tau}F(s)$\\\hline
        \end{tabular}
        \endgroup
        \caption{Tabela de Transformadas de Laplace}\label{table:Laplace}
    \end{table} \noindent
    Conside que as funções \textbf{Trigonométricas Hiperbólicas} são definidas pelas equações abaixo:
    \begin{equation}
        \boxed{
            \sinh(ax) = \frac{e^{ax} - e^{-ax}}{2}
        }
        \qquad
        \boxed{
            \cosh(ax) = \frac{e^{ax} + e^{-ax}}{2}
        }
    \end{equation}

\subsubsection{Degrau Unitário}
    \paragraph{Definição}Representação de descontinuidade unitária, normalmente utilizada para representar mudanças instantâneas em sistemas. Formalmente descrita pela seguinte equação:
    \begin{equation}
        \boxed{
            u(x - a) = 
            \begin{cases}
                0, & x < a;\\
                \frac{1}{2}, & x = a;\\
                1, & x > a;\\
            \end{cases}
        }
    \end{equation}

\subsubsection{Impulso Unitário}
    \paragraph{Definição}Distribuição infinita no ponto zero e nula no restante da reta. Formalmente descrita pela seguinte equação:
    \begin{equation}
        \boxed{
            \delta(x-a) = 
            \begin{cases}
                0, & x \neq a;\\
                \infty, & x = a;\\
            \end{cases}
            }
        \end{equation}
    Obedecendo:
    \begin{equation*}
        \int_{-\infty}^{+\infty} \delta(x) \; \text{d}x = 1
        \quad\text{e}\quad
        \boxed{
            \int_{a}^{b} f(t) \delta(t - \tau)\;\text{d}t = 
            \begin{cases}
                f(\tau);    & \text{se } \tau\in[a,b]\\
                0;          & \text{se } \tau\notin[a,b]\\
            \end{cases}
        }
    \end{equation*}
    Aplica-se o impulso unitário para se extrair uma \textbf{Amostra} do valor de uma função em um determinado ponto.

\subsubsection{Rampa Unitário}
    \paragraph{Definição}Representação de uma função linear com coeficiente angular unitário. Formalmente descrita pela seguinte equação:
    \begin{equation}
        \boxed{
            r(x - a) = 
            \begin{cases}
                0, & x < a;\\
                x, & x > a;\\
            \end{cases}
        }
    \end{equation}

\subsubsection{Transformada da Deriva}
    \paragraph{Definição}Quando aplicada em uma derivada de ordem $n$ será necessário utilizar da recursão e integração por partes, obtendo a seguinte equação geral:
    \begin{equation}
        \boxed{
            \mathcal{L}\left\{\diff[n]{f(\varphi)}{\varphi}\right\} = 
            s^{n}\cdot F(s) - 
            s^{n-1} \cdot f(0) - 
            s^{n-2} \cdot f'(0) - \dots - 
            s \cdot f^{n-2}(0) - 
            f^{n-1}(0)
        }
    \end{equation}
\end{document}