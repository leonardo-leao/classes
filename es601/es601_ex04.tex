\documentclass{article}
\usepackage{tpack}


\titleformat
    {\subsection}           % Part
    [block]                 % Part Shape
    {\normalfont\Large}     % Font Size
    {}                      % Label Numbering
    {0mm}                   % Part Separation
    {}                      % Code Before
    []                      % Code After

    \titlespacing*{\subsection}{0mm}{5mm}{2.5mm}


\title{ES601 - Análise Linear de Sistemas}
\author{Guilherme Nunes Trofino}
\authorRA{217276}
\project{Atividade Teórica}


\begin{document}
    \maketitle
\newpage

    \section{Atividade Teórica}
        \paragraph{Apresentação}Resolução das questões de Análise Linear de Sistemas por Guilherme Nunes Trofino, 217276, sobre \textbf{Simulado da P1}.

        \subsection{Questão 1}
            \begin{exercise}
                Circuito RC alimentado por uma fonte de tensão é um Sistema Linear Invariante no tempo regido pela seguinte equação:
                    \begin{equation*}
                        RC\dot{y}(t) + y(t) = u(t)
                    \end{equation*}
                Onde:
                    \begin{enumerate}[noitemsep]
                        \item $R$ é a Resistência;
                        \item $C$ é a Capacitância;
                            \begin{enumerate}
                                \item Considera-se $RC$ = 1s;
                            \end{enumerate}
                        \item $u(t)$ é um Pulso Retangular expresso por:
                            \begin{equation*}
                                u(t) = 
                                \begin{cases}
                                    0V  & \text{quando $t < 0$ s}\\
                                    10V & \text{quando $0 < t < 3$ s}\\
                                    0V  & \text{quando $t > 3$ s}\\
                                \end{cases}
                            \end{equation*}
                    \end{enumerate}
                Deseja-se obter:
                    \begin{enumerate}[label=(\alph*)]
                        \item \label{ex:1A}Calcular a resposta forçada com condições iniciais nulas por Laplace;

                        \item \label{ex:1B}Implementar uma simulação Simulink;

                        \item \label{ex:1C}Comparar os resultados dos itens anteriores;

                        \item \label{ex:1D}Calcular a resposta para entrada nula e condição inicial $y(0)= 10$V
                    \end{enumerate}
            \end{exercise}
\newpage
            \begin{resolution}
                \textbf{\ref{ex:1A}.} Analisando a equação apresentada:
                    \begin{align}
                        \dot{y}(t) + y(t) &= u(t)\label{eq:Definition}\\
                                          &= 10\mu(t) - 10\mu(t-3)\nonumber
                    \end{align}
                Note que como trata-se de um Sistema Linear Invariante no tempo as equações podem ser analisadas separadamente e posteriormente somadas como demonstrada a seguir:
                    \begin{equation}
                        y(t) = y_{1}(t) + y_{2}(t)
                        \quad\text{onde:}\quad
                        \begin{cases}
                            y_{1}(t): & \dot{y}_{1}(t) + y_{1}(t) = 10\mu(t)\\
                            y_{2}(t): & \dot{y}_{2}(t-3) + y_{2}(t-3) =-10\mu(t-3)\\
                        \end{cases}
                        \label{eq:Cases}
                    \end{equation}
                Nota-se que tratam-se de sistemas semelhantes salvo uma translação em $t$. Desta forma, pode-se analizar $y_{1}(t)$ e posteriomente obter $y_{2}(t)$. Desta forma obtem-se a seguinte equação:
                    \begin{align}
                        \dot{y}_{1}(t) + y_{1}(t) &= 10\mu(t) & \text{Aplicação de Laplace}\nonumber\\
                        (sY_{1} - y_{01}) + Y_{1} &= 10\frac{1}{s}\nonumber\\
                        Y_{1}(s + 1)     &= 10\frac{1}{s} + y_{01}\nonumber\\
                        \Aboxed{Y_{1}(s) &= 10\frac{1}{s(s + 1)} + \frac{y_{01}}{s + 1}} & \text{Equação de Laplace}
                    \end{align}
                Nota-se que será necessário aplicar frações parciais para simplificar a equação:
                    \begin{equation*}
                        \frac{10}{s(s + 1)} = 
                            \frac{A}{s} + 
                            \frac{B}{s + 1}
                        \quad
                        \begin{cases}
                            As + Bs = 0 & \rightarrow \boxed{B =-10}\\[1.5mm]
                            A = 10      & \rightarrow \boxed{A = 10}\\
                        \end{cases}
                    \end{equation*}
                Calcula-se a Anti-Transformada de Laplace:
                    \begin{align}
                        Y_{1}(s) &= \frac{10}{s} - \frac{10}{s + 1} + \frac{y_{01}}{s + 1}\nonumber\\
                        y_{1}(t) &= 10\mu(t) - 10\text{e}^{-1t} + y_{01}\text{e}^{-1t}\nonumber\\
                        \Aboxed{y_{1}(t) &=(+10 + (y_{01} - 10)\text{e}^{-1t})\mu(t)}\\
                        \Aboxed{y_{2}(t) &=(-10 - (y_{02} - 10)\text{e}^{-1(t-3)})\mu(t-3)}
                    \end{align}
                Desta forma, tem-se que para Equação (\ref{eq:Cases}) sobre as condições iniciais necessárias na Alternativa \ref{ex:1A} será obtido pela seguinte equação:
                    \begin{equation}
                        \boxed{
                            y(t) = (10 - 10\text{e}^{-1t})\mu(t) + (-10 +P 10\text{e}^{-1(t-3)})\mu(t-3)
                        }\label{eq:General1}
                    \end{equation}
            \end{resolution}
\newpage
            \begin{resolution}
                \textbf{\ref{ex:1B}.} Desta forma, a Equação (\ref{eq:General1}) será modelada no Simulink com o seguinte diagrama:
                    \begin{figure}[H]
                        \centering
                        \includegraphics[height = 5cm]{es601_ex04_im01.png}
                        \caption{Representação da Simulação no Simulink}
                    \end{figure}
                Obtem-se assim a seguinte entrada:
                    \begin{figure}[H]
                        \centering
                        \includegraphics[height = 5cm]{es601_ex04_im02.png}
                        \caption{Entrada Simulação no Simulink}
                    \end{figure}
                Obtem-se assim a seguinte saída:
                    \begin{figure}[H]
                        \centering
                        \includegraphics[height = 5cm]{es601_ex04_im03.png}
                        \caption{Saída Simulação no Simulink}
                    \end{figure}
            \end{resolution}
\newpage
            \begin{resolution}
                \textbf{\ref{ex:1C}.} Equação (\ref{eq:General1}) será modelada em Matlab através do seguinte algoritmo:
                    \begin{scriptsize}
                        \myOctave
                        \lstinputlisting{es601_ex04_01m.m}
                    \end{scriptsize}
                Obtem-se assim a seguinte saída:
                    \begin{figure}[H]
                        \centering
                        \includegraphics[height = 5cm]{es601_ex04_im04.png}
                        \caption{Comparação Matlab e Simulink}
                    \end{figure}
            \end{resolution}
\newpage
            \begin{resolution}
                \textbf{\ref{ex:1D}.} Analisando a Equação (\ref{eq:Definition}) apresenta sobre as condições necessárias para a Alternativa \ref{ex:1D}:
                    \begin{align}
                        \dot{y}_{1}(t) + y_{1}(t) &= 0 & \text{Aplicação de Laplace}\nonumber\\
                        (sY_{1} - 10) + Y_{1}     &= 0\nonumber\\
                        Y_{1}(s + 1)     &= 10\nonumber\\
                        \Aboxed{Y_{1}(s) &= \frac{10}{s + 1}} & \text{Equação de Laplace}\\
                        \Aboxed{y(t)     &= 10\text{e}^{-1t}\mu(t)}
                    \end{align}
            \end{resolution}
\newpage

        \subsection{Questão 2}
            \begin{exercise}
                Considere um sistema discreto descrito pelas seguintes equações:
                    \begin{equation*}
                        y(k+1) - 0.2y(k) = u(k)
                    \end{equation*}
                Deseja-se obter:
                    \begin{enumerate}[label=(\alph*), rightmargin = \leftmargin]
                        \item \label{ex:2A}Calcular por solução da equação, Homogênea e Particular, por recursão para a seguinte entrada com condições iniciais nulas:
                            \begin{equation*}
                                u(k) = 2r(k)\mu(k)
                            \end{equation*}
                        Onde $r(k)$ representa a \textbf{Função Rampa}. Plote o resultado usando o comando \texttt{stairs(k,y)}.

                        \item \label{ex:2B}Calcular e plotar, utilizando o comando \texttt{stairs}, a reposta para entrada nula e condição inicial $y(0) = 5$.
                    \end{enumerate}
            \end{exercise}
\newpage
            \begin{resolution}
                \textbf{\ref{ex:2A}.} Analisando a equação apresentada:
                    \begin{equation*}
                        y(k+1) -ay(k) = u(k)
                        \quad
                        \text{onde: }
                        \begin{cases}
                            u(k) = 2r(k) \mu(k)\\
                            a    = 0.2\\
                            y(0) = 0
                        \end{cases}
                    \end{equation*}
                Desta forma propõem-se a seguinte solução:
                    \begin{equation*}
                        y(k) = y_{h}(k) + y_{p}(k)
                        \quad
                        \text{onde: }
                        \begin{cases}
                            y_{h}(k), & \text{Solução Homogênea}\\
                            y_{p}(k), & \text{Solução Particular}\\
                        \end{cases}
                    \end{equation*}
                Primeiramente propõem-se a seguinte solução homogênea $y_{h}(k) = C\lambda^{k}$:
                    \begin{align}
                        C\lambda^{k + 1} - aC\lambda^{k} &= 0 \nonumber\\
                        C\lambda^{k} \underbrace{(\lambda - a)}_{=0} &= 0 \nonumber\\
                        \Aboxed{\lambda &= a}
                    \end{align}
                Desta forma obtêm-se a seguinte solução homogênea $y_{h}(k) = C0.2^{k}$.
                \\\\
                Na sequência será necessário analisar a solução particular. Nota-se que $r(k)$ é a função rampa e $r(k)\mu(k)$ será um polinômio descrito para $k\ge0$. Desta forma, propõem-se a seguinte solução particular $y_{p}(k) = Br(k) + D$:
                    \begin{align}
                        Br(k+1) + C - aBr(k) - aC &= 2r(k)\mu(k) & \text{Considerando \texttt{Heaviside}}\nonumber\\
                        B(k+1)  + C - aBk - aC    &= 2k\nonumber\\
                        \underbrace{(B-aB-2)}_{=0}k + \underbrace{(B+C-aC)}_{=0} &= 0\nonumber
                    \end{align}
                Onde:
                    \begin{equation*}
                        B=\frac{2}{1-a}
                        \quad\text{e}\quad
                        C=\frac{-B}{1-a}=\frac{-2}{(1-a)^2}
                        \qquad\text{logo:}\quad
                        \boxed{B=2.5}\quad\text{e}\quad\boxed{C=-3.125}
                    \end{equation*}
                Desta forma obtêm-se a seguinte solução particular $y_{p}(k) = 2.5k - 3.125$.
                \\\\
                Finalmente têm-se:
                    \begin{align}
                        y(k) = y_{h}(k) + y_{p}(k) &= C0.2^{k} + 2.5k - 3.125 & \text{Aplica-se Condições Iniciais Nulas}\nonumber\\
                        y(0) = 0 &= C - 3.125\nonumber\\
                        C &= 3.125\nonumber\\
                        \Aboxed{y(k) &= 3.125 (0.2)^{k} + 2.5k - 3.125}\label{eq:1A}
                    \end{align}
            \end{resolution}
            \begin{resolution}
                Obtêm-se o seguinte gráfico a partir da Equação (\ref{eq:1A}):
                \begin{figure}[H]
                    \centering
                    \includegraphics[height = 7cm]{es601_ex04_im05.png}
                    \caption{Gráfico \texttt{stairs}}
                \end{figure}
                Implementa-se o seguinte algoritmo:
                \begin{scriptsize}
                    \myOctave
                    \lstinputlisting{es601_ex04_02m.m}
                \end{scriptsize}
            \end{resolution}
\newpage
            \begin{resolution}
                \textbf{\ref{ex:2B}.}  Analisando a equação apresentada:
                    \begin{equation*}
                        y(k+1) -ay(k) = 0
                        \quad
                        \text{onde: }
                        \begin{cases}
                            a    = 0.2\\
                            y(0) = 5
                        \end{cases}
                    \end{equation*}
                Desta forma propõem-se apenas a solução homogênea $y_{h}(k) = C\lambda^{k}$:
                    \begin{align}
                        C\lambda^{k + 1} - aC\lambda^{k} &= 0 \nonumber\\
                        C\lambda^{k} \underbrace{(\lambda - a)}_{=0} &= 0 \nonumber\\
                        \Aboxed{\lambda &= a}
                    \end{align}
                Desta forma obtêm-se a seguinte solução homogênea $y_{h}(k) = C0.2^{k}$.
                \\\\
                Finalmente têm-se:
                    \begin{align}
                        y(k) = y_{p}(k) &= C0.2^{k} & \text{Aplica-se Condições Iniciais Nulas}\nonumber\\
                        y(0) = 5 &= C \nonumber\\
                        \Aboxed{y(k) &= 5 (0.2)^{k}}
                    \end{align}
                Obtêm-se o seguinte gráfico a partir da Equação (\ref{eq:1A}):
                    \begin{figure}[H]
                        \centering
                        \includegraphics[height = 7cm]{es601_ex04_im06.png}
                        \caption{Gráfico \texttt{stairs} Entrada Nula}
                    \end{figure}
            \end{resolution}
\end{document}