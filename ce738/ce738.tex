\documentclass{article}
\usepackage{tpack}

\title{CE738 - Economia para Engenharia}
\author{Guilherme Nunes Trofino}
\authorRA{217276}
\project{Resumo Teórico}

\begin{document}
    \maketitle
\newpage

    \tableofcontents
\newpage

    \section{Introdução}
        \paragraph{Apresentação}Neste documento será descrito as informações necessárias para compreensão e solução de exercícios relacionados a disciplina \thetitle . Note que este documento são notas realizadas por \theauthor , em \today.

        \paragraph{Definição}Economia é um estudo social da humanidade nos negócios comuns da vida que estuda administração dos recursos escassos entre usos alternativos e fins competitivos tanto no nível de indivídu quanto da sociedade em seu conjunto.

        \subsection{Sistemas Econômicos}
            \paragraph{Definição}Forma política, social e econômica de organização de uma determinada sociedade que envolverá os seguintes elementos básicos
                \begin{enumerate}[noitemsep, rightmargin = \leftmargin]
                    \item \textbf{Agentes Econômicos:} Responsáveis pela movimentação do mercado através de compra e venda de produtos e serviços;

                    \item \textbf{Instituições:} Responsáveis por oferecer produtos e serviços ou regulando o funcionamento do mercado;

                    \item \textbf{Recursos Produtivos:} Produtos ou serviços disponíveis ao consumo ou a utilização;
                \end{enumerate}
            Normalmente classificados nas seguintes categorias:
                \begin{multicols}{2}
                    \begin{enumerate}[rightmargin = \leftmargin]
                        \item \textbf{Capitalista:} Economia de mercado;
                            \begin{enumerate}[noitemsep]
                                \item \texttt{Propriedade:} Privada dos meios de produção;
                                \item \texttt{Organização:} Regida pelas forças do mercado;
                            \end{enumerate}

                        \columnbreak

                        \item \textbf{Socialista:} Economia centralizada;
                            \begin{enumerate}[noitemsep]
                                \item \texttt{Propriedade:} Coletiva dos meios de produção;
                                \item \texttt{Organização:} Regida por um órgão central;
                            \end{enumerate}
                    \end{enumerate}
                \end{multicols}\noindent
            Cada produto ou serviço disponível no mercado terá seu valor determinando por uma série de fatores entre os quais os enunciados abaixo:
                \begin{enumerate}
                    \item \textbf{Raridade:}
                        \begin{enumerate}[noitemsep]
                            \item \texttt{Livres:};
                            \item \texttt{Econômicos:};
                        \end{enumerate}

                    \item \textbf{Natureza:}
                        \begin{enumerate}[noitemsep]
                            \item \texttt{Materiais:};
                            \item \texttt{Imateriais:};
                        \end{enumerate}

                    \item \textbf{Destino:}
                        \begin{enumerate}[noitemsep]
                            \item \texttt{Consumo:};
                            \item \texttt{Capital:};
                            \item \texttt{Intermediários:} Consumidos ao longo da cadeia produtiva;
                        \end{enumerate}

                    \item \textbf{Origem:}
                        \begin{enumerate}[noitemsep]
                            \item \texttt{Privado:};
                            \item \texttt{Público:} Gerenciado por instituições públicas;
                        \end{enumerate}
                \end{enumerate}

        \subsection{Microeconomia}
            \paragraph{Definição}Estudo da Teoria dos Preços para determinar o funcionamento de um sistema econômico determinando como será o custo, a oferta e demanda de um certo produto ou serviço, baseada nas seguintes divisões:
                \begin{enumerate}[noitemsep, rightmargin = \leftmargin]
                    \item \textbf{Teoria da Demanda:} Necessidade de consumo de bens;

                    \item \textbf{Teoria da Oferta:} Organização das empresas para criação de bens;

                    \item \textbf{Análise das Estrutras de Mercado:} Análise do funcionamento dos agentes envolvidos com os seguintes pressupostos básicos:
                        \begin{enumerate}[noitemsep, rightmargin = \leftmargin]
                            \item \texttt{Coeteris Paribus:} Desconsidera os elementos não relevantes a análise dos aspectos em questão; 
                            \item \texttt{Preços Relativos:} Comparação entre preços de um bem ou serviço;
                            \item \texttt{Princípio da Racionalidade:} Toda empresa terá objetivos racionnais, otimizando seus fatores para maximação de seus lucros;
                        \end{enumerate}
                \end{enumerate}

            \paragraph{Teorias e Modelos}Na economia, explicação e previsão baseiam-se em teorias desenvolvidades a partir de fenômenos observados com representações matemáticas de um empresa, um mercado ou alguma outra entidade.

            \paragraph{Aplicação}Cada entidade poderá utilizar uma análise da economia para determinar planos de ação para atingir seus objetivos com base em metas e variáveis locais.

        \subsubsection{Mercados}
            \paragraph{Definição}Grupos de compradores e vendedores que através de interações determinam o preço de um produto ou serviço.

        \subsubsection{Demanda}
            \paragraph{Definição}Necessidade de consumo pelos indivíduos que pertencem a sociedade com base nos seguintes princípios:
                \begin{enumerate}[rightmargin = \leftmargin]
                    \item \textbf{Racionalidade:} Todo indivíduo terá objetivos racionnais, otimizando seus fatores para maximação de suas satisfações e necessidades;
                    \item \textbf{Utilidade:}
                        \begin{enumerate}[noitemsep, rightmargin = \leftmargin]
                            \item \texttt{Utilidade Total:} Necessidade absoluta de um bem ou serviço;
                            \item \texttt{Utilidade Marginal:} Necessidade de um bem ou serviço quando já se possui um ou mais deste bem ou serviço. Espera-se que a real necessidade decaia a medida que a quantidade aumenta;
                        \end{enumerate}
                \end{enumerate}
            Nota-se que há uma relação inversa entre \textbf{preço} e \textbf{quantidade} demandada de um bem ou serviço por: Efeito de Novo Consumidor, Efeito de Renda, Efeito de Substituição e Utilidade Marginal Decrescente.

            \paragraph{Bens Substitutos}Produtos que quando uma demanda por um aumenta o preço do outro aumenta.

            \paragraph{Bens Complementares}Produtos que quando o preço de um aumenta a demanda do outro diminui.

            \paragraph{Bens Normais}Produtos que quando a renda aumenta e a demanda aumenta.

            \paragraph{Bens Consumo Saciado}Produtos que quando a renda aumenta e a demanda permanece estável.

            \paragraph{Bens Inferiores}Produtos que quando a renda aumenta e a demanda do diminui.
\end{document}