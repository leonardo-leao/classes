\documentclass{article}
\usepackage{tpack}

\title{CE738 - Economia para Engenharia}
\author{Guilherme Nunes Trofino}
\authorRA{217276}
\project{Resumo Teórico}

\begin{document}
    \maketitle
\newpage

    \tableofcontents
\newpage

    \section{Introdução}
        \paragraph{Apresentação}Neste documento será descrito as informações necessárias para compreensão e solução de exercícios relacionados a disciplina \thetitle . Note que este documento são notas realizadas por \theauthor , em \today.

        \subsection{Economia}
            \paragraph{Definição}Economia é um estudo social da humanidade nos negócios comuns da vida que estuda administração dos recursos escassos entre usos alternativos e fins competitivos tanto no nível de indivídu quanto da sociedade em seu conjunto.

            \subsubsection{Teoria Econômica}
                \paragraph{Definição}Conjunto de leis que explicam o comportamento humano e fazem parte do rol de conhecimentos baseados em argumentos que podem ser classificados como:
                    \begin{enumerate}[noitemsep]
                        \item \textbf{Argumentos Positivos:} "foi, é ou será"; 
                        \item \textbf{Argumentos Normativos:} "deveria ser"; 
                    \end{enumerate}

            \subsubsection{Recurso Econômico}
                \paragraph{Definição}Considera-se durante a análise econômica que os \textbf{Recursos Produtivos} disponíveis para utilização são escassos e que as \textbf{Necessidades Humanas} são ilimitadas, gerando assim uma escassez de bens. Isso implica em estudar como atender as necessidades de uma população determinando o que, quanto, como e para quem produzir.

            \subsubsection{Curva de Transformação}
                \paragraph{Definição}Visualização da correlação entre o consumo de bens distintos ao longo do tempo, buscando relacioná-los e comprende-los ao longo do tempo com os seguintes movimentos esperados:
                    \begin{enumerate}[noitemsep]
                        \item \textbf{Curto Prazo:} há deslocamentos dessa curva sem mudanças em sua forma;
                        \item \textbf{Longo Prazo:} há deslocamentos dessa curva com mudanças em sua forma;
                    \end{enumerate}

        \subsection{Sistemas Econômicos}
            \paragraph{Definição}Forma política, social e econômica de organização de uma determinada sociedade que envolverá os seguintes elementos básicos
                \begin{enumerate}[noitemsep, rightmargin = \leftmargin]
                    \item \textbf{Agentes Econômicos:} Responsáveis pela movimentação do mercado através de compra e venda de produtos e serviços;

                    \item \textbf{Instituições:} Responsáveis por oferecer produtos e serviços ou regulando o funcionamento do mercado;

                    \item \textbf{Recursos Produtivos:} Produtos ou serviços disponíveis ao consumo ou a utilização;
                \end{enumerate}
            Normalmente classificados nas seguintes categorias:
                \begin{multicols}{2}
                    \begin{enumerate}[rightmargin = \leftmargin]
                        \item \textbf{Capitalista:} Economia de mercado;
                            \begin{enumerate}[noitemsep]
                                \item \texttt{Propriedade:} Privada dos meios de produção;
                                \item \texttt{Organização:} Regida pelas forças do mercado;
                            \end{enumerate}

                        \columnbreak

                        \item \textbf{Socialista:} Economia centralizada;
                            \begin{enumerate}[noitemsep]
                                \item \texttt{Propriedade:} Coletiva dos meios de produção;
                                \item \texttt{Organização:} Regida por um órgão central;
                            \end{enumerate}
                    \end{enumerate}
                \end{multicols}\noindent
            Cada produto ou serviço disponível no mercado terá seu valor determinando por uma série de fatores entre os quais os enunciados abaixo:
                \begin{enumerate}
                    \item \textbf{Raridade:}
                        \begin{enumerate}[noitemsep]
                            \item \texttt{Livres};
                            \item \texttt{Econômicos};
                        \end{enumerate}

                    \item \textbf{Natureza:}
                        \begin{enumerate}[noitemsep]
                            \item \texttt{Materiais:};
                            \item \texttt{Imateriais:};
                        \end{enumerate}

                    \item \textbf{Destino:}
                        \begin{enumerate}[noitemsep]
                            \item \texttt{Consumo};
                            \item \texttt{Capital};
                            \item \texttt{Intermediários:} Consumidos ao longo da cadeia produtiva;
                        \end{enumerate}

                    \item \textbf{Origem:}
                        \begin{enumerate}[noitemsep]
                            \item \texttt{Privado};
                            \item \texttt{Público:} Gerenciado por instituições públicas;
                        \end{enumerate}
                \end{enumerate}
            Considera-se que o processo de produção fundamenta-se na mobilização de um conjunto dos seguintes fatores:
                \begin{enumerate}[noitemsep]
                    \item \textbf{Capital};
                    \item \textbf{Terra};
                    \item \textbf{Trabalho};
                \end{enumerate}

        \subsection{Correntes da Ciência Econômica}
            \paragraph{Definição}Economia trata-se de um estudo de sociedade e, portanto, há diferentes formas de interpretá-la e, consequentemente, há diferentes formas de estudá-la. Assim, entre as principais linhas de pensamento resalta-se as seguintes:
            % \begin{table}[H]
            %     \centering  
            %     \begin{tabular}[]{c|ccccccc}\hline
            %                            & Classical            & Neoclassical & Marxist & Developmentalist & Schumpeterian & Keynesian & Institutionalist
            %         Economy Made of    & Classes              & Individuals\\
            %         Individuals are    & Selfish Rational     & Selfish Rational\\
            %         World is           & Certain              & Certain with Risk\\
            %         Domain of Economy  & Production           & Exchange Consumption\\
            %         Economy Changes by & Capital Accumulation & Individual Choises\\
            %         Policy Advices     & Free Market          & Free Market\\\hline
            %     \end{tabular}
            % \end{table}

    \section{Microeconomia}

        \subsection{Teoria e Modelagem}
            \paragraph{Definição}Elaboração de hipótese para determinar o funcionamento de um sistema econômico, determinando como será o custo, a oferta e demanda de um certo produto ou serviço baseada nas seguintes divisões:
                \begin{enumerate}[noitemsep, rightmargin = \leftmargin]
                    \item \textbf{Teoria da Demanda:} Necessidade de consumo de bens;

                    \item \textbf{Teoria da Oferta:} Organização das empresas para criação de bens;

                    \item \textbf{Análise das Estrutras de Mercado:} Análise do funcionamento dos agentes envolvidos com os seguintes pressupostos básicos:
                        \begin{enumerate}[noitemsep, rightmargin = \leftmargin]
                            \item \texttt{Coeteris Paribus:} Desconsidera elementos não relevantes a análise em questão; 
                            \item \texttt{Preços Relativos:} Comparação entre preços de um bem ou serviço;
                            \item \texttt{Racionalidade:} Objetivos racionnais, otimizando fatores para maximação de benefício;
                        \end{enumerate}
                \end{enumerate}

                \paragraph{Aplicação}Cada entidade poderá utilizar uma análise microeconômica para determinar planos de ação para atingir seus objetivos com base em metas e variáveis locais.

        \subsubsection{Mercados}
            \paragraph{Definição}Grupos de compradores e vendedores que através de interações determinam o preço de um produto ou serviço.

        \subsection{Demanda}
            \paragraph{Definição}Necessidade de consumo pelos indivíduos que pertencem a sociedade com base nos seguintes princípios:
                \begin{enumerate}[rightmargin = \leftmargin]
                    \item \textbf{Racionalidade:} Maximação de suas satisfações e necessidades;
                    \item \textbf{Utilidade:}
                        \begin{enumerate}[noitemsep, rightmargin = \leftmargin]
                            \item \texttt{Total:} Necessidade absoluta de um bem ou serviço;
                            \item \texttt{Marginal:} Necessidade de um bem ou serviço quando já se possui um ou mais deste bem ou serviço. Espera-se que a real necessidade decaia a medida que a quantidade aumenta;
                        \end{enumerate}
                \end{enumerate}
            Isso normalmente será descrito matematicamente aravés da seguinte equação:
                \begin{equation}
                    \boxed{
                        D_{\text{x}} =
                        f(
                            P_{\text{x}},
                            P_{\text{i}},
                            R,
                            G,
                            E
                        )
                    }
                \end{equation}
            Onde:
                \begin{enumerate}[noitemsep]
                    \item $D_{\text{x}}$, \textbf{Quantidade Demandada} do bem x;
                    \item $P_{\text{x}}$, \textbf{Preço do Bem} x;
                    \item $P_{\text{i}}$, \textbf{Preço de Bens Relacionados} de x;
                    \item $R$, \textbf{Renda do Consumidor} de x;
                    \item $G$, \textbf{Gosto do Consumidor} de x;
                    \item $E$, \textbf{Expectativas do Consumidor} de x;
                \end{enumerate}

        \subsubsection{Oferta e Demanda}
            \paragraph{Definição}Nota-se a seguinte relação entre a \texttt{Oferta} e a \texttt{Demanda} de um produto qualquer:
                \begin{equation}
                    \boxed{
                        P_{\text{x}}\uparrow
                        \;
                        D_{\text{x}}\downarrow
                    }
                    \qquad
                    \boxed{
                        P_{\text{x}}\downarrow
                        \;
                        D_{\text{x}}\uparrow
                    }
                \end{equation}
            Isso ocorre em virtude dos seguintes parâmetros:
                \begin{enumerate}[noitemsep]
                    \item \textbf{Efeito Novo Consumidor};
                    \item \textbf{Efeito Renda};
                    \item \textbf{Efeito Substituição};
                    \item \textbf{Utilidade Marginal Decrescente};
                \end{enumerate}

        \subsubsection{Bens Substiutos}
            \paragraph{Definição}Produtos que apresentam a seguinte relação entre a \texttt{Oferta} e a \texttt{Demanda}:
                \begin{equation}
                    \boxed{
                        P_{\text{i}}\uparrow
                        \;
                        D_{\text{x}}\uparrow
                    }
                    \qquad
                    \boxed{
                        P_{\text{i}}\downarrow
                        \;
                        D_{\text{x}}\downarrow
                    }
                \end{equation}

        \subsubsection{Bens Complementares}
            \paragraph{Definição}Produtos que apresentam a seguinte relação entre a \texttt{Oferta} e a \texttt{Demanda}:
                \begin{equation}
                    \boxed{
                        P_{\text{i}}\uparrow
                        \;
                        D_{\text{x}}\downarrow
                    }
                    \qquad
                    \boxed{
                        P_{\text{i}}\downarrow
                        \;
                        D_{\text{x}}\uparrow
                    }
                \end{equation}

        \subsubsection{Bens Normais}
            \paragraph{Definição}Produtos que apresentam a seguinte relação entre a \texttt{Renda} e a \texttt{Demanda}:
                \begin{equation}
                    \boxed{
                        R\uparrow
                        \;
                        D_{\text{x}}\uparrow
                    }
                    \qquad
                    \boxed{
                        R\downarrow
                        \;
                        D_{\text{x}}\downarrow
                    }
                \end{equation}

        \subsubsection{Bens Consumo Saciado}
            \paragraph{Definição}Produtos que apresentam a seguinte relação entre a \texttt{Renda} e a \texttt{Demanda}:
                \begin{equation}
                    \boxed{
                        R\uparrow
                        \;
                        D_{\text{x}}
                    }
                    \qquad
                    \boxed{
                        R\downarrow
                        \;
                        D_{\text{x}}
                    }
                \end{equation}

        \subsubsection{Bens Inferiores}
            \paragraph{Definição}Produtos que apresentam a seguinte relação entre a \texttt{Renda} e a \texttt{Demanda}:
                \begin{equation}
                    \boxed{
                        R\uparrow
                        \;
                        D_{\text{x}}\downarrow
                    }
                    \qquad
                    \boxed{
                        R\downarrow
                        \;
                        D_{\text{x}}\uparrow
                    }
                \end{equation}

        \subsection{Oferta}
            \paragraph{Definição}Quantidade de um bem ou serviço que um único produtor deseja vender no mercado, por unidade de tempo. Isso normalmente será descrito matematicamente aravés da seguinte equação:
                \begin{equation}
                    \boxed{
                        O_{\text{x}} =
                        f(
                            P_{\text{x}},
                            P_{\text{i}},
                            P_{\text{p}},
                            T,
                            E,
                            C
                        )
                    }
                \end{equation}
            Onde:
                \begin{enumerate}[noitemsep]
                    \item $O_{\text{x}}$, \textbf{Quantidade Ofertada} do bem x;
                    \item $P_{\text{x}}$, \textbf{Preço do Bem} x;
                    \item $P_{\text{i}}$, \textbf{Preço de Bens Relacionados} de x;
                    \item $P_{\text{p}}$, \textbf{Preço de Produção} de x;
                    \item $T$, \textbf{Tecnologias} de x;
                    \item $E$, \textbf{Expectativas} de x;
                    \item $C$, \textbf{Condições Climáticas} de x;
                \end{enumerate}

        \subsubsection{Preço e Oferta}
            \paragraph{Definição}Nota-se a seguinte relação entre o \texttt{Preço} e a \texttt{Oferta} de um produto qualquer:
                \begin{equation}
                    \boxed{
                        P_{\text{x}}\uparrow
                        \;
                        O_{\text{x}}\uparrow
                    }
                    \qquad
                    \boxed{
                        P_{\text{x}}\downarrow
                        \;
                        O_{\text{x}}\downarrow
                    }
                \end{equation}

        \subsubsection{Equilíbrio de Mercado}
            \paragraph{Definição}Preço na economia de mercado determinado tanto pela oferta quanto pela demanda.


        \subsection{Elasicidade}
            \paragraph{Definição}Sensibilidade dos agentes econômicos em face de qualquer alteraÇão nas condições de mercado demonstrando o grau de reações a ações e situações.

        \subsubsection{Elasticidade-Preço Demanda, $E_{\text{D}}$}
            \paragraph{Definição}Variação percentual na quantidade demandada dada a variação no preço do bem obtida pela seguinte equação:
                \begin{equation}
                    \boxed{
                        E_{\text{D}} = 
                        \frac{\Delta\%\;Q_{\text{x}}}{\Delta\%\;P_{\text{x}}}
                    }
                \end{equation}
            Onde:
                \begin{enumerate}[noitemsep]
                    \item \textbf{Demanda Elástica:} $|E_{\text{D}}| > 1$;
                        \begin{enumerate}
                            \item \texttt{Demanda Perfeitamente Elástica:} $E_{\text{D}} = \infty$;
                        \end{enumerate}

                    \item \textbf{Demanda Unitária:} $|E_{\text{D}}| = 1$;

                    \item \textbf{Demanda Inelástica:} $|E_{\text{D}}| < 1$;
                        \begin{enumerate}
                            \item \texttt{Demanda Perfeitamente Inelástica:} $E_{\text{D}} = 0$;
                        \end{enumerate}
                \end{enumerate}

        \subsubsection{Elasticidade-Preço Cruzada da Demanda, $E_{\text{xy}}$}
            \paragraph{Definição}Variação percentual na quantidade demandada dada a variação da demanda de outro bem obtida pela seguinte equação:
                \begin{equation}
                    \boxed{
                        E_{\text{xy}} = 
                        \frac{\Delta\%\;Q_{\text{x}}}{\Delta\%\;P_{\text{y}}} =
                        \frac{\Delta Q_{\text{x}}}{\Delta P_{\text{y}}} \frac{P_{\text{y}}}{Q_{\text{x}}}
                    }
                \end{equation}
            Onde:
                \begin{enumerate}[noitemsep]
                    \item \textbf{Bens Substitutos:} $E_{\text{xy}} > 0$;

                    \item \textbf{Bens Independentes:} $E_{\text{xy}} = 0$;

                    \item \textbf{Bens Complementares:} $E_{\text{xy}} < 0$;
                \end{enumerate}

        \subsubsection{Elasticidade-Renda da Demanda, $E_{\text{R}}$}
            \paragraph{Definição}Variação percentual na quantidade demandada dada a variação da renda obtida pela seguinte equação:
                \begin{equation}
                    \boxed{
                        E_{\text{R}} = 
                        \frac{\Delta\%\;Q_{\text{x}}}{\Delta\%\;R_{\text{x}}} =
                        \frac{\Delta Q_{\text{x}}}{\Delta R_{\text{x}}} \frac{R_{\text{x}}}{Q_{\text{x}}}
                    }
                \end{equation}
            Onde:
                \begin{enumerate}[noitemsep]
                    \item \textbf{Bens Inferiores:} $E_{\text{R}} < 0$;

                    \item \textbf{Bens de Consumo Saciado:} $E_{\text{xy}} = 0$;

                    \item \textbf{Bens Normais:} $0 < E_{\text{xy}} < 1$;

                    \item \textbf{Bens Superiores:} $E_{\text{xy}} > 1$;
                \end{enumerate}

        \subsubsection{Elasticidade-Preço da Oferta, $E_{\text{O}}$}
            \paragraph{Definição}Variação percentual na quantidade ofertada dada a variação no preço do bem obtida pela seguinte equação:
                \begin{equation}
                    \boxed{
                        E_{\text{O}} = 
                        \frac{\Delta\%\;Q_{\text{O}}}{\Delta\%\;P_{\text{O}}} = 
                        \frac{\Delta Q_{\text{O}}}{\Delta P_{\text{O}}}\frac{P_{\text{O}}}{Q_{\text{O}}}
                    }
                \end{equation}
            Onde:
                \begin{enumerate}[noitemsep]
                    \item \textbf{Oferta Elástica:} $E_{\text{O}} > 1$;
                        \begin{enumerate}
                            \item \texttt{Oferta Perfeitamente Elástica:} $E_{\text{O}} = \infty$;
                        \end{enumerate}

                    \item \textbf{Oferta Unitária :} $E_{\text{O}} = 1$;

                    \item \textbf{Oferta Inelástica:} $E_{\text{O}} < 1$;
                        \begin{enumerate}
                            \item \texttt{Oferta Perfeitamente Inelástica:} $E_{\text{O}} = 0$;
                        \end{enumerate}
                \end{enumerate}

        \subsection{Estruturas de Mercado}
            \paragraph{Definição}Modelos utilizados para compreender o funcionamento e a interação entre os diferentes agentes que fazem parte de uma sociedade. Há diferentes cenários possíveis sendo os principais os descritos a seguir:

                \begin{table}[H]
                    \centering  
                    \begin{tabular}[]{l|lllll}hline
                                       & Concorrência Perfeita & Monopólio       & Oligopólio   & Concorrência Monopolística\\\hline
                        Qntd. Empresas & Enorme                & Unitário        & Pequeno      & Elevado\\
                        Produto        & Homogêneo             & Não Substitutos & Homogêneo    & Diferenciado\\
                        Controle       & Não Há                & Elevado         & Dificultado  & Baixo\\
                        Concorrência   & Não é Possível        & Não Há          & Intensa      & Intensa\\
                        Ingresso       & Não Há Barreiras      & Há Barreiras    & Há Barreiras & Não Há Barreiras\\\hline
                    \end{tabular}
                \end{table}

        \subsubsection{Concorrência Perfeita}
            \paragraph{Definição}Único requisito necessário é apresentar uma quantidade enorme de empresas no mercado onde a diferenciação entre produtos não afeta. \textbf{Exemplo:} Indústria de Jogos, Construção Civil.

        \subsubsection{Monopólio}
            \paragraph{Definição}. \textbf{Exemplo:} Estatais de Petróleo, Indústria Farmaceútica.

        \subsubsection{Oligopólio}
            \paragraph{Definição}Não há uma definição clara na literatura. Neste cenário há possibilidade da formação de cartéis entre as empresas dominantes pelas barreiras diretas e indiretas presentes. \textbf{Exemplo:} Indústria Automotiva.

        \subsubsection{Monopsônio}
            \paragraph{Definição}Mercado em que haja apenas um comprador para muios vendedores, também referido como um monopólio para compra de insumos.

        \subsubsection{Oligopsônio}
            \paragraph{Definição}Mercado em que haja poucos compradores para muitos vendedores, também referido como um oligopólio para compra de insumos. \textbf{Exemplo:} Indústria Automotiva para Autopeças.

        \subsubsection{spam}

            concetracao de mercado
                medida comument utilizada para verificar o grau de concentracao economica, indicide de concentracao das 4 mairores empresa 
                    <40\% é concorrencial
                    <60\% é concentrado
                    >80\% é altamente concentrado

            governo e abuso de poderá
                sistema brasileiro de defesa da concorrencia
                    cade
                        cuidado da concorrencia
                    seprac
                    seaae

            livre concorrencia
                eficiencia alocativa
                eficiencia produtiva
                capacidade de inovacao dos mercados

            reducao da concorrencia
                ma conduta empresarial
                eventuais falhas da intervencao estatal na economia

            limitacao da concorrencia
                maximizacao dos lucros
                    em detrimentos dos clientes
                diminuicação dos riscos
                    perda da qualidade
                    perda de inovacao
                
            atuacao do estado
                preventiva
                    prevenir problemas
                repressiva
                    interferir quando há problemas
                proativa
                    atencipar problemas

    \section{Macroeconomia}

    \section{Comércio Internacional}

    \section{Emprego e Desigualdade de Renda}

    \section{Desenvolvimento Econômico}
\end{document}