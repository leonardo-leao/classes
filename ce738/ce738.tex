\documentclass{article}
\usepackage{tpack}

\title{CE738 - Economia para Engenharia}
\author{Guilherme Nunes Trofino}
\authorRA{217276}
\project{Resumo Teórico}

\begin{document}
    \maketitle
\newpage

    \tableofcontents
\newpage

    \section{Introdução}
        \paragraph{Apresentação}Neste documento será descrito as informações necessárias para compreensão e solução de exercícios relacionados a disciplina \thetitle . Note que este documento são notas realizadas por \theauthor , em \today.

        \subsection{Sistemas Econômicos}
            \paragraph{Definição}Forma política, social e econômica de organização de uma determinada sociedade que envolverá os seguintes elementos básicos
                \begin{enumerate}[noitemsep, rightmargin = \leftmargin]
                    \item \textbf{Agentes Econômicos:};

                    \item \textbf{Instituições:} Conjunto de instituições políticas para disciplinar cada individuo;

                    \item \textbf{Recursos Produtivos:};
                \end{enumerate}
            Normalmente classificados nas seguintes categorias:
                \begin{multicols}{2}
                    \begin{enumerate}[rightmargin = \leftmargin]
                        \item \textbf{Capitalista:}
                            \begin{enumerate}[noitemsep]
                                \item \texttt{Propriedade:} Privada dos meios de produção;
                                \item \texttt{}
                            \end{enumerate}

                        \columnbreak

                        \item \textbf{Socialista:}
                            \begin{enumerate}[noitemsep]
                                \item \texttt{Propriedade:} Coletiva dos meios de produção;
                                \item \texttt{}
                            \end{enumerate}
                    \end{enumerate}
                \end{multicols}
\end{document}