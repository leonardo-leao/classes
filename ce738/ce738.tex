\documentclass{article}
\usepackage{tpack}

\title{CE738 - Economia para Engenharia}
\author{Guilherme Nunes Trofino}
\authorRA{217276}
\project{Resumo Teórico}

\begin{document}
    \maketitle
\newpage

    \tableofcontents
\newpage

    \section{Introdução}
        \paragraph{Apresentação}Neste documento será descrito as informações necessárias para compreensão e solução de exercícios relacionados a disciplina \thetitle . Note que este documento são notas realizadas por \theauthor , em \today.

        \subsection{Economia}
            \paragraph{Definição}Economia é um estudo social da humanidade nos negócios comuns da vida que estuda administração dos recursos escassos entre usos alternativos e fins competitivos tanto no nível de indivídu quanto da sociedade em seu conjunto.

        \subsection{Sistemas Econômicos}
            \paragraph{Definição}Forma política, social e econômica de organização de uma determinada sociedade que envolverá os seguintes elementos básicos
                \begin{enumerate}[noitemsep, rightmargin = \leftmargin]
                    \item \textbf{Agentes Econômicos:} Responsáveis pela movimentação do mercado através de compra e venda de produtos e serviços;

                    \item \textbf{Instituições:} Responsáveis por oferecer produtos e serviços ou regulando o funcionamento do mercado;

                    \item \textbf{Recursos Produtivos:} Produtos ou serviços disponíveis ao consumo ou a utilização;
                \end{enumerate}
            Normalmente classificados nas seguintes categorias:
                \begin{multicols}{2}
                    \begin{enumerate}[rightmargin = \leftmargin]
                        \item \textbf{Capitalista:} Economia de mercado;
                            \begin{enumerate}[noitemsep]
                                \item \texttt{Propriedade:} Privada dos meios de produção;
                                \item \texttt{Organização:} Regida pelas forças do mercado;
                            \end{enumerate}

                        \columnbreak

                        \item \textbf{Socialista:} Economia centralizada;
                            \begin{enumerate}[noitemsep]
                                \item \texttt{Propriedade:} Coletiva dos meios de produção;
                                \item \texttt{Organização:} Regida por um órgão central;
                            \end{enumerate}
                    \end{enumerate}
                \end{multicols}\noindent
            Cada produto ou serviço disponível no mercado terá seu valor determinando por uma série de fatores entre os quais os enunciados abaixo:
                \begin{enumerate}
                    \item \textbf{Raridade:}
                        \begin{enumerate}[noitemsep]
                            \item \texttt{Livres:};
                            \item \texttt{Econômicos:};
                        \end{enumerate}

                    \item \textbf{Natureza:}
                        \begin{enumerate}[noitemsep]
                            \item \texttt{Materiais:};
                            \item \texttt{Imateriais:};
                        \end{enumerate}

                    \item \textbf{Destino:}
                        \begin{enumerate}[noitemsep]
                            \item \texttt{Consumo:};
                            \item \texttt{Capital:};
                            \item \texttt{Intermediários:} Consumidos ao longo da cadeia produtiva;
                        \end{enumerate}

                    \item \textbf{Origem:}
                        \begin{enumerate}[noitemsep]
                            \item \texttt{Privado:};
                            \item \texttt{Público:} Gerenciado por instituições públicas;
                        \end{enumerate}
                \end{enumerate}

    \section{Microeconomia}
        \paragraph{Definição}Estudo da Teoria dos Preços para determinar o funcionamento de um sistema econômico determinando como será o custo, a oferta e demanda de um certo produto ou serviço, baseada nas seguintes divisões:
            \begin{enumerate}[noitemsep, rightmargin = \leftmargin]
                \item \textbf{Teoria da Demanda:} Necessidade de consumo de bens;

                \item \textbf{Teoria da Oferta:} Organização das empresas para criação de bens;

                \item \textbf{Análise das Estrutras de Mercado:} Análise do funcionamento dos agentes envolvidos com os seguintes pressupostos básicos:
                    \begin{enumerate}[noitemsep, rightmargin = \leftmargin]
                        \item \texttt{Coeteris Paribus:} Desconsidera os elementos não relevantes a análise dos aspectos em questão; 
                        \item \texttt{Preços Relativos:} Comparação entre preços de um bem ou serviço;
                        \item \texttt{Princípio da Racionalidade:} Toda empresa terá objetivos racionnais, otimizando seus fatores para maximação de seus lucros;
                    \end{enumerate}
            \end{enumerate}

        \paragraph{Teorias e Modelos}Na economia, explicação e previsão baseiam-se em teorias desenvolvidades a partir de fenômenos observados com representações matemáticas de um empresa, um mercado ou alguma outra entidade.

        \paragraph{Aplicação}Cada entidade poderá utilizar uma análise da economia para determinar planos de ação para atingir seus objetivos com base em metas e variáveis locais.

    \subsection{Mercados}
        \paragraph{Definição}Grupos de compradores e vendedores que através de interações determinam o preço de um produto ou serviço.

    \subsection{Demanda}
        \paragraph{Definição}Necessidade de consumo pelos indivíduos que pertencem a sociedade com base nos seguintes princípios:
            \begin{enumerate}[rightmargin = \leftmargin]
                \item \textbf{Racionalidade:} Todo indivíduo terá objetivos racionnais, otimizando seus fatores para maximação de suas satisfações e necessidades;
                \item \textbf{Utilidade:}
                    \begin{enumerate}[noitemsep, rightmargin = \leftmargin]
                        \item \texttt{Utilidade Total:} Necessidade absoluta de um bem ou serviço;
                        \item \texttt{Utilidade Marginal:} Necessidade de um bem ou serviço quando já se possui um ou mais deste bem ou serviço. Espera-se que a real necessidade decaia a medida que a quantidade aumenta;
                    \end{enumerate}
            \end{enumerate}
        Nota-se que há uma relação inversa entre \textbf{preço} e \textbf{quantidade} demandada de um bem ou serviço por: Efeito de Novo Consumidor, Efeito de Renda, Efeito de Substituição e Utilidade Marginal Decrescente.

        \paragraph{Bens Substitutos}Produtos que quando uma demanda por um aumenta o preço do outro aumenta.

        \paragraph{Bens Complementares}Produtos que quando o preço de um aumenta a demanda do outro diminui.

        \paragraph{Bens Normais}Produtos que quando a renda aumenta e a demanda aumenta.

        \paragraph{Bens Consumo Saciado}Produtos que quando a renda aumenta e a demanda permanece estável.

        \paragraph{Bens Inferiores}Produtos que quando a renda aumenta e a demanda do diminui.

        \paragraph{Equilíbrio de Mercado}Preço na economia de mercado é determinado tanto pela oferta quanto pela procura. Balanço ideal entre oferta e demanda.

        \subsection{Elasicidade}
            \paragraph{Definição}Sensibilidade dos agentes econômicos em face de qualquer alteraÇão nas condições de mercado demonstrando o grau de reações a ações e situações.

            \subsubsection{Elasticidade Demanda}
                \paragraph{Elasticidade Preço da Demanda}Variação percentual na quantidade demandada dada a variação no preço do bem.
                \paragraph{Elasticidade Renda da Demanda}Variação percentual na quantidade demandada dada a variação da renda.
                \paragraph{Elasticidade Preço Cruzada da Demanda}Variação percentual na quantidade demandada dada a variação da demanda de outro bem.

            estruturas de mercados utilizam modelos para compreender o funcionamento d
            concorrencia perfeia
                numero enorme de empresas unico requisito necessario
                diferencia entre produtos não afeta
                empresas de jogos
            monopolio
            oligopolio
                avaicao
            concorrencia monopolistica
            atomizado grande quantidade de empresas, empresas como atomos
            monopolio industria farmaceutico
                estatais de petroleo estrategico e seguracao nacional saneamento
                petroleo
            oligopolio
                pequeno grupo de empresas dominando
                não definido pela literatura
                poucas dominando
                empresas satelites, seguem as estrageias das principais
                possibilidade de carterizacao
                industtria automotiva
                barreiras invisiveis tetos de vidro de custos para competitidade
            concorrencia monopolistica
                entre perfeita e monopolio
                varias empresas
                há pequena margem de manobra
                há produtos substitutos
            
            estrutura no mercado de fatores de produçao
            concorrencia perfeita
                mercado da construçao civil, baixa capacitacao e baixo salario
            monopolio
                um responsavel pela venda de ensumos
                demanda derivada, depende da demanda de bens e servicos
            oligopolio
                poucas empresas produzindo um mesmo ensumo
            Monopsonio
                um comprador para muitos vendedores
                monopolio na compra de ensumos
            Oligopsonio
                poucos compadores para muitos vendedores
                oligopolio para compra de ensumos
                industria automobilistica para autopecas
            monopolio bilateral

            concetracao de mercado
                medida comument utilizada para verificar o grau de concentracao economica, indicide de concentracao das 4 mairores empresa 
                    <40\% é concorrencial
                    <60\% é concentrado
                    >80\% é altamente concentrado

            governo e abuso de poderá
                sistema brasileiro de defesa da concorrencia
                    cade
                        cuidado da concorrencia
                    seprac
                    seaae

            livre concorrencia
                eficiencia alocativa
                eficiencia produtiva
                capacidade de inovacao dos mercados

            reducao da concorrencia
                ma conduta empresarial
                eventuais falhas da intervencao estatal na economia

            limitacao da concorrencia
                maximizacao dos lucros
                    em detrimentos dos clientes
                diminuicação dos riscos
                    perda da qualidade
                    perda de inovacao
                
            atuacao do estado
                preventiva
                    prevenir problemas
                repressiva
                    interferir quando há problemas
                proativa
                    atencipar problemas

    \section{Macroeconomia}

    \section{Comércio Internacional}

    \section{Emprego e Desigualdade de Renda}

    \section{Desenvolvimento Econômico}
\end{document}