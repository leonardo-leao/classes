\documentclass{article}

\usepackage[a4paper, hmargin={20mm, 20mm}, vmargin={25mm, 30mm}]{geometry}
\usepackage[utf8]{inputenc}
\usepackage[english, main=portuguese]{babel}

\usepackage[hidelinks]{hyperref}
\usepackage{bookmark}
\usepackage{cancel}
\usepackage{comment}

\usepackage{array}
\usepackage{indentfirst}
\usepackage{multicol}
\usepackage{subfiles}

\usepackage{titlesec}

\usepackage{amsmath}
\usepackage{amssymb}
\usepackage{systeme}
\usepackage{float}
\usepackage{enumitem}
\restylefloat{table}

\usepackage{graphicx}
\graphicspath{ {./IMAGENS/} }

% Pacote para a definição de novas cores
\usepackage{xcolor}
% Definindo novas cores
\definecolor{darkgreen}{rgb}{0.0, 0.42, 0.24}
\definecolor{darkpurple}{rgb}{0.74, 0.2, 0.64}
\definecolor{darkblue}{rgb}{0.0, 0.28, 0.67}

% Configurando layout para mostrar codigos Java
\usepackage{listings}


\newcommand{\myStyle}{
\lstset{
    language=Octave,                            % the language of the code
    basicstyle=\ttfamily\small,               % the size of the fonts that are used for the code
    keywordstyle=\color{darkpurple}\bfseries, %
    stringstyle=\color{darkblue},             %
    commentstyle=\color{darkgreen},           %
    morecomment=[s][\color{blue}]{/**}{*/},   %
    extendedchars=true,                       %
    showtabs=false,                           % show tabs within strings adding particular underscores
    showspaces=false,                         % show spaces adding particular underscores
    showstringspaces=false,                   % underline spaces within strings
    numbers=left,                             % where to put the line-numbers
    numberstyle=\tiny\color{gray},            % the style that is used for the line-numbers
    stepnumber=1,                             % the step between two line-numbers. If it's 1, each line will be numbered
    numbersep=5pt,                            % how far the line-numbers are from the code
    frame=single,                             % adds a frame around the code
    rulecolor=\color{black},                  % if not set, the frame-color may be changed on line-breaks within not-black text
    breaklines=true,                          % sets automatic line breaking
    backgroundcolor=\color{white},            % choose the background color
    breakatwhitespace=true,                   % sets if automatic breaks should only happen at whitespace
    breakautoindent=false,                    %
    captionpos=b,                             % sets the caption-position to bottom
    xleftmargin=0pt,                          %
    tabsize=2,                                % sets default tabsize to 2 spaces
}}

%\titleformat{<command>}[<shape>]{<format>}{<label>}{<sep>}{<before-code>}[<after-code>]
\titleformat
{\section} %comand
[block]  %shape
{\normalfont\LARGE} %format
{\thesection. } %label
{0mm} %sep
{} %before-code
[{\titlerule[0.1mm]}] %after-code

\titlespacing*{\section}{0mm}{0mm}{15mm}

\titleformat
{\subsection} %comand
[block]  %shape
{\normalfont\Large} %format
{\thesubsection. } %label
{0mm} %sep
{} %before-code
[] %after-code

\titlespacing*{\subsection}{0mm}{5mm}{2.5mm}


\begin{document}
    \begin{titlepage}
        \begin{center}
            \rule{450pt}{0.5pt}\\[4mm]
            {\Huge MS211 - Cálculo Numérico}\\
            \rule{450pt}{0.5pt}\\[2mm]
            {\Large Laboratório 08}\\[200mm]
            \today\\
            \rule{250pt}{0.5pt}\\
            {\large Guilherme Nunes Trofino}\\
            {\large 217276}\\
        \end{center}
    \end{titlepage}
\newpage

    \section{Questão}
        \subsection{Introdução}
            \paragraph{Problema}Notou-se que a dimensão de certo problema estava relacionada com o tempo necessário para solucioná-lo de acordo com seguinte gráfico:
                \begin{figure}[h]
                    \includegraphics[width = 12cm]{graphic1.png}
                    \centering
                    \caption{Dimensão pelo Tempo de Execução}
                \end{figure}

            \paragraph{Determine}
                \begin{enumerate}[noitemsep]
                    \item Polinômio de grau 4 que melhor se ajusta aos dados obtidos; 
                    \item Estime o tempo necessário para resolver problemas de dimensão 500 e 1000;
                \end{enumerate}

        \subsection{Desenvolvimento}
            \paragraph{Teoria}Suponha que os valores $x_{1}, x_{2}, \dots, x_{n}$ do intervalo $[a,b]$ estejam relacionados respectivamente com $y_{1}, y_{2}, \dots, y_{n}$ de acordo com a seguinte tabela:
                \[
                    \begin{array}{c | c c c c}
                        \text{x} & x_{1} & x_{2} & \cdots & x_{n}\\\hline
                        \text{y} & y_{1} & y_{2} & \cdots & y_{n}\\
                    \end{array}
                \]
            Deseja-se encontrar os coeficientes $\alpha_{1}, \alpha_{2}, \dots, \alpha_{n}$ tais que as funções, preferencialmente contínuas, $g_{1}, g_{2}, \dots, g_{n}$, escolhidas observando o gráfico, satisfaçam a seguinte equação:
                \[\boxed{y(x) \approx \varphi(x) = \alpha_{1} g_{1}(x) + \alpha_{2} g_{2}(x) + \cdots + \alpha_{n} g_{n}(x)}\]

            \paragraph{Mínimos Quadrados}Selecionando as funções $g_{1}, g_{2}, \dots, g_{n}$ de tal forma que o \textbf{Resíduo}, a soma dos quadrados dos desvios, seja mínimo:
                \[\boxed{R(\alpha_{1}, \dots, \alpha_{n}) = \sum_{i=1}^{k} (y_{i} - (\alpha_{1} g_{1}(x_{i}) + \alpha_{2} g_{2}(x_{i}) + \cdots + \alpha_{n} g_{n}(x_{i})))^{2}}\]
            Isto ocorre quando o gradiente $\nabla R (\alpha_{1}, \dots, \alpha_{n}) = 0$. Desta maneira cada derivada parcial deverá ser zero, implicando na seguinte família de equações:
                \[
                    \begin{split}\frac{\partial R}{\partial\alpha_{i}}(\alpha_{1}, \dots, \alpha_{n})
                        &= \sum_{i=1}^{k} -2g_{j}(x_{i})[y_{i} - [\alpha_{1} g_{1}(x_{i}) +
                                                                  \cdots +
                                                                  \alpha_{n} g_{n}(x_{i})]]\\
                        &= -2\sum_{i=1}^{k} y_{i}g_{j}(x_{i}) +
                            2\sum_{i=1}^{k} g_{j}(x_{i})[\alpha_{1} g_{1}(x_{i}) +
                                                         \cdots +
                                                         \alpha_{n} g_{n}(x_{i})]\\
                    \end{split}
                \]
            Desta maneira, o gradiente poderá ser representado matricialmente pelas seguintes matrizes:
                \[\boxed{\nabla R(\alpha_{1}, \dots, \alpha_{n}) = 0 = -2A^{T}y + 2A^{T}A\alpha}\]
                \[
                    \underbrace{
                    \begin{bmatrix}
                        g_{1}(x_{1}) & g_{1}(x_{2}) & \cdots & g_{1}(x_{k})\\
                        g_{2}(x_{1}) & g_{2}(x_{2}) & \cdots & g_{2}(x_{k})\\
                        \vdots       & \vdots       & \ddots & \vdots\\
                        g_{n}(x_{1}) & g_{n}(x_{2}) & \cdots & g_{n}(x_{k})\\
                    \end{bmatrix}}_{A^{T}}
                    \underbrace{
                    \begin{bmatrix}
                        y_{1}\\ y_{2}\\ \vdots\\ y_{k}\\
                    \end{bmatrix}}_{y} =
                    \underbrace{
                    \begin{bmatrix}
                        g_{1}(x_{1}) & g_{1}(x_{2}) & \cdots & g_{1}(x_{k})\\
                        g_{2}(x_{1}) & g_{2}(x_{2}) & \cdots & g_{2}(x_{k})\\
                        \vdots       & \vdots       & \ddots & \vdots\\
                        g_{n}(x_{1}) & g_{n}(x_{2}) & \cdots & g_{n}(x_{k})\\
                    \end{bmatrix}}_{A^{T}}
                    \underbrace{
                    \begin{bmatrix}
                        g_{1}(x_{1}) & g_{2}(x_{1}) & \cdots & g_{n}(x_{1})\\
                        g_{1}(x_{2}) & g_{2}(x_{2}) & \cdots & g_{n}(x_{2})\\
                        \vdots       & \vdots       & \ddots & \vdots\\
                        g_{1}(x_{k}) & g_{2}(x_{k}) & \cdots & g_{n}(x_{k})\\
                    \end{bmatrix}}_{A}
                    \underbrace{
                    \begin{bmatrix}
                        \alpha_{1}\\ \alpha_{2}\\ \vdots\\ \alpha_{n}\\
                    \end{bmatrix}}_{\alpha}
                \]

            \paragraph{Escolha de Funções}Devem ser selecionadas de acordo com o gráfico dos pontos analisados. Funções polinomiais são aproximações suficiente eficientes para maioria dos casos. Desta maneira, temos que:
                \[\boxed{
                    y(x) \approx\underbrace{a_{1} 1}_{\alpha_{0} g_{0}(x)} +
                                \underbrace{a_{2} x}_{\alpha_{1} g_{1}(x)} +
                                \cdots +
                                \underbrace{a_{n-1} x^{n-1}}_{\alpha_{n} g_{n}(x)}
                                \underbrace{a_{n}   x^{n}  }_{\alpha_{n} g_{n}(x)}}\]


        \subsection{Solução}
            \paragraph{Formulação de Ajuste de Dados}Definiu-se que as funções necessárias para aproximar o polinômio de grau 4 seriam as diferentes potências da função. Assim o polinômio de grau 4 que melhor se ajusta as dados obtidos será:
                \[
                    \boxed{
                        \varphi(x) = (0.000074423 \cdot x^{4} +
                                      0.017349587 \cdot x^{3} +
                                      0.162313064 \cdot x^{2}
                                     -2.225595244 \cdot x^{1}
                                     -0.223032780)\cdot 10^{-5}}
                \]
            Considerando essa aproximação são necessários \textbf{68.596} e \textbf{919.33} segundos para solucionar problemas de dimensão 500 e 1000, respectivamente.
                \begin{figure}[h]
                    \includegraphics[width = 12cm]{graphic2.png}
                    \centering
                    \caption{Dimensão pelo Tempo de Execução com Aproximação}
                \end{figure}
\newpage

        \subsection{Códigos}
            \begin{scriptsize}
                \myStyle
                \lstinputlisting{AT8_1.m}
            \end{scriptsize}
\newpage

    \section{Questão}
        \subsection{Introdução}
            \paragraph{Problema}Casos graves de COVID19 por semana em Campinas, representados na tabela abaixo, não contém dados para os dias 27/12, 03/01, 10/01 e 17/01.
                \[
                    \begin{tabular}{c | c c}
                        Data  & Casos\\ \hline
                        02/08 & 170 & $x_{1 }$\\
                        09/08 & 162 & $x_{2 }$\\
                        16/08 & 145 & $x_{3 }$\\
                        23/08 & 124 & $x_{4 }$\\
                        30/08 & 127 & $x_{5 }$\\
                        06/09 & 116 & $x_{6 }$\\
                        13/09 & 126 & $x_{7 }$\\
                        20/09 & 106 & $x_{8 }$\\
                        27/09 & 95  & $x_{9 }$\\
                        04/10 & 77  & $x_{10}$\\
                        11/10 & 45  & $x_{11}$\\
                        18/10 & 67  & $x_{12}$\\
                        25/10 & 48  & $x_{13}$\\
                        01/11 & 50  & $x_{14}$\\
                        08/11 & 83  & $x_{15}$\\
                        15/11 & 101 & $x_{16}$\\
                        22/11 & 101 & $x_{17}$\\
                        29/11 & 100 & $x_{18}$\\
                        06/12 & 116 & $x_{19}$\\
                        13/12 & 71  & $x_{20}$\\
                        20/12 & 80  & $x_{21}$\\
                        27/12 & ?   & $x_{22}$\\
                        03/01 & ?   & $x_{23}$\\
                        10/01 & ?   & $x_{24}$\\
                        17/01 & ?   & $x_{25}$\\
                    \end{tabular}
                \]

            \paragraph{Determine}Utilizando o modelo auto-regressivo linear:
                \begin{enumerate}[noitemsep]
                    \item Considerando $k = 4$, estime o valor dos coeficientes $\alpha_{0}, \alpha_{1}, \alpha_{2}, \alpha_{3}$ e $\alpha_{4}$ do modelo auto-regressivo linear usando o método dos quadrados mínimos, utilizando todos os valores da tabela.
                    \item Utilizando os coeficientes determinados no item anterior, estime o número de casos nos dias 27/12, 03/01, 10/01 e 17/01.
                \end{enumerate}

        \subsection{Desenvolvimento}
            \paragraph{Teoria}Considera-se que os dados em um \textbf{Modelo Auto-Regressivo Linear} podem ser modelados pela seguinte relação:
                \[\boxed{
                    x_{n} = \alpha_{0} + \alpha_{1}x_{n-1} + \cdots + \alpha_{k-1}x_{n-k+1} + \alpha_{k}x_{n-k}, 
                    \hspace{5mm} \forall n\geq k+1}
                \]

                \paragraph{Mínimos Quadrados}Selecionando as funções $g_{1}, g_{2}, \dots, g_{n}$ de tal forma que o \textbf{Resíduo}, a soma dos quadrados dos desvios, seja mínimo:
                \[\boxed{R(\alpha_{1}, \dots, \alpha_{n}) = \sum_{i=1}^{k} (y_{i} - (\alpha_{1} g_{1}(x_{i}) + \alpha_{2} g_{2}(x_{i}) + \cdots + \alpha_{n} g_{n}(x_{i})))^{2}}\]
                Isto ocorre quando o gradiente $\nabla R (\alpha_{1}, \dots, \alpha_{n}) = 0$. Desta maneira cada derivada parcial deverá ser zero, implicando na seguinte família de equações:
                    \[
                        \begin{split}\frac{\partial R}{\partial\alpha_{i}}(\alpha_{1}, \dots, \alpha_{n})
                            &= \sum_{i=1}^{k} -2g_{j}(x_{i})[y_{i} - [\alpha_{1} g_{1}(x_{i}) +
                                                                    \cdots +
                                                                    \alpha_{n} g_{n}(x_{i})]]\\
                            &= -2\sum_{i=1}^{k} y_{i}g_{j}(x_{i}) +
                                2\sum_{i=1}^{k} g_{j}(x_{i})[\alpha_{1} g_{1}(x_{i}) +
                                                            \cdots +
                                                            \alpha_{n} g_{n}(x_{i})]\\
                        \end{split}
                    \]
                Desta maneira, o gradiente poderá ser representado matricialmente pelas seguintes matrizes:
                    \[\boxed{\nabla R(\alpha_{1}, \dots, \alpha_{n}) = 0 = -2A^{T}y + 2A^{T}A\alpha}\]
                    \[
                        \underbrace{
                        \begin{bmatrix}
                            g_{1}(x_{1}) & g_{1}(x_{2}) & \cdots & g_{1}(x_{k})\\
                            g_{2}(x_{1}) & g_{2}(x_{2}) & \cdots & g_{2}(x_{k})\\
                            \vdots       & \vdots       & \ddots & \vdots\\
                            g_{n}(x_{1}) & g_{n}(x_{2}) & \cdots & g_{n}(x_{k})\\
                        \end{bmatrix}}_{A^{T}}
                        \underbrace{
                        \begin{bmatrix}
                            y_{1}\\ y_{2}\\ \vdots\\ y_{k}\\
                        \end{bmatrix}}_{y} =
                        \underbrace{
                        \begin{bmatrix}
                            g_{1}(x_{1}) & g_{1}(x_{2}) & \cdots & g_{1}(x_{k})\\
                            g_{2}(x_{1}) & g_{2}(x_{2}) & \cdots & g_{2}(x_{k})\\
                            \vdots       & \vdots       & \ddots & \vdots\\
                            g_{n}(x_{1}) & g_{n}(x_{2}) & \cdots & g_{n}(x_{k})\\
                        \end{bmatrix}}_{A^{T}}
                        \underbrace{
                        \begin{bmatrix}
                            g_{1}(x_{1}) & g_{2}(x_{1}) & \cdots & g_{n}(x_{1})\\
                            g_{1}(x_{2}) & g_{2}(x_{2}) & \cdots & g_{n}(x_{2})\\
                            \vdots       & \vdots       & \ddots & \vdots\\
                            g_{1}(x_{k}) & g_{2}(x_{k}) & \cdots & g_{n}(x_{k})\\
                        \end{bmatrix}}_{A}
                        \underbrace{
                        \begin{bmatrix}
                            \alpha_{1}\\ \alpha_{2}\\ \vdots\\ \alpha_{n}\\
                        \end{bmatrix}}_{\alpha}
                    \]

            \paragraph{Sistema}Considerando o método de auto-regressão, tomando $k = 4$, aplicado a todos os dados do problema tem-se o seguinte sistema de equações:
                \[
                    \left\{
                    \begin{array}{r c l}
                        x_{5}   & = &\alpha_{0} + \alpha_{1}x_{4} + \alpha_{2}x_{3} + \alpha_{3}x_{2} + \alpha_{4}x_{1}\\
                        x_{6}   & = &\alpha_{0} + \alpha_{1}x_{5} + \alpha_{2}x_{4} + \alpha_{3}x_{3} + \alpha_{4}x_{2}\\
                                & \vdots &\\
                        x_{20}  & = & \alpha_{0} + \alpha_{1}x_{19} + \alpha_{2}x_{18} + \alpha_{3}x_{17} + \alpha_{4}x_{16}\\
                        x_{21}  & = & \alpha_{0} + \alpha_{1}x_{20} + \alpha_{2}x_{19} + \alpha_{3}x_{18} + \alpha_{4}x_{17}\\
                    \end{array}
                    \right.
                \]
            Este sistema poderá ser simplesmente representado matricialmente pela seguinte equação:
                \[
                    \begin{bmatrix}
                        1      & x_{ 4} & x_{ 3} & x_{ 2} & x_{ 1}\\ 
                        1      & x_{ 5} & x_{ 4} & x_{ 3} & x_{ 2}\\ 
                        \vdots & \vdots & \vdots & \vdots & \vdots\\ 
                        1      & x_{19} & x_{18} & x_{17} & x_{16}\\ 
                        1      & x_{20} & x_{19} & x_{18} & x_{17}\\ 
                    \end{bmatrix}
                    \begin{bmatrix}
                        \alpha_{0}\\ \alpha_{1}\\ \alpha_{2}\\ \alpha_{3}\\ \alpha_{4}\\
                    \end{bmatrix} =
                    \begin{bmatrix}
                        x_{5}\\ x_{6}\\ \vdots\\ x_{20}\\ x_{21}\\
                    \end{bmatrix}
                \]

        \subsection{Solução}
            \paragraph{Método Auto-Regressivo}Nota-se que o sistema de equações, solucionado através do método dos mínimos quadrados, possui os seguintes coeficientes:
                \[
                    \alpha_{0} = 27.55601, \hspace{2.5mm}
                    \alpha_{1} =  0.59273, \hspace{2.5mm}
                    \alpha_{2} =  0.21653, \hspace{2.5mm}
                    \alpha_{3} =  0.13406, \hspace{2.5mm}
                    \alpha_{4} = -0.26193
                \]
            Conhecidos os coeficientes, estima-se que o número de casos graves nos respectivos dias serão aproximadamente:
                \[
                    27/12 =  79.706, \hspace{2.5mm}
                    03/01 =  71.257, \hspace{2.5mm}
                    10/01 =  79.179, \hspace{2.5mm}
                    17/01 =  79.648
                \]
\newpage

        \subsection{Códigos}
            \begin{scriptsize}
                \myStyle
                \lstinputlisting{AT8_2.m}
            \end{scriptsize}
\newpage
\end{document}