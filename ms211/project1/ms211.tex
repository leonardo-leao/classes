\documentclass{article}

\usepackage[a4paper, hmargin={20mm, 20mm}, vmargin={25mm, 30mm}]{geometry}
\usepackage[utf8]{inputenc}
\usepackage[english, main=portuguese]{babel}

\usepackage[hidelinks]{hyperref}
\usepackage{bookmark}
\usepackage{cancel}
\usepackage{comment}

\usepackage{array}
\usepackage{indentfirst}
\usepackage{multicol}
\usepackage{subfiles}

\usepackage{amsmath}
\usepackage{amssymb}
\usepackage{float}
\usepackage{enumitem}

\usepackage{pythontex}
\restylefloat{table}


\title{MS211 - Cálculo Numérico}
\author{Guilherme Nunes Trofino}
\date{\today}


\begin{document}
        \maketitle
    \newpage

        \tableofcontents
    \newpage

    \section{Sistemas Lineares}
        \paragraph{Definição}Seja $A=(a_{ij})$ uma matriz $n \times n$ e $y_{1}, \cdots, y_{n}$ escalares então um \textit{Sistema Linear} com \textit{n-equações} e \textit{n-incógnitas} é dada pela família:
            \[
                \left\{
                    \begin{matrix}
                        a_{11}x_{1} & a_{12}x_{2} & \cdots & a_{1n}x_{n} = b_{1}\\
                        a_{21}x_{1} & a_{22}x_{2} & \cdots & a_{2n}x_{n} = b_{2}\\
                        \cdots      &             &        & \\
                        a_{n1}x_{1} & a_{n2}x_{2} & \cdots & a_{nn}x_{n} = b_{n}
                    \end{matrix}
                \right.
            \]

        \paragraph{Representação}Simplificadamente estes sistemas podem ser apresentados como matrizes e vetores multiplicados, $A \times X = B$, onde $A$ representa a \textit{Matriz dos Coeficientes}, $X$ representa o \textit{Vetor de Variáveis} e $B$ representa o \textit{Vetor de Resultados}.
            \[Ax=b\]
            \[
                \begin{bmatrix}
                    a_{11} & a_{12} & \cdots & a_{1n}\\
                    a_{21} & a_{22} & \cdots & a_{2n}\\
                    \vdots & \vdots & \ddots & \vdots\\
                    a_{n1} & a_{n2} & \cdots & a_{nn}
                \end{bmatrix}
                \begin{bmatrix}
                    x_{1}\\
                    x_{2}\\
                    \vdots\\
                    x_{n}\\
                \end{bmatrix}=
                \begin{bmatrix}
                    b_{1}\\
                    b_{2}\\
                    \vdots\\
                    b_{n}\\
                \end{bmatrix}
            \]

        \paragraph{Matriz Inversa}A matriz $A \in R^{n \times n}$ é não-singular, isto é, inversível,  se, e somente se, exista uma matriz $A^{-1}$, nomeada como inversa de $A$, tal que: \[A \times A^{-1} = A^{-1} \times A = I\] em que $I \in R^{n \times n}$ representa a matriz identidade. Algebricamente temos que $A$ será não-singular se, e somente se, $det(A) \neq 0$.
\newpage

    \section{Méodo da Eliminação de Gauus}
        \paragraph{Definição}Este algoritmo permite solucionar sistemas lineares quadrados de maneira sistématica independente de seu tamanho ou coeficientes.
            \begin{enumerate}[noitemsep]
                \item Primeiramente será necessário concatenar a matriz dos coeficientes com o vetor de resultados.
                    \[[A|B]^{0}=
                        \begin{bmatrix}
                            a_{11} & a_{12} & a_{13} & \cdots & a_{1n} & | & b_{1}\\
                            a_{21} & a_{22} & a_{23} & \cdots & a_{2n} & | & b_{2}\\
                            a_{31} & a_{32} & a_{33} & \cdots & a_{3n} & | & b_{3}\\
                            \vdots & \vdots & \vdots & \ddots & \vdots & | & \vdots\\
                            a_{n1} & a_{n2} & a_{n3} & \cdots & a_{nn} & | & b_{n}
                        \end{bmatrix} 
                    \]
                \item Em seguida aplicam-se sucessivas operações elementares pocurando transformar a concatenada em uma \textit{Matriz Triangular Superior}, para isso, modificando cada entrada de uma coluna, abaixo de seu pivô, para zero.
                    \[[A|B]^{1}=
                        \begin{bmatrix}
                            a_{11}^{(1)} & a_{12}^{(1)} & a_{13}^{(1)} & \cdots & a_{1n}^{(1)} & | & b_{1}^{(1)}\\
                            0            & a_{22}^{(1)} & a_{23}^{(1)} & \cdots & a_{2n}^{(1)} & | & b_{2}^{(1)}\\
                            0            & a_{32}^{(1)} & a_{33}^{(1)} & \cdots & a_{3n}^{(1)} & | & b_{3}^{(1)}\\
                            \vdots       & \vdots       & \vdots       & \ddots & \vdots       & | & \vdots\\
                            0            & a_{n2}^{(1)} & a_{n3}^{(1)} & \cdots & a_{nn}^{(1)} & | & b_{n}^{(1)}
                        \end{bmatrix}
                    \]
                    \[[A|B]^{2}=
                        \begin{bmatrix}
                            a_{11}^{(2)} & a_{12}^{(2)} & a_{13}^{(2)} & \cdots & a_{1n}^{(2)} & | & b_{1}^{(2)}\\
                            0            & a_{22}^{(2)} & a_{23}^{(2)} & \cdots & a_{2n}^{(2)} & | & b_{2}^{(2)}\\
                            0            & 0            & a_{33}^{(2)} & \cdots & a_{3n}^{(2)} & | & b_{3}^{(2)}\\
                            \vdots       & \vdots       & \vdots       & \ddots & \vdots       & | & \vdots\\
                            0            & 0            & a_{n3}^{(2)} & \cdots & a_{nn}^{(2)} & | & b_{n}^{(2)}
                        \end{bmatrix}
                    \]\vspace{5mm}
                    \[\boxed{[A|B]^{n}=
                        \begin{bmatrix}
                            a_{11}^{(n)} & a_{12}^{(n)} & a_{13}^{(n)} & \cdots & a_{1n}^{(n)} & | & b_{1}^{(n)}\\
                            0            & a_{22}^{(n)} & a_{23}^{(n)} & \cdots & a_{2n}^{(n)} & | & b_{2}^{(n)}\\
                            0            & 0            & a_{33}^{(n)} & \cdots & a_{3n}^{(n)} & | & b_{3}^{(n)}\\
                            \vdots       & \vdots       & \vdots       & \ddots & \vdots       & | & \vdots\\
                            0            & 0            & 0     ^{(n)} & \cdots & a_{nn}^{(n)} & | & b_{n}^{(n)}
                        \end{bmatrix}}
                    \]
                \item Posteriormente, se for necessário, deve-se realizar a substituição reversa para obter as soluções, no caso de um sistema de equações.
            \end{enumerate}
        
        Este processo possui complexidade $O(n^{3})$ e a substituição reversa possui complexidade $O^{2}$

        \paragraph{Algoritmo}

        \subsection{Pivoteamento Parcial}
            \paragraph{Motivação}No algoritmo implementado anteriormente a eliminação irá falhar quando ao menos um dos pivôs na matriz considerada for igual a zero, pois este elemento será o denominador de uma das operações.

            \paragraph{Definição}Afim de evitar este problema antes de iniciar o j-ésimo estágio do algoritmo, permutam-se as linhas da matriz $A$ de modo a obter a seguinte organização:
                \[|a_{jj}^{(j-1)}|\geq|a_{ij}^{(j-1)}|; \forall i = j, \dots, n\]

            \paragraph{Algoritmo}
        
        \subsection{Substituição Reversa}
            \paragraph{Definição}
\newpage

    \section{Fatoração LU}
        \paragraph{Definição}Este algoritmo permite solucionar sistemas lineares quadrados de maneira sistématica independente de seu tamanho ou coeficientes. Diferentemente da \textit{Eliminação de Gauss}, este método permite resolver diferentes sistemas lineares sem refazer todos os cálculos.

        \paragraph{}São geradas duas matrizes: L, \textit{Triangular Inferior} originada da Identidade, e U, \textit{Triangular Superior}. Tais que $LU = A$, representadas genericamente como:
            \begin{enumerate}[noitemsep]
                \item Primeiramente incializamos as matrizes bases para este método. $L$ será a matriz identidade enquanto $U$ será uma cópia da matriz inicial $A$. 
                \[
                U^{0}=
                    \begin{bmatrix}
                        a_{11} & a_{12} & a_{13} & \cdots & a_{1n}\\
                        a_{21} & a_{22} & a_{23} & \cdots & a_{2n}\\
                        a_{31} & a_{32} & a_{33} & \cdots & a_{3n}\\
                        \vdots & \vdots & \vdots & \ddots & \vdots\\
                        a_{n1} & a_{n2} & a_{n3} & \cdots & a_{nn}
                    \end{bmatrix}\hspace{5mm}
                L^{0}=
                    \begin{bmatrix}
                        1      & 0      & 0      & \cdots & 0\\
                        0      & 1      & 0      & \cdots & 0\\
                        0      & 0      & 1      & \cdots & 0\\
                        \vdots & \vdots & \vdots & \ddots & \vdots\\
                        0      & 0      & 0      & \cdots & 1
                    \end{bmatrix}
                \]
                \item Em seguida realizamos operações elementares para transformar $U$ em uma matriz diagonal superior e os coeficientes do método de gauss são salvos na respectiva entrada de $L$.
                \[
                U^{1}=
                    \begin{bmatrix}
                        u_{11} & u_{12} & u_{13} & \cdots & u_{1n}\\
                        0      & u_{22} & u_{23} & \cdots & u_{2n}\\
                        0      & u_{32} & u_{33} & \cdots & u_{3n}\\
                        \vdots & \vdots & \vdots & \ddots & \vdots\\
                        0      & u_{n2} & u_{n3} & \cdots & u_{nn}
                    \end{bmatrix}\hspace{5mm}
                L^{1}=
                    \begin{bmatrix}
                        1      & 0      & 0      & \cdots & 0\\
                        i_{21} & 1      & 0      & \cdots & 0\\
                        i_{31} & 0      & 1      & \cdots & 0\\
                        \vdots & \vdots & \vdots & \ddots & \vdots\\
                        i_{n1} & 0      & 0      & \cdots & 1
                    \end{bmatrix}
                \]
                \[
                U^{2}=
                    \begin{bmatrix}
                        u_{11} & u_{12} & u_{13} & \cdots & u_{1n}\\
                        0      & u_{22} & u_{23} & \cdots & u_{2n}\\
                        0      & 0      & u_{33} & \cdots & u_{3n}\\
                        \vdots & \vdots & \vdots & \ddots & \vdots\\
                        0      & 0      & u_{n3} & \cdots & u_{nn}
                    \end{bmatrix}\hspace{5mm}
                L^{2}=
                    \begin{bmatrix}
                        1      & 0      & 0      & \cdots & 0\\
                        i_{21} & 1      & 0      & \cdots & 0\\
                        i_{31} & i_{32} & 1      & \cdots & 0\\
                        \vdots & \vdots & \vdots & \ddots & \vdots\\
                        i_{n1} & i_{n2} & 0      & \cdots & 1
                    \end{bmatrix}
                \]\vspace{5mm}
                \[
                \boxed{U^{n}=
                    \begin{bmatrix}
                        u_{11} & u_{12} & u_{13} & \cdots & u_{1n}\\
                        0      & u_{22} & u_{23} & \cdots & u_{2n}\\
                        0      & 0      & u_{33} & \cdots & u_{3n}\\
                        \vdots & \vdots & \vdots & \ddots & \vdots\\
                        0      & 0      & 0      & \cdots & u_{nn}
                    \end{bmatrix}}\hspace{5mm}
                \boxed{L^{n}=
                    \begin{bmatrix}
                        1      & 0      & 0      & \cdots & 0\\
                        i_{21} & 1      & 0      & \cdots & 0\\
                        i_{31} & i_{32} & 1      & \cdots & 0\\
                        \vdots & \vdots & \vdots & \ddots & \vdots\\
                        i_{n1} & i_{n2} & i_{n3} & \cdots & 1
                    \end{bmatrix}}
                \]
                \item 
            \end{enumerate}
        
        \subsection{Pivoteamento Parcial}
            \paragraph{Motivação}A fatoração, como implementada anteriormente, irá falhar quando qualquer um dos pivôs da matriz for igual a zero, pois este elemento será posteriormente o denominador de uma das operações.

            \paragraph{Definição}Afim de evitar este problema antes de iniciar o j-ésimo estágio do algoritmo, permutam-se as linhas da matriz $A$ de modo a satisfazer:
                \[PA = LU\]
            Onde $P$ é a \textit{Matriz de Permutação}, obtida permutando as linhas da matriz identidade.

            \paragraph{Teorema}Qualquer matriz $A\in\mathbb{R}^{n\times n}$ não-singular pode ser fatorada como \[PA=LU\] em que $U$ é traingular superior, $L$ é triangular inferior com diagonal unitária e $P$ é a matriz de permutação.

        \subsection{Solução de Sistemas}
            \paragraph{Definição}
\newpage

    \section{Ponto Flutuante}
        \paragraph{Definição}Computadores não operam com números reais, estes utilizam da representação por \textit{Ponto Flutuante Normalizado} que torna suas operações mais eficientes. Entretanto essa conversão apresenta limitações e consequentemente haverá erros pela maneira como os números são representados.
            \[x = \pm 0.d_1d_2\cdots d_t\times\beta^{e}\]

        \paragraph{Notação}Denotamos por $F(\beta, t, m, M)$ o conjunto de todos os pontos flutuantes representados nessas condições.
            \begin{enumerate}[noitemsep]
                \item $\beta$ é a base em que o número será representado;
                \item $t$ é o número de dígitos representados do número de acordo com as regras que seguem:
                    \begin{enumerate}[noitemsep]
                        \item $d_1 \neq 0$, do contrário o ponto poderia ser deslocado e expoente incrementado;
                    \end{enumerate}
                \item $e$ é o expoente da base do número representado, variando entre $-m\geq e \geq M$
            \end{enumerate}

        \subsection{Erro Absoluto}
            \paragraph{Definição}Essa representação numérica não compreende todos os números reais e portanto quando a função $F(\beta, t, m, M)$ é aplicada para um número real $X$ será retornado $\overline{X}$, o arredondamento de $X$ em ponto flutuante tal que o \textit{Erro Relativo}, denotado abaixo, seja mínimo:
                \[\boxed{|X-\overline{X}|}\]

        \subsection{Erro Relativo}
            \paragraph{Definição}Há situações em que a representação absoluta não será útil pois possirá uma esquema distinta dos números. Isso é solucionado pela normalização do erro com relação ao número utilizado. O \textit{Erro Absoluto} pode ser obtido a partir da seguinte equação:
                \[\boxed{\frac{|X-\overline{X}|}{|X|}}\]

        \subsection{Erro de Márquina}
            \paragraph{Definição}Afim de estimar a precissão dos computadores denota-se o \textit{Épsilon da Márquina} como metade da distância entre 1 e o menor ponto flutuante estritamente maior que 1.
                \[\boxed{\epsilon_{mach}=\frac{1}{2}\beta^{1-t}}\]
\newpage

    \section{Atividades}
        \subsection{Laboratórios 1}
            \paragraph{Questão 1}Quando $n = 10^{0}$ temos erro negativo, visto que $2 - 2.71828 = -0.71828$. Conforme $n$ cresce esta diferença se reduz até que, em $n = 10^{9}$, obtemos um valor calculado superior ao esperado, pois quando $n$ se torna demasiado grande as aproximações computacionais se tornam relevantes e influenciam no resultado.

            \paragraph{}Tal comportamento se mantém até $n = 10^{15}$ quando obtemos um "erro" aproximado de 3.035 e, em $n = 10^{18}$, obtemos um "erro" e = 1. Neste ponto o valor computado para 1/n é considerado como 0 e portanto se torna desprezível. Nesta simulação está claro que a capacidade computacional é limitada e deve ser considerando em cálculos de grande precisão.

            \paragraph{Questão 2}Considerando que este algoritmo é baseado no método de Eliminação de Gauss são efetuadas $O(n^{3})$ operações. A Eliminação de Gauss realiza $O(n^{3})$ para solucionar um sistema linear $Ax = b$ assim como a Fatoração LU, $O(n^{3})$, entretanto a solução dos sistemas triangulares $Ux=y$ e $Ly=b$ demandam apenas $O(n^{2})$.

            \paragraph{Questão 3}A matrix não singular a seguir apresenta falha ao ser solucionada:
                \[\begin{bmatrix}
                    0 & 1 & 1\\
                    1 & 1 & 1\\
                    1 & 1 & 0\\
                \end{bmatrix}\]
            
            Isso se deve ao fato de que tanto a Fatoração LU quanto a Eliminação de Gauss, implementadas sem pivoteamento parcial, utilizam os pivôs da matriz no denominador dos cálculos em certa altura dos algoritmos. Essa divisão é matematicamente impossível e assim retorna erro de execução do código.

            \paragraph{Questão 4}Primeiramente a equação pode ser reescrita como:
                \[Bx = (2A+I)(C^{-1}+A)b\]

            Simplificando a notação adotando: $G = (2A+I)$ e $h = (C^{-1}+A)b$, "G" não representa problemas computacionais e pode ser calculada diretamente. Por outro lado, "h" representa problemas computacionais por apresentar o cálculo de uma inversa, o que não é desejado. Pode-se evitar esse desgaste resolvendo o seguinte sistema:
                \[C(h-Ab)=b\]

            Aplicando Eliminação de Gauss ou Fatoração LU, mais eficientes do que o cálculo da inversa e, finalmente, é necessário solucionar o sistema novamente implementando um dos métodos citados anteiormente:
                \[Bx = (Gh)\]
            
            \paragraph{Questão 5}Solucionando algebricamente obtemos:
                \[Hx = Hu \rightarrow H^{-1}Hx = H^{-1}Hu \rightarrow x = u\]
            
            Portanto esperava-se que $x = [1, 1, ..., 1]^T$, independente do valor escolhido de n, entretanto ao realizar as operações descobrimos que os resultados obtidos flutuam e dificilmente são como esperado. Isso ocorre pois quanto maior a matriz de Hilbert menor os coeficientes em suas entradas, até que em certo momento essas sofreram com arredondamentos da máquina. Isso gera uma reação em cadeia que traz resultados equivocados.
\end{document}