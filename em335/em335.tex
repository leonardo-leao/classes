\documentclass{article}

\usepackage[a4paper, hmargin={20mm, 20mm}, vmargin={25mm, 30mm}]{geometry}
\usepackage[utf8]{inputenc}
\usepackage[english, main=portuguese]{babel}

\usepackage[hidelinks]{hyperref}
\usepackage{bookmark}
\usepackage{cancel}
\usepackage{comment}

\usepackage{array}
\usepackage{indentfirst}
\usepackage{multicol}
\setlength{\multicolsep}{2pt}% 50% of original values
\usepackage{subfiles}

\usepackage{titlesec}

\usepackage{amsmath}
\usepackage{amssymb}
\usepackage{systeme}
\usepackage{float}
\usepackage{enumitem}
\usepackage[thinc]{esdiff} %parcial derivatives
\restylefloat{table}

\usepackage{graphicx}
\usepackage{subcaption}
\graphicspath{ {./images/} }

% Pacote para a definição de novas cores
\usepackage{xcolor}
% Definindo novas cores
\definecolor{darkgreen}{rgb}{0.0, 0.42, 0.24}
\definecolor{darkpurple}{rgb}{0.74, 0.2, 0.64}
\definecolor{darkblue}{rgb}{0.0, 0.28, 0.67}

% Configurando espaços entre paragrafos
%\setlength{\parskip}{0.5em}


% Configurando pacote de circuitos
\usepackage{circuitikz}

%Configurando pacote de Gráficos
\usepackage{tikz}

% Configurando layout para mostrar códigos
\usepackage{listings}

\newcommand{\myStyle}{
\lstset{
    language=Octave,                            % the language of the code
    basicstyle=\ttfamily\small,               % the size of the fonts that are used for the code
    keywordstyle=\color{darkpurple}\bfseries, %
    stringstyle=\color{darkblue},             %
    commentstyle=\color{darkgreen},           %
    morecomment=[s][\color{blue}]{/**}{*/},   %
    extendedchars=true,                       %
    showtabs=false,                           % show tabs within strings adding particular underscores
    showspaces=false,                         % show spaces adding particular underscores
    showstringspaces=false,                   % underline spaces within strings
    numbers=left,                             % where to put the line-numbers
    numberstyle=\tiny\color{gray},            % the style that is used for the line-numbers
    stepnumber=1,                             % the step between two line-numbers. If it's 1, each line will be numbered
    numbersep=5pt,                            % how far the line-numbers are from the code
    frame=single,                             % adds a frame around the code
    rulecolor=\color{black},                  % if not set, the frame-color may be changed on line-breaks within not-black text
    breaklines=true,                          % sets automatic line breaking
    backgroundcolor=\color{white},            % choose the background color
    breakatwhitespace=true,                   % sets if automatic breaks should only happen at whitespace
    breakautoindent=false,                    %
    captionpos=b,                             % sets the caption-position to bottom
    xleftmargin=0pt,                          %
    tabsize=2,                                % sets default tabsize to 2 spaces
}}

%\titleformat{<command>}[<shape>]{<format>}{<label>}{<sep>}{<before-code>}[<after-code>]
\titleformat
{\section} %comand
[block]  %shape
{\normalfont\LARGE} %format
{\thesection. } %label
{0mm} %sep
{} %before-code
[{\titlerule[0.1mm]}] %after-code

\titlespacing*{\section}{0mm}{0mm}{15mm}

\titleformat
{\subsection} %comand
[block]  %shape
{\normalfont\Large} %format
{\thesubsection. } %label
{0mm} %sep
{} %before-code
[] %after-code

\titlespacing*{\subsection}{0mm}{5mm}{2.5mm}


\begin{document}
    \begin{titlepage}
        \begin{center}
            \rule{450pt}{0.5pt}\\[4mm]
            {\Huge EM335 - Tecnologia Mecânica}\\
            \rule{450pt}{0.5pt}\\[2mm]
            {\Large Resumo Teórico}\\[200mm]
            \today\\
            \rule{250pt}{0.5pt}\\
            {\large Guilherme Nunes Trofino}\\
            {\large 217276}\\
        \end{center}
    \end{titlepage}
\newpage

    \tableofcontents
\newpage

    \section{Introdução}
        \subsection{Conceitos Básicos}
            \begin{enumerate}[noitemsep]
                \item \textbf{Produção:} Conjunto de processos que agrega valor a um produto;
                \item \textbf{Produto:} Resultado final de um processo produtivo;
                \item \textbf{Processo:} Atividade que agrega valor a um produto;
            \end{enumerate}

        \subsection{Planejamento e Controle da Produção}
            \paragraph{Definição}Sistema auxiliar para conciliar o fornecimento com a demanda entre recursos e insumos para planejar a produção e entrega do produto dentro de uma quantidade, qualidade e data adequada.

        \subsection{Histórico de Sistemas Produtivos}
            \paragraph{Definição}Diferentes organizações produtivas existiram ao longo do tempo dentro da industria, adequando-se as necessidades da sociedade entre os quais os listados abaixo:
                \begin{enumerate}[noitemsep]
                    \item \textbf{Produção Artesanal:} Taylor Made Production, produtos altamente personalizados e únicos;
                        \begin{enumerate}[noitemsep]
                            \item \texttt{Custo:} Alto;
                            \item \texttt{Despepesa:} Alta;
                            \item \texttt{Ferramenta:} Generalista;
                            \item \texttt{Mão de Obra:} Especializada;
                            \item \texttt{Volume:} Baixo;
                        \end{enumerate}
                    \item \textbf{Produção em Massa:} Mass Production, resultado do mercado consumidor em massa;
                        \begin{enumerate}[noitemsep]
                            \item \texttt{Custo:} Baixo;
                            \item \texttt{Despepesa:} Média;
                            \item \texttt{Ferramenta:} Média;
                            \item \texttt{Mão de Obra:} Generalista;
                            \item \texttt{Volume:} Alta;
                        \end{enumerate}
                    \item \textbf{Produção Enxuta:} Lean Production, redução de desperdícios do sistema através do planejamento;
                        \begin{enumerate}[noitemsep]
                            \item \texttt{Custo:} Alta;
                            \item \texttt{Despepesa:} Alta;
                            \item \texttt{Ferramenta:} Especializada;
                            \item \texttt{Mão de Obra:} Especializada;
                            \item \texttt{Volume:} Baixo;
                        \end{enumerate}
                    \item \textbf{Produção 4.0:} Smart Production, baseado em informações para maximizar a eficiência;
                        \begin{enumerate}[noitemsep]
                            \item \texttt{Custo:} Médio;
                            \item \texttt{Despepesa:} Médio;
                            \item \texttt{Ferramenta:} Média;
                            \item \texttt{Mão de Obra:} Generalista;
                            \item \texttt{Volume:} Alto;
                        \end{enumerate}
                \end{enumerate}\noindent
            Atualmente o consumo de produtos e bens de serviço, visam conforto para os consumidores. Isso demanda fabricação seriada; quantidades elevadas com custos reduzidos, produção do operário; velocidade da linha de montagem, e sem ajustes suplementares; manter a programação da linha.

        \subsection{Atualidade de Sistemas Produtivos}
            \paragraph{Definição}Produtos são compostos por peças separadamente produzidas e unificadas sem ajustes locais em virtude de desvio de processos, equipamentos ou medições ao longo da cadeia produtiva. Assim, projetos não necessariamente corresponderam ao resultado final esperado, sendo necessário desenvolver alternativas para minimizar os riscos de incompatibilidades como as listadas.
                \begin{enumerate}[noitemsep]
                    \item \textbf{Intercambiabilidade:} Possibilidade de montar peças sem necessidade de retravalhos posteriores, podendo ser alcançada através dos seguintes métodos:
                        \begin{enumerate}[noitemsep]
                            \item \texttt{Medida Nominal:} Garantir peças mais próximas das dimensões de projeto, implicando em elevado gasto de tempo, baixa produtividade e alto custo;
                            \item \texttt{Desvio Nominal:} Faixa de tolerância dimensional aceitável pelo projeto, possibilitando substituição e ajustes adequados;
                            \item \texttt{Tolerância:} Diferença entre os limites máximos e mínimos das dimensões do projeto;
                        \end{enumerate}
                \end{enumerate}
\newpage

    \section{Normas}
        \subsection{Sistemas de Tolerâncias e Ajustes}
            \paragraph{Definição}Normas são criadas para padronização de produtos dentro de um mesmo país ou mercado. No Brasil a Agência de Brasileira de Normas Técnicas, \href{http://www.abnt.org.br/}{ABNT}, regulam como os sistemas de tolerância são aplicados no território nacional como descritos pelas seguintes normas:
                \begin{enumerate}[noitemsep]
                    \item \textbf{Dimensões entre $[0,500]$ mm:} Designadas pelas IT01, IT0, IT1 a IT16;
                    \item \textbf{Dimensões entre $]500,3150]$ mm:} Designadas pelas IT1 a IT16;
                \end{enumerate}
            Os graus de tolerância padrão são classificados de acordo com a aplicação e a peça como descrito abaixo:
                \begin{enumerate}
                    \item \textbf{Calibradores}
                        \begin{enumerate}[noitemsep]
                            \item \texttt{Eixo:} IT1 a IT3;
                            \item \texttt{Furo:} IT01 à IT4;
                        \end{enumerate}
                    \item \textbf{Peças em Conjuntos}
                        \begin{enumerate}[noitemsep]
                            \item \texttt{Eixo:} IT4 à IT11;
                            \item \texttt{Furo:} IT5 à IT11;
                        \end{enumerate}
                    \item \textbf{Peças em Avulsas}
                        \begin{enumerate}[noitemsep]
                            \item \texttt{Eixo:} IT11 à IT16;
                            \item \texttt{Furo:} IT11 à IT16;
                        \end{enumerate}
                \end{enumerate}

        \subsection{Cálculos de Tolerâncias}
            \paragraph{Definição}Primeiramente será necessário identificar qual a faixa da dimensão avaliada de acordo com a seguinte tabela:
                \begin{table}[H]
                    \centering
                    \begin{tabular}[]{c | c}\hline
                        $d_{i} \text{ [mm]}$ & $d_{s} \text{ [mm]}$\\[1mm]\hline
                        0         & 1\\
                        1         & 3\\
                        3         & 6\\
                        6         & 10\\
                        10        & 18\\
                        18        & 30\\
                        30        & 50\\\hline
                    \end{tabular}
                    \hspace{20mm}
                    \begin{tabular}[]{c | c}\hline
                        $d_{i} \text{ [mm]}$ & $d_{s} \text{ [mm]}$\\[1mm]\hline
                        50        & 80\\
                        80        & 120\\
                        120       & 180\\
                        180       & 250\\
                        250       & 315\\
                        315       & 400\\
                        400       & 500\\\hline
                    \end{tabular}
                    \caption{Quadro de Dimensões}\label{table:dimesion}
                \end{table} \noindent
            Onde:
                \begin{enumerate}[noitemsep]
                    \item $d_{i}$, Distância Inferior: Exclusiva;
                    \item $d_{s}$, Distância Superior: Inclusiva;
                \end{enumerate}
            Em seguida será necessário identificar qual a tolerância-padrão empregada e realizar, a partir dos ajustes mais exigentes, as seguintes operações:
                \[
                    \begin{cases}
                        IT01 &= 0,3 + 0,001 \cdot D, \text{ [$\mu$m]};\\
                        IT0  &= 0,5 + 0,012 \cdot D, \text{ [$\mu$m]};\\
                        IT1  &= 0,8 + 0,020 \cdot D, \text{ [$\mu$m]};\\
                    \end{cases}
                \]
            Onde:
                \begin{enumerate}[noitemsep]
                    \item $D$: \textbf{Média Geométrica}, em milímetros, dos valores extremos da faixa de dimensão avaliada em \ref{table:dimesion}:
                        \begin{equation}
                            \boxed{
                                D = \sqrt{d_{i} \cdot d_{s}}
                            }
                        \end{equation}
                \end{enumerate}
            Note que as tolerâncias-padrões entre IT2 e IT4 são obtidos como termos de uma progressão geométrica formada pelos termos IT1 e IT5, mostradas nas seguintes equações:
                \[
                    \begin{cases}
                        IT1 = 0,8 + 0,020 \cdot D, & \text{ [$\mu$m]};\\
                        IT2 = IT1 \cdot q^{1},     & \text{ [$\mu$m]};\\
                        IT3 = IT1 \cdot q^{2},     & \text{ [$\mu$m]};\\
                        IT4 = IT1 \cdot q^{3},     & \text{ [$\mu$m]};\\
                        IT5 = 7i,                  & \text{ [$\mu$m]};\\
                    \end{cases}
                    \hspace{15mm}
                    \boxed{
                        q = \sqrt[n-1]{\frac{a_{n}}{a_{1}}} = \sqrt[4]{\frac{a_{5}}{a_{1}}}
                    }
                \]
            Note que as tolerância-padrão entre IT5 e IT18 são obtidos de acordo com a seguinte tabela:
                \begin{table}[H]
                    \centering
                    \begin{tabular}[]{c | c}\hline
                        IT   & Função \\[1mm]\hline
                        IT5  & 7i \\
                        IT6  & 10i\\
                        IT7  & 16i\\
                        IT8  & 25i\\
                        IT9  & 40i\\
                        IT10 & 64i\\\hline
                    \end{tabular}
                    \hspace{10mm}
                    \begin{tabular}[]{c | c}\hline
                        IT   & Função \\[1mm]\hline
                             & \\
                        IT11 & 100i\\
                        IT12 & 160i\\
                        IT13 & 250i\\
                        IT14 & 400i\\
                        IT15 & 640i\\\hline
                    \end{tabular}
                    \hspace{10mm}
                    \begin{tabular}[]{c | c}\hline
                        IT   & Função \\[1mm]\hline
                             & \\
                        IT17 & 1000i\\
                        IT18 & 1600i\\
                         & \\
                         & \\
                         & \\\hline
                    \end{tabular}
                    \caption{Quadro de Tolerância Padrão IT5 à IT18}\label{table:tolerance}
                \end{table}\noindent
            Onde:
                \begin{enumerate}[noitemsep]
                    \item $i$: \textbf{Fator de Tolerância-Padrão}, dos valores extremos da faixa de dimensão avaliada em \ref{table:tolerance}:
                        \begin{equation}
                            \boxed{
                                i = 0,45 \cdot \sqrt[3]{D} + 0,001 \cdot D
                            }
                        \end{equation}
                \end{enumerate}
\newpage

    \section{Campos de Tolerância}
        \subsection{Afastamento Fundamental}
            \paragraph{Definição}Posição relativa a partir da qual referência-se o \textbf{Campo de Tolerância}, região entre a medição máxima e mínima aceitável pelo ajuste, com relação a linha mais próxima, superior ou inferior, do afastamento zero. Estes afastamentos são representados por letras como descritos a seguir:
                \begin{enumerate}[noitemsep]
                    \item \textbf{Furos:} A, B, C, ..., X, Y, Z, ZA, ZB, ZC;
                    \item \textbf{Eixos:} a, b, c, ..., x, y, z, za, zb, zc;
                \end{enumerate}

        \subsection{Posição dos Campos de Tolerância}
            \paragraph{Definição}Divisão de Campos de Tolerância ao longo da linha de afastamento zero divida entre eixos e furos com afastamentos superiores e inferiores, apresentando Afastamentos Fundamentais distintos como descritos a seguir:
                \begin{figure}[H]
                    \centering
                    \includegraphics[height = 6cm]{ima0.png}
                    \caption{Posição dos Campos de Tolerância}\label{fig:fields}
                \end{figure} \noindent
            Onde:
                \begin{enumerate}[noitemsep]
                    \item \textbf{Furos:} Ajustes Fundamentais serão obtidos através da seguinte tabela.
                        \begin{table}[H]
                            \centering
                            \begin{tabular}[]{c c c c | c c c}\hline
                                            & A  & ... & H  & J  & ... & Zc\\\hline
                                Superior    & As & ... & Hs & Js & ... & Zcs\\
                                Inferior    & Ai & ... & Hi & Ji & ... & Zci\\\hline
                                Fundamental & Ai & ... & Hi & Js & ... & Zcs\\\hline
                            \end{tabular}
                            \caption{Ajuste Fundamental para Furos}\label{table:holes}
                        \end{table}
                    \item \textbf{Eixos:} Ajustes Fundamentais serão obtidos através da seguinte tabela.
                        \begin{table}[H]
                            \centering
                            \begin{tabular}[]{c c c c | c c c}\hline
                                            & a  & ... & h  & j  & ... & zc\\\hline
                                Superior    & as & ... & hs & js & ... & zcs\\
                                Inferior    & ai & ... & hi & ji & ... & zci\\\hline
                                Fundamental & as & ... & hs & ji & ... & zci\\\hline
                            \end{tabular}
                            \caption{Ajuste Fundamental para Eixos}\label{table:axis}
                        \end{table}
                \end{enumerate}
            Apesar das tabelas representarem os ajustes mostrados na figura \ref{fig:fields} há exceções, sendo as mais relevantes demonstradas a seguir:
                \begin{enumerate}
                    \item \textbf{Eixos:} Utilizada quando o afastamento js terá seu afastamento fundamental dado por $\pm 0,5 \cdot t$ onde t será a Tolerância dada por:
                        \begin{equation}
                            \boxed{
                                t = As - Ai
                            }
                            \hspace{10mm}
                            \boxed{
                                t = as - ai
                            }
                        \end{equation}
                    \item \textbf{Rolamentos:} Utilizada nestas peças que possuiram As = 0 para N9 à N16;
                    \item \textbf{Regra Especial:} Utilizada para furos com dimensões superiores a 3 mm sobre as seguintes condições, onde n é a Qualidade de Trabalho:
                        \begin{equation}
                            \boxed{
                                As_{(n)} = - ai_{(n)} + [ IT(n) - IT(n-1) ]
                            }
                        \end{equation}
                        \begin{enumerate}[noitemsep]
                            \item Campos tolerâncias de K à N com Tolerância Padrão até IT8 inclusive; 
                            \item Campos tolerâncias de P à Zc com Tolerância Padrão até IT7 inclusive; 
                        \end{enumerate}
                    \item \textbf{Simetria:} Utilizada quando o acoplamento possuir a mesma letra, sobre as seguintes condições:
                        \begin{equation}
                            \boxed{
                                Ai = - as
                            }
                        \end{equation}
                        \begin{enumerate}[noitemsep]
                            \item Furos de A à H; 
                            \item Furos fora da Regra Especial de M à Zc com dimensões superiores a 3mm; 
                        \end{enumerate}
                \end{enumerate}
\newpage

    \section{Seleção de Ajustes}
        \subsection{Sistemas de Ajustes}
            \paragraph{Definição}Configuração do acoplamento entre um eixo e um furo, analisando o comportamento dos Campos de Tolerância das peças como descritos a seguir:
                \begin{figure}[H]
                    \centering
                    \includegraphics[height = 4cm]{ima1.png}
                    \caption{Posição dos Sistemas de Ajustes}\label{fig:ajusts}
                \end{figure} \noindent
                \begin{enumerate}[noitemsep]
                    \item \textbf{Ajustes com Folga:} ;
                        \begin{enumerate}[noitemsep]
                            \item \texttt{Furo-Base:} H acoplados com Eixos de a à h;
                            \item \texttt{Eixo-Base:} h acoplados com Furos de A à H;
                        \end{enumerate}
                    \item \textbf{Ajustes Incertos:} ;
                        \begin{enumerate}[noitemsep]
                            \item \texttt{Furo-Base:} H acoplados com Eixos de j à n;
                            \item \texttt{Eixo-Base:} h acoplados com Furos de J à N;
                        \end{enumerate}
                    \item \textbf{Ajustes com Interferência:} ;
                        \begin{enumerate}[noitemsep]
                            \item \texttt{Furo-Base:} H acoplados com Eixos de p à zc;
                            \item \texttt{Eixo-Base:} h acoplados com Furos de P à Zc;
                        \end{enumerate}
                \end{enumerate}
            Onde:
                \begin{enumerate}[noitemsep]
                    \item \textbf{Furo-Base:} Associação de várias classes de tolerâncias de eixos com uma única classe de tolerância de furo: Ai = 0;
                    \item \textbf{Eixo-Base:} Associação de várias classes de tolerâncias de furos com uma única classe de tolerância de eixo: as = 0;
                \end{enumerate}
            Como descritos nas figuras a seguir:
                \begin{figure}[h]
                    \begin{subfigure}[t]{0.5\textwidth}
                        \centering
                        \includegraphics[height = 3cm]{ima2.png}
                        \caption{Sistema Furo Base}
                    \end{subfigure}
                    \begin{subfigure}[t]{0.5\textwidth}
                        \centering
                        \includegraphics[height = 3cm]{ima3.png}
                        \caption{Sistema Eixo Base}
                    \end{subfigure}
                    \caption{Sistemas de Acoplamento}
                \end{figure}

        \subsection{Seleção de Ajustes}
            \paragraph{Definição}Diferentes configurações de ajustes podem ser combinados de acordo com a qualidade de trabalho necessária para aplicação desejada como descrito a seguir:
                \begin{enumerate}[noitemsep]
                    \item \textbf{Qualidade de Trabalho:}
                        \begin{enumerate}[noitemsep]
                            \item \texttt{Ajustes Mecânica Muita Precisão};
                            \item \texttt{Ajustes Mecânica Precisão};
                            \item \texttt{Ajustes Mecânica Precisão Média};
                            \item \texttt{Ajustes Mecânica Comum};
                        \end{enumerate}
                    \item \textbf{Ajustes:}
                        \begin{enumerate}[noitemsep]
                            \item \texttt{Interferência Forte:} Peças solidamente acopladas, mediante a pressão;
                            \item \texttt{Interferência Leve:} Peças de acoplamento fixo, golpe de martelo pesado;
                            \item \texttt{Incerto Forte:} Peças de acoplamento fixo, montagem e desmontagem não tão frequente;
                            \item \texttt{Incerto Leve:} Peças que devem acoplar-se e desacoplar-se, mão ou martelo de borracha;
                            \item \texttt{Folga Leve:} Peças que, quando bem lubrificadas, pode-se montá-las e desmontá-las a mão;
                            \item \texttt{Folga Rotativo:} Peças que devam ter folga bastante mínima;
                            \item \texttt{Folga Rotativo Livre:} Peças que devem ter folga bastante perceptível;
                            \item \texttt{Folga Rotativo Semi-Rotativo:} Peças que necessitam de folga perceptível;
                            \item \texttt{Folga Rotativo Rotativo Forte:} Peças que devem ter ampla folga;
                            \item \texttt{Folga Rotativo Rotativo Livre:} Peças que devem ter folga bastante perceptível;
                            \item \texttt{Mecânica Comum:} Peças que devam ter ampla folga e grande tolerância de fabricação;
                        \end{enumerate}
                \end{enumerate}
\newpage

    \section{Transferência de Cotas}
        \subsection{Cotas}
            \paragraph{Definição}Medidas representadas em desenho técnicos para identificar quais as dimensões e tolerâncias esperadas para uma peça ou conjunto mecânico, sendo classificadas de acordo com sua origem como demonstrado a seguir:
                \begin{enumerate}[noitemsep]
                    \item \textbf{Cota de Projeto:} Definida pelo projetista;
                    \item \textbf{Cota de Fabricação:} Resultante do somatório das demais cotas;
                        \begin{enumerate}[noitemsep]
                            \item \texttt{Processo:} Quando uma dimensão não é fornecida em um desenho técnico está será obtido pela subtração entre o máximo superior e mínimo inferior e mínimo superior e máximo superior;
                        \end{enumerate}
                \end{enumerate}

        \subsection{Referências}
            \paragraph{Definição}Superfície ou linha de uma peça a partir da qual as medições serão representadas, podendo utilizar cotas definidas a partir das seguintes partes do mecanismo que, idealmente, seriam equivalentes:
                \begin{enumerate}[noitemsep]
                    \item \textbf{Referência de Projeto:} Face ou linha do projeto será referência;
                    \item \textbf{Referência de Fabricação:} Face ou linha do equipamento de manufatura utilizado;
                    \item \textbf{Referência de Medição:} Face ou linha do equipamento de medição utilizado;
                \end{enumerate}
            Sempre que possível o processo de fabricação deve utilizar as cotas de projeto como referência, entretanto limitações físicas podem inviabilizar esta abordagem. Assim as cotas devem ser determinadas respeitando os procedimentos de produção necessários cujos afastamentos podem ser determinados por:
                \begin{equation}
                    \boxed{
                        a_{sR} = \sum\text{AS} - \sum\text{AI}
                    }
                \end{equation}
                \begin{equation}
                    \boxed{
                        a_{iR} = \sum\text{AI} - \sum\text{AS}
                    }
                \end{equation}
            Onde:
                \begin{enumerate}[noitemsep]
                    \item \textbf{AS:} Afastamento Superior:
                    \item \textbf{AI:} Afastamento Inferior:
                \end{enumerate}
\end{document}