\documentclass{article}
\usepackage{tpack}


% \usepgfplotslibrary{external} 
% \tikzexternalize


% \titleformat
%     {\subsection}           % Part
%     [block]                 % Part Shape
%     {\normalfont\Large}     % Font Size
%     {}                      % Label Numbering
%     {0mm}                   % Part Separation
%     {}                      % Code Before
%     []                      % Code After

%     \titlespacing*{\subsection}{0mm}{5mm}{2.5mm}


\title{EM607 - Vibrações Mecânicas}
\author{Guilherme Nunes Trofino}
\authorRA{217276}
\project{Atividade Teórica}


\begin{document}
    \maketitle
\newpage

\section{Atividade Teórica}
\paragraph{Apresentação}Resolução das questões de Vibrações Mecânicas por Guilherme Nunes Trofino, 217276, sobre a \textbf{Resposta em Frequência} de um Sistema de 1-DOF com Amortecimento Viscoso Forçado com as seguintes características:
\begin{enumerate}
    \item $m$, \textbf{Massa}: 1 kg;
    \item $c$, \textbf{Constante de Amortecimento}: 1 N.s/m;
    \item $k$, \textbf{Constante Elástica}: 1 N/m;
    \item $F$, \textbf{Magnitude da Força}: 1 N;
\end{enumerate}
\begin{exercise}\label{ex1}
    Verifique que a Receptância é dada por:
    \begin{equation}
        \alpha (\omega) = \frac{1}{k - \omega^2 m + j\omega c}
    \end{equation}
\end{exercise}
\begin{resolution}
    Considera-se a seguinte equação para um sistema de 1-DOF com Amortecimento Viscoso Forçado por uma Entrada Harmônica dada por $f(t) = F e^{j\omega t}$:
    \begin{equation}
        m\ddot{x} + c\dot{x} + k x = F e^{j\omega t}\label{eq:1DOFAF}
    \end{equation}
    \noindent Propõem-se a seguinte solução: $x(t) = X e^{j\omega t}$. Na sequência substitui-se na Equação \ref{eq:1DOFAF} obtendo:
    \begin{align*}
        m\ddot{x}(t) + c\dot{x}(t) + k x(t) &= f(t)\\
        (-\omega^2 m + j \omega c + k) X e^{j\omega t} &= F e^{j\omega t}\\
        (-\omega^2 m + j \omega c + k) X \cancel{e^{j\omega t}} &= F \cancel{e^{j\omega t}}\\
        \frac{X}{F} &= \frac{1}{(k -\omega^2 m) + j \omega c}
    \end{align*}
    Nota-se que há o Deslocamento Harmônico divido pelo módulo da Força Harmônica o que configura a Receptância. Dessa forma conclui-se que:
    \begin{equation}
        \boxed{
            \alpha (\omega) = 
            \frac{x(t)}{f(t)} = 
            \frac{X e^{j\omega t}}{F e^{j\omega t}} = 
            \frac{X}{F} = 
            \frac{1}{(k -\omega^2 m) + j \omega c}
        }
    \end{equation}
    Onde a Magnitude e o Ângulo de fase são dados pelas respectivas equações:
    \begin{equation}
        |\alpha(\omega)| = \frac{1}{\sqrt{(k-\omega^2m)^2 + (\omega c)^2}}
        \qquad \text{e} \qquad
        \theta_{\alpha} = - \tan^{-1}\left( \frac{-\omega c}{k - \omega^2 m} \right)
    \end{equation}
\end{resolution}

\newpage
\begin{exercise}\label{ex2}
    Determine a Receptância e Plote os respectivos gráficos de módulo e fase.
\end{exercise}
\begin{resolution}
    Parte-se da dedução apresentada no Exercício \ref{ex1} e implementa-se a função \texttt{receptivity} em MATLAB:
    \begin{scriptsize}
        \myMatlab\lstinputlisting[]{receptivity.m}
    \end{scriptsize}
    Na sequência implementa-se o seguinte algoritmo para determinar os gráficos de módulo e fase:
    \begin{scriptsize}
        \myMatlab\lstinputlisting[]{ex2.m}
    \end{scriptsize}
    Nota-se que a Magnitude será extraída através da função \href{https://www.mathworks.com/help/matlab/ref/abs.html}{\texttt{abs}} e o Ângulo através da função \href{https://www.mathworks.com/help/matlab/ref/angle.html}{\texttt{angle}}. Em ambos o eixo X estará na escala \href{https://www.mathworks.com/help/matlab/ref/loglog.html}{\texttt{loglog}}\\

    Nota-se que, no caso da Magnitude, é necessário utilizar a conversão \href{https://www.mathworks.com/help/signal/ref/mag2db.html}{\texttt{mag2db}} para converter a magnitude em decibels. Nesse caso o eixo Y estará na escala decibels.\\

    Nota-se que, no caso do Ângulo, é necessário utilizar a conversão \href{https://www.mathworks.com/help/matlab/ref/rad2deg.html}{\texttt{rad2deg}} para converter os ângulos de radianos, padrão do MATLAB, para graus. Nesse caso o eixo Y estará na escala de graus.\\

    \noindent Obtêm-se os seguintes gráficos:
    \begin{figure}[H]
        \centering
        \includegraphics[width=0.45\textwidth]{images/m1c1k1_rcpAbs.png}
        \includegraphics[width=0.45\textwidth]{images/m1c1k1_rcpAng.png}
    \end{figure}
    Nota-se que a função \href{https://www.mathworks.com/help/control/ref/lti.bode.html}{\texttt{bode}} poderia ser utilizada ao se adaptar como o sistema está descrito.
\end{resolution}

\newpage
\begin{exercise}\label{ex3}
    Determine a Mobilidade e Plote os respectivos gráficos de módulo e fase.
\end{exercise}
\begin{resolution}
    Define-se a Mobilidade como a razão entre a Velocidade e a amplitude da Força Harmônica que exita o sistema. Dessa forma tem-se, a partir do Exercício \ref{ex1}, que:
    \begin{equation}
        \boxed{
            y (\omega) = 
            \frac{\dot{x}(t)}{f(t)} = 
            -j\omega\frac{X}{F} = 
            -j\omega \cdot \alpha(\omega) = 
            \frac{-j\omega}{(k -\omega^2 m) + j \omega c}
        }
    \end{equation}
    Dessa forma implementa-se a função \texttt{mobility} em MATLAB:
    \begin{scriptsize}
        \myMatlab\lstinputlisting[]{mobility.m}
    \end{scriptsize}
    Na sequência implementa-se o seguinte algoritmo para determinar os gráficos de módulo e fase:
    \begin{scriptsize}
        \myMatlab\lstinputlisting[]{ex3.m}
    \end{scriptsize}
    Observações realizadas para o Exercício \ref{ex2} seguem pertinentes, obtendo os seguintes gráficos:
    \begin{figure}[H]
        \centering
        \includegraphics[width=0.45\textwidth]{images/m1c1k1_mblAbs.png}
        \includegraphics[width=0.45\textwidth]{images/m1c1k1_mblAng.png}
    \end{figure}
\end{resolution}

\newpage
\begin{exercise}\label{ex4}
    Determine a Inertância e Plote os respectivos gráficos de módulo e fase.
\end{exercise}
\begin{resolution}
    Define-se a Mobilidade como a razão entre a Velocidade e a amplitude da Força Harmônica que exita o sistema. Dessa forma tem-se, a partir do Exercício \ref{ex1}, que:
    \begin{equation}
        \boxed{
            A (\omega) = 
            \frac{\ddot{x}(t)}{f(t)} = 
            -j\omega   \cdot     y (\omega) = 
            - \omega^2 \cdot \alpha(\omega) = 
            \frac{-\omega^2}{(k -\omega^2 m) + j \omega c}
        }
    \end{equation}
    Dessa forma implementa-se a função \texttt{inertia} em MATLAB:
    \begin{scriptsize}
        \myMatlab\lstinputlisting[]{inertia.m}
    \end{scriptsize}
    Na sequência implementa-se o seguinte algoritmo para determinar os gráficos de módulo e fase:
    \begin{scriptsize}
        \myMatlab\lstinputlisting[]{ex4.m}
    \end{scriptsize}
    Observações realizadas para o Exercício \ref{ex2} seguem pertinentes aqui obtendo os seguintes gráficos:
    \begin{figure}[H]
        \centering
        \includegraphics[width=0.45\textwidth]{images/m1c1k1_inrAbs.png}
        \includegraphics[width=0.45\textwidth]{images/m1c1k1_inrAng.png}
    \end{figure}
\end{resolution}

\newpage
\begin{exercise}\label{ex5}
    Identifica-se as respectivas assíntotas de baixa e alta frequência nas curvas de Receptância, Mobilidade e Inertância?
\end{exercise}
\begin{resolution}
    Analisa-se a Receptância, Mobilidade e Inertância a partir dos extremos de frequência. Nesta análise desconsidera-se a parte imaginária da expressão:
    \begin{enumerate}
        \item \textbf{Receptância}:
        \begin{equation}
            \alpha(\omega) = \frac{1}{k - \omega^2m} = 
            \begin{cases}\nonumber
                \text{quando } \omega\to 0,     \;\alpha(\omega) \approx \frac{1}{k}         & \log_{10} |\alpha(\omega)| = -1\log_{10} k\\
                \text{quando } \omega\to \infty,\;\alpha(\omega) \approx \frac{1}{\omega^2m} & \log_{10} |\alpha(\omega)| = -1\log_{10} m -2\log_{10} \omega\\
            \end{cases}
        \end{equation}

        \item \textbf{Mobilidade}:
        \begin{equation}
            y(\omega) = \frac{i\omega}{k - \omega^2m} = 
            \begin{cases}\nonumber
                \text{quando } \omega\to 0,     \;y(\omega) \approx \frac{i\omega}{k}   & \log_{10} |y(\omega)| = -1\log_{10} k +1\log_{10} \omega \\
                \text{quando } \omega\to \infty,\;y(\omega) \approx \frac{-i}{\omega m} & \log_{10} |y(\omega)| = -1\log_{10} m -1\log_{10} \omega\\
            \end{cases}
        \end{equation}

        \item \textbf{Inertância}:
        \begin{equation}
            A(\omega) = \frac{-\omega^2}{k - \omega^2m} = 
            \begin{cases}\nonumber
                \text{quando } \omega\to 0,     \;A(\omega) \approx \frac{-\omega^2}{k} & \log_{10} |A(\omega)| = -1\log_{10} k +2\log_{10} \omega\\
                \text{quando } \omega\to \infty,\;A(\omega) \approx \frac{1}{m}         & \log_{10} |A(\omega)| = -1\log_{10} m\\
            \end{cases}
        \end{equation}
    \end{enumerate}
    Desconsidera-se a Constante de Amortecimento pois esta influência apenas a concavidade da Assíntota vertical entorno da Frequência Natural do sistema.\\

    Na sequência implementa-se o seguinte algoritmo para determinar os gráficos de módulo e fase:
    \begin{scriptsize}
        \myMatlab\lstinputlisting[]{ex5.m}
    \end{scriptsize}
    Assíntota de Baixa Frequência está em vermelho enquanto a Assíntota de Alta Frequência está laranja como representado abaixo:
    \begin{figure}[H]
        \centering
        \includegraphics[width=0.30\textwidth]{images/m1c1k1_rcpAss.png}
        \includegraphics[width=0.30\textwidth]{images/m1c1k1_mblAss.png}
        \includegraphics[width=0.30\textwidth]{images/m1c1k1_inrAss.png}
    \end{figure}
\end{resolution}

\newpage
\begin{exercise}\label{ex6}
    Varie o Fator de Amortecimento e observe o comportamento dos gráficos de módulo e fase.
\end{exercise}
\begin{resolution}
    Nota-se que o Fator de Amortecimento $\xi$ será obtido através da seguinte expressão:
    \begin{equation*}
        \xi = \frac{c}{2 \cdot \omega_n \cdot m}, 
        \quad \text{onde:} \quad
        \omega_n = \sqrt{\frac{k}{m}}
    \end{equation*}
    Dessa forma, altera-se a Constante de Amortecimento $c$, mantendo as demais variáveis constantes, para se estudar o comportamento dos gráficos de Módulo e Fase para a Receptância, Mobilidade e Inertância.\\

    \noindent Na sequência implementa-se o seguinte algoritmo para determinar os gráficos de módulo e fase:
    \begin{scriptsize}
        \myMatlab\lstinputlisting[]{ex6.m}
    \end{scriptsize}
    Obtêm-se os seguintes gráficos:
    \begin{figure}[H]
        \centering
        \includegraphics[width=0.3\textwidth]{images/m1c5k100_rcpAbs_fatorAmortecimento.png}
        \includegraphics[width=0.3\textwidth]{images/m1c5k100_mblAbs_fatorAmortecimento.png}
        \includegraphics[width=0.3\textwidth]{images/m1c5k100_inrAbs_fatorAmortecimento.png}
    \end{figure}
    \begin{figure}[H]
        \centering
        \includegraphics[width=0.3\textwidth]{images/m1c5k100_rcpAng_fatorAmortecimento.png}
        \includegraphics[width=0.3\textwidth]{images/m1c5k100_mblAng_fatorAmortecimento.png}
        \includegraphics[width=0.3\textwidth]{images/m1c5k100_inrAng_fatorAmortecimento.png}
    \end{figure}
\end{resolution}

\newpage
\begin{exercise}\label{ex7}
    Varie os parâmetros de massa e rigidez e verifique a mudança nas assíntotas da Receptância, Mobilidade e Inertância.
\end{exercise}
\begin{resolution}
    \noindent Mantém-se a Constante de Amortecimento constante e variam-se a Constante Elástica e a Massa como implementa-se o seguinte algoritmo para determinar os gráficos de módulo e fase:
    \begin{scriptsize}
        \myMatlab\lstinputlisting[]{ex7.m}
    \end{scriptsize}
    Obtêm-se os seguintes gráficos:
    \begin{figure}[H]
        \centering
        \includegraphics[width=0.3\textwidth]{images/m10c5k100_rcpAss_mass.png}
        \includegraphics[width=0.3\textwidth]{images/m10c5k100_mblAss_mass.png}
        \includegraphics[width=0.3\textwidth]{images/m10c5k100_inrAss_mass.png}
    \end{figure}
    \begin{figure}[H]
        \centering
        \includegraphics[width=0.3\textwidth]{images/m10c5k100_rcpAss_stiffness.png}
        \includegraphics[width=0.3\textwidth]{images/m10c5k100_mblAss_stiffness.png}
        \includegraphics[width=0.3\textwidth]{images/m10c5k100_inrAss_stiffness.png}
    \end{figure}
\end{resolution}

\newpage
\begin{exercise}\label{ex8}
    Determine a função de transferência que relaciona a força ao deslocamento.
\end{exercise}
\begin{resolution}
    Define-se a Função de Transferência como a razão entre uma Entrada e uma Saída para sistemas SISO, Single Input Single Output. Usualmente trabalha-se no espaço de Laplace para essa dedução. Dessa forma tem-se que:
    \begin{align*}
        m\ddot{x}(t) + c\dot{x}(t) + k x(t) &= f(t)\\
        m(s^2 X - sx(0) - \dot{x}(0)) + c(s X - x(0)) + kX &= F(s)\\
        X(s)(s^2m + sc + k) &= F(s) + x(0)(sm + c) + \dot{x}(0)m\\
        X(s) &= 
        \underbrace{\frac{1}{s^2m + sc + k}}_{G(s)} F(s) + 
        \underbrace{\frac{x(0)(sm + c) + \dot{x}(0)m}{s^2m + sc + k}}_{W(s)}
    \end{align*}
    Considera-se que $G(s)$, Função de Transferência, representa a Resposta Forçada da solução, Regime Permanente, enquanto $W(s)$ representa a Resposta Livre da solução, Regime Transiente.
    \begin{equation}
        \boxed{G(s) = \frac{X}{F} = \frac{1}{s^2m + sc + k}}
    \end{equation}
    Usualmente considera-se condições iniciais nulas implicando em $W(s) = 0$. Dessa forma implementa-se a função \texttt{ex8} em MATLAB:
    \begin{scriptsize}
        \myMatlab\lstinputlisting[]{ex8.m}
    \end{scriptsize}
    Note que a Função de Transferência implementada em MATLAB através da função \href{https://www.mathworks.com/help/control/ref/tf.html}{\texttt{tf}} necessita de 2 vetores representando os coeficientes do Numerador e do Denominador, respectivamente. Detalhes através do \texttt{help} do MATLAB.\\

    Obtém-se o seguinte gráfico:
    \begin{figure}[H]
        \centering
        \includegraphics[width=0.50\textwidth]{images/m1c1k1_tf_space.png}
    \end{figure}
\end{resolution}

\newpage
\begin{exercise}\label{ex9}
    Determine a função de transferência que relaciona a força à velocidade.
\end{exercise}
\begin{resolution}
    Baseado na solução apresentada no Exercício \ref{ex8} nota-se que, quando as condições iniciais são nulas, a Transformada de Laplace será simplificada para:
    \begin{equation}
        \mathcal{L} \lbrace\dot{x}\rbrace = sX 
    \end{equation}
    Observa-se que, ao multiplicar o valor de $X$ por $s$ será obtido, no espaço de Laplace, a derivada de $x$ que neste caso será a Velocidade. Assim têm-se que:
    \begin{equation}
        \boxed{G_{1}(s) = s G(s) = \frac{s}{s^2m + sc + k}}
    \end{equation}
    Dessa forma implementa-se a função \texttt{ex9} em MATLAB:
    \begin{scriptsize}
        \myMatlab\lstinputlisting[]{ex9.m}
    \end{scriptsize}
    Obtém-se o seguinte gráfico:
    \begin{figure}[H]
        \centering
        \includegraphics[width=0.50\textwidth]{images/m1c1k1_tf_speed.png}
    \end{figure}
\end{resolution}

\newpage
\begin{exercise}\label{ex10}
    Determine a função de transferência que relaciona a força à aceleração.
\end{exercise}
\begin{resolution}
    Baseado na solução apresentada no Exercício \ref{ex8} e na propriedade apresentada no Exercício \ref{ex9} têm-se que:
    \begin{equation}
        \boxed{G_{2}(s) = s^2 G(s) = \frac{s^2}{s^2m + sc + k}}
    \end{equation}
    Dessa forma implementa-se a função \texttt{ex10} em MATLAB:
    \begin{scriptsize}
        \myMatlab\lstinputlisting[]{ex10.m}
    \end{scriptsize}
    Obtém-se o seguinte gráfico:
    \begin{figure}[H]
        \centering
        \includegraphics[width=0.50\textwidth]{images/m1c1k1_tf_acceleration.png}
    \end{figure}
\end{resolution}

\newpage
\begin{exercise}\label{ex11}
    Gere os respectivos diagramas de Bode dessas funções.
\end{exercise}
\begin{resolution}
    Baseado nos códigos implementados nos Exercícios \ref{ex8}, \ref{ex9} e \ref{ex10} obteve-se os seguintes gráficos:
    \begin{figure}[H]
        \centering
        \includegraphics[width=0.30\textwidth]{images/m1c1k1_tf_space.png}
        \includegraphics[width=0.30\textwidth]{images/m1c1k1_tf_speed.png}
        \includegraphics[width=0.30\textwidth]{images/m1c1k1_tf_acceleration.png}
        % \caption{Bodes}
    \end{figure}
    Nota-se que os gráficos acima estão reduzidos para comparação lado a lado. Detalham-se também nos respectivos Exercícios. 
\end{resolution}

\newpage
\begin{exercise}\label{ex12}
    Compare os resultados desses diagramas de Bode com as respectivas Receptância, Mobilidade e Inertância. Discuta seus resultados. Note que a função Bode apresenta o eixo das ordenadas em dB. 
\end{exercise}
\begin{resolution}
    Baseado nos resultados obtidos nos Exercícios \ref{ex2}, \ref{ex3}, \ref{ex4}, \ref{ex8}, \ref{ex9} e \ref{ex10} compara-se os seguintes gráficos: 
    \begin{figure}[H]
        \centering
        \includegraphics[width=0.30\textwidth]{images/m1c1k1_rcpAbs.png}
        \includegraphics[width=0.32\textwidth]{images/m1c1k1_tf_space.png}
        \includegraphics[width=0.30\textwidth]{images/m1c1k1_rcpAng.png}
        % \caption{Bode Espaço}
    \end{figure}
    \begin{figure}[H]
        \centering
        \includegraphics[width=0.30\textwidth]{images/m1c1k1_mblAbs.png}
        \includegraphics[width=0.32\textwidth]{images/m1c1k1_tf_speed.png}
        \includegraphics[width=0.30\textwidth]{images/m1c1k1_mblAng.png}
        % \caption{Bode Espaço}
    \end{figure}
    \begin{figure}[H]
        \centering
        \includegraphics[width=0.30\textwidth]{images/m1c1k1_inrAbs.png}
        \includegraphics[width=0.32\textwidth]{images/m1c1k1_tf_acceleration.png}
        \includegraphics[width=0.30\textwidth]{images/m1c1k1_inrAng.png}
        % \caption{Bode Espaço}
    \end{figure}
    Nota-se que, apesar da diferença nas escalas visto que o Bode apresenta o eixo das ordenadas em dB e os ângulos de fase nas escalas não são os mesmos, os resultados são os mesmos o que é esperado pela explicação apresentada no próximo Exercício.
\end{resolution}

\newpage
\begin{exercise}\label{ex13}
    O que pode-se concluir sobre o paralelo de que $s = j\omega$, relacionando Transformada de Laplace com resposta em frequência? Comente sua resposta.
\end{exercise}
\begin{resolution}
    Parte-se das Funções de Transferência obtidas nos Exercícios \ref{ex8}, \ref{ex9} e \ref{ex10} e realiza-se a substituição de $s = j\omega$ obtendo as seguintes equações:
    \begin{enumerate}
        \item \textbf{Receptância}:
        \begin{equation}
            G(s) = \frac{1}{s^2m + sc + k} \to \boxed{G(j\omega) = \frac{1}{-\omega^2m + j\omega c + k} = \alpha(\omega)}
        \end{equation}

        \item \textbf{Mobilidade}:
        \begin{equation}
            G_1(s) = \frac{s}{s^2m + sc + k} \to \boxed{G_1(j\omega) = \frac{j\omega}{-\omega^2m + j\omega c + k} = y(\omega)}
        \end{equation}

        \item \textbf{Inertância}:
        \begin{equation}
            G_2(s) = \frac{s^2}{s^2m + sc + k} \to \boxed{G_2(j\omega) = \frac{-\omega^2}{-\omega^2m + j\omega c + k} = A(\omega)}
        \end{equation}
    \end{enumerate}
    Nota-se que ao aplicar a substituição saísse da Transformada de Laplace e chegasse na Resposta em Frequência. Isso ocorre por uma mudança de base saindo do espaço de Laplace e chegando no espaço do plano complexo.
\end{resolution}

\newpage
\begin{exercise}\label{ex14}
    Montar o respectivo Modelo de Estados tendo como entrada a força e como saídas a Posição, Velocidade e Aceleração.
\end{exercise}
\begin{resolution}
    Define-se o Modelo de Estados como um sistema de equações matriciais capaz de descreve a relação entre os Estados, conjunto arbitrário de variáveis do sistema estudado, e as derivadas desses Estados; e a relação entre os Estados, a Entrada e Saída. Genericamente representado através de:
    \begin{equation}
    \begin{cases}
    \begin{aligned}
        \lbrace \dot{x}(t) \rbrace &= 
        \lbrack A \rbrack \lbrace x(t) \rbrace + \lbrack B \rbrack f(t)\\
        \lbrace y(t) \rbrace &= 
        \lbrack C \rbrack \lbrace x(t) \rbrace + \lbrack D \rbrack f(t)
    \end{aligned}
    \end{cases}
    \end{equation}
    Onde:
    \begin{enumerate}
        \item \textbf{Vetores}:
        \begin{enumerate}
            \item $\lbrace x(t) \rbrace_{n\times i}$: Vetor de Estados;
            \item $\lbrace y(t) \rbrace_{j\times i}$: Vetor de Saídas;
        \end{enumerate}
        \item \textbf{Matrizes}:
        \begin{enumerate}
            \item $[A]_{n\times n}$: ;
            \item $[B]_{n\times i}$: ;
            \item $[C]_{j\times n}$: ;
            \item $[D]_{j\times i}$: ;
        \end{enumerate}
        \item \textbf{Dimensões}:
        \begin{enumerate}[noitemsep]
            \item $n$: Número de Estados;
            \item $i$: Número de Entradas;
            \item $j$: Número de Saídas;
        \end{enumerate}
    \end{enumerate}
    No caso de um Sistema de 1DOF Forçado com Amortecimento Viscoso será válida a seguinte expressão:
    \begin{equation}
    \begin{cases}
    \begin{aligned}
        \begin{Bmatrix} \dot{x}\\ \ddot{x} \end{Bmatrix} &=
        \begin{bmatrix} 0 & 1\\ -k/m & -c/m\end{bmatrix}\begin{Bmatrix} x\\ \dot{x} \end{Bmatrix} + 
        \begin{bmatrix} 0\\ 1/m \end{bmatrix} f\\
        \begin{Bmatrix} x\\ \dot{x}\\ \ddot{x} \end{Bmatrix} &=
        \begin{bmatrix} 1 & 0\\ 0 & 1\\ -k/m & -c/m\end{bmatrix}\begin{Bmatrix} x\\ \dot{x} \end{Bmatrix} + 
        \begin{bmatrix} 0\\ 0\\ 1/m \end{bmatrix} f\\
    \end{aligned}
    \end{cases}
    \end{equation}
    Nota-se que neste caso tomou-se como Vetor de Estados:
    \begin{equation*}
        \lbrace x(t) \rbrace = \begin{Bmatrix} x\\ \dot{x} \end{Bmatrix} 
        \to \begin{cases} x(t), &\text{Espaço}\\ \dot{x}(t),&\text{Velocidade} \end{cases}
    \end{equation*}
    Nota-se que neste caso tomou-se como Vetor de Saídas:
    \begin{equation*}
        \lbrace y(t) \rbrace = \begin{Bmatrix} x\\ \dot{x}\\ \ddot{x} \end{Bmatrix} 
        \to \begin{cases} x, &\text{Espaço}\\ \dot{x},&\text{Velocidade}\\ \ddot{x},&\text{Aceleração} \end{cases}
    \end{equation*}
\end{resolution}

\newpage
\begin{exercise}\label{ex15}
    Plotar os respectivos diagramas de Bode desse Modelo de Estados.
\end{exercise}
\begin{resolution}
    Baseado nos resultados obtidos no Exercício \ref{ex14} implementa-se a função \texttt{ex15} em MATLAB:
    \begin{scriptsize}
        \myMatlab\lstinputlisting[]{ex15.m}
    \end{scriptsize}
    Obtém-se o seguinte gráfico:
    \begin{figure}[H]
        \centering
        \includegraphics[width=0.75\textwidth]{images/m1c1k1_ss.png}
    \end{figure}
\end{resolution}

\newpage
\begin{exercise}\label{ex16}
    Comparar esses diagramas de Bode com os resultados anteriores. É possível verificar que as saídas são as mesmas que as respectivas Receptância, Mobilidade e Inertância?
\end{exercise}
\begin{resolution}
    Compara-se os diagramas obtidos:
    \begin{figure}[H]
        \centering
        \includegraphics[width=0.75\textwidth]{images/m1c1k1_ss.png}
    \end{figure}
    \begin{figure}[H]
        \centering
        \includegraphics[width=0.30\textwidth]{images/m1c1k1_tf_space.png}
        \includegraphics[width=0.30\textwidth]{images/m1c1k1_tf_speed.png}
        \includegraphics[width=0.30\textwidth]{images/m1c1k1_tf_acceleration.png}
    \end{figure}
    Nota-se que a solução de Estados é igual as soluções para Receptância, Mobilidade e Inertância.
\end{resolution}

\newpage
\begin{exercise}\label{ex17}
    Caso a amplitude da Força Harmônica seja $F = 10$, o que precisa ser adaptado nas formulações anteriores?
\end{exercise}
\begin{resolution}
    Nada precisaria ser alterado, pois nas soluções apresentadas não é considerada, pelo menos durante as deduções, as características da Entrada do sistema.
\end{resolution}
\end{document}