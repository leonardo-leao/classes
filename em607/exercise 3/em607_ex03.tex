\documentclass{article}
\usepackage{tpack}


% \usepgfplotslibrary{external} 
% \tikzexternalize


% \titleformat
%     {\subsection}           % Part
%     [block]                 % Part Shape
%     {\normalfont\Large}     % Font Size
%     {}                      % Label Numbering
%     {0mm}                   % Part Separation
%     {}                      % Code Before
%     []                      % Code After

%     \titlespacing*{\subsection}{0mm}{5mm}{2.5mm}


\title{EM607 - Vibrações Mecânicas}
\author{Guilherme Nunes Trofino}
\authorRA{217276}
\project{Atividade Teórica}


\begin{document}
    \maketitle
\newpage

\section{Atividade Teórica}
\paragraph{Apresentação}Resolução das questões de Vibrações Mecânicas por Guilherme Nunes Trofino, 217276, sobre a \textbf{Desacoplamento e Superposição Modal} de um Sistema de 1-DOF com Amortecimento Viscoso Forçado com as seguintes características:
\begin{enumerate}[noitemsep]
    \item $m_1$ e $m_2$, \textbf{Massa}: 1 kg;
    \item $k_2$, \textbf{Constante Elástica}: 0.8 MN/m;
    \item $k_1$ e $k_3$, \textbf{Constante Elástica}: 0.4 MN/m;
\end{enumerate}

\begin{exercise}\label{ex1}
    Determine as respectivas matrizes do Modelo Modal:
    \begin{enumerate}
        \item Matriz dos Autovalores: $[\bar{\omega}_n]$;
        \item Matriz dos Autovetores: $[\psi]$;
    \end{enumerate}
\end{exercise}
\begin{resolution}
    Considera-se as seguintes equações para o sistema de 2-DOF sem Amortecimento Viscoso Forçado por duas Entradas Harmônicas dadas por $f_1(t) = F_1 e^{j\omega t}$ e $f_2(t) = F_2 e^{j\omega t}$:
    \begin{equation*}
        \begin{cases}
            m_1 \ddot{x}_1 = -k_1 x_1 + k_2 (x_2 - x_1) + f_1\\
            m_2 \ddot{x}_2 = -k_3 x_2 - k_2 (x_2 - x_1) + f_2\\
        \end{cases}
        \to
        \underbrace{\begin{bmatrix} m_1 & 0\\ 0 & m_2\end{bmatrix}}_{\mathbf{M}} 
        \begin{Bmatrix} \ddot{x}_1\\ \ddot{x}_2\end{Bmatrix} + 
        \underbrace{\begin{bmatrix} k_1 + k_2 & -k_2\\ -k_2 & k_2 + k_3\end{bmatrix}}_{\mathbf{K}}
        \begin{Bmatrix} x_1\\ x_2\end{Bmatrix} =
        \begin{Bmatrix} f_1\\f_2\end{Bmatrix} 
    \end{equation*}
    Deseja-se encontrar as matrizes do Modelo Modal, dessa forma será necessário solucionar o seguinte Autoproblema $|\mathbf{K} - \omega^2\mathbf{M}| = 0$ como demonstrado a seguir:
    \begin{equation*}
        \begin{vmatrix}
            \begin{bmatrix} k_1 + k_2 & -k_2\\ -k_2 & k_2 + k_3\end{bmatrix} 
            -\omega^2 \begin{bmatrix} m_1 & 0\\ 0 & m_2\end{bmatrix}
        \end{vmatrix} = 0
        \quad\to\quad
        \begin{vmatrix}
            (k_1 + k_2) - \omega^2 m_1 & -k_2\\
            -k_2 & (k_2 + k_3) - \omega^2 m_2\\
        \end{vmatrix} = 0
    \end{equation*}
    Obtém-se a seguinte equação:
    \begin{equation*}
        \omega^4 m_1 m_2 - \omega^2 (m_2(k_1 + k_2) + m_1(k_2 + k_3)) + k_1 k_3 + k_2 k_3 + k_1 k_2 = 0
    \end{equation*}
    Quando considera-se os valores do exercício têm-se:
    \begin{equation*}
        \omega^4 - 2.4\cdot 10^{6} \omega^2 + 0.8\cdot 10^{12} = 0
        \quad \to \quad
        \begin{cases}
            \omega_1^2 = 0.4\cdot 10^{6}\text{ rad/s}\\
            \omega_2^2 = 2.0\cdot 10^{6}\text{ rad/s}
        \end{cases}
    \end{equation*}
    Na sequência será necessário substituir as Frequências Naturais obtidas na equação para determinar os respectivos Autovalores capazes de solucionar a seguinte equação $[\mathbf{K} - \omega^2\mathbf{M}]{x} = {0}$ como demonstrado a seguir:
    \begin{equation*}
        \begin{bmatrix} (k_1 + k_2) - \omega^2 m_1 & -k_2\\ -k_2 & (k_2 + k_3) - \omega^2 m_2\\\end{bmatrix} 
        \begin{Bmatrix} x_1 \\ x_2\end{Bmatrix} =
        \begin{Bmatrix} 0 \\ 0\end{Bmatrix}
        \quad \to \quad
        \begin{cases}
            \text{quando } \omega_1^2 \to x_1 = +x_2 & \psi_1 = \begin{Bmatrix} +1\\ +1\end{Bmatrix}\\
            \text{quando } \omega_2^2 \to x_1 = -x_2 & \psi_2 = \begin{Bmatrix} +1\\ -1\end{Bmatrix}
        \end{cases}
    \end{equation*}
    Dessa forma têm-se:
    \begin{equation}
        \boxed{
            [\bar{\omega}_n] = 
            \begin{bmatrix} \omega_1^2 & 0\\ 0 & \omega_2^2\end{bmatrix} = 
            \begin{bmatrix} 0.4 & 0\\ 0 & 2.0\end{bmatrix} \cdot 10^6
        }
        \quad\text{e}\quad
        \boxed{
            [\psi] = 
            \begin{bmatrix} \psi_1 & \psi_2\end{bmatrix} = 
            \begin{bmatrix} +1 & +1\\ +1 & -1\end{bmatrix}
        }
    \end{equation}
\end{resolution}


\newpage\begin{exercise}\label{ex2}
    Determine as respectivas matrizes na Base Modal:
    \begin{enumerate}
        \item Matriz de Massa Diagonalizadas: $\bar{m}_r$;
        \item Matriz de Rigidez Diagonalizadas: $\bar{k}_r$;
    \end{enumerate}
\end{exercise}
\begin{resolution}
    Parte-se da solução obtida no Exercício \ref{ex1} e realiza-se as seguintes operações:
    \begin{equation*}
        [\psi]^\intercal \mathbf{M} [\psi] = [\bar{m}_r] \to 
        \begin{bmatrix} +1 & +1\\ +1 & -1\end{bmatrix} 
        \begin{bmatrix} m_1 & 0\\ 0 & m_2\end{bmatrix} 
        \begin{bmatrix} +1 & +1\\ +1 & -1\end{bmatrix}
    \end{equation*}
    \begin{equation*}
        [\psi]^\intercal \mathbf{K} [\psi] = [\bar{k}_r] \to 
        \begin{bmatrix} +1 & +1\\ +1 & -1\end{bmatrix} 
        \begin{bmatrix} k_1 + k_2 & -k_2\\ -k_2 & k_2 + k_3\end{bmatrix} 
        \begin{bmatrix} +1 & +1\\ +1 & -1\end{bmatrix}
    \end{equation*}
    Dessa forma obtêm-se os seguintes resultados quando considera-se os valores do exercício:
    \begin{equation*}
        [\bar{m}_r] = 
        \begin{bmatrix} m_1+m_2 & m_1-m_2\\ m_1-m_2 & m_1+m_2\end{bmatrix} \to 
        \boxed{[\bar{m}_r] = \begin{bmatrix} 2 & 0\\ 0 & 2\end{bmatrix}}
    \end{equation*}
    \begin{equation*}
        [\bar{k}_r] = 
        \begin{bmatrix} k_1+k_3 & k_1-k_3\\ k_1-k_3 & k_1 + 4k_2 + k_3\end{bmatrix} \to 
        \boxed{[\bar{k}_r] = \begin{bmatrix} 0.8 & 0\\ 0 & 4.0\end{bmatrix}\cdot 10^6}
    \end{equation*}
    Nota-se que a Matriz dos Autovalores $[\psi]$ já considera os valores numéricos propostos pelo exercício. Dessa forma, os termos fora da diagonal principal devem ser zerados ao longo da dedução.\\

    Evidência-se que a Matriz dos Autovetores $[\psi]$ será sujeita a fatores de escala arbitrários. Dessa forma padroniza-se estas grandezas através da Normalização da Massa definida pela seguintes relação:
    \begin{equation}
        \boxed{[\phi] = [\psi] \times [\bar{m}_r]^{-1/2}}
        \quad \to \quad
        \boxed{[\phi]^\intercal \mathbf{M} [\phi] = [I]}
        \quad\text{e}\quad
        \boxed{[\phi]^\intercal \mathbf{K} [\phi] = [\bar{\omega}_r]}
    \end{equation}
    Onde $[\phi]$ será a Matriz de Autovetores Normalizada que será dada por:
    \begin{equation*}
        [\phi] = [\psi] \times [\bar{m}_r]^{-1/2}
        \to
        [\phi] = \begin{bmatrix} +1 & +1\\ +1 & -1\end{bmatrix} \begin{bmatrix} 1/\sqrt{2} & 0\\ 0 & 1/\sqrt{2}\end{bmatrix}
        \to 
        \boxed{[\phi] = \begin{bmatrix} \sqrt{2}/2 & \sqrt{2}/2\\ \sqrt{2}/2 & -\sqrt{2}/2\end{bmatrix}}
    \end{equation*}
    Na sequência implementa-se o seguinte algoritmo para determinar os gráficos de módulo e fase:
    \begin{scriptsize}
        \myMatlab\lstinputlisting[]{ex2.m}
    \end{scriptsize}
    Obtém-se seguinte resultado será obtido:
    \begin{scriptsize}
        \myMatlab\begin{lstlisting}
    MR =                KR = 1.0e+06 *          WR =                        PHI =
    
    1.0000         0        0.4000         0        400000           0          -0.7071   -0.7071
         0    1.0000             0    2.0000             0     2000000          -0.7071    0.7071
    \end{lstlisting}
    \end{scriptsize}
    Nota-se que o MATLAB já retorna a solução Normalizada do Autoproblema visto que a Matriz da Massa Normalizada será igual a Identidade e a Matriz da Rigidez será igual a Matriz das Frequências Naturais.\\

    Nota-se que a Matriz dos Autovalores Normalizada pela Massa possui o sinal oposto da obtida por cálculos. Entretanto isso não influenciou os resultados finais, pois a Normalização garante a compatibilidade dos resultados.
\end{resolution}


\newpage\begin{exercise}\label{ex3}
    Considere para os demais exercícios que $f_1 \neq 0$, Harmônica e não nula, e $f_2 = 0$.
\end{exercise}
\begin{exercise}\label{ex4}
    Determine as respectivas Equações do Movimento Desacopladas.
\end{exercise}
\begin{resolution}
    Considera-se genericamente as seguintes equações Matriciais para um sistema de n-DOF com Amortecimento Viscoso Forçado por Entradas Harmônicas:
    \begin{equation}
        \mathbf{M}\{\ddot{x}(t)\} + \mathbf{C}\{\dot{x}(t)\} + \mathbf{K}\{x(t)\} = \{f(t)\}\label{eq:nDOFAF}
    \end{equation}
    Na sequência realiza-se a seguinte Mudança de Base: $\{x\} = [\psi] \{p\}$, onde $[\psi]$ representa a Matriz de Autovalores não Normalizada. Dessa forma obtém-se:
    \begin{equation*}
        \mathbf{M}[\psi]\{\ddot{p}\} + \mathbf{C}[\psi]\{\dot{p}\} + \mathbf{K}[\psi]\{p\} = \{f\} \to
        \underbrace{[\psi]^\intercal\mathbf{M}[\psi]}_{[\bar{m}_{r}]}\{\ddot{p}\} + 
        \underbrace{[\psi]^\intercal\mathbf{C}[\psi]}_{[\bar{c}_{r}]}\{\dot{p}\} + 
        \underbrace{[\psi]^\intercal\mathbf{K}[\psi]}_{[\bar{k}_{r}]}\{p\} = 
        [\psi]^\intercal\{f\}
    \end{equation*}
    Nota-se que cada coordenada possui uma equação de um grau de liberdade, onde cada coordenada terá uma entrada composta por todas as entradas do sistema.\\
    
    Nota-se que a equação acima só será válida quando o Amortecimento for Proporcional, isto é, quando $[\bar{c}_r] = \alpha [\bar{m}_r] + \beta [\bar{k}_r]$.\\

    No caso do Exercício \ref{ex1} na condição do Exercício \ref{ex3} tém-se:
    \begin{equation*}
        \begin{bmatrix} 2 & 0\\ 0 & 2\end{bmatrix}\{\ddot{p}\} + 
        \begin{bmatrix} 0.8 & 0\\ 0 & 4.0\end{bmatrix}\cdot 10^6 \{p\} = 
        \begin{bmatrix} 1 & 1\\ 1 & -1\end{bmatrix}\begin{Bmatrix} f_1\\ 0\end{Bmatrix} \to
        \begin{cases}
            2 \ddot{p}_1 + 0.8 \cdot 10^6 p_1 = f_1(t)\\
            2 \ddot{p}_2 + 4.0 \cdot 10^6 p_2 = f_1(t)\\
        \end{cases}
    \end{equation*}
    Nota-se que neste caso $\alpha = \beta = 0$ para que o Desacoplamento seja possível.\\

    Na sequência implementa-se o seguinte algoritmo para determinar os gráficos de módulo e fase:
    \begin{scriptsize}
        \myMatlab\lstinputlisting[]{ex4.m}
    \end{scriptsize}
\end{resolution}


\newpage\begin{exercise}\label{ex5}
    Verifique as curvas de Resposta em Frequência associadas a cada Equação Desacopladas.
\end{exercise}
\begin{resolution}
    Considera-se a seguinte equação para um sistema de 1-DOF com Amortecimento Viscoso Forçado por uma Entrada Harmônica dada por $f(t) = F e^{j\omega t}$:
    \begin{equation}
        m_{ri}\ddot{p} + c_{ri}\dot{p} + k_{ri} p = f(t)\label{eq:1DOFAF}
    \end{equation}
    \noindent Propõem-se a seguinte solução: $p(t) = P e^{j\omega t}$. Na sequência substitui-se na Equação \ref{eq:1DOFAF} obtendo:
    \begin{align*}
        m\ddot{p}(t) + c\dot{p}(t) + k p(t) &= f(t)\\
        (-\omega^2 m + j \omega c + k) P e^{j\omega t} &= F e^{j\omega t}\\
        (-\omega^2 m + j \omega c + k) P \cancel{e^{j\omega t}} &= F \cancel{e^{j\omega t}}\\
        \frac{P}{F} &= \frac{1}{(k -\omega^2 m) + j \omega c}
    \end{align*}
    Nota-se que há o Deslocamento Harmônico divido pelo módulo da Força Harmônica o que configura a Receptância. Dessa forma conclui-se que:
    \begin{equation}
        \boxed{
            \alpha (\omega) = 
            \frac{p(t)}{f(t)} = 
            \frac{P e^{j\omega t}}{F e^{j\omega t}} = 
            \frac{P}{F} = 
            \frac{1}{(k -\omega^2 m) + j \omega c}
        }
    \end{equation}
    Define-se a Mobilidade como a razão entre a Velocidade e a amplitude da Força Harmônica que exita o sistema. Dessa forma tem-se, a partir da Equação \ref{eq:1DOFAF}, que:
    \begin{equation}
        \boxed{
            y (\omega) = 
            \frac{\dot{p}(t)}{f(t)} = 
            -j\omega\frac{P}{F} = 
            -j\omega \cdot \alpha(\omega) = 
            \frac{-j\omega}{(k -\omega^2 m) + j \omega c}
        }
    \end{equation}
    Define-se a Inertância como a razão entre a Aceleração e a amplitude da Força Harmônica que exita o sistema. Dessa forma tem-se, a partir da Equação \ref{eq:1DOFAF}, que:
    \begin{equation}
        \boxed{
            A (\omega) = 
            \frac{\ddot{p}(t)}{f(t)} = 
            -j\omega   \cdot     y (\omega) = 
            - \omega^2 \cdot \alpha(\omega) = 
            \frac{-\omega^2}{(k -\omega^2 m) + j \omega c}
        }
    \end{equation}
    Neste caso têm-se:
    \begin{equation*}
        \alpha_1(\omega) = \frac{1}{-2\omega^2 + 0.8\cdot 10^6} \qquad
         y_1(\omega) = \frac{j\omega}{-2\omega^2 + 0.8\cdot 10^6} \qquad
          A_1(\omega) = \frac{-\omega^2}{-2\omega^2 + 0.8\cdot 10^6}
    \end{equation*}    \begin{equation*}
        \alpha_2(\omega) = \frac{1}{-2\omega^2 + 4.0\cdot 10^6} \qquad
         y_2(\omega) = \frac{j\omega}{-2\omega^2 + 4.0\cdot 10^6} \qquad
          A_2(\omega) = \frac{-\omega^2}{-2\omega^2 + 4.0\cdot 10^6}
    \end{equation*}
    Na sequência implementa-se o seguinte algoritmo para determinar os gráficos de módulo e fase:
    \begin{scriptsize}
        \myMatlab\lstinputlisting[]{ex5.m}
    \end{scriptsize}
    Obtêm-se os seguintes gráficos:
    \begin{figure}[H]
        \centering
        \includegraphics[width=0.30\textwidth]{images/m1c0k2000000_rcpAbs_unc.png}
        \includegraphics[width=0.30\textwidth]{images/m1c0k2000000_mblAbs_unc.png}
        \includegraphics[width=0.30\textwidth]{images/m1c0k2000000_inrAbs_unc.png}
    \end{figure}
    \begin{figure}[H]
        \centering
        \includegraphics[width=0.30\textwidth]{images/m1c0k2000000_rcpAng_unc.png}
        \includegraphics[width=0.30\textwidth]{images/m1c0k2000000_mblAng_unc.png}
        \includegraphics[width=0.30\textwidth]{images/m1c0k2000000_inrAng_unc.png}
    \end{figure}
\end{resolution}


\newpage\begin{exercise}\label{ex6}
    Determine as Funções de Transferência da força para os deslocamentos de $m_1$ e $m_2$.
\end{exercise}
\begin{resolution}
    Considera-se a seguinte equação para um sistema de n-DOF com Amortecimento Viscoso Forçado por Entradas Harmônicas dada pela Equação \ref{eq:nDOFAF}:
    \begin{equation*}
        \mathbf{M}\{\ddot{x}(t)\} + \mathbf{C}\{\dot{x}(t)\} + \mathbf{K}\{x(t)\} = \{f(t)\}
    \end{equation*}
    Na sequência aplica-se Laplace considerando condições iniciais nulas para obter a Matriz de Transferência
    \begin{align*}
        \mathbf{M}\{\ddot{x}(t)\} + \mathbf{C}\{\dot{x}(t)\} + \mathbf{K}\{x(t)\} &= \{f(t)\}\\
        \{X(s)\}(\mathbf{M}s^2 + \mathbf{C}s + \mathbf{K}) &= \{F(s)\}\\
        \{X(s)\} &= \underbrace{(\mathbf{M}s^2 + \mathbf{C}s + \mathbf{K})^{-1}}_{\mathbf{G}(s)}\{F(s)\}
    \end{align*}
    Considera-se que $\mathbf{G}(s)$ será a Matriz de Transferência onde $G(s)_{ij}$ representa a Função de Transferência entre a Saída $i$ e a Entrada $j$.\\
    
    No caso do Exercício \ref{ex1} na condição do Exercício \ref{ex3} tém-se:
    \begin{align*}
        \mathbf{G}(s) &= \begin{bmatrix} s^2 m_1 + k_1 + k_2 & -k_2\\ -k_2 & s^2 m_2 + k_2 + k_3\end{bmatrix}^{-1}\\
        \mathbf{G}(s) &= 
        \frac{1}{|\mathbf{M}s^2 + \mathbf{K}|}
        \begin{bmatrix} s^2 m_2 + k_2 + k_3 & +k_2\\ +k_2 & s^2 m_1 + k_1 + k_3\end{bmatrix}\\
        \mathbf{G}(s) &= 
        \frac{1}{s^4 + 2.4\cdot 10^6 s^2 + 0.8\cdot 10^{12}}
        \begin{bmatrix} s^2 + 1.2\cdot 10^6 & 0.8\cdot 10^6\\ 0.8\cdot 10^6 & s^2 + 1.2\cdot 10^6\end{bmatrix}
    \end{align*}
    Nesse caso a Função de Transferência para $x_1$ e $x_2$ são, respectivamente:
    \begin{equation*}
        \boxed{G(s)_{11} = \frac{s^2 + 1.2\cdot 10^6}{s^4 + 2.4\cdot 10^6 s^2 + 0.8\cdot 10^{12}}}
        \qquad
        \boxed{G(s)_{21} = \frac{      0.8\cdot 10^6}{s^4 + 2.4\cdot 10^6 s^2 + 0.8\cdot 10^{12}}}
    \end{equation*}
    Na sequência implementa-se o seguinte algoritmo para determinar os gráficos de módulo e fase:
    \begin{scriptsize}
        \myMatlab\lstinputlisting[]{ex6.m}
    \end{scriptsize}
    Obtém-se seguinte resultado será obtido:
    \begin{scriptsize}
        \myMatlab\begin{lstlisting}
    GS =

    [ (s^2 + 1.2e6)/(s^4 + 2.4e6*s^2 + 0.8e12),         0.8e6/(s^4 + 2.4e6*s^2 + 0.8e12)]
    [         0.8e6/(s^4 + 2.4e6*s^2 + 0.8e12), (s^2 + 1.2e6)/(s^4 + 2.4e6*s^2 + 0.8e12)]
    \end{lstlisting}
    \end{scriptsize}
\end{resolution}


\newpage\begin{exercise}\label{ex7}
    Determine as curvas de Receptância de $m_1$ e $m_2$ das Equações Desacopladas. (\ref{ex3})
\end{exercise}
\begin{resolution}
    Considera-se a seguinte equação para um sistema de n-DOF com Amortecimento Viscoso Forçado por Entradas Harmônicas dada por $\{f(t)\} = \{F\} e^{i\omega t}$:
    \begin{equation}
        \mathbf{M}\{\ddot{x}(t)\} + \mathbf{C}\{\dot{x}(t)\} + \mathbf{K}\{x(t)\} = \{f(t)\}
    \end{equation}
    \noindent Propõem-se a seguinte solução: $\{x\}(t) = \{X\} e^{i\omega t}$. Na sequência substitui-se na Equação \ref{eq:nDOFAF} obtendo:
    \begin{align*}
        \mathbf{M}\{\ddot{x}(t)\} + \mathbf{C}\{\dot{x}(t)\} + \mathbf{K}\{x(t)\} &= \{f(t)\}\\
        (-\omega^2\mathbf{M} + i\omega\mathbf{C} + \mathbf{K}) \{X\} e^{i\omega t} &= \{X\}e^{i\omega t}\\
        (-\omega^2\mathbf{M} + i\omega\mathbf{C} + \mathbf{K}) \{X\} \cancel{e^{i\omega t}} &= \{X\}\cancel{e^{i\omega t}}\\
        \{X\} &= \underbrace{(-\omega^2\mathbf{M} + i\omega\mathbf{C} + \mathbf{K})^{-1}}_{[\alpha(\omega)]} \{F\}
    \end{align*}
    Nota-se que a Matriz de Receptância $[\alpha(\omega)]$ será dada pela seguinte equação:
    \begin{equation}
        [\alpha(\omega)] = (-\omega^2\mathbf{M} + i\omega\mathbf{C} + \mathbf{K})^{-1}
    \end{equation}
    Onde $\alpha_{jk}$ representa a Receptância entre a Saída $j$ e a Entrada $k$ será dada pela seguinte equação:
    \begin{equation}
        \alpha_{jk} (\omega) = 
        \sum_{r=1}^{n} \frac{(\psi_{rj})(\psi_{rk})}{(k_r -\omega^2 m_r) + i\omega c_r}
    \end{equation}
    Nesse caso a Receptância para $x_1$ e $x_2$ são, respectivamente:
    \begin{equation*}
        \alpha_{11} = \frac{x_1}{f_1} = \frac{0.5}{0.4\cdot 10^6 -\omega^2} + \frac{0.5}{2.0\cdot 10^6 -\omega^2}
        \qquad
        \alpha_{21} = \frac{x_2}{f_1} = \frac{0.5}{0.4\cdot 10^6 -\omega^2} + \frac{-0.5}{2.0\cdot 10^6 -\omega^2}
    \end{equation*}
    \begin{equation*}
        \boxed{\alpha_{11} = \frac{-\omega^2 + 1.2\cdot 10^6}{\omega^4 - 2.4\cdot 10^6 \omega^2 + 0.8\cdot 10^{12}}}
        \qquad
        \boxed{\alpha_{21} = \frac{0.8\cdot 10^6}{\omega^4 - 2.4\cdot 10^6 \omega^2 + 0.8\cdot 10^{12}}}
    \end{equation*}
    Na sequência implementa-se o seguinte algoritmo:
    \begin{scriptsize}
        \myMatlab\lstinputlisting[]{ex7.m}
    \end{scriptsize}
\end{resolution}


\newpage\begin{exercise}\label{ex8}
    Nota-se o efeito da Superposição Modal em Frequência?
\end{exercise}
\begin{resolution}
    % \begin{figure}[H]
    %     \centering
    %     \includegraphics[width=0.45\textwidth]{images/m1c0k2000000_rcpAbs_unc_zoom.png}
    %     \includegraphics[width=0.45\textwidth]{images/alpha21_rcpAbs_cpl_zoom.png}
    % \end{figure}
    Primeiramente considera-se as seguintes imagens obtidas a partir das Coordenadas Originais do problema:
    \begin{figure}[H]
        \centering
        \includegraphics[width=0.45\textwidth]{images/alpha11_rcpAbs_cpl_zoom.png}
        \includegraphics[width=0.45\textwidth]{images/alpha21_rcpAbs_cpl_zoom.png}
    \end{figure}
    Na sequência considera-se a seguinte imagem obtida a partir das Coordenadas Desacopladas:
    \begin{figure}[H]
        \centering
        \includegraphics[width=0.45\textwidth]{images/mkc_rcpAbs_unc_zoom.png}
    \end{figure}
    Nota-se que as Coordenadas Originais estão relacionadas pela seguinte relação $\{x\} = [\phi] \{p\}$, onde $[\phi]$ representa a Matriz de Autovalores Normalizada. Dessa forma tém-se:
    \begin{equation*}
        \begin{Bmatrix} x_1 \\ x_2\end{Bmatrix} = 
        \begin{bmatrix} +\sqrt{2}/2 & +\sqrt{2}/2\\ +\sqrt{2}/2 & -\sqrt{2}/2\end{bmatrix} \begin{Bmatrix} p_1\\ p_2\end{Bmatrix} = 
        -\frac{\sqrt{2}}{2} \begin{bmatrix} p_1 + p_2\\ p_1 - p_2\end{bmatrix}
    \end{equation*}
    Nota-se que, como esperado para uma mudança de base, as Coordenadas Originais são descritas como uma combinação linear das Coordenadas Desacopladas. Assumindo a que o sistema seja Linear Invariante no Tempo tém-se:
    \begin{equation*}
        \begin{Bmatrix} \alpha_{x_1} \\ \alpha_{x_2}\end{Bmatrix} = 
        -\frac{\sqrt{2}}{2} \begin{bmatrix} \alpha_{p_1} + \alpha_{p_2}\\ \alpha_{p_1} - \alpha_{p_2}\end{bmatrix}
    \end{equation*}
    Nota-se que, como demonstrado no gráfico, a Receptância nas Coordenadas Originais será uma combinação linear das Receptâncias nas Coordenadas Desacopladas.
\end{resolution}


\newpage\begin{exercise}\label{ex9}
    Formule o sistema no Modelo de Estados.
\end{exercise}
\begin{resolution}
    Considera-se a Equação \ref{eq:nDOFAF} para um sistema de n-DOF com Amortecimento Viscoso Forçado por uma Entrada Harmônica. Realiza-se o seguinte procedimento:\\

    \textbf{Equações de Equilíbrio}: Depois de realizado o DCL do sistema obtêm-se as seguintes equações de equilíbrio através da 2\textsuperscript{a} Lei de Newton:
    \begin{equation}\label{eq1}
        \begin{dcases}
            m_1 \ddot{x}_1 +k_1 x_1 - k_2 (x_2 - x_1) = f_1 & 
            \ddot{x}_1 = -\frac{k_1}{m_1} x_1 +\frac{k_2}{m_1} (x_2 - x_1) + f_1\\
            m_2 \ddot{x}_2 = +k_3 x_2 + k_2 (x_2 - x_1) = f_2 &
            \ddot{x}_2 = -\frac{k_3}{m_2} x_2 -\frac{k_2}{m_2} (x_2 - x_1) + f_2\\
        \end{dcases}
    \end{equation}

    \textbf{Definição de Estados}: Buscando evitar confusões entre as variáveis do sistema propõem-se o seguinte Vetor de Estados:
    \begin{equation*}
        \lbrace x(t) \rbrace = \begin{Bmatrix} e_1\\ e_2\\ e_3\\ e_4 \end{Bmatrix} 
        \to
        \begin{cases}
            e_1 = x_1(t), &\text{Deslocamento } m_1\\
            e_2 = x_2(t), &\text{Deslocamento } m_2\\
            e_3 = \dot{x}_1(t),&\text{Velocidade } m_1\\
            e_4 = \dot{x}_2(t),&\text{Velocidade } m_2
        \end{cases}
    \end{equation*}
    \textbf{Relação entre Estados}: Nota-se que, nesta realização, obtém-se as seguintes relações: $e_3 = \dot{e}_1$ e $e_4 = \dot{e}_2$.\\

    \textbf{Substituição nas Equações}: Realiza-se os Estados dentro da Equação \ref{eq1} obtendo a seguinte equação:
    \begin{equation*}
        \begin{dcases}
            \dot{e}_3 = -\frac{k_1}{m_1} e_1 +\frac{k_2}{m_1} (e_2 - e_1) + f_1\\
            \dot{e}_4 = -\frac{k_3}{m_2} e_2 -\frac{k_2}{m_2} (e_2 - e_1) + f_2\\
        \end{dcases}
    \end{equation*}
    
    \textbf{Equações Matriciais}: Deseja-se obter os Deslocamentos das massas. Dessa forma tomou-se como Vetor de Saídas:
    \begin{equation*}
        \lbrace y(t) \rbrace = 
        \begin{Bmatrix} y_1\\ y_2 \end{Bmatrix} 
        \to 
        \begin{cases}
            y_1(t), &\text{Deslocamento $m_1$}\\
            y_2(t), &\text{Deslocamento $m_2$}
        \end{cases}
    \end{equation*}
    Dessa forma obtêm-se a seguinte realização de Estados:
    \begin{equation}
        \begin{cases}
        \begin{aligned}
            \begin{Bmatrix} \dot{e}_{1}\\ \dot{e}_{2}\\ \dot{e}_{3}\\ \dot{e}_{4} \end{Bmatrix} &=
            \underbrace{
            \begin{bmatrix}
                0 & 0 & 1 & 0\\
                0 & 0 & 0 & 1\\
                -(k_1 + k_2)/m_1 & +k_2/m_1 & 0 & 0\\
                +k_2/m_2 & -(k_2 + k_3)/m_2 & 0 & 0
            \end{bmatrix}}_{\mathbf{A}}
            \begin{Bmatrix} e_{1}\\ e_{2}\\ e_{3}\\ e_{4} \end{Bmatrix} + 
            \underbrace{
                \begin{bmatrix}
                    0 & 0\\
                    0 & 0\\
                    1/m_1 & 0\\
                    0 & 1/m_2\\
                \end{bmatrix}}_{\mathbf{B}}
            \begin{Bmatrix} f_1\\f_2 \end{Bmatrix}\\

            \begin{Bmatrix} y_1\\ y_2 \end{Bmatrix} &=
            \underbrace{
                \begin{bmatrix}
                    1 & 0 & 0 & 0\\
                    0 & 1 & 0 & 0\\
                \end{bmatrix}}_{\mathbf{C}}
            \begin{Bmatrix} e_{1}\\ e_{2}\\ e_{3}\\ e_{4} \end{Bmatrix} + 
            \underbrace{\begin{bmatrix} 0 & 0\\ 0 & 0 \end{bmatrix}}_{\mathbf{D}}
            \begin{Bmatrix} f_1\\f_2 \end{Bmatrix}\\
        \end{aligned}
        \end{cases}
    \end{equation}
\end{resolution}


\newpage\begin{exercise}\label{ex10}
    Determine as curvas de Receptância de $m_1$ e $m_2$ no Modelo de Estados. (\ref{ex3})
\end{exercise}
\begin{resolution}
    Na sequência implementa-se o seguinte algoritmo:
    \begin{scriptsize}
        \myMatlab\lstinputlisting[]{ex10.m}
    \end{scriptsize}
    Obtém-se o seguinte gráfico:
    \begin{figure}[H]
        \centering
        \includegraphics[width=0.75\textwidth]{images/rcpAbs_ss.png}
    \end{figure}
    Nota-se que não resultado causado pela força $f_2(\omega)$ pois esta foi considerada como zero para o desenvolvimento do exercício.
\end{resolution}


\newpage\begin{exercise}\label{ex11}
    Compare as respostas do Modelo de Estados com das Equações Desacopladas. Discuta.
\end{exercise}
\begin{resolution}
    Obtém-se o seguinte gráfico:
    \begin{figure}[H]
        \centering
        \includegraphics[width=0.45\textwidth]{images/mkc_rcpAbs_unc_zoom.png}
        \includegraphics[width=0.45\textwidth]{images/mkc_rcpAng_unc.png}
    \end{figure}
    \begin{figure}[H]
        \centering
        \includegraphics[width=0.65\textwidth]{images/rcpAbs_ss.png}
    \end{figure}
    Construiu-se o Modelo de Estados para retornar a Magnitude o Ângulo da Receptância de cada uma das massas nas Coordenadas Originais.\\
    
    Nota-se que as Curvas de Magnitude são equivalentes como justificado no Exercício \ref{ex8} entretanto os Ângulos verificados não foram os mesmos.
\end{resolution}


\newpage\begin{exercise}\label{ex12}
    Inclua Amortecimento Proporcional ao problema e verifique o Desacoplamento das Equações. Analise novamente as respectivas curvas variando o amortecimento.
\end{exercise}
\begin{resolution}
    
\end{resolution}


\newpage\begin{exercise}\label{ex13}
    Inclua Amortecimento Proporcional e verifique a Diagonalização das Matrizes de Massa e de Rigidez.
\end{exercise}
\begin{resolution}
    
\end{resolution}


\newpage\begin{exercise}\label{ex14}
    Determine a Matriz de Amortecimento Modal em sua Forma Padronizada. ($2\xi\omega_n$)
\end{exercise}
\begin{resolution}
    
\end{resolution}

\end{document}