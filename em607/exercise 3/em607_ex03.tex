\documentclass{article}
\usepackage{tpack}


% \usepgfplotslibrary{external} 
% \tikzexternalize


% \titleformat
%     {\subsection}           % Part
%     [block]                 % Part Shape
%     {\normalfont\Large}     % Font Size
%     {}                      % Label Numbering
%     {0mm}                   % Part Separation
%     {}                      % Code Before
%     []                      % Code After

%     \titlespacing*{\subsection}{0mm}{5mm}{2.5mm}


\title{EM607 - Vibrações Mecânicas}
\author{Guilherme Nunes Trofino}
\authorRA{217276}
\project{Atividade Teórica}


\begin{document}
    \maketitle
\newpage

\section{Atividade Teórica}
\paragraph{Apresentação}Resolução das questões de Vibrações Mecânicas por Guilherme Nunes Trofino, 217276, sobre a \textbf{Desacoplamento e Superposição Modal} de um Sistema de 1-DOF com Amortecimento Viscoso Forçado com as seguintes características:
% \begin{enumerate}
%     \item $m_1$, \textbf{Massa}: 1 kg;
%     \item $c_1$, \textbf{Constante de Amortecimento}: 1 N.s/m;
%     \item $k_1$, \textbf{Constante Elástica}: 1 N/m;
% \end{enumerate}

\begin{exercise}\label{ex1}
    Determine as respectivas matrizes do Modelo Modal:
    \begin{enumerate}
        \item Matriz dos Autovalores: $[\bar{\omega}_n]$;
        \item Matriz dos Autovetores: $[\psi]$;
    \end{enumerate}
\end{exercise}
\begin{resolution}
    
\end{resolution}


\newpage\begin{exercise}\label{ex2}
    Determine as respectivas matrizes na Base Modal:
    \begin{enumerate}
        \item Matriz de Massa Diagonalizadas: $\bar{m}_r$;
        \item Matriz de Rigidez Diagonalizadas: $\bar{k}_r$;
    \end{enumerate}
\end{exercise}
\begin{resolution}
    
\end{resolution}


\newpage\begin{exercise}\label{ex3}
    Considere para os demais exercícios que $f_1 \neq 0$, Harmônica e não nula, e $f_2 = 0$.
\end{exercise}
\begin{exercise}\label{ex4}
    Determine as respectivas Equações do Movimento Desacopladas.
\end{exercise}
\begin{resolution}
    
\end{resolution}


\newpage\begin{exercise}\label{ex5}
    Verifique as curvas de Resposta em Frequência associadas a cada Equação Desacopladas.
\end{exercise}
\begin{resolution}
    
\end{resolution}


\newpage\begin{exercise}\label{ex6}
    Determine as Funções de Transferência da força para os deslocamentos de $m_1$ e $m_2$.
\end{exercise}
\begin{resolution}
    
\end{resolution}


\newpage\begin{exercise}\label{ex7}
    Determine as curvas de Receptância de $m_1$ e $m_2$ das Equações Desacopladas. (\ref{ex3})
\end{exercise}
\begin{resolution}
    
\end{resolution}


\newpage\begin{exercise}\label{ex8}
    Nota-se o efeito da Superposição Modal em Frequência?
\end{exercise}
\begin{resolution}
    
\end{resolution}


\newpage\begin{exercise}\label{ex9}
    Formule o sistema no Modelo de Estados.
\end{exercise}
\begin{resolution}
    
\end{resolution}


\newpage\begin{exercise}\label{ex10}
    Determine as curvas de Receptância de $m_1$ e $m_2$ no Modelo de Estados. (\ref{ex3})
\end{exercise}
\begin{resolution}
    
\end{resolution}


\newpage\begin{exercise}\label{ex11}
    Compare as respostas do Modelo de Estados com das Equações Desacopladas. Discuta.
\end{exercise}
\begin{resolution}
    
\end{resolution}


\newpage\begin{exercise}\label{ex12}
    Inclua Amortecimento Proporcional ao problema e verifique o Desacoplamento das Equações. Analise novamente as respectivas curvas variando o amortecimento.
\end{exercise}
\begin{resolution}
    
\end{resolution}


\newpage\begin{exercise}\label{ex13}
    Inclua Amortecimento Proporcional e verifique a Diagonalização das Matrizes de Massa e de Rigidez.
\end{exercise}
\begin{resolution}
    
\end{resolution}


\newpage\begin{exercise}\label{ex14}
    Determine a Matriz de Amortecimento Modal em sua Forma Padronizada. ($2\xi\omega_n$)
\end{exercise}
\begin{resolution}
    
\end{resolution}

\end{document}