\documentclass{article}
\usepackage{tpack}


% \usepgfplotslibrary{external} 
% \tikzexternalize


% \titleformat
%     {\subsection}           % Part
%     [block]                 % Part Shape
%     {\normalfont\Large}     % Font Size
%     {}                      % Label Numbering
%     {0mm}                   % Part Separation
%     {}                      % Code Before
%     []                      % Code After

%     \titlespacing*{\subsection}{0mm}{5mm}{2.5mm}


\title{EM607 - Vibrações Mecânicas}
\author{Guilherme Nunes Trofino}
\authorRA{217276}
\project{Atividade Teórica}


\begin{document}
    \maketitle
\newpage

\section{Atividade Teórica}
\paragraph{Apresentação}Resolução das questões de Vibrações Mecânicas por Guilherme Nunes Trofino, 217276, sobre a \textbf{Modelagem de Estados} de um Sistema de 1-DOF com Amortecimento Viscoso Forçado com as seguintes características:
\begin{enumerate}
    \item $m_1$, \textbf{Massa}: 1 kg;
    \item $c_1$, \textbf{Constante de Amortecimento}: 1 N.s/m;
    \item $k_1$, \textbf{Constante Elástica}: 1 N/m;
\end{enumerate}

\newpage\begin{exercise}\label{ex1}
    Determinar o Modelo de Estados com a força de entrada $f(t)$ e as saídas posição, velocidade e aceleração de $m_1$ seguindo o seguinte procedimento:
    \begin{enumerate}
        \item Equações de Equilíbrio;
        \item Definir Estados;
        \item Estabelecer a relação entre os Estados;
        \item Substituir os Estados nas Equações de Equilíbrio;
        \item Colocar na forma matricial de Estados;
        \item Explicitar as Matrizes $\mathbf{A}$, $\mathbf{B}$, $\mathbf{C}$ e $\mathbf{D}$;
    \end{enumerate}
\end{exercise}
\begin{resolution}

\end{resolution}

\newpage\begin{exercise}\label{ex2}
    Implementar o Modelo de Estados no MATLAB usando o comando \texttt{ss}.
\end{exercise}
\begin{resolution}

\end{resolution}

\newpage\begin{exercise}\label{ex3}
    Obter a matriz de transferência do sistema usando o comando \texttt{tf}.
\end{exercise}
\begin{resolution}

\end{resolution}

\newpage\begin{exercise}\label{ex4}
    Utilize a matriz de transferência para obter o modelo de Estados correspondente usando o comando \texttt{ss}. Verifique se o sistema obtido é igual ao inicial.\\

    Utilize o comando \texttt{damp} para verificar as propriedades dos sistemas. Verifique os polos do sistema usando o comando \texttt{eig}.
\end{exercise}
\begin{resolution}

\end{resolution}

\newpage\begin{exercise}\label{ex5}
    Obter a resposta do sistema ao impulso unitário e ao degrau unitário usando os comandos \texttt{impulse} e \texttt{step} respectivamente.
\end{exercise}
\begin{resolution}

\end{resolution}

\newpage\begin{exercise}\label{ex6}
    Obter a resposta do sistema a rampa unitária usando o comando \texttt{lsim} e usando artificialmente o comando \texttt{step}. Compare as respostas.
\end{exercise}
\begin{resolution}

\end{resolution}
\end{document}