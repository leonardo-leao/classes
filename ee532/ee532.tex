\documentclass{article}

\usepackage[a4paper, hmargin={20mm, 20mm}, vmargin={25mm, 30mm}]{geometry}
\usepackage[utf8]{inputenc}
\usepackage[english, main=portuguese]{babel}

\usepackage[hidelinks]{hyperref}
\usepackage{bookmark}
\usepackage{cancel}
\usepackage{comment}

\usepackage{array}
\usepackage{indentfirst}
\usepackage{multicol}
\setlength{\multicolsep}{2pt}% 50% of original values
\usepackage{subfiles}

\usepackage{titlesec}

\usepackage{amsmath}
\usepackage{amssymb}
\usepackage{systeme}
\usepackage{float}
\usepackage{enumitem}
\usepackage[thinc]{esdiff} %parcial derivatives
\restylefloat{table}

\usepackage{wrapfig}
\usepackage{graphicx}
\usepackage{subcaption}
\graphicspath{ {./IMAGENS/} }

% Pacote para a definição de novas cores
\usepackage{xcolor}
% Definindo novas cores
\definecolor{darkgreen}{rgb}{0.0, 0.42, 0.24}
\definecolor{darkpurple}{rgb}{0.74, 0.2, 0.64}
\definecolor{darkblue}{rgb}{0.0, 0.28, 0.67}

% Configurando espaços entre paragrafos
%\setlength{\parskip}{0.5em}

%Configurando numeração de objetos; figuras, equações e etc., em ambientes; section, subsection e etc..
\usepackage{chngcntr}
\counterwithin{figure}{section}
\counterwithin{equation}{subsection}

% Configurando pacote de circuitos
\usepackage{circuitikz}

%Configurando pacote de Gráficos
\usepackage{tikz}

% Configurando layout para mostrar códigos
\usepackage{listings}

\newcommand{\myStyle}{
\lstset{
    language=Octave,                            % the language of the code
    basicstyle=\ttfamily\small,               % the size of the fonts that are used for the code
    keywordstyle=\color{darkpurple}\bfseries, %
    stringstyle=\color{darkblue},             %
    commentstyle=\color{darkgreen},           %
    morecomment=[s][\color{blue}]{/**}{*/},   %
    extendedchars=true,                       %
    showtabs=false,                           % show tabs within strings adding particular underscores
    showspaces=false,                         % show spaces adding particular underscores
    showstringspaces=false,                   % underline spaces within strings
    numbers=left,                             % where to put the line-numbers
    numberstyle=\tiny\color{gray},            % the style that is used for the line-numbers
    stepnumber=1,                             % the step between two line-numbers. If it's 1, each line will be numbered
    numbersep=5pt,                            % how far the line-numbers are from the code
    frame=single,                             % adds a frame around the code
    rulecolor=\color{black},                  % if not set, the frame-color may be changed on line-breaks within not-black text
    breaklines=true,                          % sets automatic line breaking
    backgroundcolor=\color{white},            % choose the background color
    breakatwhitespace=true,                   % sets if automatic breaks should only happen at whitespace
    breakautoindent=false,                    %
    captionpos=b,                             % sets the caption-position to bottom
    xleftmargin=0pt,                          %
    tabsize=2,                                % sets default tabsize to 2 spaces
}}

%\titleformat{<command>}[<shape>]{<format>}{<label>}{<sep>}{<before-code>}[<after-code>]
\titleformat
{\section} %comand
[block]  %shape
{\normalfont\LARGE} %format
{\thesection. } %label
{0mm} %sep
{} %before-code
[{\titlerule[0.1mm]}] %after-code

\titlespacing*{\section}{0mm}{0mm}{15mm}

\titleformat
{\subsection} %comand
[block]  %shape
{\normalfont\Large} %format
{\thesubsection. } %label
{0mm} %sep
{} %before-code
[] %after-code

\titlespacing*{\subsection}{0mm}{5mm}{2.5mm}


\begin{document}
    \begin{titlepage}
        \begin{center}
            \rule{450pt}{0.5pt}\\[4mm]
            {\Huge EE532 - Eletrônica Aplicada}\\
            \rule{450pt}{0.5pt}\\[2mm]
            {\Large Resumo Teórico}\\[200mm]
            \today\\
            \rule{250pt}{0.5pt}\\
            {\large Guilherme Nunes Trofino}\\
            {\large 217276}\\
        \end{center}
    \end{titlepage}
\newpage

    \tableofcontents
\newpage

    \section{Semicondutores}
        \subsection{Introdução}
            \paragraph{Definição}Semicondutores são substâncias, ou elementos químicos, com características intermediárias entre isolantes, baixa condutividade elétrica, e condutores, alta condutividade elétrica, tornando-se a base da eletrônica moderna. Diferentemente dos metais, portadores de elétrons livres em qualquer temperatura, estes materiais necessitam de energia para possuirem elétrons livres.


        \subsection{Geração Térmica}
            \paragraph{Definição}Movimentações de elétrons ao longo das camadas de condução e valência liberam energia como calor que poderá ser estimado de acordo com as seguintes equações:
                \[
                    n = p = n_{i} \hspace{2.5mm} \text{[cm\textsuperscript{-3}]}
                \]
            Onde:
                \begin{itemize}[noitemsep]
                    \item \textit{n} é a concentração dos elétrons livres na camada de condução;
                    \item \textit{p} é a concentração de lacunas livres na camada de valência;
                    \item \textit{p\textsubscript{i}} é a concentração ou densidade intrínseca de portadores;
                \end{itemize}
            Assim:
                \begin{equation}
                    \boxed{
                        n_{i} = B T^{3/2} e^{-E_{G}/2kT}
                    }\label{eq:8}
                \end{equation}
            Onde:
                \begin{itemize}[noitemsep]
                    \item \textit{B} é propriedade do material: $5,2 \times 10^{15}$ para o Silício;
                    \item \textit{k} é a constante de Boltzmann: $1,38 \times 10^{-23}$ [J/K];
                    \item \textit{E\textsubscript{G}} é a energia de geração: $1,792 \times 10^{-19}$ [J];
                    \item \textit{T} é temperatura absoluta em Kelvins;
                \end{itemize}

        \subsection{Intrínsecos}
            \paragraph{Definição}Nesta organização a substância não apresenta impurezas, assim lacunas são geradas a partir da quebra de ligações covalentes através do fornecimento de energia externa.

        \subsection{Extrínsecos}
            \paragraph{Definição}Nesta organização a substância apresenta impurezas, nomeadas dopagens que, apesar de possuirem diferentes composições, apresentam cargas mensuradas, classificadas de acordo com as cargas disponíveis, ou ausentes, das impurezas como descrito a seguir: 
                \begin{enumerate}
                    \item \textbf{Dopagem Negativa:} Inclusão de impurezas com excesso de elétrons;
                        \begin{enumerate}[noitemsep]
                            \item Maioria de Carga:
                                \[
                                    p \approx N_{A}
                                \]
                            \item Minoria de Carga:
                                \[
                                    n \approx \frac{n_{i}^{2}}{N_{A}}
                                \]
                        \end{enumerate}
                    \item \textbf{Dopagem Positiva:} Inclusão de impurezas com falta de elétrons;
                        \begin{enumerate}[noitemsep]
                            \item Maioria de Carga:
                                \[
                                    n \approx N_{D}
                                \]
                            \item Minoria de Carga:
                                \[
                                    p \approx \frac{n_{i}^{2}}{N_{D}}
                                \]
                        \end{enumerate}
                \end{enumerate}

        \subsection{Corrente de Deriva}
            \paragraph{Definição}Quando submetidos a diferenças de potencial haverá fluxo de elétrons trocando de posição ao longo das lácunas a velocidade aproximadamente constante, expressa pelas seguintes equações:
                \begin{multicols}{2}
                    \begin{equation}
                        \boxed{
                            \vec{v_{p}} = \mu_{p} \vec{E}
                        }\label{eq:1}
                    \end{equation}
                    \begin{equation}
                        \boxed{
                            \vec{v_{n}} = - \mu_{n} \vec{E}
                        }\label{eq:2}
                    \end{equation}
                \end{multicols}\noindent
            Onde:
                \begin{enumerate}[noitemsep]
                    \item $\mu$: Mobilidade dos Portadores [cm\textsuperscript{2}/Vs];
                    \item $\vec{E}$: Campo Elétrico [V/m];
                \end{enumerate}

            \paragraph{Densidade de Corrente}Analisar o comportamento da corrente relativa as cargas presentes no material, expressa pelas seguintes equações:
                \begin{multicols}{2}
                    \begin{equation}
                        \boxed{
                            \vec{J}_{pd} = q p \vec{v}_{p} = q \mu_{p} p \vec{E}
                        }\label{eq:3}
                    \end{equation}
                    \begin{equation}
                        \boxed{
                            \vec{J}_{nd} = - q n \vec{v}_{n} = q \mu_{n} n \vec{E}
                        }\label{eq:4}
                    \end{equation}
                \end{multicols}\noindent
            Onde:
                \begin{enumerate}[noitemsep]
                    \item $\mu$: Mobilidade dos Portadores [cm\textsuperscript{2}/Vs];
                    \item $q$: Carga Elementar do Elétron: $1,6 \times 10^{-19} C$;
                    \item $p$: Densidade de Lacunas [lacunas/cm\textsuperscript{3}];
                    \item $n$: Densidade de Elétrons [elétrons/cm\textsuperscript{3}];
                \end{enumerate}
            Assim, a densidade de corrente total, $\vec{J}_{T}$, de um semicondutore intrínseco será igual ao somatório das densidades de lacunas, $\vec{J}_{p}$, e densidade de elétrons, $\vec{J}_{n}$, como na seguinte equação:
                \begin{equation}
                    \boxed{
                        \vec{J}_{Td} = \vec{J}_{pd} + \vec{J}_{nd} = q (\mu_{n} n + \mu_{p} p)\vec{E}
                    }\label{eq:5}
                \end{equation}

        \subsection{Corrente de Difusão}
            \paragraph{Difusão}Quando submetidos a diferenças de potencial estes matériais apresentaram fluxo de elétrons para regiões com menores concentrações de cargas, como nas seguintes equações:
                \begin{multicols}{2}
                    \begin{equation}
                        \boxed{
                            \vec{J}_{p} = - q D_{p} \nabla p(x, y, z)
                        }\label{eq:6}
                    \end{equation}
                    \begin{equation}
                        \boxed{
                            \vec{J}_{n} = q D_{n} \nabla n(x, y, z)
                        }\label{eq:7}
                    \end{equation}
                \end{multicols}\noindent
            Onde:
                \begin{enumerate}[noitemsep]
                    \item $q$: Carga Elementar do Elétron: $1,6 \times 10^{-19} C$;
                    \item $D_{p}$: Constante de Difusão das Lacunas;
                    \item $D_{n}$: Constante de Difusão dos Elétrons;
                    \item $\nabla p$: Gradiente de Densidade de Lacunas;
                    \item $\nabla n$: Gradiente de Densidade de Elétrons;
                \end{enumerate}
\newpage

    \section{Diodo}
        \paragraph{Definição}Junção de semicondutores extrínsecos do tipo N, excesso de elétrons, e semicondutores extrínsecos do tipo P, excesso de lácunas. Nesta configuração a energia só poderá fluir em um sentido, sendo bloqueada no sentido contrário, comumente representado por um Diodo como mostrado a seguir:
            \begin{figure}[H]
                \centering
                \begin{circuitikz}
                    \ctikzset{component text=left}
                    \draw
                    (0,0) to[empty diode, o-o] (-2,0);
                \end{circuitikz} 
                \caption{Representação de Diodo}
            \end{figure}

        \paragraph{Região de Depleção}Inicialmente os materiais dopados serão eletroneutros, apresentaram cargas elétricas em equilíbrio. Correntes de difusão trocam elétrons e lacunas na interface de acordo com o gradiente de concentração. Ao longo do tempo, esgotam-se portadores livres, criando uma barreira para o fluxo de energia. Nesta região a junção se comporta como um capacitor controlado por tensão cuja largura $W$ será dada pela seguinte relação:
            \begin{equation}
                \boxed{
                    W = \sqrt{\frac{2\epsilon_{Si}}{q} (V_{0} - V_{a}) \frac{N_{A} + N_{D}}{N_{A} N_{D}}}
                    \hspace{5mm}
                    \begin{cases}
                        V_{A} > 0, & \text{W diminui}\\
                        V_{A} < 0, & \text{W aumenta}
                    \end{cases}
                }
            \end{equation}

        \paragraph{Potencial Interno}Representa a diferença de potencial $V_{0}$ entre a região N e a região P quando a Corrente de Deriva, \eqref{eq:3}, e a Corrente de Difusão, \eqref{eq:6}, se anulam, calculado através das seguintes equações:
            \begin{multicols}{2}
                \begin{equation}
                    \boxed{
                        V_{0} = V_{T} \ln\left(\frac{N_{A}N_{D}}{n_{i}^{2}}\right)
                    }
                \end{equation}
                \begin{equation}
                    \boxed{
                        V_{T} = \frac{kT}{q}
                    }
                \end{equation}
            \end{multicols}


        \subsection{Polarização Direta}
            \paragraph{Definição}Quando ligado diretamente há decremento no campo entre os terminais dopados, aproximando as cargas da região de depleção, reduzindo-a. Neste cenário as cargas minoritárias carregaram a corrente entre os terminais do diodo.
                \begin{figure}[H]
                    \centering
                    \begin{circuitikz}
                        \ctikzset{component text=left}
                        \draw
                        (0,2) to[battery1] (0,0)
                        (0,2) -- (2,2)
                        (2,2) to[empty diode] (2,0)
                        (2,0) -- (0,0);
                    \end{circuitikz} 
                    \caption{Polarização Direta Diodo}
                \end{figure}

            \paragraph{Lacunas Difundidas}Quando um campo é aplicado na polarização direta as concentrações de lacunas, consequentemente de elétrons livres, se distribuem ao longo das regiões dopadas, reduzindo assim a Região de Depleção, de acordo com as seguintes equações: 
                \begin{multicols}{2}
                    \begin{equation}
                        \boxed{
                            p_{n}(x) = p_{n0} + [p_{n}(x_{n}) - p_{n0}]e^{- \frac{x - x_{n}}{L_{p}}}
                        }
                    \end{equation}
                    \begin{equation}
                        \boxed{
                            p_{n}(x_{n}) = p_{n0} e^{\frac{V}{V_{T}}}
                        }
                    \end{equation}
                \end{multicols}\noindent
            Onde:
                \begin{enumerate}[noitemsep]
                    \item $p_{n0}$, \textbf{Relação de Boltzmann:} Concentração relativa de lacunas, dada pela seguinte equação:
                        \[
                            \boxed{
                                p_{n0} = \frac{n_{i}^{2}}{N_{A}}
                                }
                        \]
                    \item $V_{T}$, \textbf{Tensão Termodinâmica:} Constante relacionando temperatura as concentrações, dada pelas seguintes equações:
                        \[
                            \boxed{
                            \frac{D_{n}}{\mu_{n}} = \frac{D_{p}}{\mu_{p}} = \frac{k_{B} T}{q} = V_{T}
                            }
                        \]
                    \item $L_{p}$, \textbf{Comprimento de Difusão:} Distância entre $x_{n}$ e $x_{a}$ onde $p_{n}(x_{a}) = p_{n0}$, dado pela seguinte equação:
                        \[
                            \boxed{
                                L_{p} = \sqrt{D_{p} \tau_{p}}
                            }
                        \]
                    Onde:
                        \begin{enumerate}[noitemsep]
                            \item $\tau$, \textbf{Tempo de Vida de Portadores Minoritários:} Tempo de decaimento da concentração;
                            \item $D_{p}$, \textbf{Constante de Difusão:} Característico do material; 
                        \end{enumerate}
                \end{enumerate}

            \paragraph{Densidade de Corrente}Quando um campo é aplicado na polarização direta haverá fluxo de corrente a inversamente proporcional Região de Depleção, conforme está reduz mais elétrons podem circular pela junção, de acordo com as seguintes equações:
                \begin{multicols}{2}
                    \begin{equation*}
                        J_{p} = q \frac{D_{p}}{L_{p}} p_{n0} (e^{\frac{V}{V_{T}}} - 1) e^{- \frac{x - x_{n}}{L_{p}}}
                    \end{equation*}
                    \begin{equation}
                        \boxed{
                            J_{Mp} = q \frac{D_{p}}{L_{p}} p_{n0} (e^{\frac{V}{V_{T}}} - 1)
                        }
                    \end{equation}
                    \begin{equation*}
                        J_{n} = q \frac{D_{n}}{L_{n}} n_{p0} (e^{\frac{V}{V_{T}}} - 1) e^{- \frac{x - x_{n}}{L_{p}}}
                    \end{equation*}
                    \begin{equation}
                        \boxed{
                            J_{Mn} = q \frac{D_{n}}{L_{n}} n_{p0} (e^{\frac{V}{V_{T}}} - 1)
                        }
                    \end{equation}
                \end{multicols}\noindent
            Onde:
                \begin{enumerate}[noitemsep]
                    \item $J_{Mp}$, \textbf{Densidade Máxima de Corrente de Lacunas};
                    \item $J_{Mn}$, \textbf{Densidade Máxima de Corrente de Elétrons};
                \end{enumerate}

            \paragraph{Corrente Total}Quando a junção passa a conduzir haverá uma corrente proporcional ao campo aplicado, obtido pela soma das densidades de lacunas e elétrons máximas, de acordo com as seguintes realações:
                \[
                    I_{T} = J_{Mp} + J_{Mn}
                \]
                \begin{multicols}{2}
                    \begin{equation}
                        \boxed{
                            I_{T} = I_{S} (e^{\frac{V_{F}}{V_{T}}} - 1)
                        }
                    \end{equation}
                    \begin{equation}
                        \boxed{
                            I_{S} = A q n_{i}^{2} \left(\frac{D_{p}}{L_{p}N_{D}} + \frac{D_{n}}{L_{n}N_{A}}\right)
                        }
                    \end{equation}
                \end{multicols}\noindent
            Onde:
                \begin{enumerate}[noitemsep]
                    \item $I_{S}$, \textbf{Corrente Saturação Reversa:} Corrente que flui nas regiões dopadas quando na saturação reversa.
                \end{enumerate}
            Nota-se que com aumento na temperatura o aumento de corrente começará antes, gráfico será deslocado horizontalmente para a esquerda. Assim o diodo será depende de sua temperatura de operação.

            \paragraph{Capacitância de Difusão}Quando diretamente polarizado o Diodo apresentará uma capacitância intrínseca causada pelo acumulo de portadores minoritários em regiões quase neutras, normalmente desconsideradas nas análises gerais.
                    \begin{equation}
                        \boxed{
                            C_{d} = \frac{\tau_{T}}{V_{T}}I
                        }
                    \end{equation}

        \subsection{Polarização Reversa}
            \paragraph{Definição}Quando ligado reversamente há incremento no campo entre os terminais dopados, aproximando as cargas das extremidades e consequentemente expandindo a região de depleção até que não seja mais possível fluxo de corrente. Neste cenário o diodo será considerado como um capacitor controlado por tensão.
                \begin{figure}[H]
                    \centering
                    \begin{circuitikz}
                        \ctikzset{component text=left}
                        \draw
                        (0,2) to[battery1] (0,0)
                        (0,2) -- (2,2)
                        (2,0) to[empty diode] (2,2)
                        (2,0) -- (0,0);
                    \end{circuitikz} 
                    \caption{Polarização Reversa Diodo}
                \end{figure}

            \paragraph{Capacitância de Depleção}Quando reversamente polarizado o Diodo apresentará uma capacitância intrínseca causada pelo acumulo de cargas em sua camada de depleção, normalmente desconsideradas nas análises gerais.
                \begin{multicols}{2}
                    \begin{equation}
                        \boxed{
                            C_{j} = \frac{C_{j0}}{\sqrt{1 + \frac{V_{R}}{V_{0}}}} \hspace{2.5mm} \text{[F/cm\textsuperscript{2}]}
                        }
                    \end{equation}
                    \begin{equation}
                        \boxed{
                            C_{j0} = \sqrt{\frac{\epsilon_{Si} q}{2} \frac{N_{A} N_{D}}{N_{A} + N_{D}} \frac{1}{V_{0}}} \hspace{2.5mm} \text{[F/cm\textsuperscript{2}]}
                        }
                    \end{equation}
                \end{multicols}\noindent

        \subsection{Efeito Zener}
            \paragraph{Definição}Ocorre quando o diodo está submetido a um campo elétrico elevado, aumentando a região de depleção que poderá romper ligações covalentes liberando elétrons e lacunas.

        \subsection{Avalanche}
            \paragraph{Definição}Ocorre quando o diodo está submetido a um campo elétrico elevado, causando cruzamentos de portadores minoritários na região de depleção que poderá acelerar elétrons até que estes rompam ligações covalentes

        \subsection{Diodo Zener}
            \paragraph{Definição}Quando diretamente polarizado este apresentará o mesmo funcionamento de um diodo comum, entretanto quando inversamente polarizado este se comportará como um regulador de tensão, representado pelo seguinte circuito:
                \begin{figure}[H]
                    \centering
                    \begin{circuitikz}
                        \ctikzset{component text=left, resistors/width=0.25, resistors/zigs=1}
                        \draw
                        (0,0) to[empty Zener diode, l=$D_{Z}$] (0,3)
    
                        (3,3) to[empty diode, l=$D_{1}$] (3,2)
                        (3,2) to[battery1, l=$V_{Z}$] (3,1)
                        (3,1) to[resistor, l=$R_{Z}$] (3,0)
                        (3,3) -- (6,3)
                        (6,0) to[empty diode, l=$D_{2}$] (6,3)
                        (3,0) -- (6,0);
                    \end{circuitikz} 
                    \caption{Equivalente Zener}
                \end{figure} \noindent
\newpage

    \section{Aplicações Diodo}
        \subsection{Modelagem}
            \paragraph{Definição}Conhecido o comportamento e estrutura dos diodos será necessário definir modelos adequados para as aplicações empregadas. O circuito utilizado como referência para o restante do desenvolvimento será o seguinte:

            % \begin{tikzpicture}[domain=0:2]
            %     \draw[thick,color=gray,step=.5cm,dashed] (-0.5,-.5) grid (3,3);
            %     \draw[->] (-1,0) -- (3.5,0)node[below right] {$x$};
            %     \draw[->] (0,-1) -- (0,3.5)node[left] {$y$};
            %     \draw plot[id=x] function{x*x};
            % \end{tikzpicture}

                \begin{figure}[H]
                    \centering
                    \begin{circuitikz}
                        \ctikzset{component text=left}
                        \draw
                        (0,3) to[battery1, l=$V$] (0,0)
                        (0,3) to[resistor, l=$R$] (3,3)
                        (3,3) to[empty diode, l=$D$] (3,0)
                        (3,0) -- (0,0);
                    \end{circuitikz} 
                    \caption{Circuito Básico}
                \end{figure} \noindent
            Onde: $V_{DD} = 5V$, $V_{T} = 26mV$, $R = 1000\Omega$ e $I_{S} = 10^{-14}A$. Desta maneira, utilizam-se as seguinte aproximações para seu funcionamento:

                \begin{enumerate}
                    \item \textbf{Modelo Ideal:}\\
                    Considera-se que quando em polarização direta o diodo se comporta como uma chave fechada, enquanto quando em polarização reversa o diodo se comporta como uma chave aberta.
                        \[
                            \begin{cases}
                                R_{D} \to \infty \hspace{2.5mm} I_{D} = 0, &
                                \text{Polarização Revera}\\
                                R_{D} = 0 \hspace{2.5mm} I_{D} \to \infty, &
                                \text{Polarização Direta}
                            \end{cases}
                        \]
                    Obtendo os seguintes resultados:
                        \[
                            \boxed{
                                \begin{cases}
                                    I_{D} = 5 mA\\
                                    V_{D} = 0 V
                                \end{cases}
                            }
                        \]

                    \item \textbf{Modelo Tensão Constante:}\\
                    Considera-se configuração similar ao ideal que quando em polarização direta o diodo se comporta como uma fonte de tensão constante, enquanto quando em polarização reversa o diodo se comporta como uma chave aberta.
                        \[
                            \begin{cases}
                                R_{D} \to \infty \hspace{2.5mm} I_{D} = 0, &
                                \text{Polarização Revera: } V_{D} < V_{D_{on}}\\
                                R_{D} = 0 \hspace{2.5mm} I_{D} \to \infty, &
                                \text{Polarização Direta: } V_{D} > V_{D_{on}}
                            \end{cases}
                        \]
                    Obtendo os seguintes resultados:
                        \[
                            \boxed{
                                \begin{cases}
                                    I_{D} = 4,3 mA\\
                                    V_{D} = 0,7 V
                                \end{cases}
                            }
                        \]

                    \item \textbf{Modelo Linear:}\\
                    Considera-se configuração similar a tensão constant eque quando em polarização direta o diodo se comporta como uma fonte de tensão constante em série com um resistor, enquanto quando em polarização reversa o diodo se comporta como uma chave aberta.
                        \[
                            \begin{cases}
                                I_{D} = 0 &
                                \text{Polarização Revera: } V_{D} < V_{D_{on}}\\
                                I_{D} = \frac{V_{D} - V_{D_{on}}}{r_{D}} &
                                \text{Polarização Direta: } V_{D} \ge V_{D_{on}}
                            \end{cases}
                        \]
                    Obtendo os seguintes resultados:
                        \[
                            \boxed{
                                \begin{cases}
                                    I_{D} = 5 mA\\
                                    V_{D} = 0,72 V
                                \end{cases}
                            }
                        \]

                    \item \textbf{Modelo Exponencial:}\\
                    Partindo das equações para o Diodo será necessário uma segunda equação, obtida pela Lei das Malhas no Circuito básico.
                        \[
                            \begin{cases}
                                I_{D} = I_{S} \cdot (e^{\frac{V_{D}}{V_{T}}} - 1)\\
                                I_{D} = \frac{V_{DD} - V_{D}}{R}
                            \end{cases} \to \hspace{2.5mm}
                            \frac{V_{DD} - V_{D}}{R} - I_{S} \cdot e^{\left(\frac{V_{D}}{V_{T}} - 1\right)} = 0
                        \]
                    Este conjunto de equações não possuem solução analítica, sendo necessário empregar um método iterativo para sua resolução, obtendo os seguintes resultados:
                        \[
                            \boxed{
                                \begin{cases}
                                    I_{D} = 4,3035 mA\\
                                    V_{D} = 0,6965 V
                                \end{cases}
                            }
                        \]

                    \item \textbf{Modelo Pequenos Sinais:}\\
                    Empregado quando os circuitos possuem uma amplitude de operação, ou seja, a fonte de tensão apresenta um valor de operação fixo e uma parcela variável, normalmente simétrica, como representado no seguinte circuito:
                        \begin{figure}[H]
                            \centering
                            \begin{circuitikz}
                                \ctikzset{component text=left}
                                \draw
                                (0,2) to[battery1, l=$V$] (0,0)
                                (0,2) to[sV, l=$v_{S}$] (0,4)
                                (0,4) to[resistor, l=$R$] (4,4)
                                (4,4) to[empty diode, l=$D$] (4,0)
                                (4,0) -- (0,0);
                            \end{circuitikz} 
                            \caption{Circuito com Amplitude de Operação}
                        \end{figure} \noindent
                    Neste cenário será necessário analisar separadamente cada fonte e realizar a supreposição dos resultados. O circuito a esquerda poderá ser analisado utilizando qualquer um dos métodos acima, enquanto o circuito a direita utilizará Pequenos Sinais.
                        \begin{figure}[H]
                            \centering
                            \begin{circuitikz}
                                \ctikzset{component text=left}
                                \draw
                                (0,3) to[battery1, l=$V$] (0,0)
                                (0,3) to[resistor, l=$R$] (3,3)
                                (3,3) to[empty diode, l=$D$] (3,0)
                                (3,0) -- (0,0);
                            \end{circuitikz} \hspace{20mm}
                            \centering
                            \begin{circuitikz}
                                \ctikzset{component text=left}
                                \draw
                                (0,0) to[sV, l=$v_{S}$] (0,3)
                                (0,3) to[resistor, l=$R$] (3,3)
                                (3,3) to[resistor, l=$r_{d}$] (3,0)
                                (3,0) -- (0,0);
                            \end{circuitikz} 
                        \end{figure} \noindent
                    Pequenos Sinais aproximam um sinal do Ponto de Operação através de retas tangentes de tal forma que os resultados obtidos sejam simétricos. Assim a tangente é aproximada pela expansão pela expansão de Séries Taylor de primeira ordem obtendo:
                        \[
                            \frac{1}{r_{d}} =
                            \diff{I_{D}}{V_{D}} =
                            \frac{I_{S}}{V_{T}} e^{\frac{V_{D}}{V_{T}}} =
                            \frac{I_{D}}{V_{T}}\to
                            \boxed{
                                r_{d} = \frac{V_{T}}{I_{D}}
                            }
                        \]
                \end{enumerate}

        \subsection{Retificador de Meia Onda}
            \paragraph{Definição}Circuito responsável por converter um sinal de tensão alternada, AC, em um sinal de tensão contínua, DC. Considera-se o modelo de tensão constante para o diodo, assim este dispositivo pode ser obtido pelo seguinte circuito:
                \begin{figure}[H]
                    \centering
                    \begin{circuitikz}
                        \ctikzset{component text=left}
                        \draw
                        (0,0) to[sV, l=$V$] (0,3)
                        (0,3) to[empty diode, l=$D$] (3,3)
                        (3,3) to[resistor, o-o, l=$R$] (3,0)
                        (3,0) -- (0,0);
                    \end{circuitikz} 
                    \caption{Retificador de Meia Onda Positiva}
                \end{figure} \noindent
            Entretanto, este apresentaria uma tensão com grande amplitude, pois durante pouco mais da metade da frequência da fonte alternada o circuito possuiria tensão e no restante do período estaria desligado. Desta maneira a introdução de um capacitor no local do resistor permite um sinal cause contínuo, podendo ser obtido pelo seguinte circuito:
                \begin{figure}[H]
                    \centering
                    \begin{circuitikz}
                        \ctikzset{component text=left}
                        \draw
                        (0,0) to[sV, l=$V$] (0,3)
                        (0,3) to[empty diode, l=$D$] (3,3)
                        (3,3) to[capacitor, o-o, l=$R$] (3,0)
                        (3,0) -- (0,0);
                    \end{circuitikz} 
                    \caption{Retificador de Meia Onda Positiva com Filtro}
                \end{figure} \noindent
            Isso seria um cenário idealizado, pois a saída não apresentaria carga. Em situações usuais haverá presença de uma resistência paralela ao capacitor que fará com que este descarregue ao longo do tempo, podendo ser obtido pelo seguinte circuito:
                \begin{figure}[H]
                    \centering
                    \begin{circuitikz}
                        \ctikzset{component text=left}
                        \draw
                        (0,0) to[sV, l=$V$] (0,3)
                        (0,3) to[empty diode, l=$D$] (3,3)
                        (3,3) to[capacitor, l=$C$] (3,0)
                        (3,0) -- (0,0)
                        (3,3) -- (6,3)
                        (6,3) to[resistor, o-o, l=$R$] (6,0)
                        (6,0) -- (3,0);
                    \end{circuitikz} 
                    \caption{Retificador de Meia Onda com Filtro e Carga}
                \end{figure} \noindent
            Esta configuração apresentará um decaimento na tensão inversamente proporcional ao tamanho do capacitor, sendo expressa pelas seguintes equações:
                \begin{equation}
                    \boxed{V_{max} = V_{p} - V_{D_{on}}}
                    \hspace{20mm}
                    \boxed{V_{R} = V_{max} - V_{min}}
                \end{equation}
            Onde:
                \begin{enumerate}[noitemsep]
                    \item $V_{max}$, \textbf{Tensão Máxima Retificada};
                    \item $V_{min}$, \textbf{Tensão Mínima Retificada};
                    \item $V_{p}$, \textbf{Tensão de Pico Alternada};
                    \item $V_{D_{on}}$, \textbf{Tensão do Diodo Ligado};
                    \item $V_{R}$, \textbf{Tensão de Ripple};
                        \begin{equation}
                            \boxed{
                                V_{R} \approx \frac{V_{p} - V_{D_{on}}}{R} \cdot \frac{T}{C} \approx \frac{i_{R}}{ C\cdot f}
                            }
                        \end{equation}
                    \item $i_{D}$, \textbf{Corrente no Diodo};
                        \begin{equation}
                            \boxed{
                                i_{D} = i_{D} + i_{R} \to i_{D} = C\diff{v_{0}}{t} + i_{R}
                            }
                        \end{equation}
                        \[
                            I_{D_{M}} = i_{R} \left(1 + \frac{1}{2\pi} \cdot \sqrt{\frac{2V_{R}}{V_{C}}}\right)
                        \]
                        \[
                            I_{D_{P}} = i_{R} \left(1 + 2\pi \cdot \sqrt{\frac{2V_{C}}{V_{R}}}\right)
                        \]
                \end{enumerate}


        \subsection{Retificador de Onda Completa}
            \paragraph{Definição}Circuito responsável por converter um sinal de tensão alternada, AC, em um sinal de tensão contínua, DC. Considera-se o modelo de tensão constante para o diodo, assim este dispositivo pode ser obtido pelo seguinte circuito:
                \begin{figure}[H]
                    \centering
                    \begin{circuitikz}
                        \ctikzset{component text=left}
                        \draw
                        (0,0) to[sV, l=$V$] (0,4)
                        (0,4) -- (4,4)
                        (2,2) to[empty diode, l=$D4$] (4,4)
                        (4,4) to[empty diode, l=$D1$] (6,2)
                        (2,2) to[empty diode, l=$D2$] (4,0)
                        (4,0) to[empty diode, l=$D3$] (6,2)
                        (2,2) to[resistor, o-o, l=$R$] (6,2)
                        (0,0) -- (4,0);
                    \end{circuitikz} 
                    \caption{Retificador de Onda Completa e Carga}
                \end{figure} \noindent
            Onde:
                \begin{enumerate}[noitemsep]
                    \item $V_{max}$, \textbf{Tensão Máxima Retificada};
                    \item $V_{min}$, \textbf{Tensão Mínima Retificada};
                    \item $V_{p}$, \textbf{Tensão de Pico Alternada};
                    \item $V_{D_{on}}$, \textbf{Tensão do Diodo Ligado};
                    \item $V_{R}$, \textbf{Tensão de Ripple};
                        \begin{equation}
                            \boxed{
                                V_{R} = \frac{V_{p}}{ 2 \cdot R \cdot C\cdot f}
                            }
                        \end{equation}
                    \item $i_{D}$, \textbf{Corrente no Diodo};
                        \[
                            I_{D_{M}} = i_{R} \left(1 + \pi \cdot \sqrt{\frac{V_{p}}{2 V_{R}}}\right)
                        \]
                        \[
                            I_{D_{P}} = i_{R} \left(1 + 2\pi \cdot \sqrt{\frac{V_{p}}{2 V_{R}}}\right)
                        \]
                \end{enumerate}

        \subsection{Deslocador de Tensão}
            \paragraph{Definição}
                \begin{figure}[H]
                    \centering
                    \begin{circuitikz}
                        \ctikzset{component text=left, resistors/width=0.25, resistors/zigs=1}
                        \draw
                        (0,2) to[resistor, o-, l=$R$] (2,2)
                        (2,2) to[empty diode, l=$D$] (2,0)
                        (2,2) -- (4,2)
                        (0,0) -- (4,0);
                    \end{circuitikz} 
                    \caption{Limitador de Tensão Positivo}
                \end{figure} \noindent

                \begin{figure}[H]
                    \centering
                    \begin{circuitikz}
                        \ctikzset{component text=left, resistors/width=0.25, resistors/zigs=1}
                        \draw
                        (0,2) to[resistor, o-, l=$R$] (2,2)
                        (2,0) to[empty diode, l=$D$] (2,2)
                        (2,2) -- (4,2)
                        (0,0) -- (4,0);
                    \end{circuitikz} 
                    \caption{Limitador de Tensão Negativa}
                \end{figure} \noindent

                \begin{figure}[H]
                    \centering
                    \begin{circuitikz}
                        \ctikzset{component text=left, resistors/width=0.25, resistors/zigs=1}
                        \draw
                        (0,2) to[resistor, o-, l=$R$] (2,2)
                        (2,0) to[empty diode, l=$D_{1}$] (2,2)
                        (4,2) to[empty diode, l=$D_{2}$] (4,0)
                        (2,2) -- (6,2)
                        (0,0) -- (6,0);
                    \end{circuitikz} 
                    \caption{Limitador de Tensão}
                \end{figure} \noindent

        \subsection{Clamper de Tensão}
            \paragraph{Definição}
                \begin{figure}[H]
                    \centering
                    \begin{circuitikz}
                        \ctikzset{component text=left, resistors/width=0.25, resistors/zigs=1}
                        \draw
                        (0,0) to[sV, l=$v_{in}$] (0,2)
                        (0,2) to[capacitor, l=$C_{1}$] (2,2)
                        (2,0) to[empty diode, l=$D_{1}$] (2,2)
                        (2,2) -- (4,2)
                        (0,0) -- (4,0);
                    \end{circuitikz} 
                    \caption{Clamper de Tensão}
                \end{figure} \noindent

        \subsection{Duplicador de Tensão}
            \paragraph{Definição}
                \begin{figure}[H]
                    \centering
                    \begin{circuitikz}
                        \ctikzset{component text=left, resistors/width=0.25, resistors/zigs=1}
                        \draw
                        (0,0) to[sV, l=$v_{in}$] (0,2)
                        (0,2) to[capacitor, l=$C_{1}$] (2,2)
                        (2,0) to[empty diode, l=$D_{1}$] (2,2)
                        (2,2) to[empty diode, l=$D_{2}$] (4,2)
                        (4,0) to[capacitor, l=$C_{2}$] (4,2)
                        (4,2) -- (6,2)
                        (0,0) -- (6,0);
                    \end{circuitikz} 
                    \caption{Duplicador de Tensão}
                \end{figure} \noindent
\newpage

    \section{Transistor Bipolar de Junção}
        \paragraph{Definição}Dispositivo elétrico capaz de controlar corrente em sua saída através de uma tensão na entrada. Construído através da combinação em série de duas junções de semicondutores positivamente e negativamente dopados classificados de acordo com as ligações realizados, comumente representados como mostrado a seguir:
            \begin{figure}[H]
                \centering
                \begin{subfigure}[t]{0.45\textwidth}
                    \centering
                    \begin{circuitikz}
                        \ctikzset{component text=left}
                        \draw
                        node[npn] (myComp) {$Q_{1}$}
                        (myComp.B) node [anchor = east] {B}
                        (myComp.C) node [anchor = south] {C}
                        (myComp.E) node [anchor = north] {E};
                    \end{circuitikz} 
                    \caption{Transistor NPN}
                \end{subfigure}
                \begin{subfigure}[t]{0.45\textwidth}
                    \centering
                    \begin{circuitikz}
                        \ctikzset{component text=left}
                        \draw
                        node[pnp] (myComp) {$Q_{2}$}
                        (myComp.B) node [anchor = east] {B}
                        (myComp.C) node [anchor = north] {C}
                        (myComp.E) node [anchor = south] {E};
                        %(myComp.E) to[open, v^>=$v_o(t)$] (myComp.C)
                    \end{circuitikz} 
                    \caption{Transistor PNP}
                \end{subfigure}
                \caption{Transistores Bipolares de Junção}
            \end{figure} \noindent
        Onde:
            \begin{enumerate}[noitemsep]
                \item \textbf{B:} Base do Transistor;
                \item \textbf{C:} Coletor do Transistor;
                \item \textbf{E:} Emissor do Transistor;
            \end{enumerate}

        \subsection{Transistores NPN}
            \paragraph{Definição}Transistores em que a região central será composta por uma região P, positivamente dopada, ligada em duas regiões N, negativamente dopadas, apresentando as seguintes tensões:
                \begin{figure}[H]
                    \centering
                    \begin{circuitikz}
                        \ctikzset{component text=left}
                        \draw
                        node[npn] (myComp) {}
                        (myComp.B) node [anchor = east] {B}
                        (myComp.C) node [anchor = south] {C}
                        (myComp.E) node [anchor = north] {E}
                        (myComp.C) to[open, v^<=$V_{CE}$] (myComp.E)
                        (myComp.E) to[open, v^>=$V_{BE}$] (myComp.B)
                        (myComp.B) to[open, v^>=$V_{CB}$] (myComp.C);
                    \end{circuitikz} 
                    \caption{Tensões Transistor NPN}
                \end{figure} \noindent
            Este componente apresentará 4 situações de operação, obtidas através das direntes combinações entre a polarização direta e indireta dos diodos intrínsecos, sendo descritas abaixo:
                \begin{enumerate}[rightmargin = \leftmargin]
                    \item \textbf{Região de Corte:} Ocorre quando $V_{BE}$ e $V_{BC}$ são inversamente polarizados;
                    \item \textbf{Região de Saturação:} Ocorre quando $V_{BE}$ e $V_{BC}$ são diretamente polarizados;
                    \item \textbf{Região Ativa:} Ocorre quando $V_{BE}$ está diretamente polaraizada e $V_{BC}$ está inversamente polarizada, haverá um pequeno fluxo de corrente vinda da Base que diminui as regiões de depleção entre o Coletor e o Emissor, possibilitando assim o fluxo de corrente entre estes terminais dada como a corrente através de um diodo como representado na seguinte equação:
                        \[
                            I_{D} = I_{S} \cdot (e^{\frac{V_{D}}{V_{T}}} - 1)
                        \]
                    Esta poderá ser simplicada pela a seguinte equação:
                        \begin{equation}
                            I_{C} \approx I_{E} \approx I_{S} \cdot (e^{\frac{V_{BE}}{V_{T}}})
                            \hspace{5mm}
                            \text{quando $V_{BE} \gg V_{T}$}
                        \end{equation}
                    Entretato, esta modelagem desconsidera o fluxo de corrente através da Base. Uma abordagem completa será descrita abaixo:
                        \begin{align}
                            \boxed{
                                \begin{cases}
                                    I_{E} = I_{B} + I_{C}\\
                                    I_{C} = \beta \cdot I_{B}\\
                                    I_{C} = \alpha \cdot I_{E}\\
                                \end{cases}
                                \hspace{5mm}
                                \text{onde: $\beta = 50 \sim 200$ e $\alpha = \frac{\beta}{\beta + 1} \approx 0,99$}
                            }
                        \end{align}
                    \item \textbf{Região Ativa Inversa:} Ocorre quando $v_{BE}$ está inversamente polaraizada e $v_{BC}$ está diretamente polarizada;
                \end{enumerate}

        \subsection{Efeito de Early}
            \paragraph{Definição}Descoberto por James M. Early, descrevendo como a largura de banda da base do transistor se altera com a variação de tensão aplicada entre base e coletor. Normalmente está influência será desconsiderada para simulações e cálculos, entretanto poderá ser visualizada em gráficos presentes nos datasheets, onde as curvas não são perfeitamente retas e sim anguladas.

        \subsection{Modelagem}
            \paragraph{Definição}Cada diferente circuito possuirá uma possível modelagem que incluirá, ou excluirá, informações dos transistores. Entre as principais modelagens estão as descritas abaixo:
                \begin{enumerate}[rightmargin = \leftmargin, noitemsep]
                    \item \textbf{Modelo Linear:}\\
                    Quando o transistor estiver na região ativa seu comportamento poderá ser descrito por meio do seguinte recurso:
                        \begin{figure}[H]
                            \centering
                            \begin{circuitikz}
                                \ctikzset{component text=left}
                                \draw
                                (-2,2) node [ocirc, anchor = west] {B} -- (-1,2)
                                (-1,2) to[empty diode, v<=$V_{BE}$] (-1,0)
                                
                                ( 2,2) node [ocirc, anchor = east] {C} -- ( 1,2)
                                ( 1,2) to[cI, v^>=${I_{C}}$] (1,0)

                                (-1,0) -- (1,0)
                                (0,-1) node [ocirc, anchor = south] {E} -- (0,0);
                            \end{circuitikz} 
                            \caption{Modelagem Linear para Transistor NPN}
                        \end{figure} \noindent
                    Onde:
                        \begin{equation}
                            \boxed{
                                I_{C} = I_{S} \cdot e^{\frac{V_{BE}}{V_{T}}}
                            }
                        \end{equation}

                    \item \textbf{Modelo Linear de Early:}\\
                    Quando analisado transistores percebe-se que o haverá influência por uma Tensão de Early, normalmente desprezada por sua pequena magnitude, gerando uma pequena variação linear como representado pelo seguinte circuito:
                        \begin{figure}[H]
                            \centering
                            \begin{subfigure}[t]{0.45\textwidth}
                                \centering
                                \begin{circuitikz}
                                    \ctikzset{component text=left, resistors/width=0.25, resistors/zigs=1}
                                    \draw
                                    (-2,2) node [ocirc, anchor = west] {B} -- (-1,2)
                                    (-1,2) to[empty diode, v<=$V_{BE}$] (-1,0)
                                    
                                    (4,2) node [ocirc, anchor = east] {C} -- ( 1,2)
                                    (1,2) to[cI, v^>=${I_{C}}$] (1,0)
                                    (3,2) to[resistor, l=$R_{O}$] (3,0)
    
                                    (-1,0) -- (3,0)
                                    (0,-1) node [ocirc, anchor = south] {E} -- (0,0);
                                \end{circuitikz} 
                                \caption{Grandes Sinais}
                            \end{subfigure}
                            \begin{subfigure}[t]{0.45\textwidth}
                                \centering
                                \begin{circuitikz}
                                    \ctikzset{component text=left, resistors/width=0.25, resistors/zigs=1}
                                    \draw
                                    (-2,2) node [ocirc, anchor = west] {B} -- (-1,2)
                                    (-1,2) to[resistor, l=$r_{\pi}$, v<=$v_{BE}$] (-1,0)
                                    
                                    (4,2) node [ocirc, anchor = east] {C} -- ( 1,2)
                                    (1,2) to[cI, v^>=${i_{C}}$] (1,0)
                                    (3,2) to[resistor, l=$r_{O}$] (3,0)
    
                                    (-1,0) -- (3,0)
                                    (0,-1) node [ocirc, anchor = south] {E} -- (0,0);
                                \end{circuitikz} 
                                \caption{Pequenos Sinais}
                            \end{subfigure}
                            \caption{Modelagem Linear de Early para Transistor NPN}
                        \end{figure} \noindent
                    Pelo modelo de grandes sinais temos:
                        \begin{equation}
                            \boxed{
                                I_{C} \approx I_{E} = I_{S} \cdot (e^\frac{V_{BE}}{V_{T}})\cdot \left(1 + \frac{V_{CE}}{V_{A}}\right)
                            }
                            \hspace{10mm}
                            \boxed{
                                R_{O} = \frac{V_{A}}{I_{S} \cdot e^{\frac{V_{BE}}{V_{T}}}}
                            }
                            \hspace{10mm}
                            \boxed{
                                \beta_{O} = \beta \left(1 + \frac{V_{CE}}{V_{A}}\right)
                            }
                        \end{equation}
                    Pelo modelo de pequenos sinais temos:
                        \begin{equation}
                            \boxed{
                                i_{C} = I_{C} + \frac{I_{C}}{V_{A}}\cdot v_{ce}
                            }
                            \hspace{10mm}
                            \boxed{
                                r_{O} = \frac{V_{A}}{I_{C}}
                            }
                            \hspace{10mm}
                            \boxed{
                                g_{m} = \frac{I_{C}}{V_{T}}
                            }
                        \end{equation}

                    \item \textbf{Modelo Pequenos Sinais:}\\
                    Empregado quando os circuitos possuem uma amplitude de operação, ou seja, a fonte de tensão apresenta um valor de operação fixo e uma parcela variável, normalmente simétrica, como representada no seguinte circuito:
                        \begin{figure}[H]
                            \centering
                            \begin{subfigure}[t]{0.3\textwidth}
                                \centering
                                \begin{circuitikz}
                                    \ctikzset{component text=left, resistors/width=0.25, resistors/zigs=1}
                                    \draw
                                    (-2,1) to[battery1, l=$V_{B}$] (-2,0)
                                    (-2,2) to[sV, l=$v_{b}$] (-2,1)
        
                                    (0,2) node[npn] (myComp) {}
                                    (myComp.B) -- (-2,2)
                                    (myComp.C) -- (0,3)
                                    (myComp.E) -- (0,0)
                                    
                                    (0,3) to[resistor, l=$R$] (0,4)
                                    (0,4) node[vcc]{}
                                    (-2,0) node[tlground]{}
                                    (0,0)  node[tlground]{};
                                \end{circuitikz}
                                \caption{Circuito Completo}
                            \end{subfigure}
                            \begin{subfigure}[t]{0.3\textwidth}
                                \centering
                                \begin{circuitikz}
                                    \ctikzset{component text=left, resistors/width=0.25, resistors/zigs=1}
                                    \draw
                                    (-2,2) to[battery1, l=$V_{B}$] (-2,0)
        
                                    (0,2) node[npn] (myComp) {}
                                    (myComp.B) -- (-2,2)
                                    (myComp.C) -- (0,3)
                                    (myComp.E) -- (0,0)
                                    
                                    (0,3) to[resistor, l=$R$] (0,4)
                                    (0,4) node[vcc]{}
                                    (-2,0) node[tlground]{}
                                    (0,0)  node[tlground]{};
                                \end{circuitikz}
                                \caption{Circuito com Fonte Constante}
                            \end{subfigure}
                            \begin{subfigure}[t]{0.3\textwidth}
                                \centering
                                \begin{circuitikz}
                                    \ctikzset{component text=left, resistors/width=0.25, resistors/zigs=1}
                                    \draw
                                    (-2,2) to[sV, l=$v_{B}$] (-2,0)
        
                                    (0,2) node[npn] (myComp) {}
                                    (myComp.B) -- (-2,2)
                                    (myComp.C) -- (0,3)
                                    (myComp.E) -- (0,0)
                                    
                                    (0,3) to[resistor, l=$R$] (0,4)
                                    (0,4) node[vcc]{}
                                    (-2,0) node[tlground]{}
                                    (0,0)  node[tlground]{};
                                \end{circuitikz}
                                \caption{Circuito com Fonte Variável}
                            \end{subfigure}
                            \par\bigskip
                            \begin{subfigure}[t]{0.3\textwidth}
                                \centering
                                \begin{circuitikz}
                                    \ctikzset{component text=left, resistors/width=0.25, resistors/zigs=1}
                                    \draw
                                    (-2,2) node [ocirc, anchor = west] {B} -- (-1,2)
                                    (-1,2) to[resistor, l=$r_{\pi}$, v<=$V_{BE}$] (-1,0)
                                    
                                    ( 2,2) node [ocirc, anchor = east] {C} -- ( 1,2)
                                    ( 1,2) to[cI, v^>=${g_{m}v_{be}}$] (1,0)
    
                                    (-1,0) -- (1,0)
                                    (0,-1) node [ocirc, anchor = south] {E} -- (0,0);
                                \end{circuitikz}
                                \caption{Pequenos Sinais Linear}
                            \end{subfigure}
                            \caption{Modelagem de Pequenos Sinais para Transistor NPN}
                        \end{figure} \noindent
                    Considera-se que uma reta tangente a exponencial, onde sua inclinação será dada pela derivada da curva. Como $V_{BE} \gg v_{be}$ pode-se considerar que $V_{BE} \approx v_{BE}$, desconsiderando sua interferência no circuito. Desta maneira, como $i_{E} \approx i_{C}$ e $I_{C} = \beta I_{B}$, obtém-se a seguintes relações:
                        \begin{equation}
                            \boxed{
                                i_{E} = I_{E} + \frac{I_{E}}{V_{T}}\cdot v_{be}
                            }
                            \hspace{10mm}
                            \boxed{
                                i_{C} = I_{C} + g_{m}\cdot v_{be}
                            }
                            \hspace{10mm}
                            \boxed{
                                i_{B} = I_{B} + \frac{v_{be}}{r_{\pi}}
                            }
                        \end{equation}
                    Onde:
                        \begin{enumerate}[rightmargin = \leftmargin, noitemsep]
                            \item \texttt{Transcondutância:} Capacidade da junção de conduzir corrente elétrica;
                                \begin{equation}
                                    \boxed{
                                        g_{m} = \frac{I_{C}}{V_{T}}
                                    }
                                \end{equation}

                            \item \texttt{Resistência $\pi$};
                                \begin{equation}
                                    \boxed{
                                        \frac{1}{r_{\pi}} = \frac{g_{m}}{\beta}
                                    }
                                \end{equation}
                        \end{enumerate}
                \end{enumerate}
\newpage

    \section{Aplicação Transistor Bipolar de Junção}
        \paragraph{Definição}Há diferentes maneiras de aplicar os transistores para que estes operem dentro da região ativa. Entretanto essa configuração demanda parâmetros mínimos, principalmente para Base, que podem não ser atingidos nas circunstâncias desejadas, inviabilizando sua operação. 

        \subsection{Polarização do Transistor}
            \paragraph{Definição}Afim de se contornar eventuais restrições para a operação direita de transistores em circuitos que possuam condições fora das necessidades da configuração, empregam-se os diferentes circuitos representados abaixo:
                \begin{enumerate}[rightmargin = \leftmargin, noitemsep]
                    \item \textbf{Alimentação Comum:} Fonte de tensão comum em que os resistores abaixam o nível de tensão. Será necessário supor resultados para os pontos do circuito e realizar operações interativas até que a solução esteja estável;
                        \begin{figure}[H]
                            \centering
                            \begin{circuitikz}
                                \ctikzset{component text=left, resistors/width=0.25, resistors/zigs=1}
                                \draw
                                (0,0) node[npn] (myNPN) {}
                                (myNPN.B) -- (-2,0)
                                (myNPN.E) -- (0,-2)
                                
                                (-2,0) to[resistor, l=$R_{1}$] (-2,2)
                                (-2,2) node[vcc]{VCC}
                    
                                (myNPN.C) to[resistor, l=$R_{C}$] (0,2)
                                (0,2) node[vcc]{VCC}
                                
                                (0,-2) node[tlground]{};
                            \end{circuitikz}
                            \caption{Alimentação para Transistor NPN}
                        \end{figure} \noindent

                    \item \textbf{Divisor Resistivo:} Fonte de tensão comum em que a associação dos resistores abaixam o nível de tensão. Esta abordagem apresentará instabilidade de corrente na base, assim recomenda-se a inclusão de um \textbf{Resistor Degenerativo};
                        \begin{figure}[H]
                            \centering
                            \begin{subfigure}[t]{0.3\textwidth}
                                \centering
                                \begin{circuitikz}
                                    \ctikzset{component text=left, resistors/width=0.25, resistors/zigs=1}
                                    \draw
                                    (0,0) node[npn] (myNPN) {}
                                    (myNPN.B) -- (-2,0)
                                    (myNPN.E) -- (0,-2)
                                    
                                    (-2,0) to[resistor, l=$R_{1}$] (-2,2)
                                    (-2,2) node[vcc]{VCC}
                                    
                                    (-2,0) to[resistor, l_=$R_{2}$] (-2,-2)
                                    (-2,-2) node[tlground]{}
                        
                                    (myNPN.C) to[resistor, l=$R_{C}$] (0,2)
                                    (0,2) node[vcc]{VCC}
                                    
                                    (0,-2) node[tlground]{};
                                \end{circuitikz}
                                \caption{Divisor de Tensão}
                            \end{subfigure}
                            \begin{subfigure}[t]{0.3\textwidth}
                                \centering
                                \begin{circuitikz}
                                    \ctikzset{component text=left, resistors/width=0.25, resistors/zigs=1}
                                    \draw
                                    (0,0) node[npn] (myNPN) {}
                                    (myNPN.B) -- (-2,0)
                                    
                                    (-2,0) to[resistor, l=$R_{1}$] (-2,2)
                                    (-2,2) node[vcc]{VCC}
                                    
                                    (-2,0) to[resistor, l_=$R_{2}$] (-2,-2)
                                    (-2,-2) node[tlground]{}
                        
                                    (myNPN.C) to[resistor, l=$R_{C}$] (0,2)
                                    (0,2) node[vcc]{VCC}
                        
                                    (myNPN.E) to[resistor, l_=$R_{E}$] (0,-2)
                                    (0,-2) node[tlground]{};
                                \end{circuitikz}
                                \caption{Inclusão Resistor Degenerativo}
                            \end{subfigure}
                            \begin{subfigure}[t]{0.3\textwidth}
                                \centering
                                \begin{circuitikz}
                                    \ctikzset{component text=left, resistors/width=0.25, resistors/zigs=1}
                                    \draw
                                    (0,0) node[npn] (myNPN) {}
                                    (myNPN.B) -- (-1,0)
                                    (myNPN.E) -- (0,-2)
                                    (myNPN.C) -- (0,1)

                                    (-1,0) to[resistor, l=$R_{1}$] (-1,1)
                                    (-1,1) -- (0,1)

                                    (0,1) to[resistor, o- , l=$R_{C}$] (0,2)
                                    (0,2) node[vcc]{VCC}

                                    (0,-2) node[tlground]{};
                                \end{circuitikz}
                                \caption{Alternativa ao Divisor}
                            \end{subfigure}
                            \caption{Divisor Resistivo para Transistor NPN}
                        \end{figure} \noindent
                \end{enumerate}

        \subsection{Amplificadores}
            \paragraph{Definição}Circuito que potencializa o sinal de saída com base no sinal de entrada a partir de uma relação entre os demais componentes que o constituem. Genericamente amplificadores são representados da seguinte forma:
                \begin{figure}[H]
                    \centering
                    \begin{circuitikz}[american]
                        \ctikzset{component text=center, resistors/width=0.25, resistors/zigs=1}
                        \draw
    
                        (0,0) node[plain mono amp] (AMP) {amp}
                        (AMP.in) node[anchor = east] {in}
                        (AMP.out) node[anchor = west] {out};
                    \end{circuitikz}
                    \caption{Representação Amplificador}
                \end{figure}\noindent
            Onde:
                \begin{enumerate}[rightmargin = \leftmargin, noitemsep]
                    \item $v_{i}$, \textbf{Tensão de Entrada};
                    \item $v_{o}$, \textbf{Tensão de Saída};
                    \item $r_{i}$, \textbf{Impedância de Entrada:} Idealmente deseja-se que $r_{i}\to\infty$;
                    \item $r_{o}$, \textbf{Impedância de Saída:} Idealmente deseja-se que $r_{o}\to 0$;
                    \item $G$, \textbf{Ganho:} Relação entre a proporcão do sinal de entrada com o sinal de saída, formalmente representada pela seguinte equação:
                        \begin{equation}
                            \boxed{
                                G = \frac{v_{o}}{v_{i}}
                            }
                        \end{equation}
                \end{enumerate}
            Transistores NPN podem ser utilizados para construir amplificadores, cada configuração resultará em um comportamento distintos com restrições. Entradas e Saídas não podem estar no mesmo pino, Base não poderá ser uma saída e Coletor não poderá ser uma entrada. Restando assim as seguintes configurações possíveis:
                \begin{table}[H]
                    \centering
                    \begin{tabular}[]{c c | c c c}\hline
                           &   &   & out &\\
                           &   & B & C   & E\\\hline
                           & B & X & 1   & 2\\
                        in & C & X & X   & X\\
                           & E & X & 3   & X\\\hline
                    \end{tabular}
                    \caption{Possibilidades de Ligações}
                \end{table}\noindent
            Assim, define-se as possíveis organizações como:
                \begin{enumerate}[rightmargin = \leftmargin]
                    \item \textbf{Emissor Comum:} Base como entrada, Coletor como saída e Emissor ligado ao negativo, como representado pelo circuito abaixo:
                        \begin{figure}[H]
                            \centering
                            \begin{subfigure}[t]{0.45\textwidth}
                                \centering
                                \begin{circuitikz}[american]
                                    \ctikzset{component text=left, resistors/width=0.25, resistors/zigs=1}
                                    \draw
                                    (0,2) node[npn] (myNPN) {}
                                    (myNPN.B) -- (-2,2)
                                    (myNPN.E) -- (0,0)
    
                                    (myNPN.C) to[short, -o] (0.5,2.75)
                                    (0.5, 2.75) node[right]{$v_{o}$}
    
                                    (myNPN.C) to[resistor, l=$R_{C}$] (0,4)
                                    (0,4) node[vcc]{VCC}
    
                                    (-2,0) to[sV<=$v_{i}$] (-2,2)
    
                                    (-2,0) node[tlground]{}
                                    (0,0)  node[tlground]{};
                                \end{circuitikz}
                                \caption{Emissor Comum Geral}
                            \end{subfigure}
                            \begin{subfigure}[t]{0.45\textwidth}
                                \centering
                                \begin{circuitikz}[american]
                                    \ctikzset{component text=left, resistors/width=0.25, resistors/zigs=1}
                                    \draw
                                    (-3,0) to[sV<=$v_{i}$] (-3,2)
                                    (-3,2) -- (-2,2)
                                    (-3,0) node[tlground]{}

                                    (-2,2) node [ocirc, anchor = west] {B} -- (-1,2)
                                    (-1,2) to[resistor, l=$r_{\pi}$, v=$V_{\pi}$] (-1,0)

                                    (-1,0) -- (1,0)
                                    (0,-1) node [ocirc, anchor = south] {E} -- (0,0)
                                    (0,-1) node[tlground]{}

                                    ( 2,2) node [ocirc, anchor = east] {C} -- ( 1,2)
                                    ( 1,2) to[cI, v^>=${g_{m}v_{\pi}}$] (1,0)
                                    
                                    (3,0) to[resistor, l_=$R_{C}$] (3,2)
                                    (2,2) -- (3,2)
                                    (3,0) node[tlground]{};
                                \end{circuitikz}
                                \caption{Equivalente Pequenos Sinais}
                            \end{subfigure}
                            \caption{Configuração de Emissor Comum}
                        \end{figure}\noindent
                    Nesta configuração de amplificador serão obtidos os seguintes resultados:
                        \begin{equation}
                            \boxed{
                                r_{i} = r_{\pi}
                            }
                            \qquad
                            \boxed{
                                r_{o} = R_{C}
                            }
                            \qquad
                            \boxed{
                                G = - g_{m}R_{C} = -\beta\frac{r_{o}}{r_{i}} = -\frac{V_{R_{C}}}{V_{T}}
                            }
                        \end{equation}

                    \item \textbf{Coletor Comum:} Base como entrada, Coletor ligado ao negativo e Emissor como saída, como representado pelo circuito abaixo:
                        \begin{figure}[H]
                            \centering
                            \begin{circuitikz}[american]
                                \ctikzset{component text=left, resistors/width=0.25, resistors/zigs=1}
                                \draw
                                (0,2) node[npn] (myNPN) {}
                                (myNPN.B) -- (-2,2)
                                (myNPN.E) -- (0,1)

                                (myNPN.E) to[short, -o] (0.5,1.25)
                                (0.5, 1.25) node[right]{$v_{o}$}

                                (-2,0) to[sV<=$v_{i}$] (-2,2)

                                (myNPN.C) -- (-1,2.75)
                                node at (-1,2)[jump crossing, rotate=90](J){}
                                (J.east) -- (-1,2.75)
                                (J.west) -- (-1,0)
                                (-1,0) node[tlground]{}

                                (-2,0) node[tlground]{};
                            \end{circuitikz}
                            \caption{Configuração de Coletor Comum}
                        \end{figure}

                    \item \textbf{Base Comum:} Base ligada ao negativo, Coletor como saída e Emissor como entrada, como representado pelo circuito abaixo:;
                        \begin{figure}[H]
                            \centering
                            \begin{circuitikz}[american]
                                \ctikzset{component text=left, resistors/width=0.25, resistors/zigs=1}
                                \draw
                                (0,2) node[npn] (myNPN) {}
                                (myNPN.B) -- (-2,2)
                                (myNPN.C) -- (0,3)

                                (myNPN.C) to[short, -o] (0.5,2.75)
                                (0.5, 2.75) node[right]{$v_{o}$}

                                (0,0) to[sV<=$v_{i}$] (myNPN.E)

                                (-2,0) -- (-2,2)
                                (0,0)  node[tlground]{}
                                (-2,0)  node[tlground]{};
                            \end{circuitikz}
                            \caption{Configuração de Base Comum}
                        \end{figure}
                \end{enumerate}
            Apesar das configurações apresentadas desempenharem como esperado para amplificadores ainda haverá vulnerabilidade do sistema a alterações das cargas que influenciam o ganho, e consequentemente, o funcionamento do circuito. Afim de proteger os amplificadores implementam-se o seguinte circuito:
                \begin{enumerate}[rightmargin = \leftmargin]
                    \item \textbf{Seguidor de Emissor:} Base como entrada, Coletor ligado a alimentação e Emissor como saída, como representado pelo circuito abaixo:
                        \begin{figure}[H]
                            \centering
                            \begin{subfigure}[t]{0.45\textwidth}
                                \centering
                                \begin{circuitikz}[american]
                                    \ctikzset{component text=left, resistors/width=0.25, resistors/zigs=1}
                                    \draw
                                    (0,2) node[npn] (myNPN) {}
                                    (myNPN.B) -- (-2,2)
                                    (myNPN.C) -- (0,3)

                                    (myNPN.E) to[short, -o] (0.5,1.25)
                                    (0.5, 1.25) node[right]{$v_{o}$}

                                    (myNPN.E) to[resistor, l=$R_{E}$] (0,0)
                                    (0,3) node[vcc]{VCC}

                                    (-2,0) to[sV<=$v_{i}$] (-2,2)

                                    (-2,0) node[tlground]{}
                                    (0,0)  node[tlground]{};
                                \end{circuitikz}
                                \caption{Seguidor de Emissor}
                            \end{subfigure}
                            \begin{subfigure}[t]{0.45\textwidth}
                                \centering
                                \begin{circuitikz}[american]
                                    \ctikzset{component text=left, resistors/width=0.25, resistors/zigs=1}
                                    \draw
                                    (-3,0) to[sV<=$v_{i}$] (-3,2)
                                    (-3,2) -- (-2,2)
                                    (-3,0) node[tlground]{}

                                    (-2,2) node [ocirc, anchor = west] {B} -- (-1,2)
                                    (-1,2) to[resistor, l=$r_{\pi}$, v=$V_{\pi}$] (-1,0)

                                    (-1,0) -- (1,0)
                                    (0,0) node [ocirc, anchor = south] {E} -- (0,0)
                                    (0,-0.5) to[short, -o] (0.5,-0.5)
                                    (0.5,-0.5) node[right]{$v_{o}$}

                                    (0,0) to[resistor, l=$R_{E}$] (0,-2)
                                    (0,-2) node[tlground]{}

                                    ( 2,2) node [ocirc, anchor = east] {C} -- ( 1,2)
                                    ( 1,2) to[cI, v^>=${g_{m}v_{\pi}}$] (1,0)

                                    (3,0) -- (3,2)
                                    (2,2) -- (3,2)
                                    (3,0) node[tlground]{};
                                \end{circuitikz}
                                \caption{Equivalente Pequenos Sinais}
                            \end{subfigure}
                            \caption{Configuração de Emissor Comum}
                        \end{figure}\noindent
                    Nesta configuração de amplificador serão obtidos os seguintes resultados:
                        \begin{equation}
                            \boxed{
                                r_{i} = r_{\pi} + (\beta + 1)R_{E}
                            }
                            \qquad
                            \boxed{
                                r_{o} = R_{E} \parallel \frac{r_{\pi} + R_{S}}{\beta + 1}
                            }
                            \qquad
                            \boxed{
                                G = \frac{R_{E}}{R_{E} + \frac{1}{g_{m}}}
                            }
                        \end{equation}
                \end{enumerate}

        \subsection{Classes de Amplificadores}
            \paragraph{Definição}Nestes circuitos sempre haverá uma troca, circuitos com maior eficiência terão maiores distorções e vice e versa. Cada configuração possuirá diferentes objetivos como mostrados abaixo:
                \begin{enumerate}[rightmargin = \leftmargin]
                    \item \textbf{Classe A:} Amplificadores com \texttt{Baixa} distorção e \texttt{Baixa} eficiência, obtidos pelo seguinte circuito:
                        \begin{figure}[H]
                            \centering
                            \begin{circuitikz}[american]
                                \ctikzset{component text=left, resistors/width=0.25, resistors/zigs=1}
                                \draw
                                (0,2) node[npn] (myNPN) {}
                                (myNPN.E) -- (0,1)

                                (myNPN.B) to[short, -o] (-1,2)
                                (-1, 2) node[left]{$v_{i}$}

                                (myNPN.C) to[short, -o] (0.5,2.75)
                                (0.5, 2.75) node[right]{$v_{o}$}

                                (myNPN.C) to[resistor, l=$R_{C}$] (0,4)
                                (0,4) node[vcc]{}
                                (0,1)  node[tlground]{};
                            \end{circuitikz}
                            \caption{Amplificador Classe A}
                        \end{figure}

                    \item \textbf{Classe B:} Amplificadores com \texttt{Alta} distorção e \texttt{Alta} eficiência, obtidos pelo seguinte circuito:
                        \begin{figure}[H]
                            \centering
                            \begin{circuitikz}[american]
                                \ctikzset{component text=left, resistors/width=0.25, resistors/zigs=1}
                                \draw
                                (0,3) node[npn] (myNPN) {}
                                (0,1) node[pnp] (myPNP) {}
                                (myNPN.E) -- (myPNP.E)
                
                                (myNPN.B) -- (-1,3)
                                (myPNP.B) -- (-1,1)
                                (-1,3) -- (-1,1)
                
                                (-1.5,2) to[short, o-*] ( -1,2)
                                (   0,2) to[short, *-o] (0.5,2)
                
                                (-1.5,2) node[left]{$v_{i}$}
                                ( 0.5,2) node[right]{$v_{o}$}
                
                                (myNPN.C) node[vcc]{}
                                (myPNP.C) node[tlground]{};
                            \end{circuitikz}
                            \caption{Amplificador Classe B}
                        \end{figure}

                    \item \textbf{Classe AB:} Amplificadores com \texttt{Média} distorção e \texttt{Média} eficiência, obtidos pelo seguinte circuito:
                        \begin{figure}[H]
                            \centering
                            \begin{circuitikz}[american]
                                \ctikzset{
                                    component text=left, 
                                    diodes/scale=0.5, 
                                    resistors/width=0.25, 
                                    resistors/zigs=1
                                }
                                \draw
                                (0,3) node[npn] (myNPN) {}
                                (0,1) node[pnp] (myPNP) {}
                                (myNPN.E) -- (myPNP.E)
                
                                (myNPN.B) -- (-1,3)
                                (myPNP.B) -- (-1,1)
                
                                (-1,3) to[empty diode] (-1,2)
                                (-1,2) to[empty diode] (-1,1)
                
                                (-1.5,2) to[short, o-*] ( -1,2)
                                (   0,2) to[short, *-o] (0.5,2)
                
                                (-1.5,2) node[left]{$v_{i}$}
                                ( 0.5,2) node[right]{$v_{o}$}
                
                                (myNPN.C) node[vcc]{}
                                (myPNP.C) node[tlground]{};
                            \end{circuitikz}
                            \caption{Amplificador Classe AB}
                        \end{figure}
                \end{enumerate}
\newpage

    \section{Transistores de Efeito de Campo}
        \paragraph{Definição}Dispositivo elétrico capaz de controlar corrente entre seus terminais de uma tensão em seu terminal de controle. Construído através da combinação em óxidos e junções de semicondutores positivamente e negativamente dopados classificados de acordo com as ligações realizados, comumente representados como mostrado a seguir:
            \begin{figure}[H]
                \centering
                \begin{subfigure}[t]{0.45\textwidth}
                    \centering
                    \begin{circuitikz}[american]
                        \ctikzset{
                            component text=left, 
                            diodes/scale=0.5, 
                            resistors/width=0.25, 
                            resistors/zigs=1
                        }
                        \draw
                        (0,0) node[nigfete] (myNFET) {$Q_{1}$}
                        (myNFET.G) node [anchor = east] {G}
                        (myNFET.D) node [anchor = south] {D}
                        (myNFET.S) node [anchor = north] {S};
                    \end{circuitikz}
                    \caption{MOSFET N}
                \end{subfigure}
                \centering
                \begin{subfigure}[t]{0.45\textwidth}
                    \centering
                    \begin{circuitikz}[american]
                        \ctikzset{
                            component text=left, 
                            diodes/scale=0.5, 
                            resistors/width=0.25, 
                            resistors/zigs=1
                        }
                        \draw
                        (0,0) node[pigfete] (myPFET) {$Q_{2}$}
                        (myPFET.G) node [anchor = east] {G}
                        (myPFET.D) node [anchor = north] {D}
                        (myPFET.S) node [anchor = south] {S};
                    \end{circuitikz}
                    \caption{MOSFET P}
                \end{subfigure}
                \caption{Transistores de Efeito de Campo}
            \end{figure}\noindent
        Onde:
            \begin{enumerate}[noitemsep]
                \item \textbf{D:} Drain do Transistor;
                \item \textbf{G:} Gate do Transistor;
                \item \textbf{S:} Source do Transistor;
            \end{enumerate}

        \subsection{Transistores MOSFET}
            \paragraph{Definição}Transistores em que a região central será composta por uma região N, MOSFET N, negativamente dopada ou uma região P, MOSFET P, positivamente dopada; ligada duas regiões nas extremidades de dopagem oposta. Na região intermediária haverá uma placa polarizável separada das demais regiões por um óxido, apresentando as seguintes tensões:
                \begin{figure}[H]
                    \centering
                    \begin{subfigure}[t]{0.45\textwidth}
                        \centering
                        \begin{circuitikz}[]
                            \ctikzset{
                                component text=left, 
                                diodes/scale=0.5, 
                                resistors/width=0.25, 
                                resistors/zigs=1
                            }
                            \draw
                            (0,0) node[nigfete] (myNFET) {}
                            (myNFET.G) node [anchor = east] {G}
                            (myNFET.D) node [anchor = south] {D}
                            (myNFET.S) node [anchor = north] {S};
                        \end{circuitikz}
                        \caption{MOSFET N}
                    \end{subfigure}
                    \centering
                    \begin{subfigure}[t]{0.45\textwidth}
                        \centering
                        \begin{circuitikz}[]
                            \ctikzset{
                                component text=left, 
                                diodes/scale=0.5, 
                                resistors/width=0.25, 
                                resistors/zigs=1
                            }
                            \draw
                            (0,0) node[pigfete] (myPFET) {}
                            (myPFET.G) node [anchor = east] {G}
                            (myPFET.D) node [anchor = north] {D}
                            (myPFET.S) node [anchor = south] {S}
                            (myPFET.G) to[open, v_<=$V_{GD}$] (myPFET.D)
                            (myPFET.S) to[open, v^<=$V_{GD}$] (myPFET.D);
                        \end{circuitikz}
                        \caption{MOSFET P}
                    \end{subfigure}
                    \caption{Tensões de Transistores de Efeito de Campo}
                \end{figure}\noindent
            Este componente apresentará 4 situações de operação, obtidas através das diferentes combinações entre a polarização dos terminais, sendo descritas abaixo:
                \begin{enumerate}[rightmargin = \leftmargin]
                    \item \textbf{Região de Corte:} ;
                    \item \textbf{Região de Saturação:} ;
                        \begin{equation}
                            \boxed{
                                I_{D} = \frac{1}{2} \mu_{n} C_{ox} \frac{W}{L} (V_{GS} - V_{TH})^{2}
                            }
                        \end{equation}
                    \item \textbf{Região de Triodo:} ;
                        \begin{equation}
                            \boxed{
                                I_{D} = \mu_{n} C_{ox} \frac{W}{L}
                                \left[
                                    (V_{GS} - V_{TH})V_{DS} - \frac{V_{DS}^{2}}{2}
                                \right]
                            }
                        \end{equation}
                    Onde:
                        \begin{enumerate}[rightmargin = \leftmargin, noitemsep]
                            \item $\mu_{n}$, \texttt{Mobilidade dos Portadores} $[\frac{\text{m}^{2}}{\text{V s}}]$
                            \item $C_{ox}$, \texttt{Capacitância por Área} $[\frac{\text{F}}{\text{m}^{2}}]$
                            \item $W$, \texttt{Largura Canal de Condução} $[\text{m}]$
                            \item $L$, \texttt{Comprimento Canal de Condução} $[\text{m}]$
                        \end{enumerate}
                \end{enumerate}
\end{document}