\documentclass{article}

\usepackage[a4paper, hmargin={20mm, 20mm}, vmargin={25mm, 30mm}]{geometry}
\usepackage[utf8]{inputenc}
\usepackage[english, main=portuguese]{babel}

\usepackage[hidelinks]{hyperref}
\usepackage{bookmark}
\usepackage{cancel}
\usepackage{comment}

\usepackage{array}
\usepackage{indentfirst}
\usepackage{multicol}
\setlength{\multicolsep}{2pt}% 50% of original values
\usepackage{subfiles}

\usepackage{titlesec}

\usepackage{amsmath}
\usepackage{amssymb}
\usepackage{systeme}
\usepackage{float}
\usepackage{enumitem}
\usepackage[thinc]{esdiff} %parcial derivatives
\restylefloat{table}

\usepackage{graphicx}
\usepackage{subcaption}
\graphicspath{ {./images/} }

% Pacote para a definição de novas cores
\usepackage{xcolor}
% Definindo novas cores
\definecolor{darkgreen}{rgb}{0.0, 0.42, 0.24}
\definecolor{darkpurple}{rgb}{0.74, 0.2, 0.64}
\definecolor{darkblue}{rgb}{0.0, 0.28, 0.67}

% Configurando espaços entre paragrafos
%\setlength{\parskip}{0.5em}


% Configurando pacote de circuitos
\usepackage{circuitikz}

%Configurando pacote de Gráficos
\usepackage{tikz}

% Configurando layout para mostrar códigos
\usepackage{listings}

\newcommand{\myStyle}{
\lstset{
    language=Octave,                            % the language of the code
    basicstyle=\ttfamily\small,               % the size of the fonts that are used for the code
    keywordstyle=\color{darkpurple}\bfseries, %
    stringstyle=\color{darkblue},             %
    commentstyle=\color{darkgreen},           %
    morecomment=[s][\color{blue}]{/**}{*/},   %
    extendedchars=true,                       %
    showtabs=false,                           % show tabs within strings adding particular underscores
    showspaces=false,                         % show spaces adding particular underscores
    showstringspaces=false,                   % underline spaces within strings
    numbers=left,                             % where to put the line-numbers
    numberstyle=\tiny\color{gray},            % the style that is used for the line-numbers
    stepnumber=1,                             % the step between two line-numbers. If it's 1, each line will be numbered
    numbersep=5pt,                            % how far the line-numbers are from the code
    frame=single,                             % adds a frame around the code
    rulecolor=\color{black},                  % if not set, the frame-color may be changed on line-breaks within not-black text
    breaklines=true,                          % sets automatic line breaking
    backgroundcolor=\color{white},            % choose the background color
    breakatwhitespace=true,                   % sets if automatic breaks should only happen at whitespace
    breakautoindent=false,                    %
    captionpos=b,                             % sets the caption-position to bottom
    xleftmargin=0pt,                          %
    tabsize=2,                                % sets default tabsize to 2 spaces
}}

%\titleformat{<command>}[<shape>]{<format>}{<label>}{<sep>}{<before-code>}[<after-code>]
\titleformat
{\section} %comand
[block]  %shape
{\normalfont\LARGE} %format
{\thesection. } %label
{0mm} %sep
{} %before-code
[{\titlerule[0.1mm]}] %after-code

\titlespacing*{\section}{0mm}{0mm}{15mm}

\titleformat
{\subsection} %comand
[block]  %shape
{\normalfont\Large} %format
{\thesubsection. } %label
{0mm} %sep
{} %before-code
[] %after-code

\titlespacing*{\subsection}{0mm}{5mm}{2.5mm}


\begin{document}
    \begin{titlepage}
        \begin{center}
            \rule{450pt}{0.5pt}\\[4mm]
            {\Huge EE532 - Eletrônica Aplicada}\\
            \rule{450pt}{0.5pt}\\[2mm]
            {\Large Resumo de Fórmulas}\\[200mm]
            \today\\
            \rule{250pt}{0.5pt}\\
            {\large Guilherme Nunes Trofino}\\
            {\large 217276}\\
        \end{center}
    \end{titlepage}
\newpage

    \tableofcontents
\newpage

    \section{Introdução}
        \[
            x_{RMS} = \sqrt{\frac{1}{T_{2} - T_{1}} \int_{T_{1}}^{T_{2}} f^{2}(t) dt}
        \]
        \[
            v_{s}(t) = V_{m} \cdot \sin(\omega \cdot t) \to
            v_{s}(t) = v_{d}(t) \to
            v_{s}(t) = v_{0}(t)
        \]
        \[
            v_{0}(t) = i(t) \cdot R \to
            i(t) = \frac{v_{0}(t)}{R} \to
            i(t) = \frac{V_{m}(t)}{R} \cdot \sin(\omega \cdot t)
        \]
        \[
            i_{RMS} = \sqrt{\frac{1}{T_{2} - T_{1}} \int_{T_{1}}^{T_{2}} i^{2}(t) dt}
        \]
        \[
            i_{RMS} = \sqrt{\frac{1}{2\pi - 0} \int_{0}^{2\pi} i^{2}(t) dt}
                    =   \sqrt{
                            \left[\frac{1}{2\pi - \pi} \int_{\pi}^{2\pi} i^{2}(t) dt \right] +
                            \left[\frac{1}{ \pi -   0} \int_{  0}^{ \pi} i^{2}(t) dt \right]
                        }
                    =   \sqrt{
                            \left[\frac{1}{ \pi -   0} \int_{  0}^{ \pi} i^{2}(t) dt \right]
                        }
        \]
        \[
            i_{RMS} = 
                    \sqrt{
                        \left[
                            \frac{1}{\pi} \int_{  0}^{ \pi} {
                                \left(
                                    \frac{V_{m}}{R} \cdot \sin(\omega \cdot t)
                                \right)}^{2} dt 
                        \right]
                    }
                    = \frac{V_{m}}{R} \cdot
                    \sqrt{
                        \frac{1}{\pi} \cdot \int_{0}^{ \pi} {
                            \sin^{2}(\omega \cdot t)
                        } dt
                    }
                    = 
        \]
        \[
            i_{RMS} = \sqrt{\frac{1}{2\pi - 0} \int_{0}^{2\pi} i^{2}(t) dt}
                    =   \sqrt{ \frac{1}{2\pi}
                            \left[
                                \int_{\pi}^{2\pi} i^{2}(t) dt +
                                \int_{  0}^{ \pi} i^{2}(t) dt
                            \right]
                        }
                    =   \sqrt{ \frac{1}{ 2\pi}
                            \left[
                                \int_{0}^{ \pi} i^{2}(t) dt
                            \right]
                        }
        \]
        \[
            i_{RMS} = 
                    \sqrt{ \frac{1}{2 \pi}
                        \left[
                            \int_{0}^{\pi} {
                                \left(
                                    \frac{V_{m}}{R} \cdot \sin(\omega \cdot t)
                                \right)}^{2} dt 
                        \right]
                    }
                    = \frac{V_{m}}{R} \cdot
                    \sqrt{ \frac{1}{2 \pi}
                        \left[
                            \int_{0}^{ \pi} {
                                \sin^{2}(\omega \cdot t)
                            } dt
                        \right]
                    }
                    = \frac{V_{m}}{R} \cdot
                    \sqrt{ \frac{1}{2 \pi}
                        \left[
                            \pi - \frac{1}{2}\left(\pi + \frac{1}{2 \omega} \sin(2 \pi \omega)\right)
                        \right]
                    }
        \]
        \[
            i_{RMS} = 
                    \frac{V_{m}}{R} \cdot
                    \sqrt{ \frac{1}{2 \pi}
                        \left[
                            \pi - \frac{\pi}{2}
                        \right]
                    }
                    = 
                    \frac{V_{m}}{R} \cdot
                    \sqrt{
                        \frac{1}{4}
                    }
                    = 
                    \frac{1}{2}
                    \frac{V_{m}}{R}
        \]
        \[
            v_{s} = V_{m} \sin(\omega \cdot t + \theta)
        \]
\end{document}