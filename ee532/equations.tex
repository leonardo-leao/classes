\documentclass{article}

\usepackage[a4paper, hmargin={20mm, 20mm}, vmargin={25mm, 30mm}]{geometry}
\usepackage[utf8]{inputenc}
\usepackage[english, main=portuguese]{babel}

\usepackage[hidelinks]{hyperref}
\usepackage{bookmark}
\usepackage{cancel}
\usepackage{comment}

\usepackage{array}
\usepackage{indentfirst}
\usepackage{multicol}
\setlength{\multicolsep}{2pt}% 50% of original values
\usepackage{subfiles}

\usepackage{titlesec}

\usepackage{amsmath}
\usepackage{amssymb}
\usepackage{systeme}
\usepackage{float}
\usepackage{enumitem}
\usepackage[thinc]{esdiff} %parcial derivatives
\restylefloat{table}

\usepackage{graphicx}
\usepackage{subcaption}
\graphicspath{ {./images/} }

% Pacote para a definição de novas cores
\usepackage{xcolor}
% Definindo novas cores
\definecolor{darkgreen}{rgb}{0.0, 0.42, 0.24}
\definecolor{darkpurple}{rgb}{0.74, 0.2, 0.64}
\definecolor{darkblue}{rgb}{0.0, 0.28, 0.67}

% Configurando espaços entre paragrafos
%\setlength{\parskip}{0.5em}


% Configurando pacote de circuitos
\usepackage{circuitikz}

%Configurando pacote de Gráficos
\usepackage{tikz}

% Configurando layout para mostrar códigos
\usepackage{listings}

\newcommand{\myStyle}{
\lstset{
    language=Octave,                            % the language of the code
    basicstyle=\ttfamily\small,               % the size of the fonts that are used for the code
    keywordstyle=\color{darkpurple}\bfseries, %
    stringstyle=\color{darkblue},             %
    commentstyle=\color{darkgreen},           %
    morecomment=[s][\color{blue}]{/**}{*/},   %
    extendedchars=true,                       %
    showtabs=false,                           % show tabs within strings adding particular underscores
    showspaces=false,                         % show spaces adding particular underscores
    showstringspaces=false,                   % underline spaces within strings
    numbers=left,                             % where to put the line-numbers
    numberstyle=\tiny\color{gray},            % the style that is used for the line-numbers
    stepnumber=1,                             % the step between two line-numbers. If it's 1, each line will be numbered
    numbersep=5pt,                            % how far the line-numbers are from the code
    frame=single,                             % adds a frame around the code
    rulecolor=\color{black},                  % if not set, the frame-color may be changed on line-breaks within not-black text
    breaklines=true,                          % sets automatic line breaking
    backgroundcolor=\color{white},            % choose the background color
    breakatwhitespace=true,                   % sets if automatic breaks should only happen at whitespace
    breakautoindent=false,                    %
    captionpos=b,                             % sets the caption-position to bottom
    xleftmargin=0pt,                          %
    tabsize=2,                                % sets default tabsize to 2 spaces
}}

%\titleformat{<command>}[<shape>]{<format>}{<label>}{<sep>}{<before-code>}[<after-code>]
\titleformat
{\section} %comand
[block]  %shape
{\normalfont\LARGE} %format
{\thesection. } %label
{0mm} %sep
{} %before-code
[{\titlerule[0.1mm]}] %after-code

\titlespacing*{\section}{0mm}{0mm}{15mm}

\titleformat
{\subsection} %comand
[block]  %shape
{\normalfont\Large} %format
{\thesubsection. } %label
{0mm} %sep
{} %before-code
[] %after-code

\titlespacing*{\subsection}{0mm}{5mm}{2.5mm}


\begin{document}
    \begin{titlepage}
        \begin{center}
            \rule{450pt}{0.5pt}\\[4mm]
            {\Huge EE532 - Eletrônica Aplicada}\\
            \rule{450pt}{0.5pt}\\[2mm]
            {\Large Resumo de Fórmulas}\\[200mm]
            \today\\
            \rule{250pt}{0.5pt}\\
            {\large Guilherme Nunes Trofino}\\
            {\large 217276}\\
        \end{center}
    \end{titlepage}
\newpage

    \section{test}
    \begin{enumerate}[rightmargin = \leftmargin, noitemsep]
        \item \textbf{Impedância de Entrada:} Alta, idealmente $\infty$;
        \item \textbf{Impedância de Saída:} Baixa, idealmente 0; 
        \item \textbf{Ganho em Malha Aberta:} Alta, idealmente $\infty$
    \end{enumerate}
    \begin{figure}[H]
        \centering
        \begin{circuitikz}[]
            \ctikzset{
                component text=left, 
                diodes/scale=0.5, 
                resistors/width=0.25, 
                resistors/zigs=1
            }
            \draw
            (0,0) node[op amp] (opamp) {}
            (opamp.+) node[left] {$v_+$}
            (opamp.-) node[left] {$v_-$}
            (opamp.out) node[right] {$v_o$}
            (opamp.up) --++(0,0.5) node[vcc]{$V_+$}
            (opamp.down) --++(0,-0.5) node[vee]{$V_-$};
        \end{circuitikz}
        \caption{Representação Amplificador Operacional}
    \end{figure}
Onde:
    \begin{enumerate}[rightmargin = \leftmargin, noitemsep]
        \item $v_{-}$, \textbf{Entrada Inversora:} ;
        \item $v_{+}$, \textbf{Entrada Não-Inversora:} ;
        \item $V_{+}$, \textbf{Terminais de Alimentação Positiva:} ;
        \item $V_{-}$, \textbf{Terminais de Alimentação Negativa:} ;
        \item $v_{o}$, \textbf{Saída:} ;
    \end{enumerate}

    \subsection{Amplificador Inversor}
        \paragraph{Definição}
            \begin{figure}[H]
                \centering
                \begin{circuitikz}[]
                    \ctikzset{
                        component text=left, 
                        diodes/scale=0.5, 
                        resistors/width=0.25, 
                        resistors/zigs=1
                    }
                    \draw
                    (0,0) node[op amp] (opamp) {}

                    (opamp.out) to[short, -o] (2,0)
                    (2,0) node[right] {$v_{\text{out}}$}

                    (opamp.+) --++(0,-1) node[tlground] {}
                    (opamp.-) to[R, l_=$R_{i}$] (-3, 0.5)
                    (-3, 0.5) to[sV<=$v_{i}$] (-3,-1.5) node[tlground] {}

                    (opamp.-) to[short,*-] ++(0,1) coordinate (leftR)
                    to[R, l=$R_{f}$] (leftR -| opamp.out)
                    to[short,-*] (opamp.out);
                \end{circuitikz}
                \caption{Amplificador Inversor}
            \end{figure}
\end{document}