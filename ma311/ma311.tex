\documentclass{article}
\usepackage{tpack}

\title{MA311 - Cálculo III}
\author{Guilherme Nunes Trofino}
\authorRA{217276}
\project{Resumo Teórico}

\begin{document}
    \maketitle
\newpage

    \tableofcontents
\newpage

    \section{Introdução}
        \paragraph{Apresentação}Neste documento será descrito as informações necessárias para compreensão e solução de exercícios relacionados a disciplina \thetitle . Note que este documento são notas realizadas por \theauthor , em \today.

    \section{Equações Diferenciais Ordinais, E.D.O.}
        \paragraph{Definição}Família de equações construídas a partir de uma função $f(t)$ qualquer e suas derivadas como representada na seguinte equação:
            \begin{equation}
                \boxed{
                    f(t) = 
                    b_{n}(t) \cdot y^{n}(t) + 
                    b_{n-1}(t) \cdot y^{n-1}(t) + \dots + 
                    b_{1}(t) \cdot y'(t) + 
                    b_{0}(t) \cdot y(t)
                }
            \end{equation}
        Onde:
            \begin{enumerate}[noitemsep]
                \item $b_{n}(t)$: Representa uma função na variável $t$;
                \item $y^{n}(t)$: Representa a ésima derivada da função $y(t)$;
            \end{enumerate}
        Tais equações ocorrem com frequência durante a análise e descrita de problemas físicos, visto que várias variáveis são denotadas através da derivação ou integração de uma propriedade. Desta forma, podem-se classificá-las como descrito a seguir:
            \begin{enumerate}[rightmargin = \leftmargin]
                \item \textbf{E.D.O. não Linear não Homogênea de Ordem N:} Quando a função possui $b_{n}(t) \neq 1$ e $f(t) \neq 0$ como representado pela seguinte equação:
                    \begin{equation*}
                        f(t) = 
                        b_{n}(t)   \cdot y^{n}(t) + 
                        b_{n-1}(t) \cdot y^{n-1}(t) + \dots +
                        b_{1}(t) \cdot y'(t) + 
                        b_{0}(t) \cdot y(t)
                    \end{equation*}

                \item \textbf{E.D.O. não Linear Homogênea de Ordem N:} Quando a função possui $b_{n}(t) \neq 1$ e $f(t) = 0$ como representado pela seguinte equação:
                    \begin{equation*}
                        0 = 
                        b_{n}(t)   \cdot y^{n}(t) + 
                        b_{n-1}(t) \cdot y^{n-1}(t) + \dots +
                        b_{1}(t) \cdot y'(t) + 
                        b_{0}(t) \cdot y(t)
                    \end{equation*}

                \item \textbf{E.D.O. Linear não Homogênea de Ordem N:} Quando a função possui $b_{n}(t) = 1$ e $f(t) \neq 0$ como representado pela seguinte equação:
                    \begin{equation*}
                        f(t) = 
                        1          \cdot y^{n}(t) + 
                        b_{n-1}(t) \cdot y^{n-1}(t) + \dots +
                        b_{1}(t) \cdot y'(t) + 
                        b_{0}(t) \cdot y(t)
                    \end{equation*}

                \item \textbf{E.D.O. Linear Homogênea de Ordem N:} Quando a função possui $b_{n}(t) = 1$ e $f(t) = 0$ como representado pela seguinte equação:
                    \begin{equation*}
                        0 = 
                        1          \cdot y^{n}(t) + 
                        b_{n-1}(t) \cdot y^{n-1}(t) + \dots +
                        b_{1}(t) \cdot y'(t) + 
                        b_{0}(t) \cdot y(t)
                    \end{equation*}
            \end{enumerate}



        \subsection{E.D.O.'s Lineares de 1º Ordem}
            \paragraph{Definição}E.D.O.'s Lineares de 1º Ordem serão equações construídas a partir de uma função $f(t)$ qualquer e sua derivada de 1º Ordem como representada na seguinte equação:
                \begin{equation}
                    \boxed{
                        f(t) = q(t) \cdot y'(t) + p(t) \cdot y(t)
                    }
                \end{equation}


            \subsubsection{Teorema da Existência e da Unicidade}
                \paragraph{Definição}E.D.O.'s Lineares de 1º Ordem existirão se, e somente se, atendem as condições enunciadas a seguir. Primeiramente, considera-se as seguintes equações:
                    \begin{equation}
                        f(t) =
                        \begin{cases}
                            f(x,y(x)) = y', & \text{Equação Geral}\\
                            y(a) = b,       & \text{Condição Inicial}\\
                        \end{cases}
                    \end{equation}
                Onde:
                    \begin{itemize}[noitemsep]
                        \item Se $f(x,y(x))$ é contínua em uma região $R$ qualquer contida em $R^2$ então existe solução em $R$;

                        \item Se $f_y(x,y(x))$ é contínua em uma região $R$ qualquer contida em $R^2$ então a solução em $R$ é única;
                    \end{itemize}
                Caso a equação atenda a estes requisitos, então as seguintes classificações serão válidas:

            \subsubsection{E.D.O. Linear não Homogênea}
                \paragraph{Definição}Quando a função possui $q_{n}(t) = 1$ e $f(t) \neq 0$ como representado pela seguinte equação:
                    \begin{equation}
                        f(t) = 1 \cdot y'(t) + p(t) \cdot y(t)
                    \end{equation}

                \paragraph{Resolução}E.D.O.'s Lineares e Homogêneas de 1º Ordem são solucionadas através da aplicação do \texttt{Fator Integrante} apresentado na seguinte equação:
                    \begin{equation*}
                        u(t) = e^{\int p(t)\;\text{d}t}
                    \end{equation*}
                Quando este termo é apicado a equação a mesma pode ser simplificada através da derivada do produto como demonstrado a seguir:
                    \begin{equation*}
                        \diff{[u(t) \cdot y(t)]}{t} = u(t) \cdot f(t)
                    \end{equation*}
                Quando o problema valores iniciais será necessário realizar substituições para obter a constante correspondente. Consequentemente a solução de tal E.D.O. será:
                    \begin{equation}
                        \boxed{
                            y(t) = \frac{\int{f(t) \cdot u(t)\;\text{d}t + C}}{u(t)}
                        }
                    \end{equation}

            \subsubsection{E.D.O. Linear Homogênea}
                \paragraph{Definição}Quando a função possui $q_{n}(t) = 1$ e $f(t) = 0$ como representado pela seguinte equação:
                    \begin{equation}
                        0 = 1 \cdot y'(t) + p(t) \cdot y(t)
                    \end{equation}

            \subsection{Equações Separáveis}
                \paragraph{Definição}Quando a função possui a forma $y'(x)=f(x) \cdot g(y)$, isto é,  x e y separáveis em funções independentes.
                \paragraph{}Equação Geral:
                    \begin{equation}
                        y'(x)=f(x) \cdot g(y)
                    \end{equation}
                
                \paragraph{}Solução Geral:
                    \begin{equation}
                        \int{\frac{1}{g(y)}dy} = \int{f(x)dx}
                    \end{equation}
                \paragraph{}Equações desta forma possuem resolução direta pois basta isolar as funções correspondentes e aplicar a integração adequada.

            \subsection{Equações Exatas}
                \paragraph{Definição}Uma E.D.O. $F(x,y)$ será exata quando puder ser rescrita como demonstrado:
                    \begin{equation}
                        M(x,y)dx+N(x,y)dy=0
                    \end{equation}
                \paragraph{}Onde:
                
                \quad$F_{x}(x,y)=M$
                
                \quad$F_{y}(x,y)=N$
                \paragraph{}Caso $M_{y}=N_{x}$ então a equação $F(x,y)$ será exata e poderá ser encontrada pelo método que segue:
                    \begin{equation}
                        F(x,y)=\int{M}dx=F_{0}(x,y)+g(y)
                    \end{equation}
                    \begin{equation}
                        N= F_{0y}(x,y)+g'(y)
                    \end{equation}
                \paragraph{}Note que o processo poderia ser aplicado em função de $N$ obtendo assim $g(x)$ como constante resultante da integração.
                \paragraph{}\textbf{Fator Integrante:} Caso $M_{y}\neq N_{x}$ a equação não seria exata, sendo necessário aplicar um fator integrante para torná-la exata. Há dois casos possíveis para tal fator, sendo eles:
                
                \paragraph{}Dependente de $x$, $u(x)$:
                    \begin{equation}
                        u(x)=e^{\int{\frac{M_{y}-N_{x}}{N}dx}}
                    \end{equation}
                \paragraph{}Dependente de $y$, $u(y)$:
                    \begin{equation}
                        u(y)=e^{\int{\frac{N_{x}-M_{y}}{M}dy}}
                    \end{equation}

            \subsubsection{Substituição Linear}
                \paragraph{Definição}Quando a função possui a forma $y'(x)=F(ax+by(x)+c)$, isto é, todos os termos, com excessão de $y'(x)$, estão contidos em uma única função.
                \paragraph{}Equação Geral:
                    \begin{equation}
                        y'(x)=F(ax+by(x)+c)
                    \end{equation}
                \paragraph{}Substituição:
                    \begin{equation}
                        V(x)=ax+by(x)+c
                    \end{equation}
                \paragraph{}Como consequência de qualquer substituição será necessário deriva-la para encontrar $y'(x)$ em função da substituição.
                \paragraph{}Derivação:
                    \begin{equation}
                        y'(x)=\frac{V'(x)-a}{b}
                    \end{equation}
                \paragraph{}Solução Geral:
                    \begin{equation}
                        V'(x)=bF(V(x))+a
                    \end{equation}
                \paragraph{}Note que a solução geral obtida sempre será separável ápos a substituição.

            \subsubsection{Substituição Homogênea}
                \paragraph{Definição}Quando a função possui os termos $x$ e $y$ descritos por quocientes.
                \paragraph{}Equação Geral:
                    \begin{equation}
                        y'(x)=f(\frac{y}{x})
                    \end{equation}
                \paragraph{}Substituição:
                    \begin{equation}
                        V(x)=\frac{y}{x} \Leftrightarrow y=V(x) \cdot x
                    \end{equation}
                \paragraph{}Derivação:
                    \begin{equation}
                        y'(x)=V(x)+V'(x) \cdot x
                    \end{equation}
                \paragraph{}Solução Geral:
                    \begin{equation}
                        V(x)+V'(x)x=f(V(x))
                    \end{equation}
                \paragraph{}Note que a solução geral obtida sempre será separável ápos a substituição.

            \subsubsection{Substituição de Bernoulli}
                \paragraph{Definição}Quando a função possui a forma $y'(x)+p(x) \cdot y(x)=f(x) \cdot y^{n}(x)$, isto 
                é, $f(x)$ está multiplicada por uma potência $n$ de $y(x)$.
                \paragraph{}Equação Geral:
                    \begin{equation}
                        y'(x)+p(x) \cdot y(x)=f(x) \cdot y^{n}(x)
                    \end{equation}
                \paragraph{}Substituição:
                    \begin{equation}
                        V(x)=y^{n-1}(x)
                    \end{equation}
                \paragraph{}Derivação:
                    \begin{equation}
                        V'(x)=y'y^{n}
                    \end{equation}
                \paragraph{}Solução:
                    \begin{equation}
                        \frac{v'(x)}{1-n}+p(x)v(x)=f(x)
                    \end{equation}


    \section{Sequências Infinitas}
        \paragraph{Definição}Funções definidas em $f:Z\to R$ onde $f(n)=a_{n}$, $n$ é o índice da sequência e $a_{n}$ é o n-ésimo termo da sequência. Todas as propriedades demonstradas e aprendidas em Cálculo I para Limites podem ser aplicadas no estudo de sequências. 
        
        \subsection{Convergência}
            \paragraph{Definição}Uma sequência, $a_{n}$, converge para $L$ se dado qualquer $\epsilon > 0$ existe $N \ge 0$ tal que $|a_{n}-L|<\epsilon$ para todo $n \ge N$.
            \paragraph{}\textbf{Exemplo:}
                \begin{equation}
                    \lim_{n\to\infty}\frac{n-1}{n}\rightarrow1
                \end{equation}
            \paragraph{}Dado $\epsilon>0$ existe $N$ tal que $|\frac{n-1}{n}-1|<\epsilon$ então $\epsilon>\frac{1}{n}$ logo $N>\frac{1}{\epsilon}$.
            \paragraph{} Toda sequência convergente é limitada, ou seja, existe $K$ tal que $|a_{n}| \le K$. Todavia uma sequência limitada não implica em convergência.
        
        \subsection{Divergência}
            \paragraph{Definição}Uma sequência, $a_{n}$, diverge se dado qualquer $K>0$ existe $N$ tal que $a_{n} \ge K$ para todo $n \ge N$.
            \paragraph{}\textbf{Exemplo:} 
                \begin{equation}
                    \lim_{n\to\infty}n^{2} \rightarrow \infty
                \end{equation}
        
        \subsection{Ordem de Crescimento}
            \paragraph{Definição}Considerando duas funções quaisquer $f(n)$ e $g(n)$ pode-se analisar a rapidez de crescimento para inferir o resultado. Tomando o seguinte quociente apenas como suposição:
                \begin{equation}
                    \lim\limits_{n\to\infty}\frac{f(n)}{g(n)}\left \{ \begin{matrix} 0\\ K\ne0\\ \infty\end{matrix}\right.
                \end{equation}
            \paragraph{}Se $\lim\limits_{n\to\infty}\frac{f(n)}{g(n)}=0$ implica que $g(n)$ tende ao infinito mais rapidamente do que $f(n)$.
            \paragraph{}Se $\lim\limits_{n\to\infty}\frac{f(n)}{g(n)}=K$ implica que $f(n)$ e $g(n)$ possuem a mesma ordem de crescimento.
            \paragraph{}Se $\lim\limits_{n\to\infty}\frac{f(n)}{g(n)}=\infty$ implica que $f(n)$ tende ao infinito mais rapidamente do que $g(n)$.
            \paragraph{}Assim sendo pode-se ordenar em ordem crescente de crescimento da principais funções, considerando $a>1$:
                \begin{equation}
                    log_{a}n < n^k < a^n < n! < n^n 
                \end{equation}
                
        \subsection{Sequências Monótonas}
            \paragraph{Definição}Sequências cujos termos podem ser comparados com apenas um símbolo serão sequências monótonas. Há subclassificações em função dos diferentes símbolos possíveis, sendo elas:
            \paragraph{}\textbf{Sequências Crescentes:} Sequências cujos termos podem ser comparados apenas com $<$.
                \begin{equation}
                    a_1<a_2<\cdots<a_{n-1}<a_n
                \end{equation}
            \paragraph{}\textbf{Sequências não Decrescentes:} Sequências cujos termos podem ser comparados apenas com $\le$.
                \begin{equation}
                    a_1\le a_2\le \cdots \le a_{n-1} \le a_n
                \end{equation}
            \paragraph{}\textbf{Sequências Decrescentes:} Sequências cujos termos podem ser comparados apenas com $>$.
                \begin{equation}
                    a_1>a_2>\cdots>a_{n-1}>a_n
                \end{equation}
            \paragraph{}\textbf{Sequências não Decrescentes:} Sequências cujos termos podem ser comparados apenas com $\ge$.
                \begin{equation}
                    a_1\ge a_2\ge \cdots \ge a_{n-1} \ge a_n
                \end{equation}
                
        \subsection{Formas Indeterminadas}
            \paragraph{Definição}Há expressões que não são resultados válidos, demandando manipulação. Entre as principais estão $\infty\cdot0=\frac{\infty}{\infty}$ e $\frac{0}{0}$ às quais aplica-se L'Hospital. Há também $\infty-\infty$, $\infty^0$ e $0^{\infty}$ os quais demandam modificação da expressão. Outra forma incomum é $1^{\infty}$ o qual aplica-se $e^{ln}$
        
        \subsection{Séries Numéricas}
            \paragraph{Definição}São consideradas séries numéricas as sequências que envolvem o somatório de uma expressão no infinito.
            \begin{equation}
                \sum\limits_{n=1}^{\infty}a_n = a_1 + a_2 + \cdots
            \end{equation}
            \paragraph{}Assim como na integração imprópria o infinito é uma impossibilidade, demandando a aplicação de limite tornando-a finita. Durante este processo considera-se a soma parcial, ou seja, até um número qualquer $N$ finito.
            \begin{equation}
                \sum\limits_{n=1}^{N}a_n = a_1 + a_2 + \cdots + a_{N-1} + a_N
            \end{equation}
            \paragraph{}Em seguida toma-se o limite de $N$ tendendo ao infinito.
            \begin{equation}
                \sum\limits_{n=1}^{\infty}a_n=\lim\limits_{N\to\infty}\sum\limits_{n=1}^{N}a_n
            \end{equation}
            \paragraph{}O limite só poderá ser calculado se a série for estritamente definida, isto é, todos os termos estejam descritos na expressão. Sendo assim, serão necessárias modificações nas exxpressões para que as mesmas estejam propriamente definidas e possam assim ser calculadas.
        
        \subsection{Séries Geométricas}
            \paragraph{Definição}Serão séries geométricas aquelas em que a expressão somada é da forma $x^{n}$ onde $n$ serão os números do domínio e $x$ será uma constante.
                \begin{equation}
                    \sum\limits_{n=1}^{\infty}x^n
                \end{equation}
            \paragraph{}Há sequências, em sua maioria, em que a simples atribuição de limite não será suficiente para que a mesma possa ser solucionada. Para torná-la estritamente definida será necessário manipular a expressão da seguinte forma:
                \begin{equation}
                    S_n=x+x^2+\cdots+x^n
                \end{equation}
                \begin{equation}
                    xS_n=x^2+x^3+\cdots+x^{n+1}
                \end{equation}
            \paragraph{}Note que todos os termos serão comuns a ambas somas parciais com exceção do primeiro e do último, sendo assim a subtração entre tais somas eliminará todos os termos desconhecidos do somatório.
                \begin{equation}
                    S_n-xS_n=x-x^{n+1}
                \end{equation}
                \begin{equation}
                    S_n=\frac{x-x^{n+1}}{1-x}
                \end{equation}
            \paragraph{}Com essa manipulação será possível isolar a soma parcial, obtendo uma série estritamente definida. Isso possibilita a aplicação do limite e consequente solução do somatório inicial.
                \begin{equation}
                    \sum\limits_{n=1}^{\infty} x^n = \lim\limits_{n\to\infty}\frac{x-x^{n+1}}{1-x}
                \end{equation}
            \paragraph{}Consequentemente haverão quatro possibilidades a serem analisadas quanto ao valor de $x$ que influenciarão o resultado final do somatório.
                \begin{equation}
                    \lim\limits_{n\to\infty}\frac{x-x^{n+1}}{1-x}\left \{ \begin{matrix} \frac{x}{1-x}; |x|<1\\ -\infty; |x|>1\\ \infty; x=1\\ Diverge; x=-1\end{matrix}\right.
                \end{equation}
            
    \section{Testes de Série}
        \paragraph{}Como as séries possuem a noção de divergência e convergência, explicadas anteriormente, se faz necessário descobrir como se enquadra cada sequência. Para tal existem diversos testes que avaliam, sobre condições especificas, o comportamento da equação trabalhada.
        
        \subsection{Teste de Divergência}
            \paragraph{}\textbf{Teorema:} Se a série converge então $\lim_{n\to\infty}a_{n} = 0$, cuja demonstração segue:
                \begin{equation}
                    S_{n-1}=a_{1}+a_{2}+\cdots+a_{n-2}+a_{n-1}, \lim_{n\to\infty}S_{n-1} = S
                \end{equation}
                \begin{equation}
                    S_{n}=a_{1}+a_{2}+\cdots+a_{n-1}+a_{n}, \lim_{n\to\infty}S_{n} = S
                \end{equation}
            \paragraph{}Note que assumindo que a série seja convergente não deve importar o fim do limite, pois calcular soma no infinito deve tender para o valor da série. Assim pode-se dizer que:
                \begin{equation}
                    \lim_{n\to\infty}(S_{n}-S_{n-1}) = S - S = 0
                \end{equation}
            \paragraph{}Note que o resultado da subtração será $a_{n}$ visto que os demais termos são eliminados com a subtração. Assim obtém-se o resultado:
                \begin{equation}
                    \lim_{n\to\infty}a_{n} = 0, Converge
                \end{equation}
                \begin{equation}
                    \lim_{n\to\infty}a_{n} \ne 0, Diverge
                \end{equation}
        
        \subsection{Teste da Comparação}
            \paragraph{}Este testes vem como consequência da comparação de limites do Cálculo I, como sequências nada mais são do que somas de finitas avaliadas no infinito o resultado pode ser estendido para o Cálculo III com os devidos ajustes.
            \paragraph{}\textbf{Teorema:} Sejam $a_{n}$ e $b_{n}$ os termos gerais de duas séries distintas tais que $0 \le a_{n} \le b_{n}$ temos que:
            \paragraph{}Se $\sum b_{n}$ converge então $\sum a_{n}$ converge. Claramente se uma série com termo geral com crescimento mais acelerado converge outra série com crescimento semelhante ou inferior deve convergir.
            \paragraph{}Se $\sum a_{n}$ diverge então $\sum b_{n}$ diverge. Claramente se uma série com termo geral com crescimento menos acelerado diverge outra série com crescimento semelhante ou superior deve divergir.
        \subsection{Teste da Comparação do Limite}
            \paragraph{}Assim como o Teste da Comparação este teste vem como consequência da comparação de limites do Cálculo I, porém este expende o resultado.
            \paragraph{}\textbf{Teorema:} Sejam $a_{n}>0$ e $b_{n}>0$ os termos gerais de duas séries distintas tais que:
                \begin{equation}
                    \lim_{n\to\infty}\frac{a_{n}}{b_{n}}=p
                \end{equation}
                \begin{equation}
                    0<p<\infty
                \end{equation}
            \paragraph{}Em outras palavras, as séries $\sum a_{n}$ e $\sum a_{n}$ possuem a mesma ordem de crescimento. Logo existem $r$ e $R$ tais que $r<p<R$ implicando em:
                \begin{equation}
                    r<\frac{a_{n}}{b_{n}}<R
                \end{equation}
                \begin{equation}
                    rb_{n}<a_{n}<Rb_{n}
                \end{equation}
            \paragraph{}Conclui-se, por meio da comparação, que:
            \paragraph{}Se $\sum Rb_{n}$ converge então $\sum a_{n}$ converge. Claramente se uma série com termo geral com crescimento mais acelerado converge outra série com crescimento semelhante ou inferior deve convergir.
            \paragraph{}Se $\sum rb_{n}$ diverge então $\sum a_{n}$ diverge. Claramente se uma série com termo geral com crescimento menos acelerado diverge outra série com crescimento semelhante ou superior deve divergir.

        \subsection{Teste da Integral}
            \paragraph{}\textbf{Teorema:} Considere uma função $f(x)$ decrescente e positiva, isto é $f(x) \ge 0$,  com $\lim_{x\to\infty}f(x)=0$, então temos:
            \paragraph{}$\sum\limits_{n=1}^{\infty}a_{n}$ converge se e somente se $\int_{1}^{\infty}f(x)dx$ convergir.
            \paragraph{}$\sum\limits_{n=1}^{\infty}a_{n}$ diverge se e somente se $\int_{1}^{\infty}f(x)dx$ divergir.
            
        \subsection{Teste da Razão}
            \paragraph{}\textbf{Teorema:} Sejam $\sum\limits_{n=1}^{\infty}a_{n}$ e $a_{n}>0$ define-se $L$ como:
                \begin{equation}
                    \lim_{n\to\infty}\frac{a_{n+1}}{a_{n}}=L
                \end{equation}
            \paragraph{}Se $L<1$ então $\sum a_{n}$ converge.
            \paragraph{}Se $L>1$ então $\sum a_{n}$ diverge.
            \paragraph{}Note que o teorema não estabelece nenhuma conclusão para $L=1$, assim não é possível inferir nada sobre a sequência. 
            
        \subsection{Teste da Raiz}
            \paragraph{}\textbf{Teorema:} Sejam $\sum\limits_{n=1}^{\infty}a_{n}$ e $a_{n}>0$ define-se $L$ como:
                \begin{equation}
                    \lim_{n\to\infty}{(a_{n})}^{\frac{1}{n}}=L
                \end{equation}
            \paragraph{}Se $L<1$ então $\sum a_{n}$ converge.
            \paragraph{}Se $L>1$ então $\sum a_{n}$ diverge.
            \paragraph{}Este teste é normalmente aplicado quando todos os termos da sequência estão elevados a $n$, possibilitando simplificar drasticamente a equação.

        \subsection{Convergência Absoluta}
            \paragraph{Definição}Uma série $\sum a_{n}$ converge absolutamente se sua série absoluta equivalente, $\sum |a_{n}|$, converge.
            \paragraph{}\textbf{Teorema:}
            
        \subsection{Séries Alternadas}
            \paragraph{Definição}Uma será alternada se seus termos possuírem sinais alternados ao longo de toda a sequência, formalmente descrito como:
                \begin{equation}
                    \sum\limits_{n=1}^{\infty}(-1)^{n}a_{n}
                \end{equation}
                \begin{equation}
                    \sum\limits_{n=1}^{\infty}(-1)^{n+1}a_{n}
                \end{equation}
        
        \subsection{Teste das Séries Alternadas}
            \paragraph{}\textbf{Teorema:} Considerando uma série alternada da forma $\sum {(-1)}^{n}a_{n}$ será convergente se $a_{n}$ for decrescente e:
                \begin{equation}
                    \lim_{n\to\infty}a_{n}=0
                \end{equation}
                
        \section{Séries de Potência}
            \paragraph{Definição}Será uma série de potência aquela que puder ser representada como:
                \begin{equation}
                    \sum\limits_{n=1}^{\infty}c_{n}{(x-x_{0})}^{n}=c_{0}+c_{1}{(x-x_{0})}+\cdots+c_{n}{(x-x_{0})}^{n}
                \end{equation}
            \paragraph{}Analisando o somatório nota-se como a constante é atribuída:
                \begin{equation}
                    f(x)=\sum\limits_{n=1}^{\infty}c_{n}{(x-x_{0})}^{n}=\sum\limits_{n=1}^{\infty}\frac{f^{n}(x_{0})}{n!}{x}^{n}
                \end{equation}
            \paragraph{}Geralmente considera-se no caso geral a translação de $x$ em $x_{0}$, entretanto pode-se, por conveniência, analisar séries dessa forma em $x_{0}=0$. Nota-se que estas funções podem ser infinitamente diferenciáveis.
                \begin{equation}
                    \sum\limits_{n=1}^{\infty}c_{n}{x}^{n}
                \end{equation}
                
        \subsection{Funções Analíticas}
            \paragraph{Definição}Uma função $f(x)$ é analítica em $x\in J$, onde $J$ é um intervalo simétrico aberto qualquer, se sua Série de Taylor converge em $J$.
            \paragraph{}O intervalo de convergência pode ser obtido pelo teste da razão:
                \begin{equation}
                    \lim_{n\to\infty}\frac{c_{n+1}{|x|}^{n+1}}{c_{n}{|x|}^{n}}
                \end{equation}
                \begin{equation}
                    |x|=\lim_{n\to\infty}\frac{c_{n+1}}{c_{n}}
                \end{equation}
                \begin{equation}
                    \begin{matrix} \underbrace{\lim_{n\to\infty}\frac{c_{n+1}}{c_{n}}}\\I\end{matrix}
                \end{equation}
            \paragraph{}Segundo o teste da razão sabe-se que para $|x|\ne 0$ deve-se analisar $I$.
            \paragraph{}

        \subsection{Método de Séries de Potência}

\end{document}