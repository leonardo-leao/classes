\documentclass{article}

\usepackage[a4paper, hmargin={20mm, 20mm}, vmargin={25mm, 30mm}]{geometry}
\usepackage[utf8]{inputenc}
\usepackage[english, main=portuguese]{babel}

\usepackage[hidelinks]{hyperref}
\usepackage{bookmark}
\usepackage{cancel}
\usepackage{comment}

\usepackage{array}
\usepackage{indentfirst}
\usepackage{multicol}
\setlength{\multicolsep}{2pt}% 50% of original values
\usepackage{subfiles}

\usepackage{titlesec}

\usepackage{amsmath}
\usepackage{amssymb}
\usepackage{systeme}
\usepackage{float}
\usepackage{enumitem}
\usepackage[thinc]{esdiff} %parcial derivatives
\restylefloat{table}

\usepackage{graphicx}
\usepackage{subcaption}
\graphicspath{ {./images/} }

% Pacote para a definição de novas cores
\usepackage{xcolor}
% Definindo novas cores
\definecolor{darkgreen}{rgb}{0.0, 0.42, 0.24}
\definecolor{darkpurple}{rgb}{0.74, 0.2, 0.64}
\definecolor{darkblue}{rgb}{0.0, 0.28, 0.67}

%Configurando numeração de objetos; figuras, equações e etc., em ambientes; section, subsection e etc..
\usepackage{chngcntr}
\counterwithin{figure}{section}
\counterwithin{equation}{subsection}

%Configurando espaços entre paragrafos
%\setlength{\parskip}{0.5em}

%Configurando pacote de Gráficos plots
\usepackage{pgfplots}
\usepackage{tikz}

%Configurando pacote de circuitos
\usepackage{circuitikz}

% Configurando layout para mostrar códigos
\usepackage{listings}

%Configurando multiple files
\usepackage{filecontents}

%Configurando quotes
\usepackage{csquotes}

\newcommand{\myStyle}{
\lstset{
    language=Octave,                            % the language of the code
    basicstyle=\ttfamily\small,               % the size of the fonts that are used for the code
    keywordstyle=\color{darkpurple}\bfseries, %
    stringstyle=\color{darkblue},             %
    commentstyle=\color{darkgreen},           %
    morecomment=[s][\color{blue}]{/**}{*/},   %
    extendedchars=true,                       %
    showtabs=false,                           % show tabs within strings adding particular underscores
    showspaces=false,                         % show spaces adding particular underscores
    showstringspaces=false,                   % underline spaces within strings
    numbers=left,                             % where to put the line-numbers
    numberstyle=\tiny\color{gray},            % the style that is used for the line-numbers
    stepnumber=1,                             % the step between two line-numbers. If it's 1, each line will be numbered
    numbersep=5pt,                            % how far the line-numbers are from the code
    frame=single,                             % adds a frame around the code
    rulecolor=\color{black},                  % if not set, the frame-color may be changed on line-breaks within not-black text
    breaklines=true,                          % sets automatic line breaking
    backgroundcolor=\color{white},            % choose the background color
    breakatwhitespace=true,                   % sets if automatic breaks should only happen at whitespace
    breakautoindent=false,                    %
    captionpos=b,                             % sets the caption-position to bottom
    xleftmargin=0pt,                          %
    tabsize=2,                                % sets default tabsize to 2 spaces
}}

%\titleformat{<command>}[<shape>]{<format>}{<label>}{<sep>}{<before-code>}[<after-code>]
\titleformat
{\section} %comand
[block]  %shape
{\normalfont\LARGE} %format
{\thesection. } %label
{0mm} %sep
{} %before-code
[{\titlerule[0.1mm]}] %after-code

\titlespacing*{\section}{0mm}{0mm}{15mm}

\titleformat
{\subsection} %comand
[block]  %shape
{\normalfont\Large} %format
{\thesubsection. } %label
{0mm} %sep
{} %before-code
[] %after-code

\titlespacing*{\subsection}{0mm}{5mm}{2.5mm}


\begin{document}
    \begin{titlepage}
        \begin{center}
            \rule{450pt}{0.5pt}\\[4mm]
            {\Huge EM404 - Dinâmica}\\
            \rule{450pt}{0.5pt}\\[2mm]
            {\Large Resumo Teórico}\\[200mm]
            \today\\
            \rule{250pt}{0.5pt}\\
            {\large Guilherme Nunes Trofino}\\
            {\large 217276}\\
        \end{center}
    \end{titlepage}
\newpage

    \tableofcontents
\newpage

    \section{Coordenadas}
        \subsection{Coordenadas Retangulares}
            \paragraph{Definição}Decomposição vetorial das vairáveis envolvidas no movimento da partícula ao longo do plano cartesiano, abssissas $x$ e ordenadas $y$, sendo descritos pelas seguintes equações:
                \begin{align*}
                    \vec{r} &= x\vec{i} + y\vec{j}\\
                    \vec{v} = \dot{\vec{r}} &= \dot{x}\vec{i} + \dot{y}\vec{j}\\
                    \vec{a} = \dot{\vec{v}} = \ddot{\vec{r}} &= \ddot{x}\vec{i} + \ddot{y}\vec{j}\\
                \end{align*}

        \subsection{Coordenadas Normal e Tangencial}
            \paragraph{Definição}Decomposição vetorial das variáveis envolvidas no movimento da partícula ao longo da tangente, $t$, e ao longo da normal, $n$, da trajetória percorrida, onde os sentidos são descritos a seguir:
                \begin{enumerate}[noitemsep]
                    \item \textbf{Direção Normal:} Positivo em direção ao centro de curvatura;
                        \begin{enumerate}[noitemsep]
                            \item $\vec{e_{n}}$, \texttt{Vetor Unitário Normal}: Cuja derivada será dada por:
                                \[
                                    \boxed{
                                        \diff{\vec{e_{t}}}{\beta} = \vec{e_{n}}
                                    }
                                    \hspace{20mm}
                                    \boxed{
                                        \dot{\vec{e_{t}}} = \dot{\beta}\vec{e_{n}}
                                    }
                                \]
                        \end{enumerate}
                    \item \textbf{Direção Tangencial:} Positivo em direção ao movimento;
                        \begin{enumerate}[noitemsep]
                            \item $\vec{e_{t}}$: \texttt{Vetor Unitário Tangencial};
                        \end{enumerate}
                \end{enumerate}
            Assim pode-se definir as seguintes variáveis:
                \begin{enumerate}[noitemsep]
                    \item \textbf{Velocidade:} Sempre será tangente a trajetória, sendo descrita pelas seguintes equações:
                        \begin{equation}
                            \boxed{
                                v = \sqrt{v_{n}^{2} + v_{t}^{2}} =
                                \begin{cases}
                                    v_{n} = 0\\
                                    v_{t} = \rho \diff{\beta}{t} = \rho \dot{\beta}\\
                                \end{cases}
                            }
                            \hspace{20mm}
                            \boxed{
                                \vec{v} = \diff{s}{t} = \rho \diff{\beta}{t} \vec{e_{t}} = \rho \dot{\beta} \vec{e_{t}}
                            }
                        \end{equation}
                    Onde:
                        \begin{enumerate}[noitemsep]
                            \item $\beta$: \texttt{Ângulo Percorrido na Curvatura};
                            \item $\rho$: \texttt{Raio de Curvatura da Trajetória};
                        \end{enumerate}

                    \item \textbf{Aceleração:} Sempre terá componente normal, modificando a direção da trajetória da partícula, e componente tangencial, modificando o módulo da velocidade, sendo descritas pelas seguintes equações:
                        \begin{equation}
                            \boxed{
                                a = \sqrt{a_{n}^{2} + a_{t}^{2}} =
                                \begin{cases}
                                    a_{n} = \frac{v^{2}}{\rho} = \rho \dot{\beta}^{2}\\
                                    a_{t} = \dot{v} = \ddot{s}\\
                                \end{cases}
                            }
                            \hspace{20mm}
                            \boxed{
                                \vec{a} = \diff{v}{t} = \diff{v\dot{\vec{e_{t}}}}{t} = v \dot{\vec{e_{t}}} + \dot{v}\vec{e_{t}} = \frac{v^{2}}{\rho}\vec{e_{n}} + \dot{v}\vec{e_{t}}
                            }
                        \end{equation}
                \end{enumerate}

        \subsection{Coordenadas Polares}
            \paragraph{Definição}Decomposição vetorial das variáveis envolvidas no movimento da partícula a uma distância radial $r$, medida a partir de um ponto fixo $O$, e por um ângulo $\theta$ da linha radial, onde os sentidos são descritos a seguir:
                \begin{enumerate}[noitemsep]
                    \item \textbf{Direção Angular:} Positivo em direção antihorária;
                        \begin{enumerate}
                            \item $\vec{e_{\theta}}$, \texttt{Vetor Unitário Angular}: Cuja derivada será dada por:
                            \[
                                \boxed{
                                    \diff{\vec{e_{r}}}{\theta} = \vec{e_{\theta}}
                                }
                                \hspace{20mm}
                                \boxed{
                                    \dot{\vec{e_{r}}} = \dot{\theta} \vec{e_{\theta}}
                                }
                            \]
                        \end{enumerate}
                    \item \textbf{Direção Radial:} Positivo do ponto de referência até o ponto referenciado;
                        \begin{enumerate}
                            \item $\vec{e_{r}}$, \texttt{Vetor Unitário Radial}: Cuja derivada será dada por:
                            \[
                                \boxed{
                                    \diff{\vec{e_{\theta}}}{\theta} = - \vec{e_{r}}
                                }
                                \hspace{20mm}
                                \boxed{
                                    \dot{\vec{e_{\theta}}} = -\dot{\theta} \vec{e_{r}}
                                }
                            \]
                        \end{enumerate}
                \end{enumerate}
            Assim pode-se definir as seguintes variáveis:
                \begin{enumerate}[noitemsep]
                    \item \textbf{Velocidade:} Diferenciando o espaço com $\vec{r} = r \vec{e_{r}}$:
                        \begin{equation}
                            \boxed{
                                v = \sqrt{v_{r}^{2} + v_{\theta}^{2}} =
                                \begin{cases}
                                    v_{r} = \dot{r}\\
                                    v_{\theta} = r\dot{\theta}\\
                                \end{cases}
                            }
                            \hspace{20mm}
                            \boxed{
                                \vec{v} = \diff{s}{t} = \diff{r \vec{e_{r}}}{t} = \dot{r}\vec{e_{r}} + r\dot{\theta}\vec{e_{\theta}}
                            }
                        \end{equation}

                    \item \textbf{Aceleração:} Diferenciando a velocidade:
                        \begin{equation}
                            \boxed{
                                a = \sqrt{a_{r}^{2} + a_{\theta}^{2}} =
                                \begin{cases}
                                    a_{r} = \ddot{r} - r\dot{\theta}^{2}\\
                                    a_{\theta} = r\ddot{\theta} + 2\dot{r}\dot{\theta}\\
                                \end{cases}
                            }
                            \hspace{20mm}
                            \boxed{
                                \vec{a} = \diff{v}{t} = \diff{v\dot{\vec{e_{t}}}}{t} = v \dot{\vec{e_{t}}} + \dot{v}\vec{e_{t}} = \frac{v^{2}}{\rho}\vec{e_{n}} + \dot{v}\vec{e_{t}}
                            }
                        \end{equation}
                \end{enumerate}

        \subsection{Coordenadas Cilíndricas}
            \paragraph{Definição}

        \subsection{Coordenadas Esféricas}
            \paragraph{Definição}
\newpage

    \section{Cinemática}
        \subsection{Movimento Retilíneo}
            \paragraph{Definição}Caso geral do movimento de uma partícula ao longo de uma linha reta, onde sua posição em qualquer instante $t$ poderá ser determinada por sua distância $s$ a partir de um referêncial fixo $O$ sobre a linha, sendo descrito pelas seguintes equações:
                \begin{enumerate}[noitemsep]
                    \item \textbf{Coordenadas Retangulares:}
                        \begin{enumerate}[noitemsep]
                            \item \texttt{Velocidade Instantânea:}
                                \begin{equation}
                                    \boxed{
                                        v = \lim_{\Delta t \to 0} \frac{\Delta s}{\Delta t} = \diff{s}{t} =\dot{s}
                                    }
                                \end{equation}
                            \item \texttt{Aceleração Instantânea:}
                                \begin{equation}
                                    \boxed{
                                        a = \lim_{\Delta t \to 0} \frac{\Delta v}{\Delta t} = \diff{v}{t} =\dot{v} = \ddot{s}
                                    }
                                \end{equation}
                            Onde:
                                \begin{enumerate}[noitemsep]
                                    \item \textbf{Aceleração Constante:}
                                        \[
                                            \boxed{v = v_{0} + a \cdot t}
                                            \hspace{10mm}
                                            \boxed{v^{2} = v_{0}^{2} + 2 a (s - s_{0})}
                                            \hspace{10mm}
                                            \boxed{s = s_{0} + v_{0} \cdot t + \frac{1}{2} a t^{2}}
                                        \]
                                    \item \textbf{Aceleração Função do Tempo:}
                                        \[
                                            \boxed{v = v_{0} + \int_{0}^{t} f(t) dt}
                                            \hspace{10mm}
                                            \boxed{s = s_{0} + \int_{0}^{t} v dt}
                                        \]
                                    \item \textbf{Aceleração Função da Velocidade:}
                                        \[
                                            \boxed{t = \int_{v_{0}}^{v} \frac{dv}{f(v)}}
                                            \hspace{10mm}
                                            \boxed{s = s_{0} + \int_{v_{0}}^{v} \frac{v dv}{f(v)}}
                                        \]
                                \end{enumerate}
                        \end{enumerate}
                \end{enumerate}
            Onde:
                \begin{equation}
                    \boxed{
                        \dot{s} d\dot{s} = \ddot{s} ds
                    }
                \end{equation}

        \subsection{Movimento Curvilíneo Plano}
            \paragraph{Definição}Caso geral do movimento de uma partícula ao longo de uma curva em um plano com suas variáveis decompostas nas coordenadas cartesianas, sendo descrito pelas seguintes equações:
                \begin{enumerate}[noitemsep]
                    \item \textbf{Coordenadas Retangulares:} Comumente aplicada quando se tratam de lançamentos oblíquos:
                        \[
                            \boxed{x = x_{0} + v_{0_{x}} \cdot t}
                        \]
                        \[
                            \boxed{v_{y} = v_{0_{y}} - g \cdot t}
                            \hspace{10mm}
                            \boxed{y = y_{0} + v_{0_{y}} \cdot t - \frac{1}{2}g t^{2}}
                            \hspace{10mm}
                            \boxed{v_{y}^{2} = v_{0_{y}}^{2} - 2g(y - y_{0})}
                        \]
                \end{enumerate}

        \subsection{Movimento Circular}
            \paragraph{Definição}Caso especial do movimento curvilíneo plano em que o raio de curvatura $\rho$ será constante durante toda trajetória, um círculo de raio $r$, e seu ângulo $\beta$ será dado pelo ângulo $\theta$ a partir de um referência conveniente, sendo descritos pelas seguintes equações:
                \begin{enumerate}
                    \item \textbf{Coordenadas Normal e Tangencial:}
                        \begin{equation}
                            \boxed{
                                v = r \dot{\theta}
                            }
                            \hspace{20mm}
                            \boxed{
                                a = 
                                \begin{cases}
                                    a_{n} = v \dot{\theta}\\
                                    a_{t} = r \ddot{\theta}\\
                                \end{cases}
                            }
                        \end{equation}
                    \item \textbf{Coordenadas Polares:}
                        \begin{equation}
                            \boxed{
                                v = 
                                \begin{cases}
                                    v_{r} = \dot{r}\\
                                    v_{\theta} = r \dot{\theta}\\
                                \end{cases}
                            }
                            \hspace{20mm}
                            \boxed{
                                a = 
                                \begin{cases}
                                    a_{r} = - r\dot{\theta}^{2}\\
                                    a_{\theta} = r \ddot{\theta}\\
                                \end{cases}
                            }
                        \end{equation}
                \end{enumerate}

        \subsection{Movimento Curvilíneo Espacial}
            \paragraph{Definição}Caso geral de movimento tridimensional de uma partícula ao longo de uma curva espacial, sendo descrito pelas seguintes equações:
                \begin{enumerate}[noitemsep]
                    \item \textbf{Coordenadas Retangulares:}
                        \begin{align*}
                            \vec{R} &= x\vec{i} + y\vec{j} + z\vec{k}\\
                            \vec{v} = \dot{\vec{R}} &= \dot{x}\vec{i} + \dot{y}\vec{j} + \dot{z}\vec{k}\\
                            \vec{a} = \dot{\vec{v}} = \ddot{\vec{R}} &= \ddot{x}\vec{i} + \ddot{y}\vec{j} + \ddot{z}\vec{k}\\
                        \end{align*}
                    \item \textbf{Coordenadas Cilíndricas:}
                        \begin{align*}
                            \vec{R} &= r\vec{e_{r}} + z\vec{k}\\
                            \vec{v} = \dot{\vec{R}} &= \dot{r}\vec{e_{r}} + r\dot{\theta}\vec{e_{\theta}} + \dot{z}\vec{k}\\
                            \vec{a} = \dot{\vec{v}} = \ddot{\vec{R}} &= (\ddot{r} - r\dot{\theta}^{2})\vec{e_{r}} + (r\ddot{\theta} + 2\dot{r}\dot{\theta})\vec{j} + \ddot{z}\vec{k}\\
                        \end{align*}
                    \item \textbf{Coordenadas Esféricas:}
                        \[
                            \vec{v} = v_{R}\vec{e_{R}} + v_{\theta}\vec{e_{\theta}} + v_{\phi}\vec{e_{\phi}}
                            \hspace{2.5mm}
                            \begin{cases}
                                v_{R} = \dot{R}\\
                                v_{\theta} = R\dot{\theta}\cos(\phi)\\
                                v_{\phi} = R\dot{\phi}\\
                            \end{cases}
                            \hspace{5mm}
                            \vec{a} = d_{R}\vec{e_{R}} + a_{\theta}\vec{e_{\theta}} + a_{\phi}\vec{e_{\phi}}
                            \hspace{2.5mm}
                            \begin{cases}
                                a_{R} = \ddot{R} - R\dot{\phi}^{2} - R\dot{\theta}^{2}\cos^{2}(\phi)\\
                                a_{\theta} = \frac{\cos(\phi)}{R} \frac{d(R^{2} \dot{\theta})}{dt} - 2R\dot{\theta}\dot{\phi}\sin(\phi)\\
                                a_{\phi} = \frac{1}{R} \frac{dR^{2}\dot{\phi}}{dt} + R\dot{\theta}^{2}\sin(\phi)\cos(\phi)\\
                            \end{cases}
                        \]
                \end{enumerate}

        \subsection{Movimento Relativo}
            \paragraph{Definição}Caso geral do movimento medido a partir da combinação entre referências movéis, $B$, e referências absolutas, $A$, de uma partícula no espaço, sendo descrito pelas seguintes equações: 
                \begin{equation}
                    \boxed{
                        \vec{r_{A}} = \vec{r_{B}} + \vec{r_{AB}}
                    }
                    \hspace{10mm}
                    \boxed{
                        \vec{v_{A}} = \vec{v_{B}} + \vec{v_{AB}}
                    }
                    \hspace{10mm}
                    \boxed{
                        \vec{a_{A}} = \vec{a_{B}} + \vec{a_{AB}}
                    }
                \end{equation}
            O \textbf{Sistema Inercial} ocorre quando a velocidade do sistema for constante.

        \subsection{Movimento Restritro de Partículas Conectadas}
            \paragraph{Definição}Caso geral do movimento de partículas cujas variáveis são interdependentes que podem apresentar diferentes graus de restrição, sendo descrito pelas seguintes equações:
                \begin{enumerate}[noitemsep]
                    \item \textbf{Um Grau de Liberdade:} Cenário em que a restrição em um dispositivo impede o movimento do sistema.
                    \item \textbf{Dois Graus de Liberdade:} Cenário em que a restrição em dois dispositivo impede o movimento do sistema.
                \end{enumerate}
            Em casos envolvendo cordas será necessário obter um \textbf{Comprimento Total} do cabo de ligação, assumindo que seja inextensivo, para que, com derivações sucessivas, obtenha-se equações de velocidades e acelerações dos corpos.
\newpage

    \section{Cinética}
        \subsection{2\textsuperscript{a} Lei de Newton}
            \paragraph{Definição}Quando uma massa sofrer a ação de uma força esta experimentará uma aceleração proporcional, sendo descrita pela seguinte equação:
                \begin{equation}
                    \boxed{
                        \vec{F} = m\vec{a}
                    }
                \end{equation}
                \begin{enumerate}[noitemsep]
                    \item \textbf{Movimento Geral:} Casos gerais:
                        \begin{equation}
                            \boxed{
                                \vec{a} = a_{x}\vec{i} + a_{y}\vec{j} + a_{z}\vec{k}
                                \hspace{5mm}
                                a = \sqrt{a_{x}^2 + a_{y}^2 + a_{z}^2}
                            }
                        \end{equation}
                        \begin{equation}
                            \boxed{
                                \vec{F} = F_{x}\vec{i} + F_{y}\vec{j} + F_{z}\vec{k}
                                \hspace{5mm}
                                F = \sqrt{F_{x}^2 + F_{y}^2 + F_{z}^2}
                            }
                        \end{equation}
                \end{enumerate}

        \subsection{Trabalho e Energia}
            \paragraph{Definição}Quando uma força atua sobre uma partícula ao longo de um deslocamento está realizará trabalho, sendo descrito pela seguinte equação:
                \begin{equation}
                    \boxed{dU = \vec{F} \cdot d\vec{r}}
                \end{equation}
            Onde:
                \begin{enumerate}[noitemsep]
                    \item \textbf{Força Elástica:} Considerando esta força, obtém-se a seguinte equação:
                        \begin{enumerate}[noitemsep]
                            \item \texttt{Trabalho:}
                                \[
                                    \boxed{
                                        U_{1-2} = \int_{1}^{2} \vec{F} \cdot d\vec{r}
                                                = \int_{1}^{2} (-kx\vec{i})\cdot dx \vec{i}
                                                = \frac{1}{2} k (x_{1}^2 - x_{2}^2)
                                    }
                                \]
                            \item \texttt{Energia:}
                                \begin{equation}
                                    \boxed{
                                        V_{e} = \frac{1}{2} k x^{2}
                                    }
                                \end{equation}
                        \end{enumerate}
                    \item \textbf{Força Gravitacional:} Considerando esta força, obtém-se a seguinte equação:
                        \begin{enumerate}[noitemsep]
                            \item \texttt{Trabalho:}
                                \[
                                    \boxed{
                                        U_{1-2} = \int_{1}^{2} \vec{F} \cdot d\vec{r}
                                                = \int_{1}^{2} (-mg\vec{j})\cdot dy \vec{j}
                                                = mg (y_{1} - y_{2})
                                    }
                                \]
                            \item \texttt{Energia:}
                                \begin{equation}
                                    \boxed{
                                        V_{g} = mgh
                                    }
                                \end{equation}
                        \end{enumerate}
                    \item \textbf{Força Cinética:} Considerando esta força, obtém-se a seguinte equação:
                        \begin{enumerate}[noitemsep]
                            \item \texttt{Trabalho:}
                                \[
                                    \boxed{
                                        U_{1-2} = \int_{1}^{2} \vec{F} \cdot d\vec{r}
                                                = \int_{1}^{2} mvdv
                                                = \frac{1}{2} m (v_{2}^{2} - v_{1}^{2})
                                    }
                                \]
                            \item \texttt{Energia:}
                                \begin{equation}
                                    \boxed{
                                        T = \frac{1}{2} m v^{2}
                                    }
                                \end{equation}
                        \end{enumerate}
                \end{enumerate}
            Assim, pelo \textbf{Princípio do Trabalho e Energia Cinética}, o trabalho total realizado por todas as forças agindo sobre uma partícula enquanto se move do ponto 1 para o ponto 2 será igual a variação da energia cinética, formalmente descrita pela seguinte equação:
                \begin{equation}
                    \boxed{
                        T_{1} + U_{1-2} = T_{2}
                    }
                \end{equation}

        \subsection{Impulso e Quantidade de Movimento}
            \paragraph{Definição}Quando uma partícula se desloca com velocidade constante haverá uma quantidade de movimento, sendo descrito pela seguinte equação:
                \begin{equation}
                    \boxed{
                        \vec{F} = m \vec{v} = \diff{(mv)}{t} = \dot{\vec{G}}
                    }
                \end{equation}
            Assim, pelo \textbf{Princípio do Impulso e Quantidade de Movimento}, a quantidade de movimento linear total de uma partícula enquanto se move do ponto 1 para o ponto 2 será igual ao impulso aplicado, formalmente descrito pela seguinte equação:
                \begin{equation}
                    \boxed{
                        \vec{G_{1}} + \int_{t_{1}}^{t_{2}} \vec{F} dt = \vec{G_{2}}
                    }
                \end{equation}

        \subsection{Impulso Angular e Quantidade de Movimento Angular}
            \paragraph{Definição}Quando uma partícula se desloca entorno de uma origem $O$, sendo descrito pela seguinte equação:
                \begin{equation}
                    \boxed{
                        \vec{H_{O}} = \vec{r} \times m\vec{v} = m \cdot
                        \begin{vmatrix}
                            \vec{i} & \vec{j} & \vec{k}\\
                            x       & y       & z\\
                            v_{x}   & v_{y}   & v_{z}\\
                        \end{vmatrix}
                    }
                \end{equation}
            Assim, pelo \textbf{Princípio do Impulso e Quantidade de Movimento Angular}, a quantidade de movimento linear angular total de uma partícula enquanto se move do ponto 1 para o ponto 2 será igual ao momento angular aplicado, formalmente descrito pela seguinte equação:
                \begin{equation}
                    \boxed{
                        \vec{H_{1}} + \int_{t_{1}}^{t_{2}} \vec{M} dt = \vec{H_{2}}
                    }
                \end{equation}

        \subsection{Colisões}
            \paragraph{Definição}Quando dois ou mais corpos se encontram este dividiram suas quantidades de movimento proporcionalmente a suas massas e as suas velocidades, sendo descrito pelas seguintes equações:
                \begin{enumerate}[noitemsep]
                    \item \textbf{Impacto Central Direto:}
                        \begin{equation}
                            \boxed{
                                m_{A}v_{A1} + m_{B}v_{B1} = m_{A}v_{A2} + m_{B}v_{B2}
                            }
                        \end{equation}
                        \begin{equation}
                            \boxed{
                                e = \frac{v_{B2} - v_{A2}}{v_{A1} - v_{B1}}
                                \hspace{2.5mm}
                                \begin{cases}
                                    e = 1, & \text{Impacto Elástico};\\
                                    e = 0, & \text{Impacto Inelástico};\\
                                \end{cases}
                            }
                        \end{equation}
                    \item \textbf{Impacto Central Oblíquo}
                        \begin{equation}
                            \boxed{
                                m_{A}v_{A1_{n}} + m_{B}v_{B1_{n}} = m_{A}v_{A2_{n}} + m_{B}v_{B2_{n}}
                            }
                        \end{equation}
                        \begin{equation}
                            \boxed{m_{A}v_{A1_{t}} = m_{A}v_{A2_{t}}}
                            \hspace{10mm}
                            \boxed{m_{B}v_{B1_{t}} = m_{B}v_{B2_{t}}}
                        \end{equation}
                        \begin{equation}
                            \boxed{
                                e = \frac{v_{B2_{n}} - v_{A2_{n}}}{v_{A1_{n}} - v_{B1_{n}}}
                                \hspace{2.5mm}
                                \begin{cases}
                                    e = 1, & \text{Impacto Elástico};\\
                                    e = 0, & \text{Impacto Inelástico};\\
                                \end{cases}
                            }
                        \end{equation}
                \end{enumerate}
\newpage

    \section{Cinética de Partículas}
        \paragraph{Definição}Análise dos princípios dinâmicos de movimento para sistemas gerais compostos por partículas, onde as equações discutidas se aplicam a mais casos estudados.

        \subsection{2\textsuperscript{a} Lei de Newton}
            \paragraph{Definição}



\end{document}