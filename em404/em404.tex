\documentclass{article}

\usepackage[a4paper, hmargin={20mm, 20mm}, vmargin={25mm, 30mm}]{geometry}
\usepackage[utf8]{inputenc}
\usepackage[english, main=portuguese]{babel}

\usepackage[hidelinks]{hyperref}
\usepackage{bookmark}
\usepackage{cancel}
\usepackage{comment}

\usepackage{array}
\usepackage{indentfirst}
\usepackage{multicol}
\setlength{\multicolsep}{2pt}% 50% of original values
\usepackage{subfiles}

\usepackage{titlesec}

\usepackage{amsmath}
\usepackage{amssymb}
\usepackage{systeme}
\usepackage{float}
\usepackage{enumitem}
\usepackage[thinc]{esdiff} %parcial derivatives
\restylefloat{table}

\usepackage{graphicx}
\usepackage{subcaption}
\graphicspath{ {./images/} }

% Pacote para a definição de novas cores
\usepackage{xcolor}
% Definindo novas cores
\definecolor{darkgreen}{rgb}{0.0, 0.42, 0.24}
\definecolor{darkpurple}{rgb}{0.74, 0.2, 0.64}
\definecolor{darkblue}{rgb}{0.0, 0.28, 0.67}

%Configurando espaços entre paragrafos
%\setlength{\parskip}{0.5em}

%Configurando pacote de Gráficos plots
\usepackage{pgfplots}
\usepackage{tikz}

%Configurando pacote de circuitos
\usepackage{circuitikz}

% Configurando layout para mostrar códigos
\usepackage{listings}

%Configurando multiple files
\usepackage{filecontents}

%Configurando quotes
\usepackage{csquotes}

\newcommand{\myStyle}{
\lstset{
    language=Octave,                            % the language of the code
    basicstyle=\ttfamily\small,               % the size of the fonts that are used for the code
    keywordstyle=\color{darkpurple}\bfseries, %
    stringstyle=\color{darkblue},             %
    commentstyle=\color{darkgreen},           %
    morecomment=[s][\color{blue}]{/**}{*/},   %
    extendedchars=true,                       %
    showtabs=false,                           % show tabs within strings adding particular underscores
    showspaces=false,                         % show spaces adding particular underscores
    showstringspaces=false,                   % underline spaces within strings
    numbers=left,                             % where to put the line-numbers
    numberstyle=\tiny\color{gray},            % the style that is used for the line-numbers
    stepnumber=1,                             % the step between two line-numbers. If it's 1, each line will be numbered
    numbersep=5pt,                            % how far the line-numbers are from the code
    frame=single,                             % adds a frame around the code
    rulecolor=\color{black},                  % if not set, the frame-color may be changed on line-breaks within not-black text
    breaklines=true,                          % sets automatic line breaking
    backgroundcolor=\color{white},            % choose the background color
    breakatwhitespace=true,                   % sets if automatic breaks should only happen at whitespace
    breakautoindent=false,                    %
    captionpos=b,                             % sets the caption-position to bottom
    xleftmargin=0pt,                          %
    tabsize=2,                                % sets default tabsize to 2 spaces
}}

%\titleformat{<command>}[<shape>]{<format>}{<label>}{<sep>}{<before-code>}[<after-code>]
\titleformat
{\section} %comand
[block]  %shape
{\normalfont\LARGE} %format
{\thesection. } %label
{0mm} %sep
{} %before-code
[{\titlerule[0.1mm]}] %after-code

\titlespacing*{\section}{0mm}{0mm}{15mm}

\titleformat
{\subsection} %comand
[block]  %shape
{\normalfont\Large} %format
{\thesubsection. } %label
{0mm} %sep
{} %before-code
[] %after-code

\titlespacing*{\subsection}{0mm}{5mm}{2.5mm}


\begin{document}
    \begin{titlepage}
        \begin{center}
            \rule{450pt}{0.5pt}\\[4mm]
            {\Huge EM404 - Dinâmica}\\
            \rule{450pt}{0.5pt}\\[2mm]
            {\Large Resumo Teórico}\\[200mm]
            \today\\
            \rule{250pt}{0.5pt}\\
            {\large Guilherme Nunes Trofino}\\
            {\large 217276}\\
        \end{center}
    \end{titlepage}
\newpage

    \tableofcontents
\newpage

    \section{Introdução}
        \subsection{Conceitos Básiscos}
\newpage

    \section{Coordenadas}
        \subsection{Coordenadas Normal e Tangencial}
            \paragraph{Definição}Decomposição vetorial das variáveis envolvidas no movimento da partícula ao longo da tangente, $t$, e ao longo da normal, $n$, da trajetória percorrida, onde os sentidos são descritos a seguir:
                \begin{enumerate}[noitemsep]
                    \item \textbf{Direção Normal:} Positivo em direção ao centro de curvatura;
                    \item \textbf{Direção Tangencial:} Positivo em direção ao movimento;
                \end{enumerate}
            Assim pode-se definir as seguintes variáveis:
                \begin{enumerate}[noitemsep]
                    \item \textbf{Velocidade:} Sempre será tangente a trajetória, sendo descrita pelas seguintes equações:
                        \begin{equation}
                            \boxed{
                                v = \diff{s}{t} = \rho \diff{\beta}{t} = \rho \dot{\beta}
                            }
                            \hspace{20mm}
                            \boxed{
                                \vec{v} = \rho \diff{\beta}{t} \vec{e_{t}} = \rho \dot{\beta} \vec{e_{t}}
                            }
                        \end{equation}
                    Onde:
                        \begin{enumerate}[noitemsep]
                            \item $\vec{e_{t}}$: \texttt{Vetor Unitário Tangencial};
                            \item $\beta$: \texttt{Ângulo Percorrido na Curvatura};
                            \item $\rho$: \texttt{Raio de Curvatura da Trajetória};
                        \end{enumerate}

                    \item \textbf{Aceleração:} Sempre terá componente normal, modificando a direção da trajetória da partícula, e componente tangencial, modificando o módulo da velocidade, sendo descritas pelas seguintes equações:
                        \begin{equation}
                            \boxed{
                                a = \sqrt{a_{n}^{2} + a_{t}^{2}} =
                                \begin{cases}
                                    a_{n} = \frac{v^{2}}{\rho} = v \beta\\
                                    a_{t} = \dot{v} = \ddot{s}\\
                                \end{cases}
                            }
                            \hspace{20mm}
                            \boxed{
                                \vec{a} = \diff{v}{t} = \diff{v\dot{\vec{e_{t}}}}{t} = v \dot{\vec{e_{t}}} + \dot{v}\vec{e_{t}} = \frac{v^{2}}{\rho}\vec{e_{n}} + \dot{v}\vec{e_{t}}
                            }
                        \end{equation}
                    Onde:
                        \begin{enumerate}[noitemsep]
                            \item $\vec{e_{n}}$: \texttt{Vetor Unitário Normal};
                                \[
                                    \boxed{
                                        \diff{\vec{e_{t}}}{\beta} = \vec{e_{n}}
                                    }
                                    \hspace{20mm}
                                    \boxed{
                                        \dot{\vec{e_{t}}} = \dot{\beta}\vec{e_{n}}
                                    }
                                \]
                        \end{enumerate}
                \end{enumerate}
\newpage

    \section{Cinemática}
        \subsection{Movimento Retilíneo}

        \subsection{Movimento Circular}
            \paragraph{Definição}Caso especial do movimento curvilíneo plano em que o raio de curvatura $\rho$ será constante durante toda trajetória, um círculo de raio $r$, e seu ângulo $\beta$ será dado pelo ângulo $\theta$ a partir de um referência conveniente, sendo descritos pelas seguintes equações:
                \begin{equation}
                    \boxed{
                        v = r \dot{\theta}
                    }
                    \hspace{20mm}
                    \boxed{
                        a = 
                        \begin{cases}
                            a_{n} = v \dot{\theta}\\
                            a_{t} = r \ddot{\theta}\\
                        \end{cases}
                    }
                \end{equation}
\newpage

\end{document}