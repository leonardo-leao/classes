\documentclass{article}

\usepackage[a4paper, hmargin={20mm, 20mm}, vmargin={25mm, 30mm}]{geometry}
\usepackage[utf8]{inputenc}
\usepackage[english, main=portuguese]{babel}

\usepackage[hidelinks]{hyperref}
\usepackage{bookmark}
\usepackage{cancel}
\usepackage{comment}

\usepackage{array}
\usepackage{indentfirst}
\usepackage{multicol}
\setlength{\multicolsep}{2pt}% 50% of original values
\usepackage{subfiles}

\usepackage{titlesec}

\usepackage{amsmath}
\usepackage{amssymb}
\usepackage{tabularx}
\usepackage{systeme}
\usepackage{float}
\usepackage{enumitem}
\usepackage[thinc]{esdiff} %parcial derivatives
\restylefloat{table}

\usepackage{graphicx}
\usepackage{subcaption}
\graphicspath{ {./images/} }

% Pacote para a definição de novas cores
\usepackage{xcolor}
% Definindo novas cores
\definecolor{darkgreen}{rgb}{0.0, 0.42, 0.24}
\definecolor{darkpurple}{rgb}{0.74, 0.2, 0.64}
\definecolor{darkblue}{rgb}{0.0, 0.28, 0.67}

%Configurando numeração de objetos; figuras, equações e etc., em ambientes; section, subsection e etc..
\usepackage{chngcntr}
\counterwithin{figure}{section}
\counterwithin{equation}{subsection}

%Configurando espaços entre paragrafos
%\setlength{\parskip}{0.5em}

%Configurando pacote de Gráficos plots
\usepackage{pgfplots}
\usepackage{tikz}

%Configurando pacote de circuitos
\usepackage{circuitikz}

% Configurando layout para mostrar códigos
\usepackage{listings}

%Configurando multiple files
\usepackage{filecontents}

%Configurando quotes
\usepackage{csquotes}

\newcommand{\myStyle}{
\lstset{
    language=Octave,                            % the language of the code
    basicstyle=\ttfamily\small,               % the size of the fonts that are used for the code
    keywordstyle=\color{darkpurple}\bfseries, %
    stringstyle=\color{darkblue},             %
    commentstyle=\color{darkgreen},           %
    morecomment=[s][\color{blue}]{/**}{*/},   %
    extendedchars=true,                       %
    showtabs=false,                           % show tabs within strings adding particular underscores
    showspaces=false,                         % show spaces adding particular underscores
    showstringspaces=false,                   % underline spaces within strings
    numbers=left,                             % where to put the line-numbers
    numberstyle=\tiny\color{gray},            % the style that is used for the line-numbers
    stepnumber=1,                             % the step between two line-numbers. If it's 1, each line will be numbered
    numbersep=5pt,                            % how far the line-numbers are from the code
    frame=single,                             % adds a frame around the code
    rulecolor=\color{black},                  % if not set, the frame-color may be changed on line-breaks within not-black text
    breaklines=true,                          % sets automatic line breaking
    backgroundcolor=\color{white},            % choose the background color
    breakatwhitespace=true,                   % sets if automatic breaks should only happen at whitespace
    breakautoindent=false,                    %
    captionpos=b,                             % sets the caption-position to bottom
    xleftmargin=0pt,                          %
    tabsize=2,                                % sets default tabsize to 2 spaces
}}

%\titleformat{<command>}[<shape>]{<format>}{<label>}{<sep>}{<before-code>}[<after-code>]
\titleformat
{\section} %comand
[block]  %shape
{\normalfont\LARGE} %format
{\thesection. } %label
{0mm} %sep
{} %before-code
[{\titlerule[0.1mm]}] %after-code

\titlespacing*{\section}{0mm}{0mm}{15mm}

\titleformat
{\subsection} %comand
[block]  %shape
{\normalfont\Large} %format
{\thesubsection. } %label
{0mm} %sep
{} %before-code
[] %after-code

\titlespacing*{\subsection}{0mm}{5mm}{2.5mm}


\begin{document}
    \begin{titlepage}
        \begin{center}
            \rule{450pt}{0.5pt}\\[4mm]
            {\Huge EM404 - Dinâmica}\\
            \rule{450pt}{0.5pt}\\[2mm]
            {\Large Resumo Teórico}\\[200mm]
            \today\\
            \rule{250pt}{0.5pt}\\
            {\large Guilherme Nunes Trofino}\\
            {\large 217276}\\
        \end{center}
    \end{titlepage}
\newpage

    \tableofcontents
\newpage

    \section{Coordenadas}
        \subsection{Coordenadas Retangulares}
            \paragraph{Definição}Decomposição vetorial das vairáveis envolvidas no movimento da partícula ao longo do plano cartesiano, abssissas $x$ e ordenadas $y$, sendo descritos pelas seguintes equações:
                \begin{align*}
                    \vec{r} &= x\vec{i} + y\vec{j}\\
                    \vec{v} = \dot{\vec{r}} &= \dot{x}\vec{i} + \dot{y}\vec{j}\\
                    \vec{a} = \dot{\vec{v}} = \ddot{\vec{r}} &= \ddot{x}\vec{i} + \ddot{y}\vec{j}\\
                \end{align*}

        \subsection{Coordenadas Normal e Tangencial}
            \paragraph{Definição}Decomposição vetorial das variáveis envolvidas no movimento da partícula ao longo da tangente, $t$, e ao longo da normal, $n$, da trajetória percorrida, onde os sentidos são descritos a seguir:
                \begin{enumerate}[noitemsep]
                    \item \textbf{Direção Normal:} Positivo em direção ao centro de curvatura;
                        \begin{enumerate}[noitemsep]
                            \item $\vec{e_{n}}$, \texttt{Vetor Unitário Normal}: Cuja derivada será dada por:
                                \[
                                    \boxed{
                                        \diff{\vec{e_{t}}}{\beta} = \vec{e_{n}}
                                    }
                                    \hspace{20mm}
                                    \boxed{
                                        \dot{\vec{e_{t}}} = \dot{\beta}\vec{e_{n}}
                                    }
                                \]
                        \end{enumerate}
                    \item \textbf{Direção Tangencial:} Positivo em direção ao movimento;
                        \begin{enumerate}[noitemsep]
                            \item $\vec{e_{t}}$: \texttt{Vetor Unitário Tangencial};
                        \end{enumerate}
                \end{enumerate}
            Assim pode-se definir as seguintes variáveis:
                \begin{enumerate}[noitemsep]
                    \item \textbf{Velocidade:} Sempre será tangente a trajetória, sendo descrita pelas seguintes equações:
                        \begin{equation}
                            \boxed{
                                v = \sqrt{v_{n}^{2} + v_{t}^{2}} =
                                \begin{cases}
                                    v_{n} = 0\\
                                    v_{t} = \rho \diff{\beta}{t} = \rho \dot{\beta}\\
                                \end{cases}
                            }
                            \hspace{20mm}
                            \boxed{
                                \vec{v} = \diff{s}{t} = \rho \diff{\beta}{t} \vec{e_{t}} = \rho \dot{\beta} \vec{e_{t}}
                            }
                        \end{equation}
                    Onde:
                        \begin{enumerate}[noitemsep]
                            \item $\beta$: \texttt{Ângulo Percorrido na Curvatura};
                            \item $\rho$: \texttt{Raio de Curvatura da Trajetória};
                        \end{enumerate}

                    \item \textbf{Aceleração:} Sempre terá componente normal, modificando a direção da trajetória da partícula, e componente tangencial, modificando o módulo da velocidade, sendo descritas pelas seguintes equações:
                        \begin{equation}
                            \boxed{
                                a = \sqrt{a_{n}^{2} + a_{t}^{2}} =
                                \begin{cases}
                                    a_{n} = \frac{v^{2}}{\rho} = \rho \dot{\beta}^{2}\\
                                    a_{t} = \dot{v} = \ddot{s}\\
                                \end{cases}
                            }
                            \hspace{20mm}
                            \boxed{
                                \vec{a} = \diff{v}{t} = \diff{v\dot{\vec{e_{t}}}}{t} = v \dot{\vec{e_{t}}} + \dot{v}\vec{e_{t}} = \frac{v^{2}}{\rho}\vec{e_{n}} + \dot{v}\vec{e_{t}}
                            }
                        \end{equation}
                \end{enumerate}

        \subsection{Coordenadas Polares}
            \paragraph{Definição}Decomposição vetorial das variáveis envolvidas no movimento da partícula a uma distância radial $r$, medida a partir de um ponto fixo $O$, e por um ângulo $\theta$ da linha radial, onde os sentidos são descritos a seguir:
                \begin{enumerate}[noitemsep]
                    \item \textbf{Direção Angular:} Positivo em direção antihorária;
                        \begin{enumerate}
                            \item $\vec{e_{\theta}}$, \texttt{Vetor Unitário Angular}: Cuja derivada será dada por:
                            \[
                                \boxed{
                                    \diff{\vec{e_{r}}}{\theta} = \vec{e_{\theta}}
                                }
                                \hspace{20mm}
                                \boxed{
                                    \dot{\vec{e_{r}}} = \dot{\theta} \vec{e_{\theta}}
                                }
                            \]
                        \end{enumerate}
                    \item \textbf{Direção Radial:} Positivo do ponto de referência até o ponto referenciado;
                        \begin{enumerate}
                            \item $\vec{e_{r}}$, \texttt{Vetor Unitário Radial}: Cuja derivada será dada por:
                            \[
                                \boxed{
                                    \diff{\vec{e_{\theta}}}{\theta} = - \vec{e_{r}}
                                }
                                \hspace{20mm}
                                \boxed{
                                    \dot{\vec{e_{\theta}}} = -\dot{\theta} \vec{e_{r}}
                                }
                            \]
                        \end{enumerate}
                \end{enumerate}
            Assim pode-se definir as seguintes variáveis:
                \begin{enumerate}[noitemsep]
                    \item \textbf{Velocidade:} Diferenciando o espaço com $\vec{r} = r \vec{e_{r}}$:
                        \begin{equation}
                            \boxed{
                                v = \sqrt{v_{r}^{2} + v_{\theta}^{2}} =
                                \begin{cases}
                                    v_{r} = \dot{r}\\
                                    v_{\theta} = r\dot{\theta}\\
                                \end{cases}
                            }
                            \hspace{20mm}
                            \boxed{
                                \vec{v} = \diff{s}{t} = \diff{r \vec{e_{r}}}{t} = \dot{r}\vec{e_{r}} + r\dot{\theta}\vec{e_{\theta}}
                            }
                        \end{equation}

                    \item \textbf{Aceleração:} Diferenciando a velocidade:
                        \begin{equation}
                            \boxed{
                                a = \sqrt{a_{r}^{2} + a_{\theta}^{2}} =
                                \begin{cases}
                                    a_{r} = \ddot{r} - r\dot{\theta}^{2}\\
                                    a_{\theta} = r\ddot{\theta} + 2\dot{r}\dot{\theta}\\
                                \end{cases}
                            }
                            \hspace{20mm}
                            \boxed{
                                \vec{a} = \diff{v}{t} = \diff{v\dot{\vec{e_{t}}}}{t} = v \dot{\vec{e_{t}}} + \dot{v}\vec{e_{t}} = \frac{v^{2}}{\rho}\vec{e_{n}} + \dot{v}\vec{e_{t}}
                            }
                        \end{equation}
                \end{enumerate}

        \subsection{Coordenadas Cilíndricas}
            \paragraph{Definição}

        \subsection{Coordenadas Esféricas}
            \paragraph{Definição}
\newpage

    \section{Cinemática}
        \subsection{Movimento Retilíneo}
            \paragraph{Definição}Caso geral do movimento de uma partícula ao longo de uma linha reta, onde sua posição em qualquer instante $t$ poderá ser determinada por sua distância $s$ a partir de um referêncial fixo $O$ sobre a linha, sendo descrito pelas seguintes equações:
                \begin{enumerate}[noitemsep]
                    \item \textbf{Coordenadas Retangulares:}
                        \begin{enumerate}[noitemsep]
                            \item \texttt{Velocidade Instantânea:}
                                \begin{equation}
                                    \boxed{
                                        v = \lim_{\Delta t \to 0} \frac{\Delta s}{\Delta t} = \diff{s}{t} =\dot{s}
                                    }
                                \end{equation}
                            \item \texttt{Aceleração Instantânea:}
                                \begin{equation}
                                    \boxed{
                                        a = \lim_{\Delta t \to 0} \frac{\Delta v}{\Delta t} = \diff{v}{t} =\dot{v} = \ddot{s}
                                    }
                                \end{equation}
                            Onde:
                                \begin{enumerate}[noitemsep]
                                    \item \textbf{Aceleração Constante:}
                                        \[
                                            \boxed{v = v_{0} + a \cdot t}
                                            \hspace{10mm}
                                            \boxed{v^{2} = v_{0}^{2} + 2 a (s - s_{0})}
                                            \hspace{10mm}
                                            \boxed{s = s_{0} + v_{0} \cdot t + \frac{1}{2} a t^{2}}
                                        \]
                                    \item \textbf{Aceleração Função do Tempo:}
                                        \[
                                            \boxed{v = v_{0} + \int_{0}^{t} f(t) dt}
                                            \hspace{10mm}
                                            \boxed{s = s_{0} + \int_{0}^{t} v dt}
                                        \]
                                    \item \textbf{Aceleração Função da Velocidade:}
                                        \[
                                            \boxed{t = \int_{v_{0}}^{v} \frac{dv}{f(v)}}
                                            \hspace{10mm}
                                            \boxed{s = s_{0} + \int_{v_{0}}^{v} \frac{v dv}{f(v)}}
                                        \]
                                \end{enumerate}
                        \end{enumerate}
                \end{enumerate}
            Onde:
                \begin{equation}
                    \boxed{
                        \dot{s} d\dot{s} = \ddot{s} ds
                    }
                \end{equation}

        \subsection{Movimento Curvilíneo Plano}
            \paragraph{Definição}Caso geral do movimento de uma partícula ao longo de uma curva em um plano com suas variáveis decompostas nas coordenadas cartesianas, sendo descrito pelas seguintes equações:
                \begin{enumerate}[noitemsep]
                    \item \textbf{Coordenadas Retangulares:} Comumente aplicada quando se tratam de lançamentos oblíquos:
                        \[
                            \boxed{x = x_{0} + v_{0_{x}} \cdot t}
                        \]
                        \[
                            \boxed{v_{y} = v_{0_{y}} - g \cdot t}
                            \hspace{10mm}
                            \boxed{y = y_{0} + v_{0_{y}} \cdot t - \frac{1}{2}g t^{2}}
                            \hspace{10mm}
                            \boxed{v_{y}^{2} = v_{0_{y}}^{2} - 2g(y - y_{0})}
                        \]
                \end{enumerate}

        \subsection{Movimento Circular}
            \paragraph{Definição}Caso especial do movimento curvilíneo plano em que o raio de curvatura $\rho$ será constante durante toda trajetória, um círculo de raio $r$, e seu ângulo $\beta$ será dado pelo ângulo $\theta$ a partir de um referência conveniente, sendo descritos pelas seguintes equações:
                \begin{enumerate}
                    \item \textbf{Coordenadas Normal e Tangencial:}
                        \begin{equation}
                            \boxed{
                                v = r \dot{\theta}
                            }
                            \hspace{20mm}
                            \boxed{
                                a = 
                                \begin{cases}
                                    a_{n} = v \dot{\theta}\\
                                    a_{t} = r \ddot{\theta}\\
                                \end{cases}
                            }
                        \end{equation}
                    \item \textbf{Coordenadas Polares:}
                        \begin{equation}
                            \boxed{
                                v = 
                                \begin{cases}
                                    v_{r} = \dot{r}\\
                                    v_{\theta} = r \dot{\theta}\\
                                \end{cases}
                            }
                            \hspace{20mm}
                            \boxed{
                                a = 
                                \begin{cases}
                                    a_{r} = - r\dot{\theta}^{2}\\
                                    a_{\theta} = r \ddot{\theta}\\
                                \end{cases}
                            }
                        \end{equation}
                \end{enumerate}

        \subsection{Movimento Curvilíneo Espacial}
            \paragraph{Definição}Caso geral de movimento tridimensional de uma partícula ao longo de uma curva espacial, sendo descrito pelas seguintes equações:
                \begin{enumerate}[noitemsep]
                    \item \textbf{Coordenadas Retangulares:}
                        \begin{align*}
                            \vec{R} &= x\vec{i} + y\vec{j} + z\vec{k}\\
                            \vec{v} = \dot{\vec{R}} &= \dot{x}\vec{i} + \dot{y}\vec{j} + \dot{z}\vec{k}\\
                            \vec{a} = \dot{\vec{v}} = \ddot{\vec{R}} &= \ddot{x}\vec{i} + \ddot{y}\vec{j} + \ddot{z}\vec{k}\\
                        \end{align*}
                    \item \textbf{Coordenadas Cilíndricas:}
                        \begin{align*}
                            \vec{R} &= r\vec{e_{r}} + z\vec{k}\\
                            \vec{v} = \dot{\vec{R}} &= \dot{r}\vec{e_{r}} + r\dot{\theta}\vec{e_{\theta}} + \dot{z}\vec{k}\\
                            \vec{a} = \dot{\vec{v}} = \ddot{\vec{R}} &= (\ddot{r} - r\dot{\theta}^{2})\vec{e_{r}} + (r\ddot{\theta} + 2\dot{r}\dot{\theta})\vec{j} + \ddot{z}\vec{k}\\
                        \end{align*}
                    \item \textbf{Coordenadas Esféricas:}
                        \[
                            \vec{v} = v_{R}\vec{e_{R}} + v_{\theta}\vec{e_{\theta}} + v_{\phi}\vec{e_{\phi}}
                            \hspace{2.5mm}
                            \begin{cases}
                                v_{R} = \dot{R}\\
                                v_{\theta} = R\dot{\theta}\cos(\phi)\\
                                v_{\phi} = R\dot{\phi}\\
                            \end{cases}
                            \hspace{5mm}
                            \vec{a} = d_{R}\vec{e_{R}} + a_{\theta}\vec{e_{\theta}} + a_{\phi}\vec{e_{\phi}}
                            \hspace{2.5mm}
                            \begin{cases}
                                a_{R} = \ddot{R} - R\dot{\phi}^{2} - R\dot{\theta}^{2}\cos^{2}(\phi)\\
                                a_{\theta} = \frac{\cos(\phi)}{R} \frac{d(R^{2} \dot{\theta})}{dt} - 2R\dot{\theta}\dot{\phi}\sin(\phi)\\
                                a_{\phi} = \frac{1}{R} \frac{dR^{2}\dot{\phi}}{dt} + R\dot{\theta}^{2}\sin(\phi)\cos(\phi)\\
                            \end{cases}
                        \]
                \end{enumerate}

        \subsection{Movimento Relativo}
            \paragraph{Definição}Caso geral do movimento medido a partir da combinação entre referências movéis, $B$, e referências absolutas, $A$, de uma partícula no espaço, sendo descrito pelas seguintes equações: 
                \begin{equation}
                    \boxed{
                        \vec{r_{A}} = \vec{r_{B}} + \vec{r_{AB}}
                    }
                    \hspace{10mm}
                    \boxed{
                        \vec{v_{A}} = \vec{v_{B}} + \vec{v_{AB}}
                    }
                    \hspace{10mm}
                    \boxed{
                        \vec{a_{A}} = \vec{a_{B}} + \vec{a_{AB}}
                    }
                \end{equation}
            O \textbf{Sistema Inercial} ocorre quando a velocidade do sistema for constante.

        \subsection{Movimento Restritro de Partículas Conectadas}
            \paragraph{Definição}Caso geral do movimento de partículas cujas variáveis são interdependentes que podem apresentar diferentes graus de restrição, sendo descrito pelas seguintes equações:
                \begin{enumerate}[noitemsep]
                    \item \textbf{Um Grau de Liberdade:} Cenário em que a restrição em um dispositivo impede o movimento do sistema.
                    \item \textbf{Dois Graus de Liberdade:} Cenário em que a restrição em dois dispositivo impede o movimento do sistema.
                \end{enumerate}
            Em casos envolvendo cordas será necessário obter um \textbf{Comprimento Total} do cabo de ligação, assumindo que seja inextensivo, para que, com derivações sucessivas, obtenha-se equações de velocidades e acelerações dos corpos.
\newpage

    \section{Cinética Partículas}
        \paragraph{Definição}

        \subsection{2\textsuperscript{a} Lei de Newton}
            \paragraph{Definição}Quando uma massa sofrer a ação de uma força esta experimentará uma aceleração proporcional, sendo descrita pela seguinte equação:
                \begin{equation}
                    \boxed{
                        \vec{F} = m\vec{a}
                    }
                \end{equation}
                \begin{enumerate}[noitemsep]
                    \item \textbf{Movimento Geral:} Casos gerais:
                        \begin{equation}
                            \boxed{
                                \vec{a} = a_{x}\vec{i} + a_{y}\vec{j} + a_{z}\vec{k}
                                \hspace{5mm}
                                a = \sqrt{a_{x}^2 + a_{y}^2 + a_{z}^2}
                            }
                        \end{equation}
                        \begin{equation}
                            \boxed{
                                \vec{F} = F_{x}\vec{i} + F_{y}\vec{j} + F_{z}\vec{k}
                                \hspace{5mm}
                                F = \sqrt{F_{x}^2 + F_{y}^2 + F_{z}^2}
                            }
                        \end{equation}
                \end{enumerate}

        \subsection{Trabalho e Energia}
            \paragraph{Definição}Quando uma força atua sobre uma partícula ao longo de um deslocamento está realizará trabalho, sendo descrito pela seguinte equação:
                \begin{equation}
                    \boxed{dU = \vec{F} \cdot d\vec{r}}
                \end{equation}
            Onde:
                \begin{enumerate}[noitemsep]
                    \item \textbf{Força Elástica:} Considerando esta força, obtém-se a seguinte equação:
                        \begin{enumerate}[noitemsep]
                            \item \texttt{Trabalho:}
                                \[
                                    \boxed{
                                        U_{1-2} = \int_{1}^{2} \vec{F} \cdot d\vec{r}
                                                = \int_{1}^{2} (-kx\vec{i})\cdot dx \vec{i}
                                                = \frac{1}{2} k (x_{1}^2 - x_{2}^2)
                                    }
                                \]
                            \item \texttt{Energia:}
                                \begin{equation}
                                    \boxed{
                                        V_{e} = \frac{1}{2} k x^{2}
                                    }
                                \end{equation}
                        \end{enumerate}
                    \item \textbf{Força Gravitacional:} Considerando esta força, obtém-se a seguinte equação:
                        \begin{enumerate}[noitemsep]
                            \item \texttt{Trabalho:}
                                \[
                                    \boxed{
                                        U_{1-2} = \int_{1}^{2} \vec{F} \cdot d\vec{r}
                                                = \int_{1}^{2} (-mg\vec{j})\cdot dy \vec{j}
                                                = mg (y_{1} - y_{2})
                                    }
                                \]
                            \item \texttt{Energia:}
                                \begin{equation}
                                    \boxed{
                                        V_{g} = mgh
                                    }
                                \end{equation}
                        \end{enumerate}
                    \item \textbf{Força Cinética:} Considerando esta força, obtém-se a seguinte equação:
                        \begin{enumerate}[noitemsep]
                            \item \texttt{Trabalho:}
                                \[
                                    \boxed{
                                        U_{1-2} = \int_{1}^{2} \vec{F} \cdot d\vec{r}
                                                = \int_{1}^{2} mvdv
                                                = \frac{1}{2} m (v_{2}^{2} - v_{1}^{2})
                                    }
                                \]
                            \item \texttt{Energia:}
                                \begin{equation}
                                    \boxed{
                                        T = \frac{1}{2} m v^{2}
                                    }
                                \end{equation}
                        \end{enumerate}
                \end{enumerate}
            Assim, pelo \textbf{Princípio do Trabalho e Energia Cinética}, o trabalho total realizado por todas as forças agindo sobre uma partícula enquanto se move do ponto 1 para o ponto 2 será igual a variação da energia cinética, formalmente descrita pela seguinte equação:
                \begin{equation}
                    \boxed{
                        T_{1} + U_{1-2} = T_{2}
                    }
                \end{equation}

        \subsection{Impulso e Quantidade de Movimento Linear}
            \paragraph{Definição}Quando uma partícula se desloca com velocidade constante haverá uma quantidade de movimento, sendo descrito pela seguinte equação:
                \begin{equation}
                    \boxed{
                        \vec{F} = m \vec{v} = \diff{(mv)}{t} = \dot{\vec{G}}
                    }
                \end{equation}
            Assim, pelo \textbf{Princípio do Impulso e Quantidade de Movimento}, a quantidade de movimento linear total de uma partícula enquanto se move do ponto 1 para o ponto 2 será igual ao impulso aplicado, formalmente descrito pela seguinte equação:
                \begin{equation}
                    \boxed{
                        \vec{G_{1}} + \int_{t_{1}}^{t_{2}} \vec{F} dt = \vec{G_{2}}
                    }
                \end{equation}

        \subsection{Impulso Angular e Quantidade de Movimento Angular}
            \paragraph{Definição}Quando uma partícula se desloca entorno de uma origem $O$, sendo descrito pela seguinte equação:
                \begin{equation}
                    \boxed{
                        \vec{H_{O}} = \vec{r} \times m\vec{v} = m \cdot
                        \begin{vmatrix}
                            \vec{i} & \vec{j} & \vec{k}\\
                            x       & y       & z\\
                            v_{x}   & v_{y}   & v_{z}\\
                        \end{vmatrix}
                    }
                \end{equation}
            Assim, pelo \textbf{Princípio do Impulso e Quantidade de Movimento Angular}, a quantidade de movimento linear angular total de uma partícula enquanto se move do ponto 1 para o ponto 2 será igual ao momento angular aplicado, formalmente descrito pela seguinte equação:
                \begin{equation}
                    \boxed{
                        \vec{H_{1}} + \int_{t_{1}}^{t_{2}} \vec{M} dt = \vec{H_{2}}
                    }
                \end{equation}

        \subsection{Colisões}
            \paragraph{Definição}Quando dois ou mais corpos se encontram este dividiram suas quantidades de movimento proporcionalmente a suas massas e as suas velocidades, sendo descrito pelas seguintes equações:
                \begin{enumerate}[noitemsep]
                    \item \textbf{Impacto Central Direto:}
                        \begin{equation}
                            \boxed{
                                m_{A}v_{A1} + m_{B}v_{B1} = m_{A}v_{A2} + m_{B}v_{B2}
                            }
                        \end{equation}
                        \begin{equation}
                            \boxed{
                                e = \frac{v_{B2} - v_{A2}}{v_{A1} - v_{B1}}
                                \hspace{2.5mm}
                                \begin{cases}
                                    e = 1, & \text{Impacto Elástico};\\
                                    e = 0, & \text{Impacto Inelástico};\\
                                \end{cases}
                            }
                        \end{equation}
                    \item \textbf{Impacto Central Oblíquo}
                        \begin{equation}
                            \boxed{
                                m_{A}v_{A1_{n}} + m_{B}v_{B1_{n}} = m_{A}v_{A2_{n}} + m_{B}v_{B2_{n}}
                            }
                        \end{equation}
                        \begin{equation}
                            \boxed{m_{A}v_{A1_{t}} = m_{A}v_{A2_{t}}}
                            \hspace{10mm}
                            \boxed{m_{B}v_{B1_{t}} = m_{B}v_{B2_{t}}}
                        \end{equation}
                        \begin{equation}
                            \boxed{
                                e = \frac{v_{B2_{n}} - v_{A2_{n}}}{v_{A1_{n}} - v_{B1_{n}}}
                                \hspace{2.5mm}
                                \begin{cases}
                                    e = 1, & \text{Impacto Elástico};\\
                                    e = 0, & \text{Impacto Inelástico};\\
                                \end{cases}
                            }
                        \end{equation}
                \end{enumerate}
\newpage

    \section{Cinética de Sistemas de Partículas}
        \paragraph{Definição}Análise dos princípios dinâmicos de movimento para problemas compostos por sistemas de partículas, sendo estes corpos rígidos ou não modelados pelas equações discutidas a seguir.

        \subsection{2\textsuperscript{a} Lei de Newton}
            \paragraph{Definição}Modelagem de um corpo através de $n$ partículas, de massa $m_{i}$, limitadas por uma superfície fechada no espaço, sofrendo ação tanto de forças externas, $F_{i}$, quanto de forças internas, $f_{i}$, a partir de um sistema de coordenadas inercial, sendo descrita pela seguinte equação:
                \begin{equation}
                    m\vec{\overline{r}} = \Sigma m_{i}\vec{\ddot{r_{i}}}
                \end{equation}
            Onde:
                \begin{enumerate}[rightmargin = \leftmargin, noitemsep]
                    \item $\vec{\overline{r}}$, \textbf{Vetor Direção CM:} Medido a partir do referencial inercial até o centro de massa do corpo;
                    \item $\vec{\ddot{r_{i}}}$, \textbf{Vetor Direção Partícula:} Medido a partir do referencial inercial até a partícula analisada;
                \end{enumerate}
            Logo, considerando as forças que agem sobre cada partícula obtêm-se o seguinte resultado:
                \begin{equation}
                    \Sigma F_{i} + \Sigma f_{i} = \Sigma m_{i}\vec{\ddot{r_{i}}}
                \end{equation}
            Entretanto, como todas as forças internas ocorrem aos pares de mesma direção e módulo, porém com sentidos opostos estas se cancelam. Assim, considera-se, pelo \textbf{Princípio de Movimento do Centro de Massa}, a força e aceleração resultantes podem ser analisadas com relação ao centro de massa do corpo, obtendo as seguintes equações:
                \begin{equation}
                    \boxed{
                        \Sigma\vec{F_{x}} = m \vec{\overline{a}_{x}}
                    }
                    \hspace{10mm}
                    \boxed{
                        \Sigma\vec{F_{y}} = m \vec{\overline{a}_{y}}
                    }
                    \hspace{10mm}
                    \boxed{
                        \Sigma\vec{F_{z}} = m \vec{\overline{a}_{z}}
                    }
                \end{equation}

        \subsection{Trabalho e Energia}
            \paragraph{Definição}Modelagem de um corpo através de $n$ partículas, de massa $m_{i}$, limitadas por uma superfície fechada no espaço com $(U_{1-2})_{i} = \Delta T_{i}$, trabalho e energia realizado tanto de forças externas, $F_{i}$, quanto de forças internas, $f_{i}$, a partir de um sistema de coordenadas inercial, sendo descrita pela seguinte equação:
                \begin{equation}
                    \Sigma (U_{1-2})_{i} = \Sigma \Delta T_{i}
                \end{equation}
            Logo, como para a 2\textsuperscript{a} Lei a energia e trabalho resultantes podem ser analisadas com relação ao centro de massa do corpo, obtendo a seguinte equação:
                \begin{equation}
                    T_{1} + U_{1-2} = T_{2}
                \end{equation}
            Entretanto, deve-se considerar as demais formas de energia potencial que podem estar presentes no sistema, obtendo a seguinte equação:
                \begin{equation}
                    \boxed{
                        T_{1} + V_{1g} + V_{1e} + U_{1-2} = T_{2} + V_{2g} + V_{2e}
                    }
                \end{equation}
            Onde:
                \begin{enumerate}[rightmargin = \leftmargin]
                    \item $T_{i}$, \textbf{Energia Cinética:} Analiza-se o sistemas coordenadas, considerando as componentes relativas e absolutas do sistema através da seguinte expressão para a energia cinética:
                        \begin{equation}
                            \boxed{
                                T = \frac{1}{2} m \vec{\overline{v}}^{2} + \Sigma \frac{1}{2} m_{i} |\dot{\rho}_{i}|^{2}
                            }
                        \end{equation}
                    Onde:
                        \begin{enumerate}[noitemsep]
                            \item $\vec{v_{i}}$, \texttt{Velocidade Total da Partícula}: Considera-se o movimento relativo, implicando que a velocidade da partícula será:
                                \begin{equation}
                                    \vec{v_{i}} = \vec{\overline{v}} + \dot{\rho_{i}}
                                \end{equation}
                            \item $\vec{\overline{v}}$, \texttt{Velocidade do Centro de Massa};
                            \item $\dot{\rho_{i}}$, \texttt{Velocidade Relativa ao Centro de Massa};
                        \end{enumerate}

                    \item $V_{i}$, \textbf{Energia Potencial}: Não modificação nas equações empregadas em tópicos anteriores;
                        \begin{enumerate}[noitemsep]
                            \item $V_{ig}$, \texttt{Gravitacional};
                            \item $V_{ie}$, \texttt{Elástica};
                        \end{enumerate}
                \end{enumerate}

            \paragraph{Lei de Conservação de Energia Dinâmica}Caso nenhum trabalho seja realizado sobre um sistema conservativo, então nenhuma energia será dissipada, implicando:
                \begin{equation}
                    \boxed{
                        T_{1} + V_{1} = T_{2} + V_{2}
                    }
                \end{equation}

        \subsection{Impulso e Quantidade de Movimento Linear}
            \paragraph{Definição}Modelagem de um corpo através de $n$ partículas, de massa $m_{i}$, limitadas por uma superfície fechada no espaço com $\vec{G}_{i} = m_{i} \vec{v}_{i}$, quantidade de movimento linear sobre um sistema de coordenadas inercial, sendo descrita pela seguinte equação:
                \begin{equation}
                    \boxed{
                        \vec{G} = m \vec{\overline{v}}
                    }
                \end{equation}
            Considerado a partir do centro de massa do sistema, extraindo a derivada no tempo da quantidade de movimento obtêm-se o impulso do sistema pela seguinte equação:
                \begin{equation}
                    \boxed{
                        \Sigma\vec{F} = \Delta\dot{\vec{G}}
                    }
                \end{equation}

            \paragraph{Princípio da Conservação da Quantidade de Movimento Linear}Caso as forças externas resultantes sobre um sistema de massas seja nula, então nenhuma quantidade de movimento linear será dissipada, implicando:
                \begin{equation}
                    \boxed{
                        \vec{G}_{1} = \vec{G}_{2}
                    }
                \end{equation}

        \subsection{Impulso e Quantidade de Movimento Angular}
            \paragraph{Definição}Modelagem de um corpo através de $n$ partículas, de massa $m_{i}$, limitadas por uma superfície fechada no espaço com momento causado pela quantidade de movimento linear sobre um sistema de coordenadas inercial avaliado em diferentes pontos, sendo descritos pelas seguintes equações:
                \begin{enumerate}[rightmargin = \leftmargin]
                    \item \textbf{Origem do Sistema Inercial:} Somatório dos momentos das quantidades de movimento linear em torno de $O$, descritos pelas seguintes equações:
                        \begin{equation}
                            \boxed{
                                \vec{H}_{O} = \Sigma \vec{r}_{i}\times m_{i}\vec{v}_{i}
                            }
                            \hspace{10mm}
                            \boxed{
                                \Sigma\vec{M}_{O} = 
                                \Sigma\vec{r}_{i}\times m_{i}\vec{\dot{v}}_{i}
                            }
                        \end{equation}
                    Note que um dos termos resultantes da derivada será nula por apresentar vetores paralelos.

                    \item \textbf{Centro de Massa:} Somatório dos momentos das quantidades de movimento linear em torno do centro de massa $G$, descritos pelas seguintes equações:
                        \begin{equation}
                            \boxed{
                                \vec{H}_{G} = \Sigma  \rho_{i}\times m_{i}\vec{\dot{r}}_{i}
                            }
                            \hspace{10mm}
                            \boxed{
                                \Sigma \vec{M}_{G} = 
                                \Sigma \rho_{i}\times\vec{F}_{i}
                            }
                        \end{equation}

                    \item \textbf{Ponto Arbitrário:} Somatório dos momentos das quantidades de movimento linear em torno de um ponto arbitrário $P$, obtido pelo teorema dos eixos paralelos, descritos pelas seguintes equações:
                        \begin{equation}
                            \boxed{
                                \vec{H}_{P} = 
                                \vec{H}_{G} + \overline{\rho}\times m_{i}\vec{\overline{v}}_{i}
                            }
                            \hspace{10mm}
                            \boxed{
                                \Sigma\vec{M}_{G} + \overline{\rho}\times\Sigma\vec{F} = 
                                \Sigma\vec{M}_{P} = 
                                \vec{\dot{H}}_{G} + \overline{\rho}\times m \vec{\overline{a}}
                            }
                        \end{equation}
                \end{enumerate}

            \paragraph{Princípio da Conservação da Quantidade de Movimento Angular}Caso o momento resultante de todas as forças de um sistema de massas em torno de um ponto fixo for nulo, então nenhuma quantidade de movimento angular será dissipada, implicando:
                \begin{equation}
                    \boxed{
                        (\vec{H}_{O})_{1} = (\vec{H}_{O})_{2}
                    }
                \end{equation}

        \subsection{Escoamento em Regime Permanente}
            \paragraph{Definição}Quantidade de massa fluidindo a uma taxa constante, isto é, toda materia entradando terá que sair do dispositivo ao longo da fronteira do sistema, podendo possuir diversas entradas e saídas, sendo descrito pela seguinte equação:
                \begin{equation}
                    \boxed{
                        \dot{m} = \rho_{1} A_{1} V_{1}
                        \quad
                        \left[\frac{\text{Kg}}{\text{s}}\right]
                    }
                \end{equation}
            Onde:
                \begin{enumerate}[rightmargin =\leftmargin, noitemsep]
                    \item $\rho_{i}$, \textbf{Massa Específica};
                    \item $A_{i}$, \textbf{Área};
                    \item $V_{i}$, \textbf{Velocidade};
                \end{enumerate}
            A partir desta definição pode-se reescrever as equações de quantidade de movimento linear e momento linear com base nos fluxos de massa ação de forças, tanto exercidas externamente quanto internas do fluído com relação a um ponto $O$, obtendo as seguintes equações:
                \begin{equation}
                    \boxed{
                        \Delta\vec{G} = \Delta m (\vec{v}_{2} - \vec{v}_{1})
                    }
                    \hspace{10mm}
                    \boxed{
                        \Delta\vec{\dot{G}} = 
                        \Sigma \vec{F} = 
                        \dot{m} (\vec{v}_{2} - \vec{v}_{1})
                    }
                    \hspace{10mm}
                    \boxed{
                        \Sigma \vec{M}_{O} = \dot{m} (\vec{d}_{2}\times\vec{v}_{2} - \vec{d}_{1}\times\vec{v}_{1})
                    }
                \end{equation}
            Nesta abordagem, reforça-se a definição de pressão pois estas geram trabalho entorno das áreas em que são aplicadas, expressa pela seguinte equação:
                \begin{equation}
                    \boxed{
                        F = P A
                    }
                \end{equation}

        \subsection{Escoamento}
            \paragraph{Definição}Quantidade de massa fluindo a uma taxa não constante ao longo da fronteira do sistema, podendo possuir diversas entradas e saídas, sendo descrito pela seguinte equação:
                \begin{equation}
                    \boxed{
                        R = \dot{m} (v - v_{0}) = \dot{m} u
                        \quad
                        [\text{N}]
                    }
                \end{equation}
            Onde:
                \begin{enumerate}[rightmargin = \leftmargin, noitemsep]
                    \item $R$, \textbf{Forças de Reação:} Par ação e reação entre os corpos envolvidos;
                    \item $v$, \textbf{Velocidade do Corpo};
                    \item $v_{0}$, \textbf{Velocidade da Massa:} Absorvida ou Expelida pelo corpo;
                    \item $m$, \textbf{Massa do Corpo};
                \end{enumerate}
            Assim, aplicando a 2\textsuperscript{a} Lei de Newton para a equação de movimento de $m$, tanto quando massa é absorvida quanto quando massa é expelida será válida a seguinte equação:
                \begin{equation}
                    \boxed{
                        \Sigma F = m \dot{v} + \dot{m} u
                    }
                    \quad
                    \text{onde: }
                    \begin{cases}
                        \dot{m} u > 0, & \text{Massa \texttt{Aumenta}};\\
                        \dot{m} u < 0, & \text{Massa \texttt{Diminui}};\\
                    \end{cases}
                \end{equation}
\newpage

    \section{Cinemática Plana de Corpos Rígidos}
        \paragraph{Definição}Análise dos princípios dinâmicos de movimento para sistemas gerais compostos por corpos rígidos, inclindo, além dos parâmetros discutidos em cinemática de partículas, o movimento rotacional do corpo.

        \paragraph{Corpos Rígidos}Sistema de partículas cujas distâncias entre as partículas permanecem inalteradas ao longo do movimento.

        \subsection{Movimento Plano}
            \paragraph{Definição}Movimento em que todas as partes do corpo se movem em planos paralelos geralmente define-se o plano de movimento como aquele que contém o centro de massa do corpo, sendo classificados como mostrado a seguir:
                \begin{enumerate}[rightmargin = \leftmargin]
                    \item \textbf{Translação:} Qualquer movimento em que cada linha do corpo permanece paralela à sua posição original em todos os instantes de tempo;
                        \begin{enumerate}[rightmargin = \leftmargin, noitemsep]
                            \item \texttt{Retilínea:} Deslocamento ao longo de retas paralelas;
                            \item \texttt{Curvilínea:} Deslocamento ao longo de curvas congruentes;
                        \end{enumerate}

                    \item \textbf{Rotação:} Qualquer movimento angular em torno de um eixo fixo, onde todas as partículas do corpo se deslocam em trajetórias circulares, girando do mesmo ângulo ao mesmo tempo, obtendo as seguintes equações rotacionais:
                        \begin{equation}
                            \boxed{
                                \omega = \diff{\theta}{t} = \dot{\theta}
                            }
                            \qquad
                            \boxed{
                                \omega d \omega =\alpha d\theta
                            }
                            \qquad
                            \boxed{
                                \alpha = \diff{\theta}{t} = \dot{\omega} = \ddot{\theta}
                            }
                        \end{equation}
                    Quando a aceleração angular for constante pode-se  integrar as equações acima, obtendo as seguintes equações do movimento angular: 
                        \begin{equation}
                            \boxed{
                                \omega = \omega_{0} + \alpha t
                            }
                            \qquad
                            \boxed{
                                \omega^{2} = \omega^{2}_{0} + 2\alpha(\theta - \theta_{0})
                            }
                            \qquad
                            \boxed{
                                \theta = \theta_{0} + \omega_{0} t + \frac{1}{2}\alpha t^{2}
                            }
                        \end{equation}
                    Onde, aplicando coordenadas normal e tangencial têm-se:
                        \begin{enumerate}[rightmargin = \leftmargin]
                            \item \texttt{Velocidade}: Neste movimento a velocidade sempre será tangente a trajetória da partícula, sendo expressa pela seguinte equação:
                                \begin{equation}
                                    \boxed{
                                        \vec{v} = \vec{\dot{r}} = \vec{\omega}\times\vec{r}
                                    }
                                \end{equation}
        
                            \item \texttt{Aceleração Normal}: Neste movimento a aceleração normal modificará a trajetória da velocidade do corpo, sendo expressa pela seguinte equação:
                                \begin{equation}
                                    \boxed{
                                        \vec{a}_{n} = \vec{\omega} \times (\vec{\omega}\times\vec{r})
                                    }
                                \end{equation}
        
                            \item \texttt{Aceleração Tangencial}: Neste movimento a aceleração tangencial modificará a magnitude da velocidade do corpo, sendo expressa pela seguinte equação:
                                \begin{equation}
                                    \boxed{
                                        \vec{a}_{t} = \vec{\alpha} \times \vec{r}
                                        \vec{a}_{n} = \vec{\omega} \times (\vec{\omega}\times\vec{r})
                                    }
                                \end{equation}
                        \end{enumerate}

                    \item \textbf{Geral}: Combinação entre a translação e a rotação simultânea do corpo rígido ao longo de sua trajetória. Cada componente pode ser analisada individualmente para compreender o resultado de sua combinação.
                \end{enumerate}

        \subsection{Movimento Absoluto}
            \paragraph{Definição}Aplicação de relações geométricas para definir o movimento do corpo ao longo de sua trajetória a partir de um referência inercial, obtendo suas derivadas no tempo para obter velocidades e acelerações. Este é um método analítico, gerando equações possivelmente complexas.

        \subsection{Movimento Relativo}
            \paragraph{Definição}Composição do movimento através de um referencial inercial e um referencial não inercial. Quando, fixado um ponto $B$ do corpo como referencial, qualquer outro ponto $A$ analisado com relação a $B$, possuirá apenas rotação visto que ambos transladaram igualmente pela imposição de corpo rígido como ilustrado pela seguinte figura:
                \begin{figure}[H]
                    \centering
                    \includegraphics[height = 4cm]{ima8.png}
                    \caption{Representação do Movimento Relativo}
                \end{figure} \noindent
            Desta forma, considera-se que, quando analisado a partir de um ponto fixo pertencente ao corpo rígido, os demais pontos possuem componentes de movimento causados pela rotação do mesmo, sendo descrita pelas seguintes equações:
                \begin{equation}
                    \boxed{
                        \vec{r}_{A} = \vec{r}_{B} + \vec{r}_{A|B}
                    }
                    \qquad
                    \boxed{
                        \vec{v}_{A} = \vec{v}_{B} + \vec{v}_{A|B}
                    }
                    \qquad
                    \boxed{
                        \vec{a}_{A} = \vec{a}_{B} + \vec{a}_{A|B}
                    }
                \end{equation}
            Onde:
                \begin{enumerate}[rightmargin = \leftmargin, noitemsep]
                    \item $\vec{v}_{A|B}$, \textbf{Velocidade Devido à Rotação:} Quando analisado a partir de $B$ o ponto $A$ terá apenas rotação, portanto haverá uma velocidade tangencial a trajetória:
                        \begin{equation}
                            \boxed{
                                \vec{v}_{A|B} = \omega_{AB}\times\vec{r}_{AB}
                            }
                        \end{equation}

                    \item $\vec{a}_{A|B}$, \textbf{Aceleração Devido à Rotação:} Quando analisado a partir de $B$ o ponto $A$ terá apenas rotação, portanto haverá uma componente normal e outra tangencial a trajetória:
                        \begin{equation}
                            \boxed{
                                \vec{a}_{A|B} = 
                                \underbrace{
                                    \vec{\omega}_{AB}\times(\vec{\omega}_{AB}\times\vec{r}_{AB})
                                }_{\vec{a}_{A|B_{n}}} + 
                                \underbrace{
                                    \alpha_{AB}\times\vec{r}_{AB}
                                }_{\vec{a}_{A|B_{t}}}
                            }
                        \end{equation}
                    Onde:
                        \begin{enumerate}[rightmargin = \leftmargin, noitemsep]
                            \item $\vec{a}_{A|B_{n}}$, \texttt{Componente Normal:} Responsável pela modificação da direção de $\vec{v}_{A|B}$;

                            \item $\vec{a}_{A|B_{t}}$, \texttt{Componente Tangencial:} Responsável pela modificação do módulo de $\vec{v}_{A|B}$;
                        \end{enumerate}
                \end{enumerate}
            Nesta análise recomenda-se realizar as operações vetorialmente, desta maneira evita-se possíveis confusões quando aos produtos vetoriais. Caso esta abordagem seja empregada será necessário conhecer os seguintes resultados vetoriais:
                \begin{table}[H]
                    \centering
                    \begin{tabular}[]{c | c c c}
                        $\times$  & $ \vec{i}$ & $ \vec{j}$ & $ \vec{k}$\\\hline
                        $\vec{i}$ & 0          & $+\vec{k}$ & $-\vec{j}$\\
                        $\vec{j}$ & $-\vec{k}$ & 0          & $+\vec{i}$\\
                        $\vec{k}$ & $+\vec{j}$ & $-\vec{i}$ & 0\\
                    \end{tabular}
                    \caption{Multiplicação Vetores Unitários}
                \end{table}

        \subsection{Centro Instantâneo de Velocidade Nula}
            \paragraph{Definição}Ponto de intersecção entre o plano do movimento e o eixo normal na qual o corpo possui apenas rotação pura. Neste ponto a velocidade translacional será instantaneamente nula, como mostrada nas seguintes figuras:
                \begin{figure}[H]
                    \centering
                    \begin{subfigure}[]{0.3\textwidth}
                        \centering
                        \includegraphics[height = 5cm]{ima1.png}
                        \caption{Centro Instantâneo Interno}
                    \end{subfigure}
                    \begin{subfigure}[]{0.3\textwidth}
                        \centering
                        \includegraphics[height = 5cm]{ima0.png}
                        \caption{Centro Instantâneo Genérico}
                    \end{subfigure}
                    \begin{subfigure}[]{0.3\textwidth}
                        \centering
                        \includegraphics[height = 5cm]{ima2.png}
                        \caption{Centro Instantâneo Externo}
                    \end{subfigure}
                    \caption{Centro Instantâneo de Velocidade Nula}
                \end{figure} \noindent
            Como consequência de sua definição, todos os pontos do corpo rígido terão velocidade translacional perpendiculares a reta que liga este ponto ao centro de velocidade nula. Reversamente este ponto especial poderá ser encontrado com a intersecção das retas perpediculares a duas velocidades, todo corpo possuirá a seguinte velocidade angular:
                \begin{equation}
                    \boxed{
                        \omega = 
                        \frac{v_{A}}{r_{A}} =
                        \frac{v_{B}}{r_{B}}
                    }
                \end{equation}

        \subsection{Movimento Relativo com Rotação de Eixos}
            \paragraph{Definição}Composição do movimento através de sistemas cinemáticos móveis, referentes ao ponto $B$, que possuem rotação com relação a um sistema inercial, fixado no ponto $O$. Considere a seguinte figura como referência desta configuração:
                \begin{figure}[H]
                    \centering
                    \includegraphics[height = 5cm]{ima3.png}
                    \caption{Representação Eixos Rotativos}
                \end{figure} \noindent
            Note que para as equações abaixo serem válidas será necessário que as variáveis do problema sejam definidas com relação ao referêncial apresentado acima, pois do contrário não funcionará. Desta maneira, será necessário utilizar a seguinte configuração:
                \begin{enumerate}[noitemsep]
                    \item $B$, \textbf{Origem do Sistema Rotacional};
                    \item $A$, \textbf{Ponto Qualquer};
                    \item $O$, \textbf{Origem Inercial};
                \end{enumerate}
            Onde:
                \begin{enumerate}[rightmargin = \leftmargin]
                    \item $\vec{r}_{A}$, \textbf{Direção:} Vetorialmente a posição do ponto $A$ será descrita como a composição entre o movimento rotacional e translacional, representado pela seguinte equação:
                        \begin{equation}
                            \boxed{
                                \vec{r}_{A} = 
                                \vec{r}_{B} + 
                                \underbrace{
                                    (x\vec{i} + y\vec{j})
                                }_{\vec{r}_{AB}}
                            }
                        \end{equation}
                    Onde:
                        \begin{enumerate}[rightmargin = \leftmargin]
                            \item \texttt{Derivadas Direcionais:} Vetores unitários $\vec{i}$ e $\vec{j}$ possuem derivadas no tempo, pois possuem rotação, como mostrado nas equações abaixo:
                                \begin{equation}
                                    \boxed{
                                        \dot{\vec{i}} = \omega\times\vec{i} = \omega\vec{j}
                                    }
                                    \qquad
                                    \boxed{
                                        \dot{\vec{j}} = \omega\times\vec{j} = -\omega\vec{i}
                                    }
                                \end{equation}\noindent
                        \end{enumerate}

                    \item $\vec{v}_{A}$, \textbf{Velocidade:} Vetorialmente a velocidade do ponto $A$ será descrita como a composição entre o movimento rotacional,  translacional e relativo, representado pela seguinte equação:
                        \begin{equation}
                            \boxed{
                                \vec{v}_{A} = 
                                \vec{v}_{B} + 
                                \underbrace{
                                    \omega_{AB}\times\vec{r}_{AB}
                                }_{\vec{v}_{AB}} + 
                                \vec{v}_{R}
                            }
                        \end{equation}
                    Onde:
                        \begin{enumerate}[rightmargin = \leftmargin, noitemsep]
                            \item $\vec{v}_{AB}$, \texttt{Velocidade Devido a Rotação:} Velocidade causada pela rotação de $AB$.

                            \item $v_{R}$, \texttt{Velocidade Relativa:} Velocidade de $A$ observada com relação a $B$, ou seja, há deslocamento entre os pontos.
                        \end{enumerate}

                    \item $\vec{a}_{A}$, \textbf{Aceleração:} Vetorialmente a aceleração do ponto $A$ será descrita como a composição entre o movimento rotacional, translacional, relativo e de \textbf{Coriolis}, representado pela seguinte equação:
                        \begin{equation}
                            \boxed{
                                \vec{a}_{A} = 
                                \vec{a}_{B} + 
                                \underbrace{
                                    \omega_{AB}\times(\omega_{AB}\times\vec{r}_{AB})
                                }_{\vec{a}_{AB_{n}}} + 
                                \underbrace{
                                    \dot{\omega}_{AB}\times\vec{r}_{AB}
                                }_{\vec{a}_{AB_{t}}} + 
                                \underbrace{
                                    2\omega_{AB}\times\vec{v}_{R}
                                }_{\vec{a}_{C}} + 
                                \vec{a}_{R}
                            }
                        \end{equation}
                    Onde:
                        \begin{enumerate}[rightmargin = \leftmargin]
                            \item $\vec{a}_{AB_{t}}$, \texttt{Aceleração Devido a Rotação Tangencial:} Causada pela rotação de $AB$, tangencial a trajetória considerada a partir do \textbf{Centro de Velocidade Nula};

                            \item $\vec{a}_{AB_{n}}$, \texttt{Aceleração Devido a Rotação Normal:} Causada pela rotação de $AB$, normal a trajetória considerada a partir do \textbf{Centro de Velocidade Nula};

                            \item $\vec{a}_{C}$, \texttt{Aceleração de Coriolis:} Causada pela rotação percebida apenas por observadores no sistema rotativo;

                            \item $\vec{a}_{R}$, \texttt{Aceleração de Relativa:} Aceleração de $A$ observada com relação a $B$;
                        \end{enumerate}
                \end{enumerate}
\newpage

    \section{Cinética Plana de Corpos Rígidos}
        \paragraph{Definição}Análise das relações entre as forças externas atuando sobre um corpo e seus movimenos correspondentes de rotação e translação, sendo necessárias duas equações de força e uma equação de momento para determinar o estado do movimento plano do corpo rígido.

        \subsection{Momento de Inércia da Massa}
            \paragraph{Definição}Propriedade constante relacionada a geometria do corpo rígido estudado, quantificando a resistência à variação na velocidade de rotação em torno do eixo $z$ a partir do centro de gravidade, como representado pela seguinte figura:
                \begin{figure}[H]
                    \centering
                    \includegraphics[height = 4cm]{ima4.png}
                    \caption{Definição Momento de Inércia}
                \end{figure}\noindent
            Desta maneira o momento de inércia de um corpo, considerado a partir de um ponto qualquer será descrito pela seguinte equação:
                \begin{equation}
                    \boxed{
                        \overline{I} = \int\rho^{2}_{i}dm
                    }
                \end{equation}
            Usualmente estes valores são tabelados para diferente eixos ao longo de algumas formas geométricas comuns, como representadas pelas seguintes imagens:
                \begin{figure}[H]
                    \centering
                    \includegraphics[width = 16cm]{ima9.png}
                \end{figure}
                \begin{figure}[H]
                    \centering
                    \includegraphics[width = 16cm]{ima10.png}
                    \includegraphics[width = 16cm]{ima11.png}
                    \includegraphics[width = 16cm]{ima13.png}
                    \caption{Momentos de Inércia para Corpos Usuais} \label{momentoInercia}
                \end{figure}\noindent
            Há outros corpos, não representados na tabela acima para se economizar espaço do documento. Nesta tabela nota-se que para um eixo $O$ que não passam pelo centro de massa aplica-se o \textbf{Teorema dos Eixos Paralelos} para obter o momento inercial observado sobre este eixo como descrito pela seguinte equação:
                \begin{equation}
                    \boxed{
                        I_{O} = I_{G} + m r^{2}
                    }
                \end{equation}
            Onde $I_{G}$ representa o momento de inercia com relação ao centro de massa do corpo e $r$ representa a distância entre o eixo $O$ e o centro de massa $G$ do corpo avaliado. Caso o corpo rígido seja considerado uma partícula desconsidera-se o momento de seu centro de massa $G$, pois este não apresentaria rotação entorno de si próprio.

        \subsection{Equações Gerais}
            \paragraph{Definição}Nesta abordagem retomam-se os resultados para cinética de sistemas de partículas, aplicando-os, porém, para o um corpo rígido genérico, representado pela seguinte figura:
                \begin{figure}[H]
                    \centering
                    \includegraphics[height = 3cm]{ima5.png}
                    \caption{Diagramas para Cinética Plana de Corpos Rígidos}
                \end{figure}\noindent
            Desta maneira, o \textbf{Diagrama Dinâmico}, à esquerda,  representa as variáveis dinâmicas como forças; enquanto o \textbf{Diagrama Cinemático}, à direita, representa as variáveis cinemáticas como momentos e acelerações. Utilizam-se as seguintes equações para descrição do movimento:
                \begin{equation}
                    \boxed{
                        \Sigma\vec{F} = m\vec{\overline{a}}
                    }
                    \qquad
                    \boxed{
                        \Sigma\vec{M}_{G} = \dot{\vec{H}}_{G}
                    }
                \end{equation}
            Onde:
                \begin{enumerate}[rightmargin = \leftmargin]
                    \item $G$, \textbf{Centro de Gravidade:} Eixo de rotação atravessa o centro de gravidade do corpo rígido, sua quantidade de movimento angular será obtida pela seguinte equação:
                        \begin{equation}
                            \boxed{
                                \Sigma\vec{M}_{G} = \overline{I}_{G}\alpha
                            }
                        \end{equation}

                    \item $P$, \textbf{Ponto Arbitrário:} Eixo de rotação não fixo e não coincidente com o centro de gravidade do corpo rígido, sua quantidade de movimento angular será obtida pela seguinte equação:
                        \begin{equation}
                            \boxed{
                                \Sigma\vec{M}_{P} = 
                                \overline{I}_{G}\alpha + m\overline{a}d
                            }
                        \end{equation}
                    Neste caso, deve-se considerar acelerações $a$ de corpos com relação ao ponto $P$, separados uma distância $d$ de momento. Esta configuração será válida para corpos interligados e movimento relativo.

                    \item $O$, \textbf{Eixo Fixo:} Eixo de rotação fixo e não conincidente com o centro de gravidade do corpo rígido, sua quantidade de movimento angular será obtida pela seguinte equação:
                        \begin{equation}
                            \boxed{
                                \Sigma\vec{M}_{O} = 
                                \underbrace{
                                    (\overline{I}_{G} + m\overline{r}^{2})
                                }_{\overline{I}_{O}}
                                \alpha
                            }
                        \end{equation}
                    Quando o corpo não apresentar um formato usual tabelado seu momento inercial com relação a um eixo fixo poderá ser obtido através do \textbf{Raio de Giração}, uma medida padronizada obtida pela seguinte equação:
                        \begin{equation}
                            \boxed{
                                \overline{I}_{O} = 
                                m k^{2}_{O}
                            }
                        \end{equation}
                        \begin{figure}[H]
                            \centering
                            \includegraphics[height = 3cm]{ima6.png}
                            \caption{Definição Raio de Giração}
                        \end{figure}\noindent
                    Onde:
                        \begin{enumerate}[rightmargin = \leftmargin, noitemsep]
                            \item $k_{O}$, \texttt{Raio de Giração};
                        \end{enumerate}
                \end{enumerate}

        \subsection{Trabalho e Energia Geral}
            \paragraph{Definição}Baseia-se nos resultados obtidos para cinética de partículas para expressar essas grandes em corpos rígidos que apresentam movimento translacional e rotacional. Recomenda-se analisar as variações individuais entre cada componente descrita pela seguinte equação:
                \begin{equation}
                    \boxed{
                        T_{1} + U_{1-2} = T_{2}
                    }
                \end{equation}
            Onde:
                \begin{enumerate}[rightmargin = \leftmargin]
                    \item $U$, \textbf{Impulso:} Causado pela presença de forças realizando trabalho ou momentos sendo aplicados. Quando um momento constante for aplicado ao corpo rígido será necessário multiplicá-lo pela comprimento da trajetória para obter as equações necessárias.\\
                    Caso não hajam forças ou momentos externos sobre o sistema então este termo será nulo, portanto a análise se dará para as variações termo a termo.

                    \item $T$, \textbf{Energia Cinética:} Analisa-se conjuntamento a parcela gerada pela translação e pela rotação, obtendo:
                        \begin{equation}
                            \boxed{
                                T_{i} = 
                                \underbrace{
                                    \frac{1}{2} m\overline{v}^{2}
                                }_{\text{Translação}} + 
                                \underbrace{
                                    \frac{1}{2}I_{G}\omega^{2}
                                }_{\text{Rotação}}
                            }
                        \end{equation}
                    Onde:
                        \begin{enumerate}[rightmargin = \leftmargin, noitemsep]
                            \item $\overline{v}$, \texttt{Velocidade do Centro de Massa};
                            \item $I_{G}$, \texttt{Momento de Inercia em Torno do Centro de Massa};
                            \item $\omega$, \texttt{Velocidade Angular em Torno do Centro de Massa};
                        \end{enumerate}
                    Essa expressão geral será simplificada a depender das situações e dos pontos avaliadas, os principais pontos são descritos a seguir:
                        \begin{enumerate}[rightmargin = \leftmargin]
                            \item \texttt{Translação}: Quando houver apenas translação do corpo rígido avaliado, aplica-se a seguinte equação:
                                \begin{equation}
                                    \boxed{
                                        T_{i} = \frac{1}{2}m\overline{v}^{2}
                                    }
                                \end{equation}
                                
                            \item \texttt{Rotação em Eixo Fixo}: Quando houver apenas rotação do corpo rígido avaliado entorno de um eixo fixo, aplica-se a seguinte equação:
                                \begin{equation}
                                    \boxed{
                                        T_{i} = \frac{1}{2}I_{O}\omega^{2}
                                    }
                                \end{equation}
                            Neste caso nota-se que será necessário aplicar o \textbf{Teorema dos Eixos Paralelos} para obter o momento inércia do corpo rígido entorno deste eixo.
                                
                            \item \texttt{Centro de Velocidade Nula}: Quando a energia cinética do corpo rígido for avaliada entorno do centro de velocidade nula $C$ haverá rotação, aplica-se a seguinte equação:
                                \begin{equation}
                                    \boxed{
                                        T_{i} = 
                                        \frac{1}{2}m\overline{v}^{2} + 
                                        \frac{1}{2}I_{C}\omega^{2}
                                    }
                                \end{equation}
                            Neste caso nota-se que se o corpo rígido gira sem deslizar ao longo de uma trajetória este terá o centro de velocidade nula no ponto tangente entre o corpo e a trajetória, $C$. Todavia o centro de massa neste caso apresentará translação e o teorema dos eixos não precisa ser aplicado.
                        \end{enumerate}

                    \item $V$, \textbf{Energia Potencial:} Analisa-se conjuntamento potenciais geradas pela gravidade e pela presença de molas, obtendo:
                        \begin{equation}
                            \boxed{
                                V_{i} = 
                                m g h +
                                \frac{1}{2} k x^{2} 
                            }
                        \end{equation}
                    Onde:
                        \begin{enumerate}[rightmargin = \leftmargin, noitemsep]
                            \item $h$, \texttt{Variação de Altura};
                            \item $x$, \texttt{Variação de Comprimento};
                            \item $k$, \texttt{Constante Elástica};
                        \end{enumerate}
                \end{enumerate}

        \subsection{Trabalho e Energia Instantâneo}
            \paragraph{Definição}Quando trata-se de um intervalo infinitesimal de movimento, a equação de trabalho-energia será obtida pela derivação das expressões associadas descritas na seção geral como mostrado nas seguintes equações:
                \begin{equation}
                    \boxed{
                        dU' = dT + dV
                    }
                \end{equation}
            Onde:
                \begin{enumerate}[rightmargin = \leftmargin, noitemsep]
                    \item $dT$, \textbf{Energia Cinética:} Analisa-se conjuntamento a parcela gerada pela translação e pela rotação, obtendo:
                        \begin{equation}
                            \boxed{
                                dT = 
                                \Sigma m_{i}\overline{v}_{i}d\overline{v}_{i} +
                                \Sigma I_{i}\omega_{i}d\omega_{i}
                            }
                        \end{equation}
                    Nesta abordagem será necessário substituir os termos diferenciais por variáveis conhecidas no problema, buscando simplificar cada termo a apenas um diferencial que poderá ser eliminado. Denotam-se as seguintes substituições:
                        \begin{enumerate}[rightmargin = \leftmargin, noitemsep]
                            \item $vdv$:
                                \begin{equation}
                                    v dv = 
                                    a_{t} ds = 
                                    (r \alpha)(r d\theta) = 
                                    r^{2} \alpha d\theta
                                \end{equation}
                                \begin{equation}
                                    v dv = 
                                    v r dw
                                \end{equation}

                            \item $\omega d\omega$:
                                \begin{equation}
                                    \omega d\omega = 
                                    \alpha d\theta
                                \end{equation}
                        \end{enumerate}

                    \item $dV$, \textbf{Energia Potencial:} Analisa-se conjuntamento potenciais geradas pela gravidade e pela presença de molas, obtendo:
                        \begin{equation}
                            \boxed{
                                dV = 
                                \Sigma m_{i} g d h_{i} +
                                \Sigma k_{j} x_{j} d x_{j}
                            }
                        \end{equation}
                \end{enumerate}
\newpage

    \section{Cinética Tridimensional}
        \paragraph{Definição}Análise das relações entre as forças externas atuando sobre um corpo e seus movimentos correspondentes de rotação e translação no espaço, sendo necessárias três equações de forças e uma equação de momento para determinar o estado do movimento do corpo rígido.

        \subsection{Movimento Translacional}
            \paragraph{Definição}Quaisquer dois pontos no corpo rígido, tais como $A$ e $B$, se deslocam ao longo de retas paralelas, se o movimento for retilíneo, ou ao longo de linhas congruentes, se o movimento for curvilíneo, como representado pela seguinte figura:
                \begin{figure}[H]
                    \centering
                    \includegraphics[height = 4cm]{ima14.png}
                    \caption{Movimento Translacional}
                \end{figure}\noindent

        \subsection{Movimento Rotacional}
            \paragraph{Definição}Considera-se $O$ como a origem do sistema de coordenadas fixo, onde qualquer ponto no corpo rígido fora do eixo de rotação, como $A$, se desloca em um arco circular em um plano normal ao eixo, como representado pela seguinte figura:
                \begin{figure}[H]
                    \centering
                    \includegraphics[height = 4cm]{ima15.png}
                    \caption{Movimento Rotacional em Eixo Fixo}
                \end{figure}\noindent
            Onde:
                \begin{enumerate}
                    \item $\vec{v}$, \textbf{Velocidade}:
                        \begin{equation}
                            \boxed{
                                \vec{v} = \vec{\omega}\times\vec{r}
                            }
                        \end{equation}

                    \item $\vec{a}$, \textbf{Aceleração}:
                        \begin{equation}
                            \boxed{
                                \vec{a} = 
                                \vec{\dot{\omega}}\times\vec{r} +
                                \vec{\omega}\times(\vec{\omega}\times\vec{r})
                            }
                        \end{equation}
                \end{enumerate}
            Note neste caso que utilizam-se coordenadas normais e tangenciais para representação das acelerações nos corpos.

        \subsection{Movimento em Planos Paralelos}
            \paragraph{Definição}Quando todos os pontos em um corpo rígido se deslocam em planos paralelos ao \textbf{Plano de Movimento} $P$, escolhido através do centro de massa $G$. Desta maneira qualquer ponto poderá ser analisado a partir da cinemática plana de seu equivalente em $P$, como representado pela seguinte figura:
                \begin{figure}[H]
                    \centering
                    \includegraphics[height = 4cm]{ima16.png}
                    \caption{Movimento em Planos Paralelos}
                \end{figure}\noindent
            Desta maneira, quando $\omega_{x} = \omega_{y} = 0$ e $\omega_{z} \ne 0$ a quantidade de movimento angular e momento angular serão dados pelas seguintes equações:
                \begin{equation}
                    \boxed{
                        H_{x} = -I_{xz}\omega_{z}
                    }
                    \qquad
                    \boxed{
                        H_{y} = -I_{yz}\omega_{z}
                    }
                    \qquad
                    \boxed{
                        H_{z} = -I_{zz}\omega_{z}
                    }
                \end{equation}
                \begin{equation}
                    \boxed{
                        \Sigma M_{x} = -I_{xz}\dot{\omega}_{z} + I_{yz}\omega_{z}^{2}
                    }
                    \qquad
                    \boxed{
                        \Sigma M_{y} = -I_{yz}\dot{\omega}_{z} + I_{xz}\omega_{z}^{2}
                    }
                    \qquad
                    \boxed{
                        \Sigma M_{z} = I_{zz}\dot{\omega}_{z}
                    }
                \end{equation}
\newpage
        \subsection{Eixo Instantâneo de Rotação}
            \paragraph{Introdução}Primeiramente será necessário analisar o comportamento de um corpo rígido em movimento no espaço, enfatizando suas rotações. Desta forma, considera-se as seguintes figuras que realizam operações idênticas em ordens distintas:
                \begin{figure}[H]
                    \centering
                    \begin{subfigure}[t]{0.45\textwidth}
                        \centering
                        \includegraphics[height = 3cm]{ima17.png}
                        \caption{Rotação em $x$ e $y$}
                    \end{subfigure}
                    \begin{subfigure}[t]{0.45\textwidth}
                        \centering
                        \includegraphics[height = 3cm]{ima18.png}
                        \caption{Rotação em $y$ e $x$}
                    \end{subfigure}
                    \caption{Rotações Finitas}
                \end{figure}\noindent
            Note que o ponto final, 3, termina em posições diferentes a depender dos movimentos realizados. Assim, torna-se evidente que rotações finitas não devem ser tratadas como vetores, pois não apresentam comutabilidade em sua execução. Entretanto quando considera-se rotações infinitesimais as rotações são equivalentes a uma única rotação, implicando na seguinte relação:
                \begin{equation}
                    \boxed{
                        \vec{\omega} = 
                        \vec{\omega}_{1} + 
                        \vec{\omega}_{2}
                    }
                \end{equation}

            \paragraph{Definição}Considera-se um corpo rígido rotacionado entorno de dois eixos com um ponto fixo $O$, este gira instaneamente em torno do \textbf{Eixo Instantâneo de Rotação} que passa por $O$, como representado na figura abaixo:
                \begin{figure}[H]
                    \centering
                    \includegraphics[height = 3cm]{ima20.png}
                    \caption{Representação Eixo Instantâneo de Rotação}\label{figEixo}
                \end{figure}\noindent
            Desta maneira, qualquer outro ponto $P$ deste corpo girará instantaneamente em arcos circulares em torno deste eixo. Representando a rotação deste eixo instantâneo entorno dos demais eixos obtêm-se as seguintes figuras:
                \begin{figure}[H]
                    \centering
                    \begin{subfigure}[t]{0.45\textwidth}
                        \centering
                        \includegraphics[height = 3cm]{ima21.png}
                        \caption{Cones Circulares}
                    \end{subfigure}
                    \begin{subfigure}[t]{0.45\textwidth}
                        \centering
                        \includegraphics[height = 3cm]{ima22.png}
                        \caption{Cones não Circulares}
                    \end{subfigure}
                    \caption{Representação dos Cones de Movimento}
                \end{figure}\noindent
            Onde:
                \begin{enumerate}[rightmargin = \leftmargin, noitemsep]
                    \item \textbf{Cone do Corpo}: Representa a rotação do eixo instantâneo entorno do corpo rígido;
                    \item \textbf{Cone Espacial}: Representa a rotação do eixo instantâneo entorno do eixo vertical;
                \end{enumerate}
            Baseado nesta definições será necessário descrever o movimento com base nesta variáveis. Assim considera-se as seguintes figuras:
                \begin{figure}[H]
                    \centering
                    \begin{subfigure}[t]{0.45\textwidth}
                        \centering
                        \includegraphics[height = 3cm]{ima23.png}
                        \caption{Cone do Corpo}
                    \end{subfigure}
                    \begin{subfigure}[t]{0.45\textwidth}
                        \centering
                        \includegraphics[height = 3cm]{ima24.png}
                        \caption{Cone Espacial}
                    \end{subfigure}
                    \caption{Definição das Variáveis dos Cones de Movimento}
                \end{figure}\noindent
            Onde:
                \begin{enumerate}[rightmargin = \leftmargin]
                    \item $\vec{\alpha}$, \textbf{Aceleração Angular}: Representa a derivada no tempo da velocidade angular do corpo rígido, refletindo a variação tanto no módulo quanto na direção de $\vec{\omega}$, sendo dada pela seguinte equação:
                        \begin{equation}
                            \boxed{
                                \vec{\alpha} = 
                                \underbrace{
                                    \vec{\dot{\omega}}_{1} +
                                    \vec{\dot{\omega}}_{2} 
                                }_{\text{Módulo}} +
                                \underbrace{
                                    \vec{\Omega}\times\vec{\omega}
                                }_{\text{Direção}}
                            }
                        \end{equation}
                    Onde:
                        \begin{enumerate}[rightmargin = \leftmargin]
                            \item $\vec{\dot{\omega}}_{1}$, \texttt{Aceleração Angular de $\vec{\omega}_{1}$}: Representa a aceleração angular com que o \textbf{Cone do Corpo} rotaciona;

                            \item $\vec{\dot{\omega}}_{2}$, \texttt{Aceleração Angular de $\vec{\omega}_{2}$}: Representa a aceleração angular com que o \textbf{Cone Espacial} rotaciona;

                            \item $\vec{\Omega}$, \texttt{Velocidade Angular de $\vec{\omega}$}: Representa a velocidade angular com que o vetor $\vec{\omega}$ gira entorno do cone espacial, considerando a figura \ref{figEixo} a rotação $\vec{\omega}_{2}$ do corpo rígido;
                        \end{enumerate}
                    Caso $\vec{\omega}$, rotação do corpo entorno do eixo instantâneo de rotação, possuir módulo constante, então equação será simplificada:
                        \begin{equation}
                            \boxed{
                                \vec{\alpha} = 
                                \underbrace{
                                    \vec{\Omega}\times\vec{\omega}
                                }_{\text{Direção}}
                            }
                        \end{equation}
                    Neste cenário a aceleração $\vec{\alpha}$ será normal a $\vec{\omega}$.
                \end{enumerate}
\newpage
        \subsection{Movimento Relativo Translacional}
            \paragraph{Definição}Análise cinemática de um corpo rígido em movimento tridimensional através do movimento relativo, onde será necessário definir um referencial inercial, $O$, e um referencial não inercial, $B$, apenas translacional como ilustrado pela seguinte figura:
                \begin{figure}[H]
                    \centering
                    \includegraphics[height = 4cm]{ima25.png}
                    \caption{Representação Movimento Relativo Translacional}
                \end{figure}\noindent
            Nesta configuração, como a distância entre $AB$ permanece constante qualquer ponto $A$ visto a partir de $B$ apresentará apenas rotação entorno de uma superfície esférica centrada em $B$. Desta maneira, pode-se descrever o movimento como uma translação do corpo em $B$ e uma rotação entorno de $B$ descrito pelas seguintes equações:
                \begin{enumerate}[rightmargin = \leftmargin]
                    \item $\vec{v}_{A}$, \textbf{Velocidade:} Vetorialmente a velocidade do ponto $A$ será descrita como a composição entre o movimento rotacional,  translacional e relativo, representado pela seguinte equação:
                        \begin{equation}
                            \boxed{
                                \vec{v}_{A} = 
                                \vec{v}_{B} + 
                                \underbrace{
                                    \omega_{AB}\times\vec{r}_{AB}
                                }_{\vec{v}_{AB}}
                            }
                        \end{equation}
                    Onde:
                        \begin{enumerate}[rightmargin = \leftmargin, noitemsep]
                            \item $\vec{v}_{AB}$, \texttt{Velocidade Devido a Rotação:} Velocidade causada pela rotação de $AB$.
                        \end{enumerate}

                    \item $\vec{a}_{A}$, \textbf{Aceleração:} Vetorialmente a aceleração do ponto $A$ será descrita como a composição entre o movimento rotacional e translacional, representado pela seguinte equação:
                        \begin{equation}
                            \boxed{
                                \vec{a}_{A} = 
                                \vec{a}_{B} + 
                                \underbrace{
                                    \omega_{AB}\times(\omega_{AB}\times\vec{r}_{AB})
                                }_{\vec{a}_{AB_{n}}} + 
                                \underbrace{
                                    \dot{\omega}_{AB}\times\vec{r}_{AB}
                                }_{\vec{a}_{AB_{t}}}
                            }
                        \end{equation}
                    Onde:
                        \begin{enumerate}[rightmargin = \leftmargin]
                            \item $\vec{a}_{AB_{t}}$, \texttt{Aceleração Devido a Rotação Tangencial:} Causada pela rotação de $AB$, tangencial a trajetória considerada a partir do \textbf{Centro de Velocidade Nula};

                            \item $\vec{a}_{AB_{n}}$, \texttt{Aceleração Devido a Rotação Normal:} Causada pela rotação de $AB$, normal a trajetória considerada a partir do \textbf{Centro de Velocidade Nula};
                        \end{enumerate}
                \end{enumerate}
            Nota-se que conexões em juntas esféricas haverão rotação entorno de seu próprio eixo que não afeta sua ação e, portanto, sua velocidade angular será normal à haste.
\newpage
        \subsection{Movimento Relativo Geral}
            \paragraph{Definição}Análise cinemática de um corpo rígido em movimento tridimensional através do movimento relativo, onde será necessário definir um referencial inercial, $O$, e um referencial não inercial, $B$, como ilustrado pela seguinte figura:
                \begin{figure}[H]
                    \centering
                    \includegraphics[height = 4cm]{ima26.png}
                    \caption{Representação Movimento Relativo Geral}
                \end{figure}\noindent
            Nesta configuração o referencial não inercial apresentará uma rotação descrita por $\vec{\Omega}$ que poderá ser diferente da velocidade angular absoluta $\vec{\omega}$ do corpo.
                \begin{enumerate}[rightmargin = \leftmargin, noitemsep]
                    \item $\vec{r}_{A}$, \textbf{Posição A:} Representação vetorial de $A$ com relação a origem inercial $O$;

                    \item $\vec{r}_{B}$, \textbf{Posição B:} Representação vetorial de $B$ com relação a origem inercial $O$;

                    \item $\vec{r}_{AB}$, \textbf{Posição AB:} Representação vetorial de $A$ com relação a origem não inercial $B$, representado pela seguinte equação:
                        \begin{equation*}
                            \vec{r}_{AB} = 
                            \underbrace{
                                (x_{B} - x_{A})
                            }_{x_{AB}} + 
                            \underbrace{
                                (y_{B} - y_{A})
                            }_{y_{AB}} + 
                            \underbrace{
                                (z_{B} - z_{A})
                            }_{z_{AB}}
                        \end{equation*}
                    Onde:
                        \begin{enumerate}[rightmargin = \leftmargin]
                            \item \texttt{Derivadas Direcionais:}
                                \begin{equation}
                                    \boxed{
                                        \dot{\vec{i}} = 
                                        \vec{\Omega}\times\vec{i}
                                    }
                                    \qquad
                                    \boxed{
                                        \dot{\vec{j}} = 
                                        \vec{\Omega}\times\vec{j}
                                    }
                                    \qquad
                                    \boxed{
                                        \dot{\vec{k}} = 
                                        \vec{\Omega}\times\vec{k}
                                    }
                                \end{equation}\noindent
                        \end{enumerate}

                    \item $\vec{v}_{A}$, \textbf{Velocidade:} Vetorialmente a velocidade do ponto $A$ será descrita como a composição entre o movimento rotacional,  translacional e relativo, representado pela seguinte equação:
                        \begin{equation}
                            \boxed{
                                \vec{v}_{A} = 
                                \vec{v}_{B} + 
                                \underbrace{
                                    \Omega\times\vec{r}_{AB}
                                }_{\vec{v}_{AB}} + 
                                \vec{v}_{R}
                            }
                        \end{equation}
                    Onde:
                        \begin{enumerate}[rightmargin = \leftmargin, noitemsep]
                            \item $\Omega$, \texttt{Aceleração Angular Eixos:} Representa a aceleração angular que os eixos relativos, representado pela seguinte equação:
                                \begin{equation}
                                    \boxed{
                                        \vec{\Omega} = 
                                        \Omega_{x}\vec{i} + 
                                        \Omega_{y}\vec{j} + 
                                        \Omega_{z}\vec{k}
                                    }
                                \end{equation}
                            % Note que sua derivada deverá realizar a regra da cadeia entre as direções e as velocidades sobre cada eixo como demonstrado:
                            %     \begin{equation*}
                            %         \dot{\vec{\Omega}} = 
                            %         \Omega_{x}\dot{\vec{i}} + 
                            %         \dot{\Omega}_{x}\vec{i} + 
                            %         \Omega_{y}\dot{\vec{y}} + 
                            %         \dot{\Omega}_{y}\vec{y} + 
                            %         \Omega_{z}\dot{\vec{z}} + 
                            %         \dot{\Omega}_{z}\vec{z}
                            %     \end{equation*}

                            \item $\vec{v}_{AB}$, \texttt{Velocidade Devido a Rotação:} Velocidade causada pela rotação de $AB$;

                            \item $\vec{v}_{R}$, \texttt{Velocidade Relativa:} Velocidade de $A$ observada com relação a $B$, ou seja, há deslocamento entre os pontos, obtido pela seguinte equação:
                                \begin{equation*}
                                    \vec{v}_{R} = 
                                    \dot{x}_{AB} + 
                                    \dot{y}_{AB} + 
                                    \dot{z}_{AB}
                                \end{equation*}
                            Note que este termo será nulo caso $\vec{\Omega} = \vec{\omega}$;
                        \end{enumerate}

                    \item $\vec{a}_{A}$, \textbf{Aceleração:} Vetorialmente a aceleração do ponto $A$ será descrita como a composição entre o movimento rotacional, translacional, relativo e de \textbf{Coriolis}, representado pela seguinte equação:
                        \begin{equation}
                            \boxed{
                                \vec{a}_{A} = 
                                \vec{a}_{B} + 
                                \underbrace{
                                    \Omega\times(\Omega\times\vec{r}_{AB})
                                }_{\vec{a}_{AB_{n}}} + 
                                \underbrace{
                                    \dot{\Omega}\times\vec{r}_{AB}
                                }_{\vec{a}_{AB_{t}}} + 
                                \underbrace{
                                    2\Omega\times\vec{v}_{R}
                                }_{\vec{a}_{C}} + 
                                \vec{a}_{R}
                            }
                        \end{equation}
                    Onde:
                        \begin{enumerate}[rightmargin = \leftmargin]
                            \item $\vec{a}_{AB_{t}}$, \texttt{Aceleração Devido a Rotação Tangencial:} Causada pela rotação de $AB$, tangencial a trajetória considerada a partir do \textbf{Centro de Velocidade Nula};

                            \item $\vec{a}_{AB_{n}}$, \texttt{Aceleração Devido a Rotação Normal:} Causada pela rotação de $AB$, normal a trajetória considerada a partir do \textbf{Centro de Velocidade Nula};

                            \item $\vec{a}_{C}$, \texttt{Aceleração de Coriolis:} Causada pela rotação percebida apenas por observadores no sistema rotativo;

                            \item $\vec{a}_{R}$, \texttt{Aceleração de Relativa:} Aceleração de $A$ observada com relação a $B$, obtido pela seguinte equação:
                                \begin{equation*}
                                    \vec{a}_{R} = 
                                    \ddot{x}_{AB} + 
                                    \ddot{y}_{AB} + 
                                    \ddot{z}_{AB}
                                \end{equation*}
                            Note que este termo será nulo caso $\vec{\Omega} = \vec{\omega}$;
                        \end{enumerate}
                \end{enumerate}
\newpage

        \subsection{Impulso e Quantidade de Movimento Angular}
            \paragraph{Definição}Considere um corpo rígido se deslocando no espaço, onde os eixos $x-y-z$ centrados com o centro de massa $G$ do corpo rígido em movimento e os eixos $X-Y-Z$ fixos na origem do sistema como representado pela seguinte figura:
                \begin{figure}[H]
                    \centering
                    \includegraphics[height = 4cm]{ima27.png}
                    \caption{Representação Movimento Angular}
                \end{figure}\noindent
            Desta forma, quantidade de movimento angular do corpo rígido com relação ao centro de massa G pode ser escrita como:
                \begin{equation}
                    \boxed{
                        \vec{H}_{G} = 
                        -\vec{\overline{v}}\times\Sigma m_{i}\vec{\rho}_{i} + 
                        \Sigma
                        \left[
                            \vec{\rho}_{i}\times m_{i} (\vec{\omega}\times\vec{\rho}_{i})
                        \right]
                    }
                \end{equation}
            Onde:
                \begin{enumerate}[rightmargin = \leftmargin, noitemsep]
                    \item $\vec{\overline{v}}$, \textbf{Velocidade Linear do Centro de Massa};
                    \item $\vec{\omega}$, \textbf{Velocidade Angular do Centro de Massa};
                \end{enumerate}
            Entretanto nota-se que a expressão será complexa e, em muitos casos, não útil para sua aplicação. Desvolve-se os somatórios através da aplicação de integrais para obter a seguinte expressão:
                \begin{equation}
                    \boxed{
                        \vec{H} = 
                        \begin{bmatrix}
                            \vec{i} & \vec{j} & \vec{k}\\
                        \end{bmatrix} \times 
                        \underbrace{
                            \begin{bmatrix}
                                +I_{xx} & -I_{xy} & -I_{xz}\\
                                -I_{yx} & +I_{yy} & -I_{yz}\\
                                -I_{zx} & -I_{zy} & +I_{zz}\\
                            \end{bmatrix}
                        }_{\text{Tensor de Inércia}} \times
                        \begin{bmatrix}
                            \omega_{x}\\
                            \omega_{y}\\
                            \omega_{z}\\
                        \end{bmatrix} = 
                        \begin{cases}
                            \vec{H}_{x} = (+I_{xx}\omega_{x} - I_{xy}\omega_{y} - I_{xz}\omega_{z})\vec{i}\\
                            \vec{H}_{y} = (-I_{yx}\omega_{x} + I_{yy}\omega_{y} - I_{yz}\omega_{z})\vec{j}\\
                            \vec{H}_{z} = (-I_{zx}\omega_{x} - I_{zy}\omega_{y} + I_{zz}\omega_{z})\vec{k}\\
                        \end{cases}
                    }
                \end{equation}
            Onde:
                \begin{enumerate}[rightmargin = \leftmargin]
                    \item $I_{xx}$ $I_{yy}$ $I_{zz}$, \textbf{Momento de Inércia nos Eixos:} Representam os momentos de inércias dos corpos com relação aos respectivos eixos e representados pelas seguintes equações:
                        \begin{equation}
                            \boxed{I_{xx} = \int(y^2 + z^2)dm} \qquad 
                            \boxed{I_{yy} = \int(x^2 + z^2)dm} \qquad 
                            \boxed{I_{zz} = \int(x^2 + y^2)dm}
                        \end{equation}
                    Note que estes momentos de inércia podem ser calculados pelas equações definidas na Figura \ref{momentoInercia}, evitando assim o cálculo das integrais.

                    \item $I_{xy}$ $I_{xz}$ $I_{yz}$, \textbf{Produtos de Inérica nos Eixos:} Representam os momentos de inércias dos corpos com relação aos respectivos eixos, sendo estes simétricos e representados pelas seguintes equações:
                        \begin{equation}
                            \boxed{I_{xy} = I_{yx} = \int(xy)dm} \qquad 
                            \boxed{I_{xz} = I_{zx} = \int(xz)dm} \qquad 
                            \boxed{I_{yz} = I_{zy} = \int(yz)dm}
                        \end{equation}
                \end{enumerate}

            \paragraph{Eixos Principais}Eixos únicos em que os valores dos momentos de inércia assumem valores estacionários e o Tensero de Inércia será diagonal.

            \paragraph{Princípio de Transferência}Quantidade de movimento angular em relação a um ponto $P$ qualquer será análogo ao desenvolvimento para forças em binário com representado na seguinte figura, obtido pela equação na sequência:
                \begin{figure}[H]
                    \centering
                    \includegraphics[height = 4cm]{ima28.png}
                    \caption{Representação Princípio de Transferência}
                \end{figure}
                \begin{equation}
                    \boxed{
                        \vec{H}_{P} = \vec{H}_{G} + \overline{r}\times\vec{G}
                    }
                \end{equation}
            Onde:
                \begin{enumerate}[rightmargin = \leftmargin, noitemsep]
                    \item $\vec{G}$, \textbf{Quantidade de Movimento Linear:} Representa a quantidade de movimento linear dos corpos com relação a seu centro de massa e representado pela seguinte equação:
                        \begin{equation}
                            \boxed{
                                \vec{G} = m\vec{\overline{v}}
                            }
                        \end{equation}
                \end{enumerate}
\newpage

        \subsection{Equações Gerais}
            \paragraph{Definição}Nesta abordagem define-se as equações gerais de movimento para um sistema de massa constante:
                \begin{equation}
                    \boxed{
                        \Sigma\vec{F} = \vec{\dot{G}}
                    }
                    \qquad
                    \boxed{
                        \Sigma\vec{M} = \vec{\dot{H}}
                    }
                \end{equation}
            Quando $\vec{H}$ é expresso em termos dos componentes medidos em relação a um sistema móvel $x-y-z$ que possui velocidade angular $\vec{\Omega}$ então o momento será:
                \begin{equation}
                    \boxed{
                        \Sigma\vec{M} = 
                        \Sigma\vec{M}_{x} + 
                        \Sigma\vec{M}_{y} + 
                        \Sigma\vec{M}_{z} = 
                        \begin{cases}
                            \Sigma\vec{M}_{x} = (\dot{H}_{x} - H_{y}\Omega_{z} + H_{z}\Omega_{y})\vec{i}\\
                            \Sigma\vec{M}_{y} = (\dot{H}_{y} - H_{z}\Omega_{x} + H_{x}\Omega_{z})\vec{j}\\
                            \Sigma\vec{M}_{z} = (\dot{H}_{z} - H_{x}\Omega_{y} + H_{y}\Omega_{x})\vec{k}\\
                        \end{cases}
                    }
                \end{equation}
            Onde:
                \begin{equation}
                    \boxed{
                        \vec{H} = 
                        \begin{cases}
                            \vec{H}_{x} = (+I_{xx}\omega_{x} - I_{xy}\omega_{y} - I_{xz}\omega_{z})\vec{i}\\
                            \vec{H}_{y} = (-I_{yx}\omega_{x} + I_{yy}\omega_{y} - I_{yz}\omega_{z})\vec{j}\\
                            \vec{H}_{z} = (-I_{zx}\omega_{x} - I_{zy}\omega_{y} + I_{zz}\omega_{z})\vec{k}\\
                        \end{cases}
                    }
                \end{equation}

            \paragraph{Planos Paralelos}Quando o movimento ocorrer ao longo de planos paralelos com $\omega_{x} = \omega_{y} = 0$ e $\omega_{z} \ne 0$ as equações serão:
                \begin{equation}
                    \boxed{
                        \Sigma\vec{M} = 
                        \Sigma\vec{M}_{x} + 
                        \Sigma\vec{M}_{y} + 
                        \Sigma\vec{M}_{z} = 
                        \begin{cases}
                            \Sigma\vec{M}_{x} = (- I_{xz}\dot{\omega}_{z} + I_{yz}\omega_{z}^{2})\vec{i}\\
                            \Sigma\vec{M}_{y} = (- I_{yz}\dot{\omega}_{z} - I_{xz}\omega_{z}^{2})\vec{j}\\
                            \Sigma\vec{M}_{z} = (+ I_{zz}\dot{\omega}_{z})\vec{k}\\
                        \end{cases}
                    }
                \end{equation}
            Onde:
                \begin{equation}
                    \boxed{
                        \vec{H} = 
                        \begin{cases}
                            \vec{H}_{x} = (- I_{xz}\omega_{z})\vec{i}\\
                            \vec{H}_{y} = (- I_{yz}\omega_{z})\vec{j}\\
                            \vec{H}_{z} = (+ I_{zz}\omega_{z})\vec{k}\\
                        \end{cases}
                    }
                \end{equation}

            \paragraph{Eixos Principais}Quando os eixos de referências coincidem com os eixos principais de inércia com origem em G ou em um ponto O fixo do corpo e no espaço terá os produtos de inércia nulos e, portanto, as seguintes equações são denominadas \textbf{Equações de Euler}:
                \begin{equation}
                    \boxed{
                        \Sigma\vec{M} = 
                        \Sigma\vec{M}_{x} + 
                        \Sigma\vec{M}_{y} + 
                        \Sigma\vec{M}_{z} = 
                        \begin{cases}
                            \Sigma\vec{M}_{x} = (I_{xx}\dot{\omega}_{x} - (I_{yy} - I_{zz})\omega_{y}\omega_{z})\vec{i}\\
                            \Sigma\vec{M}_{y} = (I_{yy}\dot{\omega}_{y} - (I_{zz} - I_{xx})\omega_{z}\omega_{x})\vec{j}\\
                            \Sigma\vec{M}_{z} = (I_{zz}\dot{\omega}_{z} - (I_{xx} - I_{yy})\omega_{x}\omega_{y})\vec{k}\\
                        \end{cases}
                    }
                \end{equation}
\newpage

        \subsection{Trabalho e Energia Geral}
            \paragraph{Definição}Baseia-se nos resultados obtidos para cinética de corpos rígidos para expressar essas grandezas no movimento tridimensional que apresentam movimento translacional e rotacional. Recomenda-se analisar as variações individuais entre cada componente descrita pela seguinte equação:
                \begin{equation}
                    \boxed{
                        T_{1} + V_{1} + U_{1-2} = T_{2} + V_{2}
                    }
                \end{equation}
            Onde:
                \begin{enumerate}[rightmargin = \leftmargin]
                    \item $U$, \textbf{Impulso:} Causado pela presença de forças realizando trabalho ou momentos sendo aplicados. Quando um momento constante for aplicado ao corpo rígido será necessário multiplicá-lo pela comprimento da trajetória para obter as equações necessárias.\\
                    Caso não hajam forças ou momentos externos sobre o sistema então este termo será nulo, portanto a análise se dará para as variações termo a termo.

                    \item $T$, \textbf{Energia Cinética:} Analisa-se conjuntamento a parcela gerada pela translação e pela rotação, obtendo:
                        \begin{equation}
                            \boxed{
                                T_{i} = 
                                \underbrace{
                                    \frac{1}{2} \vec{\overline{v}}\cdot\vec{G}
                                }_{\text{Translação}} + 
                                \underbrace{
                                    \frac{1}{2}\vec{\omega}\cdot\vec{H}_{G}
                                }_{\text{Rotação}}
                            }
                        \end{equation}
                    Onde:
                        \begin{enumerate}[rightmargin = \leftmargin, noitemsep]
                            \item $\vec{\overline{v}}$, \texttt{Velocidade Linear do Centro de Massa};

                            \item $\vec{G}$, \texttt{Quantidade de Movimento Linear no Centro de Massa}; \vspace{5mm}

                            \item $\vec{\omega}$, \texttt{Velocidade Angular do Centro de Massa};

                            \item $\vec{H}_{G}$, \texttt{Quantidade de Movimento Angular no Centro de Massa};
                        \end{enumerate}
                    Note que realiza-se o produto escalar entre os vetores envolvidos, desta maneira relembra-se sua execução através da seguinte equação:
                        \begin{equation}
                            \boxed{
                                \vec{a} \cdot \vec{b} = 
                                a_{x} \cdot b_{x} + 
                                a_{y} \cdot b_{y} + 
                                a_{z} \cdot b_{z}
                            }
                        \end{equation}

                    \item $V$, \textbf{Energia Potencial:} Analisa-se conjuntamento potenciais geradas pela gravidade e pela presença de molas, obtendo:
                        \begin{equation}
                            \boxed{
                                V_{i} = 
                                m g h +
                                \frac{1}{2} k x^{2} 
                            }
                        \end{equation}
                    Onde:
                        \begin{enumerate}[rightmargin = \leftmargin, noitemsep]
                            \item $h$, \texttt{Variação de Altura};
                            \item $x$, \texttt{Variação de Comprimento};
                            \item $k$, \texttt{Constante Elástica};
                        \end{enumerate}
                \end{enumerate}
\newpage

        \subsection{Movimento Giroscópico com Precessão}
            \paragraph{Definição}Efeito de movimento que ocorre sempre que o eixo em torno do qual um corpo rotaciona também apresenta giro em torno de outro eixo. Há abordagens mais complexas deste movimento, entretanto considera-se apenas a sua versão simplificado onde o movimento ocorre entorno dos eixos com velocidades constantes como representado pela seguinte figura:
                \begin{figure}[H]
                    \centering
                    \includegraphics[height = 4cm]{ima29.png}
                    \caption{Movimento Giroscópio}
                \end{figure}\noindent
            Onde:
                \begin{enumerate}[rightmargin = \leftmargin, noitemsep]
                    \item $\vec{p}$, \textbf{Velocidade de Giro};
                    \item $\vec{M}$, \textbf{Momento:} Somatório de momentos medidos a partir do centro avaliado $O$;
                    \item $\vec{\Omega}$, \textbf{Velocidade de Precessão};
                \end{enumerate}
            Desta maneira, considera-se que para descrever o movimento a seguinte equação poderá ser empregada:
                \begin{equation}
                    \boxed{
                        \Sigma\vec{M}_{O} = I \vec{\Omega}\times\vec{p}
                    }
                \end{equation}
                \begin{figure}[H]
                    \centering
                    \begin{subfigure}[t]{0.3\linewidth}
                        \centering
                        \includegraphics[height = 2cm]{ima31.png}
                        \caption{Precessão em x}
                    \end{subfigure}
                    \begin{subfigure}[t]{0.3\linewidth}
                        \centering
                        \includegraphics[height = 2cm]{ima30.png}
                        \caption{Precessão em y}
                    \end{subfigure}
                    \begin{subfigure}[t]{0.3\linewidth}
                        \centering
                        \includegraphics[height = 2cm]{ima32.png}
                        \caption{Precessão em z}
                    \end{subfigure}
                    \caption{Orientações Precessão}
                \end{figure}
\newpage

    \section{Mecânica Analítica}
        \paragraph{Definição}Desenvolvida por Leibnitz e Lagrange, considerando o sistema como um todo para formular problemas a partir de duas quantidades escalares fundamentais: energia cinética e energia potencial. Desta maneira será necessário introduzir e aplicar coordenadas generalizadas para expandir a versatilidade das equações do movimento independente do sistema de coordenadas utilizado.

        \subsection{Vínculos}
            \paragraph{Definição}Movimento deverá ser descrito considerando quaisquer restrições impostas pelo sistema mecânico, sejam geométricas ou cinemáticas, que devem ser adequadamente descritas, podendo ser classificados como:
                \begin{enumerate}[rightmargin = \leftmargin]
                    \item \textbf{Holônomos:} Se $\xi_{1}$, ..., $\xi_{M}$ são coordenadas arbitrárias usadas para descrever a configuração de um sistema mecânico, então estes conjunto será holônomo se puder ser descrito pela seguinte equação:
                        \begin{equation}
                            \boxed{
                                f(\xi_{1}, ..., \xi_{M}, t) = 0
                            }
                        \end{equation}
                    Note que se um vínculo, originalmente expresso em função de velocidades puder ser reduzido por uma integração a equação assim, este também será holônomo. Isto ocorre tipicamente quando um cilíndro de raio $R$ rola sem deslizamento ao longo de uma linha reta onde a posição do centro de massa será dada por $x$ e o ângulo de rotação entorno do centro demassa será dado por $\phi$ como mostrado na seguinte equação:
                        \begin{equation}
                            \boxed{
                                \dot{x} = R \dot{\phi}
                            }
                        \end{equation}
                \end{enumerate}

        \subsection{Graus de Liberdade}
            \paragraph{Definição}Representa o número mínimo de parâmetros independentes necessários e suficientes para definir completamente sua posição no espaço a cada instante de tempo. Se a posição do sistema for descrito em $n$ coordenadas e há $m$ equações independentes, então este sistema possuirá $n-m$ graus de liberdade.

        \subsection{Coordenadas Generalizadas}
            \paragraph{Definição}Conjunto de coordenadas que possam ser aplicadas como referência para descrição do movimento dos corpos envolvidos, obedecendo as seguintes características:
                \begin{enumerate}[noitemsep]
                    \item Independentes entre si;
                    \item Caracterizam univocamente a configuração do sistema;
                    \item Vínculos identicamente satisfeitos;
                \end{enumerate}

        \subsection{Deslocamento Virtual}
            \paragraph{Definição}Mudança na sua configuração que resulta de uma variação instantânea arbitrária das suas coordenadas, mantendo as forças aplicadas e as condições de vínculo constantes, obedecendo as seguintes características:
                \begin{enumerate}[rightmargin = \leftmargin, noitemsep]
                    \item Quantidade diferencial, infinitesimal;
                    \item Tempo constante durante deslocamento;
                    \item Válido enquanto atenda as restrições;
                \end{enumerate}
            Onde, $\delta$ representa o operador semelhante ao operador diferencial $d$ estudado em cálculo em que seguinte equação será aplicada:
                \begin{equation}
                    \boxed{
                        \delta f = 
                        \diffp{f}{X}\delta X + 
                        \diffp{f}{Y}\delta Y + 
                        \diffp{f}{Z}\delta Z
                    }
                \end{equation}

        \subsection{Trabalho Virtual}
            \paragraph{Definição}Trabalho realizado durante deslocamentos virtuais por forças aplicadas ao sistema como descrito pela seguinte equação:
                \begin{equation}
                    \boxed{
                        \delta W = \Sigma\vec{F}_{i} \cdot \delta\vec{r}_{i}
                    }
                \end{equation}
            Nota-se que o trabalho virtual das forças de vínculo se anulam, permitindo simplificar a análise do sistema, sendo nomeado \textbf{Vínculo Ideal}. Desta maneira poderá ser enunciado para o \textbf{Princípio do Trabalho Virtual}:
                \begin{displayquote}[][]
                    Condição necessária e suficiente para o equilíbrio estático de um sistema inicialmente em repouso com vínculos ideias é que seja nulo o trabalho virtual realizado pelas forças apicadas durante deslocamentos virtuais arbitrários
                \end{displayquote}
            Desta forma será necessário analizar cada coordenada generalizada independetemente, pois cada um constitue uma equação nula.

        \subsection{Princípio de D'Alembert}
            \paragraph{Definição}Qualquer partícula de massa $m$ quando submetida a uma força $\vec{F}$ adiquire uma aceleração absoluta $\vec{a}$ de acordo com a \textbf{2\textsuperscript{a} Lei de Newton} em um referencial inercial, considerando o \textbf{Princípio do Trabalho Virtual} obtém-se a seguinte equação:
                \begin{equation}
                    \boxed{
                        \delta W = 
                        \Sigma
                        \left(
                            \vec{F}_{i} - \dot{\vec{G}}_{i}
                        \right) \cdot \delta\vec{r}_{i} = 0
                    }
                \end{equation}
            Nota-se que uma \textbf{Força de Inércia} será responsável pelo equilíbrio dinâmico das forças envolvidas. Desta maneira poderá ser enunciado:
                \begin{displayquote}[][]
                    Cada partícula do sistema encontra-se em equilíbrio sob uma força resultante, que é a soma de uma força rela com uma força efetiva igual a $-\dot{\vec{G}}_{i}$.
                \end{displayquote}

        \subsection{Forças Generalizadas}
            \paragraph{Definição}Considere um sistema de partículas cujas posições são especificadas pelas coordenadas cartesianas $x_{1}$, $x_{2}$, ..., $x_{k}$. Se as forças $F_{1}$, $F_{2}$, ..., $F_{k}$ são aplicadas nas coordenadas correspondentes na direção positiva, desta maneira pelo \textbf{Trabalho Virtual} o deslocamento virtual arbitrário será:
                \begin{equation*}
                    \delta W = 
                    \sum_{j=1}^{k} F_{j} \cdot 
                    \delta x_{j} = 0
                \end{equation*}
            Suponha que as coordenadas cartesianas sejam relacionadas com as coordenadas generalizadas na forma das seguintes equações:
                \begin{equation*}
                    x_{1} = 
                    f_{1} (q_{1}, ..., q_{n}, t)
                    \quad
                    \cdots
                    \quad
                    x_{n} = 
                    f_{n} (q_{1}, ..., q_{n}, t)
                \end{equation*}
            Nota-se que os deslocamentos virtuais das coordenadas cartesianas $x_{j}$ serão dadas pela diferenciação das funções que as relacionam com as coordenadas generalizadas, obtendo a seguinte equação:
                \begin{equation*}
                    \delta x_{j} = 
                    \sum_{i=1}^{k} \diffp{x_{j}}{{q_{i}}}\delta q_{i} + 
                    \diffp{x_{j}}{t}\delta t
                \end{equation*}
            Entretanto, deslocamentos virtuais implicam em tempo constante e portanto a derivada parcial com relação ao tempo será nula. Desta maneira a \textbf{Força Generalizada} $Q_{i}$ associada as coordenadas generalizadas $q_{i}$ obtida pelo trabalho virtual poderá ser reescrito como:
                \begin{equation}
                    \boxed{
                        \delta W = 
                        \sum_{i=1}^{k} Q_{i} \cdot 
                        \delta q_{i}
                    }
                    \quad
                    \text{onde}
                    \quad
                    \boxed{
                        Q_{i} = 
                        \sum_{j=1}^{n} F_{j} \diffp{x_{j}}{{q_{i}}}
                    }
                \end{equation}

        \subsection{Equações de Lagrange}
            \paragraph{Definção}Equações obtidas a partir do \textbf{Princípio de d'Alembert}, relacionando as energias envolvidas no sistema denotadas pelo Lagrangiano $L = T - V$ como demonstrado pela seguinte equação:
                \begin{equation}
                    \boxed{
                        \diff{}{t} \left(\diffp{L}{{\dot{q}_{i}}}\right) - 
                        \diffp{L}{{q_{i}}} + 
                        \diffp{R}{{\dot{q}_{i}}} = 
                        Q_{NC}
                    }
                \end{equation}
            Onde:
                \begin{enumerate}[rightmargin = \leftmargin, noitemsep]
                    \item \textbf{Considerações:}
                        \begin{enumerate}[noitemsep]
                            \item \texttt{Coordenadas:} Generalizadas representados por $q_{i}$;
                            \item \texttt{Vínculos:} Ideias e Holônomos;
                            \item \texttt{Referencial:} Inercial;
                        \end{enumerate}

                    \item $R$: \textbf{Função de Dissipação de Rayleigh:} Representa forças não conservativas proporcionais à velocidade, obtido pela seguinte equação:
                        \begin{equation}
                            Q'_{i} = -\diffp{R}{{\dot{q}_{i}}}
                            \quad\text{onde}\quad
                            R = \sum_{j=1}^{k} \frac{1}{2} c_{j} \dot{x}_{j}^{2}
                        \end{equation}
                        \begin{enumerate}[rightmargin = \leftmargin, noitemsep]
                            \item $c_{j}$: Constante de Amortecimento;
                            \item $Q'_{i}$: Forças generalizadas não deriváveis de uma função potencial;
                        \end{enumerate}
                \end{enumerate}
\end{document}